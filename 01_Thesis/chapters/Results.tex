% !TEX root = master.tex

\chapter{Results}
\label{Ch:Results}
%\pagenumbering{arabic}
This chapter covers the evaluation of reconstructions \(\tilde{f}\) obtained from the FCNN and the CNN through a comparison against reconstructions obtained from POD. Additionally, an analysis of the interpretability of the intrinsic variables \(\idhy\) and \(\idrare\) is provided. This chapter concludes with the attempt to create new states of the FOM with the FCNN and the CNN, which can be viewed as an online phase of MOR.\\ 
To begin with the benchmarking of POD and neural networks, the number of parameters to obtain \(\tilde{f}\) is contrasted. Beforehand solely the number of trainable parameters that compose both neural networks were called \(\frepar\). For this comparison, \(\frepar\) is extended to also include all elements of the left and right singular vectors as well as the singular values of POD. Additionally the amount of intrinsic variables used for reconstruction is set to \(p=3\) and \(p=5\) for \(\hy\) and \(\rare\) respectively. An exception is the CNN, that uses \(p=5\) independent from rarefaction level. A summary is provided in \cref{Tab: Parameters}.   
\begin{table}[htp]
	\centering
	\caption{Amount of parameters \(\frepar\) used to reconstruct \(f\), number  of intrinsic variables \(p\) and the corresponding $\L2$-Error for POD, FCNN and CNN.}
	\begin{tabular*}{16cm}{ @{\extracolsep{\fill}} c c c c c c c @{} }
		\toprule
		Algorithm & \multicolumn{2}{c}{Parameters \(\frepar\)} & \multicolumn{2}{c}{Int. variables \(p\)}& \multicolumn{2}{c}{$\L2$-Error} \\ [.5ex]
		 & \(\hy\)&\(\rare\)&\(\hy\)&\(\rare\)&\(\hy\)&\(\rare\)\\   
		\hline
		POD     & 15129 & 25225 & 3 & 5 & 0.0205 & 0.0087 \\
		FCNN 	& 2683 & 3725 & 3 & 5 & 0.0008 & 0.0009 \\
		CNN   	& 8246 & 8246 & 5 & 5 &	0.025 & 0.027\\
		\bottomrule
	\end{tabular*} \label{Tab: Parameters}
\end{table}
POD uses with 15129 and 25225 parameters to reconstruct \(\hy\) and \(\rare\) respectively the largest amount of parameters of all three algorithms. These yield \(\L2\)-errors of 0.0205 and 0.0087 respectively. Interestingly, the elevation of \(p\) amounts to an increase of paramters by approximately 1.7 which is comparable to the FCNN with an approximate increase of 1.4. The FCNN, which holds the best \(\L2\)-errors of 0.0008 and 0.0009 for \(\hy\) and \(\rare\) respectively, does so with the least amount of parameters. For reconstructing \(\hy\) solely 2683 and for the reconstruction of \(\rare\) solely 3725 parameters are used, which is a fraction of the need for POD. The second most populous alhorithm is the CNN which uses 8246 parameters for both rarefaction levels. The resulting \(\L2\)-errors with 0.025 for \(\hy\) and 0.027 for \(\rare\) are the largest of all three algorithms. \\

Next a qualitative analysis with actual reconstructions is presented. For this purpose the \(\L2\)-error over time \(t\), seen in \cref{Fig:ErrTime}, is used to localize the most challenging snapshot for each algorithm.\\ 
\begin{figure}[tp!]
	% This file was created by tikzplotlib v0.9.6.
\begin{tikzpicture}
\definecolor{color0}{rgb}{0.12156862745098,0.466666666666667,0.705882352941177}

\begin{groupplot}[group style={group size=2 by 1,horizontal sep=2cm}]
\nextgroupplot[
legend cell align={left},
legend style={draw=none, at={(0,1)},anchor= north west},
tick align=outside,
tick pos=left,
x grid style={white!69.0196078431373!black},
xmin=-0.006, xmax=0.126,
xtick style={color=black},
y grid style={white!69.0196078431373!black},
ymin=-0.000841513772702749, ymax=0.0304345512636816,
ytick style={color=black},
ylabel={Relative Error},
xlabel={\(t\)},
axis lines=left,
width=0.47\textwidth,
height =.45\textwidth,
x tick label style={/pgf/number format/fixed}
]
\addplot [semithick, red, mark=o, mark size=2, mark options={solid}]
table {%
0 0.0041745608742312
0.005 0.00772826090461208
0.01 0.00998129651030254
0.015 0.0114814636830063
0.02 0.0127128633875819
0.025 0.0138212914705674
0.03 0.0148590810264679
0.035 0.0158480900427887
0.04 0.0167989301972853
0.045 0.0177175305866807
0.05 0.0186076563966316
0.055 0.0194719780627452
0.06 0.0203125628459335
0.065 0.021131115773516
0.07 0.0219291044243208
0.075 0.0227078270777979
0.08 0.0234684520909674
0.085 0.0242120420959511
0.09 0.0249395698621951
0.095 0.0256519293912289
0.1 0.0263499442110898
0.105 0.0270343740437529
0.11 0.0277059205812353
0.115 0.0283652327959761
0.12 0.029012911943846
};
\addlegendentry{POD}
\addplot [semithick, color0, mark=pentagon, mark size=2, mark options={solid}]
table {%
0 0.000580125547132904
0.005 0.0016071267939448
0.01 0.00188504649223781
0.015 0.00133036247337402
0.02 0.00132628189845874
0.025 0.00132716953321487
0.03 0.00138970165580601
0.035 0.00142017716840674
0.04 0.00145638808517116
0.045 0.00148681063630736
0.05 0.00153865172400739
0.055 0.0015731085384213
0.06 0.00160068094345799
0.065 0.00164077357612924
0.07 0.00168465107889107
0.075 0.00170978225630885
0.08 0.00174559426233016
0.085 0.00178917826233095
0.09 0.00182042226136015
0.095 0.0018497979559777
0.1 0.00189400391034215
0.105 0.00192810267450896
0.11 0.00196043475749194
0.115 0.00199731114052969
0.12 0.00204309455217968
};
\addlegendentry{FCNN}
\addplot [semithick, green!50!black, mark=triangle, mark size=2, mark options={solid,rotate=180}]
table {%
0 0.00649221194908023
0.005 0.00637732213363051
0.01 0.00686583481729031
0.015 0.00666404515504837
0.02 0.00614608032628894
0.025 0.0067147696390748
0.03 0.00703709293156862
0.035 0.00886726658791304
0.04 0.00913859903812408
0.045 0.00808194372802973
0.05 0.00891359616070986
0.055 0.00899100676178932
0.06 0.00881004240363836
0.065 0.009485456161201
0.07 0.00957572367042303
0.075 0.00939757097512484
0.08 0.0101738022640347
0.085 0.0107985194772482
0.09 0.0102939195930958
0.095 0.0112069239839911
0.1 0.0109974481165409
0.105 0.0116838105022907
0.11 0.0127648552879691
0.115 0.0127696730196476
0.12 0.0131859742105007
};
\addlegendentry{CNN}

\nextgroupplot[
legend cell align={left},
legend style={draw=none,at={(0,1)},anchor= north west},
tick align=outside,
tick pos=left,
x grid style={white!69.0196078431373!black},
xmin=-0.006, xmax=0.126,
xtick style={color=black},
y grid style={white!69.0196078431373!black},
ymin=-0.00230221234608083, ymax=0.0611092213046212,
ytick style={color=black},
ylabel={Relative Error},
xlabel={\(t\)},
axis lines=left,
width=0.47\textwidth,
height =.45\textwidth,
x tick label style={/pgf/number format/fixed}
]
\addplot [semithick, red, mark=o, mark size=2, mark options={solid}]
table {%
0 0.00740144721634158
0.005 0.0106893808997395
0.01 0.0149150434602834
0.015 0.0187372384735488
0.02 0.0219134756297143
0.025 0.0245262733618508
0.03 0.0266961103723662
0.035 0.0285235128498648
0.04 0.0300844182519668
0.045 0.0314352845718027
0.05 0.0326184730907696
0.055 0.0336663066570384
0.06 0.0346038953801234
0.065 0.0354510718869172
0.07 0.0362237260413102
0.075 0.0369347412336944
0.08 0.0375946650603757
0.085 0.038212200327165
0.09 0.0387945721638091
0.095 0.0393478079345977
0.1 0.0398769544960087
0.105 0.0403862495460136
0.11 0.0408792586959919
0.115 0.0413589864772983
0.12 0.0418279671630029
};
\addlegendentry{POD}
\addplot [semithick, color0, mark=pentagon, mark size=2, mark options={solid}]
table {%
0 0.000580125547132904
0.005 0.00316589483113868
0.01 0.00644069989189273
0.015 0.0094435622115595
0.02 0.0117462910277023
0.025 0.0135083335876847
0.03 0.0147541401649302
0.035 0.0156776008463723
0.04 0.016359257000976
0.045 0.0168387643606902
0.05 0.0171780580682302
0.055 0.0174023878750501
0.06 0.0175410884132009
0.065 0.0176139846919089
0.07 0.0176235927054341
0.075 0.0175937947736521
0.08 0.0175227229486254
0.085 0.0174309194625182
0.09 0.0173139350118248
0.095 0.0171816191398723
0.1 0.0170365979594161
0.105 0.0168830016501064
0.11 0.0167221999919594
0.115 0.0165578630537945
0.12 0.0163915040253931
};
\addlegendentry{FCNN}
\addplot [semithick, green!50!black, mark=triangle, mark size=2, mark options={solid,rotate=180}]
table {%
0 0.00956460367888212
0.005 0.0124108670279384
0.01 0.0172932539135218
0.015 0.0221939664334059
0.02 0.0249469373375177
0.025 0.0283007752150297
0.03 0.0311273168772459
0.035 0.0343553461134434
0.04 0.036633376032114
0.045 0.0377688743174076
0.05 0.0399965308606625
0.055 0.041291106492281
0.06 0.0428216233849525
0.065 0.0447116009891033
0.07 0.0460423566401005
0.075 0.0471654385328293
0.08 0.048737995326519
0.085 0.049955528229475
0.09 0.0508080013096333
0.095 0.0526490993797779
0.1 0.053377740085125
0.105 0.0546315461397171
0.11 0.0561474151909351
0.115 0.0573081187903881
0.12 0.0582268834114075
};
\addlegendentry{CNN}
\end{groupplot}

\end{tikzpicture}

	\caption{\(\L2\)-error over time for POD, FCNN and CNN. Results for $\hy$ are displayed on the left, the results for $\rare$ are displayed on the right.}
	\label{Fig:ErrTime}
\end{figure}
With POD and the CNN the last timestep at $t=0.12s$ for both rarefaction levels is the most rich in the $L2$-Error. In contrast, the FCNN does not show a distinct time dependence of the \(\L2\)-error. Nonetheless, struggles at the onset at around $t=0.005s$ with $\hy$ and around \(t=0.005s\) and \(t=0.0115\) (in the beginning and at the end) with $\rare$ can be observed for the FCNN. Examples of reconstructions $\tilde{f}(x,v,t_i)$ with $t_i=0.12s$ and $x \in [0.375,0.75]$ are given in \cref{Fig: ErrWorst}.\\
\begin{figure}[tp!]
	% This file was created by tikzplotlib v0.9.6.
\begin{tikzpicture}

\begin{groupplot}[
group style={group size=4 by 2,
	horizontal sep= 1.1cm,
	vertical sep = 1.5cm},
tick align=outside,
tick pos=left,
x grid style={white!69.0196078431373!black},
xmin=0.375, xmax=0.75,
xtick style={color=black},
y grid style={white!69.0196078431373!black},
ymin=-10, ymax=10,
ytick style={color=black},
height=.26\textwidth,
width=.26\textwidth,
xlabel={\(x\)},
ylabel={\(v\)},
y label style={yshift=-1.4em}
]
\nextgroupplot[
]
\addplot graphics [
includegraphics cmd=\pgfimage,
xmin=0.375, xmax=0.75,
ymin=-10, ymax=10
] {Figures/Chapter_5/ErrWorst-000.png};
\node[fill=white] at (axis cs:0.65,15) {FOM};
\nextgroupplot[
]
\addplot graphics [
includegraphics cmd=\pgfimage,
xmin=0.375, xmax=0.75,
ymin=-10, ymax=10
] {Figures/Chapter_5/ErrWorst-001.png};
\node[fill=white] at (axis cs:0.65,15) {POD};
\nextgroupplot[
]
\addplot graphics [
includegraphics cmd=\pgfimage,
xmin=0.375, xmax=0.75,
ymin=-10, ymax=10
] {Figures/Chapter_5/ErrWorst-002.png};
\node[fill=white] at (axis cs:0.65,15) {FCNN};
\nextgroupplot[
colorbar,
colorbar style={
	ylabel={$\tilde{f}$},
	ytick={0,0.1,.3,.39},
	yticklabels={0,0.1,0.3,0.39},
	y label style={yshift=1.3cm},
	ticklabel style={font=\footnotesize},
	tick align=outside,
	tick pos=right,
	width=0.1*\pgfkeysvalueof{/pgfplots/parent axis width},
	xshift=-0.2cm
},
colormap/blackwhite,
point meta max=0.397007430832041,
point meta min=8.50982895819395e-73,
]
\addplot graphics [
includegraphics cmd=\pgfimage,
xmin=0.375, xmax=0.75,
ymin=-10, ymax=10
] {Figures/Chapter_5/ErrWorst-003.png};
\node[fill=white] at (axis cs:0.65,15) {CNN};
\nextgroupplot[
]
\addplot graphics [
includegraphics cmd=\pgfimage,
xmin=0.375, xmax=0.75,
ymin=-10, ymax=10
] {Figures/Chapter_5/ErrWorst-004.png};
\node[fill=white] at (axis cs:0.65,15) {FOM};
\nextgroupplot[
]
\addplot graphics [
includegraphics cmd=\pgfimage,
xmin=0.375, xmax=0.75,
ymin=-10, ymax=10
] {Figures/Chapter_5/ErrWorst-005.png};
\node[fill=white] at (axis cs:0.65,15) {POD};
\nextgroupplot[
]
\addplot graphics [
includegraphics cmd=\pgfimage,
xmin=0.375, xmax=0.75,
ymin=-10, ymax=10
] {Figures/Chapter_5/ErrWorst-006.png};
\node[fill=white] at (axis cs:0.65,15) {FCNN};
\nextgroupplot[
colorbar,
colorbar style={
	ylabel={$\tilde{f}$},
	ytick={0,0.1,.3,.4},
	yticklabels={0,0.1,0.3,0.4},
	y label style={yshift=1.3cm},
	ticklabel style={font=\footnotesize},
	tick align=outside,
	tick pos=right,
	width=0.1*\pgfkeysvalueof{/pgfplots/parent axis width},
	xshift=-0.2cm
},
colormap/blackwhite,
point meta max=0.406101604565777,
point meta min=1.26406789996295e-34
]
\addplot graphics [
includegraphics cmd=\pgfimage,
xmin=0.375, xmax=0.75,
ymin=-10, ymax=10
] {Figures/Chapter_5/ErrWorst-007.png};
\node[fill=white] at (axis cs:0.65,15) {CNN};
\end{groupplot}

\end{tikzpicture}

	\caption{Comparison of the FOM solution \(f\) with three reconstructions \(\tilde{f}\) obtained from POD, the FCNN and the CNN. Reconstrucions are shown at \(t=0.12s\) for \(x\in [0.375,0.75]\). Case $\hy$ is displayed in the top row, $\rare$ in the bottom row. The colobars reference \(f\) and \(\tilde{f}\).}
	\label{Fig: ErrWorst}
\end{figure}
The FOM solution viewed as $f(x,v,t_i)$ has been introduced in \cref{Ch:BGK}. There, $f(x_j,v,t_i)$ is the probability distribution of the microscopic velocities $v$ at point $x_j$ in space at one moment $t_i$ in time for a gas.
With this in mind, a qualitative comparison between the three algorithms can be made considering the rendition of the velocity probabilities. Starting with \(\hy\), seen in the top row of \cref{Fig: ErrWorst} one can observe that \(\tilde{f}(x,v,t_i)\) starting around \(x=0.6\) gets defective for POD and the CNN. Noteworthy, here the probability distribution is thinner as the original with POD. This in turn leads to errors in the temperature \(T\) once passing \(x\approx 0.6\). Prominent qualitative deviations using the CNN are especially blurriness/pixelation of \(\tilde{f}(x,v,t_i)\) after \(x\approx 0.6\). In contrast the FCNN seems to reproduce the FOM solution almost exactly.\\
Continuing with a row further down in \cref{Fig: ErrWorst} and therefore \(\rare\). The FCNN seems to reproduce the FOM solution without any visible drawback. Also POD seems to reproduce all important structures, except after \(x\approx 0.7\) around the contact discontinuity, some values for velocities with \(v>0\) appear to be missing. Again the CNN struggles with blurriness making \(\tilde{f}\) for both rarefaction levels look largely similar.    
\begin{figure}[H]
	% This file was created by tikzplotlib v0.9.8.
\begin{tikzpicture}

\definecolor{color0}{rgb}{0.12156862745098,0.466666666666667,0.705882352941177}

\begin{groupplot}[
group style={group size=3 by 2,horizontal sep=1.1cm,vertical sep=1.1cm},
tick align=outside,
tick pos=left,
x grid style={white!69.0196078431373!black},
xmajorgrids,
xlabel={\(x\)},
xmin=-0.04725, xmax=1.04725,
xtick style={color=black},
y grid style={white!69.0196078431373!black},
ymajorgrids,
ymin=0.0787652333577474, ymax=1.04575335673797,
ytick style={color=black},
width=.35\textwidth,
height=.4\textwidth,
y label style={yshift=-.9em},
y label style={xshift=-.7em}
]
\nextgroupplot[
legend cell align={left},
legend style={at={(1,1)},
	anchor=north east,
	fill opacity=0.1,
	draw opacity=1,
	text opacity=1,
	draw=none},
ylabel={\(\rho\)},
ymin=0.0709843959659338, ymax=1.04776137061417,
]
\addplot [semithick, black, dashed, mark=x,mark size=2,mark repeat=25, mark options={solid}]
table {%
0 0.999999994499998
0.005 0.999999994499989
0.01 0.999999994499973
0.015 0.999999994499938
0.02 0.999999994499854
0.025 0.999999994499657
0.03 0.999999994499225
0.035 0.999999994498307
0.04 0.999999994496346
0.045 0.999999994492153
0.05 0.999999994483302
0.055 0.999999994464812
0.06 0.999999994426618
0.065 0.999999994348593
0.07 0.999999994191032
0.075 0.999999993876523
0.08 0.999999993255951
0.085 0.999999992045765
0.09 0.999999989713722
0.095 0.999999985273699
0.1 0.999999976922728
0.105 0.999999961408873
0.11 0.999999932947061
0.115 0.999999881389466
0.12 0.999999789188686
0.125 0.999999626442624
0.13 0.99999934295137
0.135 0.999998855715861
0.14 0.99999802963589
0.145 0.99999664829786
0.15 0.999994370682796
0.155 0.999990668414943
0.16 0.999984736919399
0.165 0.999975372768066
0.17 0.999960808885077
0.175 0.999938499600449
0.18 0.999904849332842
0.185 0.999854882524753
0.19 0.99978185882593
0.195 0.999676846595887
0.2 0.999528279209094
0.205 0.999321531266986
0.21 0.999038563648431
0.215 0.998657694619451
0.22 0.99815355591569
0.225 0.997497285204109
0.23 0.996656988391704
0.235 0.995598477836268
0.24 0.994286259099913
0.245 0.992684705039226
0.25 0.990759328214853
0.255 0.988478046531036
0.26 0.985812336037212
0.265 0.982738179020057
0.27 0.979236741747293
0.275 0.975294749126552
0.28 0.970904557028253
0.285 0.96606395178504
0.29 0.960775726924111
0.295 0.955047098184107
0.3 0.948889019907117
0.305 0.942315460872332
0.31 0.9353426880196
0.315 0.927988594730014
0.32 0.920272098375855
0.325 0.912212621103049
0.33 0.903829659028516
0.335 0.895142438489912
0.34 0.886169653587621
0.345 0.876929276719067
0.35 0.867438432750232
0.355 0.857713327521522
0.36 0.847769222220737
0.365 0.837620446519981
0.37 0.82728044509819
0.375 0.816761854164244
0.38 0.806076606850691
0.385 0.795236068929107
0.39 0.784251209358396
0.395 0.773132813964529
0.4 0.761891755435828
0.405 0.750539339336509
0.41 0.739087754739222
0.415 0.727550670369378
0.42 0.715944034148932
0.425 0.704287157239119
0.43 0.692604194420389
0.435 0.680926170788684
0.44 0.669293745622052
0.445 0.657760931521731
0.45 0.646399957544768
0.455 0.635307281677215
0.46 0.624610228710984
0.465 0.614472544087769
0.47 0.605094996557836
0.475 0.596704251904308
0.48 0.589521611837022
0.485 0.583707948506557
0.49 0.57929759656908
0.495 0.576156313265513
0.5 0.574000965899689
0.505 0.57248017604117
0.51 0.571264200514682
0.515 0.570086379629996
0.52 0.568721456981151
0.525 0.566925162056573
0.53 0.564365789892603
0.535 0.560570505963129
0.54 0.554906052745071
0.545 0.546613496573865
0.55 0.534908968811207
0.555 0.519142277507728
0.56 0.49897861904467
0.565 0.474548678169257
0.57 0.446512146926428
0.575 0.41600335371414
0.58 0.384466860249091
0.585 0.353428287708994
0.59 0.324264601694908
0.595 0.298031573575595
0.6 0.275379773169268
0.605 0.256558296419699
0.61 0.241481616612008
0.615 0.229826468110973
0.62 0.221130573161533
0.625 0.214876144173828
0.63 0.210551636151645
0.635 0.207691966758486
0.64 0.205900478048754
0.645 0.20485677362768
0.65 0.204314507250902
0.655 0.204092812694663
0.66 0.204064483070227
0.665 0.204143251128216
0.67 0.204271659470887
0.675 0.204410148437548
0.68 0.204527208694704
0.685 0.204589750704457
0.69 0.204552164518092
0.695 0.204341776813738
0.7 0.203837533803555
0.705 0.202838125078778
0.71 0.201017075159804
0.715 0.197870633082619
0.72 0.192691646081915
0.725 0.18466735979763
0.73 0.173289805104828
0.735 0.159203830364094
0.74 0.144954928763386
0.745 0.134090935533034
0.75 0.128237750573878
0.755 0.125970038823052
0.76 0.125265821320929
0.765 0.125069121636156
0.77 0.125016331920844
0.775 0.125002354274959
0.78 0.124998671121747
0.785 0.124997702861611
0.79 0.124997448751868
0.795 0.124997382171878
0.8 0.124997364756704
0.805 0.124997360209753
0.81 0.124997359024918
0.815 0.124997358716831
0.82 0.124997358636905
0.825 0.124997358616222
0.83 0.124997358610884
0.835 0.12499735860951
0.84 0.124997358609158
0.845 0.124997358609067
0.85 0.124997358609045
0.855 0.124997358609039
0.86 0.124997358609038
0.865 0.124997358609038
0.87 0.124997358609037
0.875 0.124997358609037
0.88 0.124997358609037
0.885 0.124997358609037
0.89 0.124997358609037
0.895 0.124997358609037
0.9 0.124997358609037
0.905 0.124997358609037
0.91 0.124997358609037
0.915 0.124997358609037
0.92 0.124997358609037
0.925 0.124997358609037
0.93 0.124997358609037
0.935 0.124997358609037
0.94 0.124997358609037
0.945 0.124997358609037
0.95 0.124997358609037
0.955 0.124997358609037
0.96 0.124997358609037
0.965 0.124997358609037
0.97 0.124997358609037
0.975 0.124997358609037
0.98 0.124997358609037
0.985 0.124997358609037
0.99 0.124997358609037
0.995 0.124997358609037
};
\addlegendentry{FOM}
\addplot [semithick, red, mark=o,mark size=2, mark repeat=25, mark options={solid}]
table {%
0 0.999548766097275
0.005 0.999548766097269
0.01 0.999548766097259
0.015 0.999548766097235
0.02 0.999548766097171
0.025 0.999548766097013
0.03 0.999548766096661
0.035 0.99954876609592
0.04 0.999548766094334
0.045 0.999548766090938
0.05 0.999548766083761
0.055 0.999548766068759
0.06 0.999548766037749
0.065 0.999548765974361
0.07 0.999548765846286
0.075 0.999548765590504
0.08 0.999548765085545
0.085 0.999548764100284
0.09 0.999548762200651
0.095 0.999548758581912
0.1 0.999548751771821
0.105 0.999548739113262
0.11 0.999548715876255
0.115 0.999548673758189
0.12 0.999548598392518
0.125 0.999548465280291
0.13 0.999548233262582
0.135 0.999547834237471
0.14 0.999547157267545
0.145 0.999546024502503
0.15 0.999544155456665
0.155 0.999541115165718
0.16 0.999536240687142
0.165 0.999528539469326
0.17 0.999516552551976
0.175 0.99949817573444
0.18 0.999470433202107
0.185 0.999429201094819
0.19 0.999368883516272
0.195 0.999282050646021
0.2 0.999159057618876
0.205 0.998987672765875
0.21 0.998752753079526
0.215 0.998436011242218
0.22 0.998015919930009
0.225 0.997467793500422
0.23 0.996764073840335
0.235 0.99587482702148
0.24 0.994768433239172
0.245 0.993412428434768
0.25 0.991774436584228
0.255 0.989823120647582
0.26 0.987529079574339
0.265 0.984865628216473
0.27 0.981809414115731
0.275 0.978340846241213
0.28 0.974444331958855
0.285 0.970108336601115
0.29 0.965325292982492
0.295 0.960091395422539
0.3 0.954406314745543
0.305 0.94827286847776
0.31 0.941696675484427
0.315 0.934685817948182
0.32 0.927250526992108
0.325 0.919402902164065
0.33 0.911156669860784
0.335 0.902526981759004
0.34 0.893530251424297
0.345 0.884184025366768
0.35 0.874506883738104
0.355 0.864518365439027
0.36 0.854238912476574
0.365 0.843689828853849
0.37 0.832893250015516
0.375 0.821872119872861
0.38 0.810650173710983
0.385 0.799251926898944
0.39 0.787702671409527
0.395 0.776028484908574
0.4 0.764256260893497
0.405 0.75241377348081
0.41 0.740529797574265
0.415 0.728634315130239
0.42 0.716758852181771
0.425 0.704937010511207
0.43 0.693205283636944
0.435 0.681604279403932
0.44 0.670180507942401
0.445 0.658988922287852
0.45 0.648096387187819
0.455 0.637586124321244
0.46 0.6275627870392
0.465 0.618156899387084
0.47 0.60952566688161
0.475 0.601844723022755
0.48 0.595283730966716
0.485 0.589962038559897
0.49 0.585893691387117
0.495 0.582950448311049
0.5 0.580875837601375
0.505 0.579352248441286
0.51 0.578077720365126
0.515 0.576800620455441
0.52 0.575296687125193
0.525 0.57330876413295
0.53 0.570477429780912
0.535 0.566285775255063
0.54 0.560040701041629
0.545 0.550913664162498
0.55 0.538054896658628
0.555 0.520771875900115
0.56 0.498731910583755
0.565 0.472125593756413
0.57 0.441728222740961
0.575 0.408825404763645
0.58 0.37501639703621
0.585 0.341952316369531
0.59 0.31108540142856
0.595 0.28349274475971
0.6 0.259803154421408
0.605 0.240218450942317
0.61 0.224596871298365
0.615 0.212561361650023
0.62 0.203604093836384
0.625 0.197171595221461
0.63 0.192725776322009
0.635 0.189782475844084
0.64 0.187931636993338
0.645 0.18684363464789
0.65 0.186265935658639
0.655 0.186013762860454
0.66 0.185957829351355
0.665 0.186011485850003
0.67 0.186118821973583
0.675 0.186244485220258
0.68 0.186365322176616
0.685 0.186463436212429
0.69 0.186519852103761
0.695 0.186507572151725
0.7 0.186382226374384
0.705 0.186067491092764
0.71 0.18543053435851
0.715 0.184239277772964
0.72 0.182088046679165
0.725 0.178277989928247
0.73 0.171696752993081
0.735 0.161099167148337
0.74 0.147018645241884
0.745 0.134243306776949
0.75 0.127109460414396
0.755 0.124412935892678
0.76 0.123592847851888
0.765 0.12336580304971
0.77 0.123305047782118
0.775 0.123288974714035
0.78 0.123284740328715
0.785 0.123283627194518
0.79 0.123283335059134
0.795 0.123283258513415
0.8 0.123283238490736
0.805 0.12328323326276
0.81 0.123283231900399
0.815 0.123283231546133
0.82 0.123283231454222
0.825 0.123283231430435
0.83 0.123283231424296
0.835 0.123283231422716
0.84 0.12328323142231
0.845 0.123283231422207
0.85 0.123283231422181
0.855 0.123283231422174
0.86 0.123283231422173
0.865 0.123283231422172
0.87 0.123283231422172
0.875 0.123283231422172
0.88 0.123283231422172
0.885 0.123283231422172
0.89 0.123283231422172
0.895 0.123283231422172
0.9 0.123283231422172
0.905 0.123283231422172
0.91 0.123283231422172
0.915 0.123283231422172
0.92 0.123283231422172
0.925 0.123283231422172
0.93 0.123283231422172
0.935 0.123283231422172
0.94 0.123283231422172
0.945 0.123283231422172
0.95 0.123283231422172
0.955 0.123283231422172
0.96 0.123283231422172
0.965 0.123283231422172
0.97 0.123283231422172
0.975 0.123283231422172
0.98 0.123283231422172
0.985 0.123283231422172
0.99 0.123283231422172
0.995 0.123283231422172
};
\addlegendentry{POD}
\addplot [semithick, color0, mark=pentagon,mark size=2, mark repeat=25, mark options={solid}]
table {%
0 1.00007958960132
0.005 1.00007958960132
0.01 1.00007958960132
0.015 1.00007958960132
0.02 1.00007958960132
0.025 1.00007958960132
0.03 1.00007958960132
0.035 1.00007958960132
0.04 1.00007958960132
0.045 1.00007958960132
0.05 1.00007958960132
0.055 1.00007958960132
0.06 1.00007958960132
0.065 1.00007958960132
0.07 1.00007958960132
0.075 1.00007958960132
0.08 1.00007958960132
0.085 1.00007958960132
0.09 1.00007958960132
0.095 1.00007960560096
0.1 1.00007967294275
0.105 1.00007964882388
0.11 1.00007973574732
0.115 1.00007942697807
0.12 1.00007952906535
0.125 1.00007935247227
0.13 1.00007910507478
0.135 1.0000786045486
0.14 1.00007773758281
0.145 1.00007644805925
0.15 1.00007424166057
0.155 1.00007047887858
0.16 1.00006458671113
0.165 1.00005566690952
0.17 1.00004139821195
0.175 1.00001937338839
0.18 0.999986432031501
0.185 0.999937488403667
0.19 0.999865611912762
0.195 0.999762540533591
0.2 0.999616785088856
0.205 0.999413966793886
0.21 0.999135611671512
0.215 0.998761416889719
0.22 0.998265151151335
0.225 0.997618268225603
0.23 0.996788949596336
0.235 0.99574209013388
0.24 0.994440960625299
0.245 0.992848293362174
0.25 0.990926713533989
0.255 0.988643694819403
0.26 0.98597447156029
0.265 0.982894724536353
0.27 0.979384886899208
0.275 0.97543120528311
0.28 0.971025352350884
0.285 0.966164945468956
0.29 0.960852690965192
0.295 0.955096454813896
0.3 0.948908228748352
0.305 0.942303409679839
0.31 0.935299842942792
0.315 0.927917993992475
0.32 0.920179602285666
0.325 0.912107081055203
0.33 0.903720666985319
0.335 0.895043348699038
0.34 0.886099242316696
0.345 0.876913020961165
0.35 0.867506658352671
0.355 0.8578984900834
0.36 0.848106184178299
0.365 0.838090457212586
0.37 0.827832277131951
0.375 0.817345481359971
0.38 0.806643968094992
0.385 0.795742298087786
0.39 0.784654691798187
0.395 0.773418127866418
0.4 0.76206761590187
0.405 0.750619441273673
0.41 0.739091075712631
0.415 0.727500537444857
0.42 0.715867658268065
0.425 0.704210523511352
0.43 0.692526782779235
0.435 0.680846175469963
0.44 0.66921524980196
0.445 0.657692996940784
0.45 0.646356617796227
0.455 0.635305164135372
0.46 0.624665874322266
0.465 0.61460028374918
0.47 0.605298140195823
0.475 0.596974775589291
0.48 0.589842921078701
0.485 0.584063832062555
0.49 0.579672229440244
0.495 0.576530995545067
0.5 0.574369383563039
0.505 0.572842073294033
0.51 0.571619951938634
0.515 0.570435925138936
0.52 0.56906307830009
0.525 0.567255114691168
0.53 0.564675626033231
0.535 0.560842670806613
0.54 0.555107028704735
0.545 0.546692667435545
0.55 0.534817340636247
0.555 0.5188380042211
0.56 0.498502658421844
0.565 0.473995488276239
0.57 0.446001658485437
0.575 0.415641278464789
0.58 0.38431216452763
0.585 0.353492774178785
0.59 0.324502245123317
0.595 0.298354483407366
0.6 0.275683994082713
0.605 0.256797161618465
0.61 0.241638212449143
0.615 0.229901758262591
0.62 0.221134202819148
0.625 0.214821467045285
0.63 0.210452181777347
0.635 0.207558450317622
0.64 0.205741919376864
0.645 0.204680016666904
0.65 0.204124898744949
0.655 0.203894491451661
0.66 0.203860315972972
0.665 0.203935182367526
0.67 0.204061199146803
0.675 0.204198292814041
0.68 0.204315014755273
0.685 0.204378813575093
0.69 0.204344825913122
0.695 0.204142378236774
0.7 0.203654290704554
0.705 0.202686372769895
0.71 0.200921816195532
0.715 0.197857485664529
0.72 0.19278071511726
0.725 0.184850776625979
0.73 0.173437997270236
0.735 0.159227317397964
0.74 0.144888745564241
0.745 0.134016824746112
0.75 0.128206411531143
0.755 0.125959403239509
0.76 0.125261302826069
0.765 0.125066164599088
0.77 0.125013926119164
0.775 0.125000055502295
0.78 0.124996353614482
0.785 0.124995377277988
0.79 0.124995147432354
0.795 0.124995057285105
0.8 0.124995140387734
0.805 0.124995080209968
0.81 0.124995080209968
0.815 0.124995080209968
0.82 0.124995080209968
0.825 0.124995080209968
0.83 0.124995080209968
0.835 0.124995080209968
0.84 0.124995080209968
0.845 0.124995080209968
0.85 0.124995080209968
0.855 0.124995080209968
0.86 0.124995080209968
0.865 0.124995080209968
0.87 0.124995080209968
0.875 0.124995080209968
0.88 0.124995080209968
0.885 0.124995080209968
0.89 0.124995080209968
0.895 0.124995080209968
0.9 0.124995080209968
0.905 0.124995080209968
0.91 0.124995080209968
0.915 0.124995080209968
0.92 0.124995080209968
0.925 0.124995080209968
0.93 0.124995080209968
0.935 0.124995080209968
0.94 0.124995080209968
0.945 0.124995080209968
0.95 0.124995080209968
0.955 0.124995080209968
0.96 0.124995080209968
0.965 0.124995080209968
0.97 0.124995080209968
0.975 0.124995080209968
0.98 0.124995080209968
0.985 0.124995080209968
0.99 0.124995080209968
0.995 0.124995080209968
};
\addlegendentry{FCNN}
\addplot [semithick, green!50!black, mark=triangle,mark size=2, mark repeat=25, mark options={solid,rotate=180}, only marks]
table {%
0 1.00258326530457
0.005 1.00256443023682
0.01 1.00046801567078
0.015 1.00242698192596
0.02 1.00336241722107
0.025 0.999016523361206
0.03 0.999636828899384
0.035 0.998747169971466
0.04 1.00108003616333
0.045 1.00217437744141
0.05 0.998163282871246
0.055 0.998933553695679
0.06 0.999087631702423
0.065 1.00104880332947
0.07 1.00219261646271
0.075 0.999145150184631
0.08 0.998864829540253
0.085 0.997559905052185
0.09 0.99945205450058
0.095 0.999408662319183
0.1 0.99824458360672
0.105 0.998392403125763
0.11 0.997295022010803
0.115 1.00039541721344
0.12 1.00017428398132
0.125 0.998519062995911
0.13 0.998890161514282
0.135 1.00138914585114
0.14 1.00031340122223
0.145 0.99955803155899
0.15 1.00110673904419
0.155 1.0005294084549
0.16 1.00117695331573
0.165 1.00021278858185
0.17 0.999607264995575
0.175 0.999183714389801
0.18 0.997960090637207
0.185 0.997788369655609
0.19 0.996238589286804
0.195 0.996689915657043
0.2 0.996785521507263
0.205 0.995399951934814
0.21 0.993049681186676
0.215 0.993473589420319
0.22 0.993443071842194
0.225 0.993543267250061
0.23 0.992028534412384
0.235 0.991612792015076
0.24 0.989040851593018
0.245 0.989743173122406
0.25 0.982356786727905
0.255 0.980083107948303
0.26 0.978865742683411
0.265 0.97463446855545
0.27 0.972721338272095
0.275 0.966095924377441
0.28 0.96215695142746
0.285 0.959598898887634
0.29 0.95321649312973
0.295 0.949223458766937
0.3 0.941951155662537
0.305 0.935739934444427
0.31 0.930762648582458
0.315 0.921872198581696
0.32 0.915408790111542
0.325 0.906080961227417
0.33 0.897238671779633
0.335 0.890177428722382
0.34 0.877592802047729
0.345 0.870674252510071
0.35 0.858479261398315
0.355 0.849721968173981
0.36 0.842632293701172
0.365 0.830277264118195
0.37 0.82226550579071
0.375 0.815447270870209
0.38 0.804872095584869
0.385 0.793931841850281
0.39 0.783296346664429
0.395 0.773645877838135
0.4 0.766008734703064
0.405 0.754041612148285
0.41 0.741739988327026
0.415 0.729966223239899
0.42 0.718774795532227
0.425 0.712926149368286
0.43 0.7003453373909
0.435 0.688562452793121
0.44 0.674932539463043
0.445 0.663553476333618
0.45 0.657922863960266
0.455 0.64715188741684
0.46 0.636059641838074
0.465 0.624775171279907
0.47 0.612429976463318
0.475 0.600696802139282
0.48 0.592281877994537
0.485 0.585568487644196
0.49 0.573356807231903
0.495 0.565456390380859
0.5 0.569030821323395
0.505 0.565917253494263
0.51 0.566354513168335
0.515 0.556866705417633
0.52 0.552515983581543
0.525 0.54369592666626
0.53 0.537129640579224
0.535 0.534266650676727
0.54 0.52167671918869
0.545 0.511373162269592
0.55 0.496925234794617
0.555 0.479478240013123
0.56 0.46316009759903
0.565 0.442822933197021
0.57 0.419572800397873
0.575 0.39087051153183
0.58 0.369901090860367
0.585 0.348668992519379
0.59 0.327757328748703
0.595 0.310218751430511
0.6 0.282398700714111
0.605 0.272575795650482
0.61 0.262125581502914
0.615 0.254319429397583
0.62 0.248346030712128
0.625 0.237663269042969
0.63 0.234263047575951
0.635 0.231954455375671
0.64 0.229010328650475
0.645 0.225278198719025
0.65 0.22056819498539
0.655 0.217902630567551
0.66 0.215957656502724
0.665 0.214296653866768
0.67 0.21147309243679
0.675 0.20885506272316
0.68 0.206284239888191
0.685 0.206196844577789
0.69 0.202801719307899
0.695 0.200748220086098
0.7 0.208443388342857
0.705 0.202578201889992
0.71 0.196301594376564
0.715 0.191382989287376
0.72 0.180745482444763
0.725 0.175989151000977
0.73 0.166682109236717
0.735 0.154879227280617
0.74 0.147852048277855
0.745 0.142320021986961
0.75 0.133487522602081
0.755 0.133048489689827
0.76 0.132668346166611
0.765 0.129774421453476
0.77 0.130361720919609
0.775 0.12637297809124
0.78 0.126359358429909
0.785 0.126889064908028
0.79 0.124303087592125
0.795 0.125612393021584
0.8 0.120551019906998
0.805 0.121369138360023
0.81 0.12182705104351
0.815 0.122782081365585
0.82 0.123911902308464
0.825 0.119310915470123
0.83 0.12048265337944
0.835 0.122536949813366
0.84 0.123518005013466
0.845 0.124819166958332
0.85 0.125285848975182
0.855 0.124968484044075
0.86 0.125277489423752
0.865 0.125845834612846
0.87 0.125722765922546
0.875 0.122230641543865
0.88 0.12384420633316
0.885 0.127788439393044
0.89 0.124791122972965
0.895 0.128436326980591
0.9 0.120557300746441
0.905 0.122165061533451
0.91 0.125625491142273
0.915 0.124156132340431
0.92 0.12682980298996
0.925 0.120922178030014
0.93 0.121201321482658
0.935 0.123363144695759
0.94 0.120900563895702
0.945 0.122320406138897
0.95 0.115383349359035
0.955 0.117032036185265
0.96 0.121062539517879
0.965 0.119826912879944
0.97 0.123609863221645
0.975 0.11955314129591
0.98 0.121047168970108
0.985 0.123612985014915
0.99 0.124337144196033
0.995 0.127016335725784
};
\addlegendentry{CNN}

\nextgroupplot[
legend cell align={left},
legend style={at={(0.0,1)},
	anchor=north west,
	fill opacity=0.1,
	draw opacity=1,
	text opacity=1,
	draw=none},
ylabel={\(\rho u\)},
ymin=-0.0300337838946434, ymax=0.480265793243131,
width=.37\textwidth,
height=.4\textwidth,
y label style={yshift=-.9em},
y label style={xshift=+.7em}
]
\addplot [semithick, black, dashed, mark=x,mark size=2, mark repeat=25, mark options={solid}]
table {%
0 6.30909939231123e-15
0.005 2.52743809394443e-14
0.01 6.71665948905054e-14
0.015 1.47797315006362e-13
0.02 3.07924882400649e-13
0.025 6.6172472758689e-13
0.03 1.42564590009118e-12
0.035 3.07415834705836e-12
0.04 6.61513596957766e-12
0.045 1.41586575371753e-11
0.05 3.00685203765747e-11
0.055 6.32804197847002e-11
0.06 1.31840368981028e-10
0.065 2.71816268178485e-10
0.07 5.54376848420904e-10
0.075 1.11825149293422e-09
0.08 2.23051508115254e-09
0.085 4.3988436607807e-09
0.09 8.57587718345818e-09
0.095 1.6525966137083e-08
0.1 3.14735316685944e-08
0.105 5.92319220162265e-08
0.11 1.1013832102345e-07
0.115 2.02317177002482e-07
0.12 3.67093736874415e-07
0.125 6.57821440259511e-07
0.13 1.16402287545304e-06
0.135 2.03362895936117e-06
0.14 3.50728899710936e-06
0.145 5.97025139178314e-06
0.15 1.00291758876392e-05
0.155 1.66233526126904e-05
0.16 2.71819698904176e-05
0.165 4.38409222097891e-05
0.17 6.97336015392367e-05
0.175 0.000109369362893115
0.18 0.000169109904238073
0.185 0.000257746595168556
0.19 0.000387169949609523
0.195 0.000573105626212421
0.2 0.000835870200910254
0.205 0.00120107654392011
0.21 0.0017001966437239
0.215 0.00237087431188884
0.22 0.00325687722538598
0.225 0.00440759236330102
0.23 0.00587700372294412
0.235 0.00772214475261246
0.24 0.0100010837355784
0.245 0.0127705675659017
0.25 0.0160835049890066
0.255 0.0199865026149812
0.26 0.024517668517973
0.265 0.0297048683221924
0.27 0.0355645634604718
0.275 0.0421012918433731
0.28 0.0493077803908931
0.285 0.0571656182273979
0.29 0.0656463765219179
0.295 0.0747130389685672
0.3 0.0843216044817968
0.305 0.0944227366293171
0.31 0.104963357117122
0.315 0.115888107826333
0.32 0.127140633030916
0.325 0.138664657456851
0.33 0.150404855140034
0.335 0.162307518113764
0.34 0.174321043116104
0.345 0.186396259514057
0.35 0.198486623435526
0.355 0.210548302610127
0.36 0.222540174456852
0.365 0.234423757164425
0.37 0.2461630903626
0.375 0.257724578799727
0.38 0.269076809419216
0.385 0.280190349466021
0.39 0.291037530789031
0.395 0.301592223328621
0.4 0.311829598863364
0.405 0.321725884407147
0.41 0.33125810319655
0.415 0.340403800052193
0.42 0.34914074723873
0.425 0.357446627257301
0.43 0.36529869126666
0.435 0.37267339798658
0.44 0.379546051649685
0.445 0.385890485406314
0.45 0.391678889528999
0.455 0.396881977899246
0.46 0.401469838716726
0.465 0.405414024841958
0.47 0.408691630863368
0.475 0.411292024184161
0.48 0.413226029159821
0.485 0.414535185870344
0.49 0.415295776454039
0.495 0.415611531428894
0.5 0.41559377314098
0.505 0.415336270392398
0.51 0.414895446220246
0.515 0.41427973531504
0.52 0.413441913351203
0.525 0.412265556853831
0.53 0.410543069713507
0.535 0.407952684100831
0.54 0.404050220477402
0.545 0.39829414322411
0.55 0.390116259686097
0.555 0.379034706827778
0.56 0.364785475907473
0.565 0.347433283903082
0.57 0.327421672581876
0.575 0.305538966019578
0.58 0.282804899447141
0.585 0.260309643125283
0.59 0.239050561287535
0.595 0.219807541815168
0.6 0.20307934526991
0.605 0.189081107060286
0.61 0.177786616825983
0.615 0.168992999905332
0.62 0.162388541129918
0.625 0.15761173233599
0.63 0.154296586879928
0.635 0.152103711892583
0.64 0.150738704005009
0.645 0.149960167756696
0.65 0.149579858547135
0.655 0.149457421963541
0.66 0.149491943424773
0.665 0.1496120131206
0.67 0.149765299386181
0.675 0.149907790936039
0.68 0.149991956492704
0.685 0.149952035270804
0.69 0.149683390951819
0.695 0.149011228649705
0.7 0.147642219487674
0.705 0.145092233973356
0.71 0.140589880124687
0.715 0.132983873988083
0.72 0.120763965024808
0.725 0.10247138733323
0.73 0.0779153045242086
0.735 0.0500927744393442
0.74 0.0255869633865413
0.745 0.0101029770077908
0.75 0.00323067177688417
0.755 0.000916685479303123
0.76 0.000247535672051547
0.765 6.57015394421608e-05
0.77 1.733476779243e-05
0.775 4.56166541654529e-06
0.78 1.19832317168437e-06
0.785 3.14296210643595e-07
0.79 8.22998342934205e-08
0.795 2.15132336229347e-08
0.8 5.61307170845078e-09
0.805 1.46156535792315e-09
0.81 3.79741250295264e-10
0.815 9.8431384383556e-11
0.82 2.54493088990962e-11
0.825 6.56196969727746e-12
0.83 1.68704996773948e-12
0.835 4.3243891523643e-13
0.84 1.10497770422186e-13
0.845 2.8177175461474e-14
0.85 7.14463267733202e-15
0.855 1.78835622885588e-15
0.86 4.08881035414389e-16
0.865 5.23823250656242e-17
0.87 2.23070011976512e-18
0.875 -1.70270269849063e-18
0.88 1.91820139524602e-19
0.885 -2.40410875285755e-18
0.89 -2.60371400131214e-18
0.895 1.34734437976866e-19
0.9 1.3517563413719e-19
0.905 1.3561001001166e-19
0.91 1.3561001001166e-19
0.915 1.3561001001166e-19
0.92 1.3561001001166e-19
0.925 1.3561001001166e-19
0.93 1.3561001001166e-19
0.935 1.3561001001166e-19
0.94 1.3561001001166e-19
0.945 1.3561001001166e-19
0.95 1.3561001001166e-19
0.955 1.3561001001166e-19
0.96 1.3561001001166e-19
0.965 1.3561001001166e-19
0.97 1.3561001001166e-19
0.975 1.3561001001166e-19
0.98 1.3561001001166e-19
0.985 1.3561001001166e-19
0.99 1.3561001001166e-19
0.995 1.3561001001166e-19
};
\addlegendentry{FOM}
\addplot [semithick, red, mark=o,mark size=2, mark repeat=25, mark options={solid}]
table {%
0 0.00393237571999395
0.005 0.00393237572000699
0.01 0.00393237572003567
0.015 0.00393237572009051
0.02 0.00393237572019905
0.025 0.00393237572043822
0.03 0.00393237572095436
0.035 0.00393237572206808
0.04 0.00393237572445993
0.045 0.00393237572955494
0.05 0.00393237574030051
0.055 0.0039323757627314
0.06 0.00393237580903555
0.065 0.0039323759035716
0.07 0.00393237609440472
0.075 0.00393237647522753
0.08 0.00393237722641135
0.085 0.00393237869081672
0.09 0.00393238151180752
0.095 0.00393238688092408
0.1 0.00393239697573214
0.105 0.00393241572214782
0.11 0.00393245010107675
0.115 0.00393251235215761
0.12 0.00393262362937609
0.125 0.00393281996185788
0.13 0.00393316180252827
0.135 0.00393374904486233
0.14 0.0039347441901554
0.145 0.00393640737919783
0.15 0.00393914825997421
0.155 0.00394360109373099
0.16 0.00395073097240658
0.165 0.00396198028695791
0.17 0.00397946526737079
0.175 0.00400623198601582
0.18 0.00404657903798491
0.185 0.00410644952294667
0.19 0.00419388743095293
0.195 0.00431954293599799
0.2 0.00449719793890273
0.205 0.00474426888743845
0.21 0.00508223083477531
0.215 0.00553689806212456
0.22 0.00613849578340267
0.225 0.00692146715778863
0.23 0.00792398099090093
0.235 0.00918713646349646
0.24 0.0107538975968649
0.245 0.0126678255279208
0.25 0.0149717041056653
0.255 0.0177061682912189
0.26 0.0209084427337273
0.265 0.0246112806802754
0.27 0.0288421651697878
0.275 0.0336228011750155
0.28 0.0389688950556044
0.285 0.0448901911684271
0.29 0.0513907175017316
0.295 0.0584691833808663
0.3 0.0661194715760273
0.305 0.0743311724784666
0.31 0.0830901170469029
0.315 0.0923788758120131
0.32 0.102177201705979
0.325 0.112462403774987
0.33 0.123209646379675
0.335 0.134392174142079
0.34 0.145981466776403
0.345 0.157947330310664
0.35 0.170257932388046
0.355 0.182879789645279
0.36 0.19577771487644
0.365 0.208914731025643
0.37 0.222251958178542
0.375 0.235748478758555
0.38 0.249361185160587
0.385 0.263044613127058
0.39 0.276750763332264
0.395 0.290428912940081
0.4 0.304025418415702
0.405 0.317483510747501
0.41 0.330743084740245
0.415 0.343740485673065
0.42 0.356408300277696
0.425 0.368675166294672
0.43 0.380465628625291
0.435 0.391700095123521
0.44 0.402294989247729
0.445 0.412163272166321
0.45 0.421215629468121
0.455 0.429362801709529
0.46 0.436519775658085
0.465 0.442612760412359
0.47 0.447589788305399
0.475 0.451434863339327
0.48 0.454183158275777
0.485 0.455930951133784
0.49 0.456831359452117
0.495 0.457070357918686
0.5 0.456828523915175
0.505 0.456244020453484
0.51 0.455389801769247
0.515 0.454263805718551
0.52 0.452778908382685
0.525 0.450740148616296
0.53 0.447808225787494
0.535 0.443463391680839
0.54 0.436996430513708
0.545 0.42755738088059
0.55 0.414281771780078
0.555 0.396486576755878
0.56 0.37389172937469
0.565 0.34679529354051
0.57 0.316129706016037
0.575 0.283360069791525
0.58 0.25024189170646
0.585 0.21850925500469
0.59 0.189589085430727
0.595 0.164421154340371
0.6 0.143416226417726
0.605 0.12653160310242
0.61 0.113410082848755
0.615 0.103525858054258
0.62 0.0963008068435049
0.625 0.0911796270368232
0.63 0.0876693014755772
0.635 0.0853544429081577
0.64 0.0838988475299465
0.645 0.0830399636601531
0.65 0.0825799089415004
0.655 0.0823749447352665
0.66 0.082324584919914
0.665 0.0823612011397087
0.67 0.0824407246577258
0.675 0.0825347278827024
0.68 0.0826238220571997
0.685 0.0826919721806419
0.69 0.0827209970781337
0.695 0.082684107270029
0.7 0.0825366463643485
0.705 0.0822008573826466
0.71 0.0815386689054063
0.715 0.0803004121356355
0.72 0.0780249722062501
0.725 0.0738497028125768
0.73 0.0662278271288958
0.735 0.0530360602215534
0.74 0.0341561497137797
0.745 0.0160919950519565
0.75 0.00586903035262873
0.755 0.00207603312015679
0.76 0.000946018721700573
0.765 0.000636416828080291
0.77 0.000553883593108021
0.775 0.00053207498359914
0.78 0.000526331642334363
0.785 0.000524822010949138
0.79 0.000524425837975289
0.795 0.000524322035967706
0.8 0.0005242948845691
0.805 0.000524287795515585
0.81 0.000524285948245756
0.815 0.000524285467905487
0.82 0.000524285343290609
0.825 0.000524285311041846
0.83 0.000524285302718498
0.835 0.00052428530057647
0.84 0.000524285300026817
0.845 0.000524285299886275
0.85 0.00052428529985036
0.855 0.000524285299841212
0.86 0.000524285299838856
0.865 0.000524285299838238
0.87 0.000524285299838151
0.875 0.000524285299838156
0.88 0.000524285299838143
0.885 0.000524285299838148
0.89 0.000524285299838144
0.895 0.000524285299838138
0.9 0.000524285299838138
0.905 0.000524285299838138
0.91 0.000524285299838138
0.915 0.000524285299838138
0.92 0.000524285299838138
0.925 0.000524285299838138
0.93 0.000524285299838138
0.935 0.000524285299838138
0.94 0.000524285299838138
0.945 0.000524285299838138
0.95 0.000524285299838138
0.955 0.000524285299838138
0.96 0.000524285299838138
0.965 0.000524285299838138
0.97 0.000524285299838142
0.975 0.000524285299838142
0.98 0.000524285299838144
0.985 0.000524285299838144
0.99 0.000524285299838144
0.995 0.000524285299838144
};
\addlegendentry{POD}
\addplot [semithick, color0, mark=pentagon,mark size=2, mark repeat=25, mark options={solid}]
table {%
0 -0.000224018744984643
0.005 -0.000224018744984643
0.01 -0.000224018744984643
0.015 -0.000224018744984643
0.02 -0.000224018744984643
0.025 -0.000224018744984643
0.03 -0.000224018744984643
0.035 -0.000224018744984643
0.04 -0.000224018744984643
0.045 -0.000224018744984643
0.05 -0.000224018744984643
0.055 -0.000224018744984643
0.06 -0.000224018744984643
0.065 -0.000224018744984643
0.07 -0.000224018744984643
0.075 -0.000224018744984643
0.08 -0.000224018744984643
0.085 -0.000224018744984643
0.09 -0.000224018744984643
0.095 -0.000224114204017176
0.1 -0.000224420971017424
0.105 -0.000224423848871476
0.11 -0.000224399111572556
0.115 -0.000224021377914952
0.12 -0.000223996242614871
0.125 -0.000223399761430633
0.13 -0.000222882727388758
0.135 -0.000222388961103418
0.14 -0.000220583597507155
0.145 -0.000218405245650912
0.15 -0.000214517724002303
0.155 -0.000207951256962543
0.16 -0.000198811069898842
0.165 -0.000182532549448376
0.17 -0.000157972483380186
0.175 -0.000120629754814146
0.18 -6.45468077751703e-05
0.185 1.94128097132156e-05
0.19 0.000142050020835177
0.195 0.000317399813851215
0.2 0.000566233682376917
0.205 0.000911136198722704
0.21 0.00138317272628902
0.215 0.00201805094390059
0.22 0.00285827773621694
0.225 0.00395250206330534
0.23 0.00535272705479511
0.235 0.00711846857027321
0.24 0.00930975956459987
0.245 0.0119877343535197
0.25 0.0152145700462701
0.255 0.0190384658616776
0.26 0.023486006956628
0.265 0.0285888621560336
0.27 0.0343668356271487
0.275 0.0408301942981457
0.28 0.0479735901058079
0.285 0.0557848471250038
0.29 0.0642390732036719
0.295 0.0733041575213734
0.3 0.0829408320345379
0.305 0.093101483810572
0.31 0.103735478966086
0.315 0.114789414264818
0.32 0.126204462695237
0.325 0.137922470816743
0.33 0.149856288696278
0.335 0.161922343022342
0.34 0.17405417015264
0.345 0.186189567965098
0.35 0.198268231517027
0.355 0.210235498357924
0.36 0.222050954078593
0.365 0.233822526473728
0.37 0.245508581475077
0.375 0.257068285670135
0.38 0.268466084457423
0.385 0.279669139654188
0.39 0.290645325132257
0.395 0.30128351815843
0.4 0.311509890508708
0.405 0.321307295897787
0.41 0.330665317075539
0.415 0.339573258160456
0.42 0.348027166971018
0.425 0.356030595024101
0.43 0.363641410975208
0.435 0.370859526743636
0.44 0.3776673689889
0.445 0.384045174629705
0.45 0.389965828769702
0.455 0.395394642704727
0.46 0.400289485355001
0.465 0.404598553234369
0.47 0.408279565124794
0.475 0.4112957652881
0.48 0.413636596928851
0.485 0.415293668469434
0.49 0.416305254573296
0.495 0.416742638645411
0.5 0.416771726371971
0.505 0.416527856893603
0.51 0.416087128785646
0.515 0.415465345538933
0.52 0.414619796982088
0.525 0.413435572513614
0.53 0.411703304513104
0.535 0.409091789631392
0.54 0.405106248364163
0.545 0.399101061056479
0.55 0.39036861576975
0.555 0.378290466986318
0.56 0.363146681986418
0.565 0.345323436214576
0.57 0.32549162805932
0.575 0.304513083434557
0.58 0.283249913988369
0.585 0.262475023127174
0.59 0.242767803704885
0.595 0.224524474631891
0.6 0.208051999031094
0.605 0.193581449267863
0.61 0.181341328977864
0.615 0.171418746395887
0.62 0.163721182217289
0.625 0.15801678880118
0.63 0.153986948144191
0.635 0.151285444173278
0.64 0.149584317095146
0.645 0.148598559150083
0.65 0.148099745476024
0.655 0.147915445005698
0.66 0.14792264607018
0.665 0.148034800207918
0.67 0.148191477993727
0.675 0.148346908903325
0.68 0.148456643561773
0.685 0.148466094832618
0.69 0.148289537902271
0.695 0.147783888314429
0.7 0.146705594838524
0.705 0.14462977813771
0.71 0.140840847912772
0.715 0.134184715481149
0.72 0.123067795069719
0.725 0.105610533503771
0.73 0.0802160459178515
0.735 0.049383167051561
0.74 0.0224540554733486
0.745 0.00768127561627308
0.75 0.00231781407386183
0.755 0.000701356647177434
0.76 0.000246654138973484
0.765 0.00012330573050479
0.77 9.08364029228138e-05
0.775 8.20810216900367e-05
0.78 7.98726666741706e-05
0.785 7.89232503624513e-05
0.79 7.88321693419298e-05
0.795 7.90195053972418e-05
0.8 7.88600600342628e-05
0.805 7.87138405539246e-05
0.81 7.87138405539246e-05
0.815 7.87138405539246e-05
0.82 7.87138405539246e-05
0.825 7.87138405539246e-05
0.83 7.87138405539246e-05
0.835 7.87138405539246e-05
0.84 7.87138405539246e-05
0.845 7.87138405539246e-05
0.85 7.87138405539246e-05
0.855 7.87138405539246e-05
0.86 7.87138405539246e-05
0.865 7.87138405539246e-05
0.87 7.87138405539246e-05
0.875 7.87138405539246e-05
0.88 7.87138405539246e-05
0.885 7.87138405539246e-05
0.89 7.87138405539246e-05
0.895 7.87138405539246e-05
0.9 7.87138405539246e-05
0.905 7.87138405539246e-05
0.91 7.87138405539246e-05
0.915 7.87138405539246e-05
0.92 7.87138405539246e-05
0.925 7.87138405539246e-05
0.93 7.87138405539246e-05
0.935 7.87138405539246e-05
0.94 7.87138405539246e-05
0.945 7.87138405539246e-05
0.95 7.87138405539246e-05
0.955 7.87138405539246e-05
0.96 7.87138405539246e-05
0.965 7.87138405539246e-05
0.97 7.87138405539246e-05
0.975 7.87138405539246e-05
0.98 7.87138405539246e-05
0.985 7.87138405539246e-05
0.99 7.87138405539246e-05
0.995 7.87138405539246e-05
};
\addlegendentry{FCNN}
\addplot [semithick, green!50!black, mark=triangle,mark size=2, mark repeat=25, mark options={solid,rotate=180}, only marks]
table {%
0 -0.00122694015063241
0.005 -0.000137009857390951
0.01 0.0015538256777313
0.015 0.00194832828492225
0.02 0.00204139539215335
0.025 0.000588879975472772
0.03 0.00128607111254231
0.035 0.00217344946233309
0.04 0.00269778076248705
0.045 0.00285834738640903
0.05 0.00137074958124482
0.055 0.00193043856418041
0.06 0.00225685383767425
0.065 0.00321828341453952
0.07 0.00338080208077022
0.075 0.00392147616324426
0.08 0.00365993972228779
0.085 0.0034538069935733
0.09 0.00307163874656639
0.095 0.0028202739890579
0.1 0.00369821101803113
0.105 0.00364719878867449
0.11 0.00354166408273786
0.115 0.00347199307205595
0.12 0.00306226306531448
0.125 0.00253607769235283
0.13 0.00258100650554095
0.135 0.00184459621780539
0.14 0.00292698677824481
0.145 0.00330252444941978
0.15 0.00280387144534229
0.155 0.0029376409614855
0.16 0.00218489817822747
0.165 0.00352262175081597
0.17 0.00398895951496746
0.175 0.00307823209398799
0.18 0.00326681064750177
0.185 0.00270435892463609
0.19 0.00396984272622468
0.195 0.00437740368821258
0.2 0.000944965802211406
0.205 0.00169578446446857
0.21 0.00197744482063867
0.215 0.00364661880922809
0.22 0.00446549198539959
0.225 0.00575588355293884
0.23 0.00684797505337374
0.235 0.00770752908516725
0.24 0.00939465010665665
0.245 0.010385450939941
0.25 0.0242470704120734
0.255 0.0273964979221798
0.26 0.0300144595032173
0.265 0.0346259877659885
0.27 0.0381339532794452
0.275 0.047049805860805
0.28 0.0519189008403496
0.285 0.056003627991845
0.29 0.0629428758410377
0.295 0.0680501180228545
0.3 0.0806411803427856
0.305 0.0865907510688713
0.31 0.0921802804496128
0.315 0.0996964319926208
0.32 0.10584304632696
0.325 0.118682138032735
0.33 0.127279189530224
0.335 0.13566890229444
0.34 0.145807210472521
0.345 0.154389368646458
0.35 0.166783012624795
0.355 0.177165055695641
0.36 0.187908632187333
0.365 0.199184448077305
0.37 0.208449760318199
0.375 0.242312911503235
0.38 0.25491264412401
0.385 0.268737319297291
0.39 0.28034079467342
0.395 0.292612266765149
0.4 0.305946032933564
0.405 0.317412082081923
0.41 0.330088361212964
0.415 0.340275351905667
0.42 0.351511679718692
0.425 0.36182875382506
0.43 0.3701108681558
0.435 0.379283495860365
0.44 0.38632879961324
0.445 0.394166423868413
0.45 0.399419775804727
0.455 0.404417122658115
0.46 0.409721649453158
0.465 0.414134123773514
0.47 0.418174002936659
0.475 0.426735062888694
0.48 0.427378858688294
0.485 0.429006415232831
0.49 0.428447226874046
0.495 0.428414976293991
0.5 0.443246206776022
0.505 0.442909332768442
0.51 0.441044949650507
0.515 0.442760288827388
0.52 0.442936073543758
0.525 0.446200121451329
0.53 0.440526027736354
0.535 0.434102015698836
0.54 0.429349332978932
0.545 0.421475249863785
0.55 0.420419933174965
0.555 0.401375474335214
0.56 0.381508620315729
0.565 0.362237492380451
0.57 0.337629088895162
0.575 0.316883676230219
0.58 0.294516644952199
0.585 0.270053610082132
0.59 0.248832183205417
0.595 0.22673149181299
0.6 0.207144968213698
0.605 0.196317355443383
0.61 0.184809446866305
0.615 0.17415784659145
0.62 0.166124607123349
0.625 0.147421706702872
0.63 0.144923092057879
0.635 0.141970508788091
0.64 0.140075485708698
0.645 0.136531896035416
0.65 0.128779962273892
0.655 0.126477968008907
0.66 0.124296779935685
0.665 0.121937756154588
0.67 0.118491451665406
0.675 0.114367228599404
0.68 0.110786887627189
0.685 0.108832384324614
0.69 0.103497539887772
0.695 0.0982090849893793
0.7 0.101039887760717
0.705 0.094619374214082
0.71 0.0883361708013591
0.715 0.0811884761838366
0.72 0.0700738015089106
0.725 0.0622668412177231
0.73 0.054317430332903
0.735 0.0435388429704633
0.74 0.0383074107303172
0.745 0.032217793582155
0.75 0.00798448445604866
0.755 0.00797083681470288
0.76 0.00893745828527709
0.765 0.00728354745510593
0.77 0.00698350508145117
0.775 0.00571581304730853
0.78 0.0054672531566156
0.785 0.00594410206853624
0.79 0.00445076080137207
0.795 0.0040718260377477
0.8 0.00331565236256237
0.805 0.0024997932180911
0.81 0.00201694906255282
0.815 0.000438528141579321
0.82 -0.000397740009353681
0.825 -0.00244781060082634
0.83 -0.0030959635939856
0.835 -0.00376685481849439
0.84 -0.0044698261561267
0.845 -0.00432970135844935
0.85 -0.00563785283714155
0.855 -0.0060070288587649
0.86 -0.00616480727068655
0.865 -0.00673375829025283
0.87 -0.00683834857019914
0.875 0.00260455935329182
0.88 0.0024291302662109
0.885 0.00273842243030865
0.89 0.00180976610078691
0.895 0.00208511722108842
0.9 0.00399502872555835
0.905 0.00350111156716226
0.91 0.00331843420654509
0.915 0.00233419292120446
0.92 0.00230775438179973
0.925 0.00565550547169083
0.93 0.0045009819270077
0.935 0.00351154531902562
0.94 0.00180616131299525
0.945 0.000961615229069817
0.95 0.00266030204000742
0.955 0.00224920633842114
0.96 0.0019149157865843
0.965 0.00101075820034622
0.97 0.00159265344024063
0.975 0.000832698142314212
0.98 0.00130437328488365
0.985 0.00208063119763468
0.99 0.00184639689723488
0.995 0.00247777506257439
};
\addlegendentry{CNN}

\nextgroupplot[
legend cell align={left},
legend style={at={(1,1)},
	anchor=north east,
	fill opacity=0.1,
	draw opacity=1,
	text opacity=1,
	draw=none},
ylabel={\(E\)},
ymin=-0.05, ymax=0.618829439145279,
y label style={yshift=-.9em},
y label style={xshift=+.7em}
]
\addplot [semithick, black, dashed, mark=x,mark size=2, mark repeat=25, mark options={solid}]
table {%
0 0.499999997249975
0.005 0.499999997249952
0.01 0.49999999724991
0.015 0.49999999724983
0.02 0.499999997249673
0.025 0.499999997249343
0.03 0.499999997248633
0.035 0.499999997247108
0.04 0.499999997243849
0.045 0.499999997236918
0.05 0.49999999722231
0.055 0.499999997191854
0.06 0.499999997129034
0.065 0.499999997000896
0.07 0.499999996742452
0.075 0.499999996227163
0.08 0.499999995211631
0.085 0.499999993233655
0.09 0.499999989426775
0.095 0.499999982187882
0.1 0.499999968590229
0.105 0.499999943362711
0.11 0.49999989714242
0.115 0.499999813531151
0.12 0.499999664219035
0.125 0.499999401043895
0.13 0.499998943288962
0.135 0.499998157735899
0.14 0.499996827939107
0.145 0.499994607841606
0.15 0.499990953221392
0.155 0.499985022618125
0.16 0.499975537529369
0.165 0.499960590121808
0.17 0.499937386002681
0.175 0.499901910468537
0.18 0.499848509989294
0.185 0.499769387444734
0.19 0.499654020597409
0.195 0.499488528760043
0.2 0.499255032002629
0.205 0.498931068658278
0.21 0.498489156918177
0.215 0.497896600156329
0.22 0.497115637784343
0.225 0.496104028964957
0.23 0.494816122733982
0.235 0.493204416224345
0.24 0.491221538805241
0.245 0.488822534483274
0.25 0.48596726066466
0.255 0.482622690300756
0.26 0.478764904226945
0.265 0.474380592224794
0.27 0.469467939022164
0.275 0.46403684380784
0.28 0.458108495428486
0.285 0.451714388097888
0.29 0.444894905918178
0.295 0.437697625542276
0.3 0.430175486289908
0.305 0.422384960603958
0.31 0.414384331017942
0.315 0.406232148833711
0.32 0.397985919383409
0.325 0.389701032374154
0.33 0.381429935098179
0.335 0.373221531647741
0.34 0.365120782228682
0.345 0.357168472254166
0.35 0.349401119983176
0.355 0.341850992947928
0.36 0.334546206351425
0.365 0.327510880277021
0.37 0.320765336408751
0.375 0.314326318665062
0.38 0.308207225493483
0.385 0.302418344457086
0.39 0.296967082133645
0.395 0.291858184256781
0.4 0.287093942489105
0.405 0.28267438527183
0.41 0.278597450881622
0.415 0.27485914117029
0.42 0.271453654478241
0.425 0.268373495894285
0.43 0.265609562373013
0.435 0.263151199234399
0.44 0.260986223393296
0.445 0.25910090775627
0.45 0.257479921790285
0.455 0.256106228028066
0.46 0.254960948347607
0.465 0.254023244450158
0.47 0.253270307401023
0.475 0.252677601550096
0.48 0.252219485718782
0.485 0.251870132202037
0.49 0.251604330194165
0.495 0.251397729084703
0.5 0.25122669129834
0.505 0.251068332781992
0.51 0.250900333775037
0.515 0.250698706378175
0.52 0.250431897599141
0.525 0.250051148444404
0.53 0.249478662542547
0.535 0.248596912673306
0.54 0.247244381726547
0.545 0.245223917214396
0.55 0.242327893303772
0.555 0.238378924611544
0.56 0.233277606118821
0.565 0.227043160220145
0.57 0.219832439446087
0.575 0.211928643264648
0.58 0.203701164573315
0.585 0.195547680233291
0.59 0.187834659876278
0.595 0.18085111996356
0.6 0.174784098494021
0.605 0.169716410533201
0.61 0.165641200169253
0.615 0.162485367499619
0.62 0.160134752274621
0.625 0.158456399070025
0.63 0.157315740975606
0.635 0.156588304507358
0.64 0.156166477645325
0.645 0.155962258709567
0.65 0.155906982680582
0.655 0.155948945374948
0.66 0.156049656989133
0.665 0.156179160619016
0.67 0.156310449279297
0.675 0.156412496184095
0.68 0.156440733162282
0.685 0.156322878793174
0.69 0.155936716265043
0.695 0.155074755703434
0.7 0.15338930096658
0.705 0.150312941539415
0.71 0.144962247510296
0.715 0.136077279136682
0.72 0.122162286306937
0.725 0.102173607088272
0.73 0.0770700772857907
0.735 0.0513960997098126
0.74 0.0317715372289992
0.745 0.0212761317050686
0.75 0.0172929078157419
0.755 0.0160879851700508
0.76 0.0157570212643047
0.765 0.0156687593965459
0.77 0.0156454185249256
0.775 0.0156392644262287
0.78 0.015637644580833
0.785 0.0156372188316351
0.79 0.0156371070941755
0.795 0.0156370778143173
0.8 0.0156370701545981
0.805 0.0156370681544042
0.81 0.0156370676331101
0.815 0.0156370674975363
0.82 0.0156370674623578
0.825 0.0156370674532522
0.83 0.0156370674509016
0.835 0.0156370674502965
0.84 0.0156370674501412
0.845 0.0156370674501014
0.85 0.0156370674500913
0.855 0.0156370674500887
0.86 0.015637067450088
0.865 0.0156370674500878
0.87 0.0156370674500877
0.875 0.0156370674500877
0.88 0.0156370674500877
0.885 0.0156370674500877
0.89 0.0156370674500877
0.895 0.0156370674500877
0.9 0.0156370674500877
0.905 0.0156370674500877
0.91 0.0156370674500877
0.915 0.0156370674500877
0.92 0.0156370674500877
0.925 0.0156370674500877
0.93 0.0156370674500877
0.935 0.0156370674500877
0.94 0.0156370674500877
0.945 0.0156370674500877
0.95 0.0156370674500877
0.955 0.0156370674500877
0.96 0.0156370674500877
0.965 0.0156370674500877
0.97 0.0156370674500877
0.975 0.0156370674500877
0.98 0.0156370674500877
0.985 0.0156370674500877
0.99 0.0156370674500877
0.995 0.0156370674500877
};
\addlegendentry{FOM}
\addplot [semithick, red, mark=o,mark size=2, mark repeat=25, mark options={solid}]
table {%
0 0.498625068252937
0.005 0.498625068252925
0.01 0.498625068252902
0.015 0.498625068252857
0.02 0.498625068252763
0.025 0.498625068252555
0.03 0.498625068252104
0.035 0.49862506825114
0.04 0.498625068249076
0.045 0.498625068244678
0.05 0.498625068235402
0.055 0.498625068216044
0.06 0.498625068176088
0.065 0.498625068094528
0.07 0.498625067929928
0.075 0.498625067601551
0.08 0.498625066953994
0.085 0.498625065691945
0.09 0.498625063261437
0.095 0.498625058636801
0.1 0.498625049944138
0.105 0.498625033806027
0.11 0.49862500421869
0.115 0.49862495065886
0.12 0.498624854944535
0.125 0.498624686117918
0.13 0.498624392251324
0.135 0.49862388756532
0.14 0.498623032562605
0.145 0.498621603999727
0.15 0.498619250445458
0.155 0.498615427955829
0.16 0.498609309146351
0.165 0.498599657873226
0.17 0.498584661175883
0.175 0.498561710536654
0.18 0.498527126433899
0.185 0.498475824177929
0.19 0.498400925587275
0.195 0.498293330326205
0.2 0.49814127226332
0.205 0.497929898827616
0.21 0.497640922968887
0.215 0.497252405198263
0.22 0.496738724279561
0.225 0.496070787001924
0.23 0.49521650910091
0.235 0.494141572051478
0.24 0.492810427829793
0.245 0.491187491436285
0.25 0.489238435215899
0.255 0.486931484994448
0.26 0.484238618478227
0.265 0.481136580751056
0.27 0.477607656641814
0.275 0.473640170003247
0.28 0.46922870997415
0.285 0.464374109509521
0.29 0.459083219059018
0.295 0.453368527409278
0.3 0.447247683150007
0.305 0.440742965755295
0.31 0.433880747042387
0.315 0.426690973811312
0.32 0.419206692392557
0.325 0.411463626723806
0.33 0.403499814028093
0.335 0.395355296367017
0.34 0.387071862212956
0.345 0.37869282948689
0.35 0.370262859940735
0.355 0.361827794015729
0.36 0.353434495102821
0.365 0.345130692231178
0.37 0.336964810428308
0.375 0.328985778177342
0.38 0.321242801419793
0.385 0.313785093300736
0.39 0.306661548208254
0.395 0.299920347474363
0.4 0.293608482185813
0.405 0.287771175631133
0.41 0.282451183611503
0.415 0.277687944667622
0.42 0.273516543602126
0.425 0.26996643985966
0.43 0.267059897017509
0.435 0.264810031667844
0.44 0.263218383442823
0.445 0.262271904825078
0.45 0.261939308710615
0.455 0.262166855174695
0.46 0.262874020016432
0.465 0.263950229135682
0.47 0.265255069845643
0.475 0.266625777123379
0.48 0.267895889725081
0.485 0.268925020408209
0.49 0.26963033801455
0.495 0.270001701728146
0.5 0.270087323008733
0.505 0.269957248169946
0.51 0.269667081142932
0.515 0.269236876720657
0.52 0.26864199262919
0.525 0.267804762948415
0.53 0.266581189987853
0.535 0.264746630091301
0.54 0.261992019047724
0.545 0.257944609792719
0.55 0.252222457110084
0.555 0.244519460754593
0.56 0.234701239723374
0.565 0.22287963361912
0.57 0.209434200102417
0.575 0.194965967612635
0.58 0.180195230922152
0.585 0.165837357045217
0.59 0.152496014738539
0.595 0.140600033767407
0.6 0.130388158417552
0.605 0.12192899814333
0.61 0.115158489101641
0.615 0.109921294886794
0.62 0.106008878468737
0.625 0.103191244826849
0.63 0.101241318676698
0.635 0.0999518169876444
0.64 0.0991451161127038
0.645 0.0986771615521078
0.65 0.0984368373210163
0.655 0.0983423281863598
0.66 0.0983358762205455
0.665 0.0983780187284478
0.67 0.0984419875996118
0.675 0.0985085282807888
0.68 0.0985609990829604
0.685 0.0985802266228475
0.69 0.0985381540461125
0.695 0.0983886973108488
0.7 0.0980532128531553
0.705 0.0973962540212734
0.71 0.0961844112612728
0.715 0.0940167791201649
0.72 0.0902125236817476
0.725 0.0836586313961077
0.73 0.072760329902892
0.735 0.0562332989542404
0.74 0.0362830769821052
0.745 0.0206913794146427
0.75 0.0135920222695203
0.755 0.0113884738418281
0.76 0.0107936391799858
0.765 0.0106369030493254
0.77 0.0105956420413167
0.775 0.0105847779578505
0.78 0.0105819193508041
0.785 0.0105811680428482
0.79 0.0105809708531453
0.795 0.0105809191772497
0.8 0.0105809056574263
0.805 0.0105809021266249
0.81 0.0105809012063222
0.815 0.0105809009669498
0.82 0.0105809009048299
0.825 0.0105809008887486
0.83 0.0105809008845965
0.835 0.0105809008835274
0.84 0.010580900883253
0.845 0.0105809008831828
0.85 0.0105809008831647
0.855 0.01058090088316
0.86 0.0105809008831587
0.865 0.0105809008831583
0.87 0.0105809008831582
0.875 0.0105809008831582
0.88 0.0105809008831581
0.885 0.0105809008831581
0.89 0.0105809008831581
0.895 0.0105809008831581
0.9 0.0105809008831581
0.905 0.0105809008831581
0.91 0.0105809008831581
0.915 0.0105809008831581
0.92 0.0105809008831581
0.925 0.0105809008831581
0.93 0.0105809008831581
0.935 0.0105809008831581
0.94 0.0105809008831581
0.945 0.0105809008831581
0.95 0.0105809008831581
0.955 0.0105809008831581
0.96 0.0105809008831581
0.965 0.0105809008831581
0.97 0.0105809008831582
0.975 0.0105809008831582
0.98 0.0105809008831582
0.985 0.0105809008831582
0.99 0.0105809008831582
0.995 0.0105809008831582
};
\addlegendentry{POD}
\addplot [semithick, color0, mark=pentagon,mark size=2, mark repeat=25, mark options={solid}]
table {%
0 0.499566434696068
0.005 0.499566434696068
0.01 0.499566434696068
0.015 0.499566434696068
0.02 0.499566434696068
0.025 0.499566434696068
0.03 0.499566434696068
0.035 0.499566434696068
0.04 0.499566434696068
0.045 0.499566434696068
0.05 0.499566434696068
0.055 0.499566434696068
0.06 0.499566434696068
0.065 0.499566434696068
0.07 0.499566434696068
0.075 0.499566434696068
0.08 0.499566434696068
0.085 0.499566434696068
0.09 0.499566434696068
0.095 0.499566370348632
0.1 0.499567510892607
0.105 0.499567527370009
0.11 0.499568461188917
0.115 0.499565495813832
0.12 0.499567820889932
0.125 0.499567439440812
0.13 0.499567908092993
0.135 0.499566768757936
0.14 0.499564687891228
0.145 0.499561228374623
0.15 0.499559679675419
0.155 0.499553906343711
0.16 0.499542880865482
0.165 0.499527347451757
0.17 0.499503439223567
0.175 0.499467262862731
0.18 0.499412237656599
0.185 0.499332065979628
0.19 0.499212074995381
0.195 0.499043806748767
0.2 0.498802532356096
0.205 0.498472301059849
0.21 0.498020814721587
0.215 0.497416882395611
0.22 0.49662143605863
0.225 0.495597837974689
0.23 0.494300632658181
0.235 0.492685228301581
0.24 0.490716071969207
0.245 0.488356221513139
0.25 0.485586777843386
0.255 0.482366202590156
0.26 0.478636349469023
0.265 0.474370902309825
0.27 0.469563017550409
0.275 0.464211038835019
0.28 0.458326442635222
0.285 0.451927760843783
0.29 0.445051297040587
0.295 0.437740810869383
0.3 0.430049988082055
0.305 0.422041759481815
0.31 0.41377940822452
0.315 0.405341573274708
0.32 0.396805402694221
0.325 0.388252174850362
0.33 0.379800179743692
0.335 0.371559981380593
0.34 0.363620787652613
0.345 0.356180819505493
0.35 0.349377926001172
0.355 0.3432583843994
0.36 0.33782942289075
0.365 0.332347496805064
0.37 0.326394980320559
0.375 0.320012823852642
0.38 0.313236933914264
0.385 0.306129853124111
0.39 0.298745550819665
0.395 0.291790480170699
0.4 0.285852767651042
0.405 0.281006252061158
0.41 0.277299036681199
0.415 0.274759348070896
0.42 0.273400023151207
0.425 0.272964188700684
0.43 0.271815780753738
0.435 0.269828438346021
0.44 0.267249841352473
0.445 0.264350149833369
0.45 0.261423391158425
0.455 0.258766695461318
0.46 0.256654295660602
0.465 0.25529988899042
0.47 0.254513976549939
0.475 0.254076380414675
0.48 0.253857447711592
0.485 0.253797470865467
0.49 0.253808752077065
0.495 0.253763714009828
0.5 0.253699368179262
0.505 0.25362855235414
0.51 0.253532433385479
0.515 0.253393969678747
0.52 0.253173501656991
0.525 0.252807950506907
0.53 0.252185353063773
0.535 0.251113538453846
0.54 0.249354667481387
0.545 0.246752242224898
0.55 0.243418878017469
0.555 0.239465106067682
0.56 0.234279419366452
0.565 0.227783804517479
0.57 0.220192690876746
0.575 0.211972309824358
0.58 0.203765609306105
0.585 0.196264027025048
0.59 0.190010807978054
0.595 0.184577986106031
0.6 0.179033526307619
0.605 0.173685600452327
0.61 0.168782162839598
0.615 0.164524354824771
0.62 0.16104175884175
0.625 0.158359583934681
0.63 0.156417219529006
0.635 0.15509608100915
0.64 0.154262423100925
0.645 0.153786435607038
0.65 0.153558933343705
0.655 0.153494577775065
0.66 0.15352806838995
0.665 0.153617849069447
0.67 0.153725001225156
0.675 0.15382214803754
0.68 0.1538752986271
0.685 0.153841953133464
0.69 0.153640439458696
0.695 0.153146652727694
0.7 0.152149559024472
0.705 0.150290444535956
0.71 0.146908177847255
0.715 0.140638036539332
0.72 0.129557863401295
0.725 0.11102893867151
0.73 0.0820449516669314
0.735 0.046488003712627
0.74 0.0207998444859785
0.745 0.0142594923706771
0.75 0.014927622493576
0.755 0.0156025802899868
0.76 0.0158366322815483
0.765 0.0158999766958207
0.77 0.0159181611108628
0.775 0.0159240230115524
0.78 0.0159260359236937
0.785 0.0159249571562401
0.79 0.0159248001028088
0.795 0.0159241176643776
0.8 0.0159246474218944
0.805 0.0159247248240804
0.81 0.0159247248240804
0.815 0.0159247248240804
0.82 0.0159247248240804
0.825 0.0159247248240804
0.83 0.0159247248240804
0.835 0.0159247248240804
0.84 0.0159247248240804
0.845 0.0159247248240804
0.85 0.0159247248240804
0.855 0.0159247248240804
0.86 0.0159247248240804
0.865 0.0159247248240804
0.87 0.0159247248240804
0.875 0.0159247248240804
0.88 0.0159247248240804
0.885 0.0159247248240804
0.89 0.0159247248240804
0.895 0.0159247248240804
0.9 0.0159247248240804
0.905 0.0159247248240804
0.91 0.0159247248240804
0.915 0.0159247248240804
0.92 0.0159247248240804
0.925 0.0159247248240804
0.93 0.0159247248240804
0.935 0.0159247248240804
0.94 0.0159247248240804
0.945 0.0159247248240804
0.95 0.0159247248240804
0.955 0.0159247248240804
0.96 0.0159247248240804
0.965 0.0159247248240804
0.97 0.0159247248240804
0.975 0.0159247248240804
0.98 0.0159247248240804
0.985 0.0159247248240804
0.99 0.0159247248240804
0.995 0.0159247248240804
};
\addlegendentry{FCNN}
\addplot [semithick, green!50!black, mark=triangle,mark size=2, mark repeat=25, mark options={solid,rotate=180}, only marks]
table {%
0 0.580351209094351
0.005 0.57077945042704
0.01 0.539566898244465
0.015 0.55106193474696
0.02 0.574102505754469
0.025 0.505732785772193
0.03 0.509329440964851
0.035 0.501065754496298
0.04 0.523745148063936
0.045 0.553555970460756
0.05 0.491158076056909
0.055 0.498286206202749
0.06 0.509936583711341
0.065 0.526394265164651
0.07 0.56322617893089
0.075 0.511250478689714
0.08 0.507527396493573
0.085 0.495623678835766
0.09 0.519408597745589
0.095 0.540947956025889
0.1 0.555902878529499
0.105 0.54945453369396
0.11 0.528396269030057
0.115 0.561606350697884
0.12 0.581694721320512
0.125 0.526519065623212
0.13 0.496948294063992
0.135 0.500369962916352
0.14 0.463117191516237
0.145 0.446407516303542
0.15 0.573506758368176
0.155 0.530642595391712
0.16 0.513895661247344
0.165 0.470505105755225
0.17 0.450856737137399
0.175 0.545205582490985
0.18 0.498133059336671
0.185 0.477500838033298
0.19 0.425331193976563
0.195 0.417816230376557
0.2 0.526838634222763
0.205 0.486946088935088
0.21 0.43581573900696
0.215 0.422335432457849
0.22 0.400775903199294
0.225 0.473036689159518
0.23 0.451567000822283
0.235 0.462898673935365
0.24 0.40679197023438
0.245 0.413189219893051
0.25 0.369818714487334
0.255 0.366056781355128
0.26 0.380184823057476
0.265 0.362221454622405
0.27 0.369854100066303
0.275 0.380801566940673
0.28 0.358284062007452
0.285 0.35241798847653
0.29 0.321141237256712
0.295 0.308630189579946
0.3 0.387301366990995
0.305 0.345046071463847
0.31 0.319324327797677
0.315 0.263788799099293
0.32 0.220805700897333
0.325 0.284568319494346
0.33 0.236051790893339
0.335 0.217995584361025
0.34 0.124150679551091
0.345 0.107738485902334
0.35 0.0666575512868527
0.355 0.0720109431337936
0.36 0.12473898169491
0.365 0.0666177757950225
0.37 0.0675841922330152
0.375 0.213746063867327
0.38 0.197151566686961
0.385 0.179358924551715
0.39 0.185420766454552
0.395 0.205239784189822
0.4 0.270208933617
0.405 0.241838500650269
0.41 0.21842642383104
0.415 0.204848744452808
0.42 0.208164746792572
0.425 0.277863157785523
0.43 0.244838279295333
0.435 0.240918737100281
0.44 0.19144537643013
0.445 0.176140461338021
0.45 0.212450752824093
0.455 0.193764641650536
0.46 0.182292476065263
0.465 0.151030283741313
0.47 0.0989682372011611
0.475 0.0168237314314365
0.48 -0.011774723602192
0.485 0.00824626950089414
0.49 -0.0947823899122481
0.495 -0.120032093160333
0.5 0.225044650544667
0.505 0.25046637052675
0.51 0.345817416118199
0.515 0.26327521270465
0.52 0.27926023351757
0.525 0.239290549054387
0.53 0.253769216944143
0.535 0.329768571440985
0.54 0.258440448599594
0.545 0.272952148539458
0.55 0.163096428049381
0.555 0.177643363976322
0.56 0.224027714940232
0.565 0.209298023817162
0.57 0.240353324839574
0.575 0.184773587843442
0.58 0.196784554535292
0.585 0.221444004916069
0.59 0.240227178933198
0.595 0.288983614620402
0.6 0.243242048272678
0.605 0.256933861303238
0.61 0.266190762259576
0.615 0.327596730914184
0.62 0.379061837980178
0.625 0.293822474729815
0.63 0.289526802813796
0.635 0.321188769797794
0.64 0.325742431771872
0.645 0.316540416470078
0.65 0.363755345327897
0.655 0.342410616712464
0.66 0.341539712436685
0.665 0.345074952495956
0.67 0.329564717759189
0.675 0.381584919264603
0.68 0.356644229093139
0.685 0.357743849574851
0.69 0.348588207759581
0.695 0.342149075071588
0.7 0.484827673689255
0.705 0.434427770168046
0.71 0.362122512836843
0.715 0.353857583348845
0.72 0.285105276953075
0.725 0.324912682348258
0.73 0.28507637004697
0.735 0.269128542036309
0.74 0.198286743566259
0.745 0.189568224582876
0.75 0.241452108749669
0.755 0.224626642818457
0.76 0.210703651219036
0.765 0.159097264156785
0.77 0.163212043113606
0.775 0.100389747531589
0.78 0.0880875631065737
0.785 0.0911350586763443
0.79 0.0330207428584245
0.795 0.0482875397760968
0.8 -0.0178879417162022
0.805 -0.0162847677331684
0.81 -0.0110312658407166
0.815 -0.0173912733983735
0.82 -0.00625119176520834
0.825 -0.0505232753206818
0.83 -0.0387546483731013
0.835 -0.00492758957930424
0.84 -0.00448533879764755
0.845 0.0121094639093998
0.85 0.0311592744210672
0.855 0.0147257080976437
0.86 0.0187220148604639
0.865 0.00299166940801596
0.87 -0.00620980373307118
0.875 0.0171002143004383
0.88 0.0314193063268115
0.885 0.0923047494572561
0.89 0.0238246273302254
0.895 0.0784464392064668
0.9 -0.00979254059967602
0.905 0.00628188064861579
0.91 0.0625897986953574
0.915 0.017842934904219
0.92 0.0553172643937923
0.925 -0.0133510284447434
0.93 -0.0240271303517264
0.935 0.00719655078936462
0.94 -0.0657238903863322
0.945 -0.0516665471184085
0.95 -0.160999635174841
0.955 -0.132952922390871
0.96 -0.0524919998885909
0.965 -0.0957160104945508
0.97 -0.0386176507744581
0.975 -0.0830408011653402
0.98 -0.0550460874524865
0.985 0.00134739991972707
0.99 -0.00867601711338034
0.995 0.0347083169386156
};
\addlegendentry{CNN}

\nextgroupplot[
legend cell align={left},
legend style={at={(1,1)},
	anchor=north east,
	fill opacity=0.1,
	draw opacity=1,
	text opacity=1,
	draw=none},
ylabel={\(\rho\)},
ymin=0.0749713271856308, ymax=1.04554586708546,
]
\addplot [semithick, black, dashed, mark=x,mark size=2, mark repeat=25, mark options={solid}]
table {%
0 0.999999807666081
0.005 0.999999748719081
0.01 0.999999674796829
0.015 0.999999578515261
0.02 0.999999452070285
0.025 0.999999285816446
0.03 0.999999067316936
0.035 0.999998780365606
0.04 0.999998403807108
0.045 0.999997910049692
0.05 0.999997263171242
0.055 0.999996416506449
0.06 0.999995309581959
0.065 0.999993864239697
0.07 0.999991979757303
0.075 0.999989526739214
0.08 0.999986339513002
0.085 0.999982206724206
0.09 0.999976859780583
0.095 0.999969958755766
0.1 0.999961075326019
0.105 0.999949672286213
0.11 0.999935079177682
0.115 0.999916463567656
0.12 0.999892797554757
0.125 0.999862819145735
0.13 0.99982498826325
0.135 0.999777437310758
0.14 0.999717916444541
0.145 0.999643733987731
0.15 0.999551692766091
0.155 0.999438023543809
0.16 0.999298317176075
0.165 0.999127457552945
0.17 0.998919557856471
0.175 0.998667903054044
0.18 0.998364901863486
0.185 0.99800205160545
0.19 0.997569919364129
0.195 0.997058142673286
0.2 0.996455452509002
0.205 0.99574972069893
0.21 0.994928032967131
0.215 0.993976787763627
0.22 0.992881819839973
0.225 0.991628546305134
0.23 0.990202131718563
0.235 0.988587667739745
0.24 0.986770362036876
0.245 0.98473573062587
0.25 0.982469787603312
0.255 0.979959226364307
0.26 0.977191586840609
0.265 0.97415540401367
0.27 0.970840333891215
0.275 0.967237254215386
0.28 0.963338338325584
0.285 0.959137101767173
0.29 0.954628422367118
0.295 0.949808535550069
0.3 0.944675007613097
0.305 0.93922669048504
0.31 0.93346366213093
0.315 0.927387157171918
0.32 0.920999492407739
0.325 0.914303991672197
0.33 0.907304913751212
0.335 0.90000738591857
0.34 0.892417344055854
0.345 0.884541478519005
0.35 0.876387183275967
0.355 0.867962504945375
0.36 0.859276088942573
0.365 0.850337122723488
0.37 0.841155281587692
0.375 0.831740690478113
0.38 0.822103924376787
0.385 0.812256077407175
0.39 0.802208932296694
0.395 0.791975252482376
0.4 0.781569195218389
0.405 0.771006805995516
0.41 0.760306509342156
0.415 0.749489472124117
0.42 0.738579698452171
0.425 0.727603729465644
0.43 0.71658986174401
0.435 0.705566850669331
0.44 0.694562135134672
0.445 0.68359978054784
0.45 0.672698740784772
0.455 0.661872775215992
0.46 0.651134050812332
0.465 0.640501842836554
0.47 0.630014074440158
0.475 0.619732881873294
0.48 0.609731506896937
0.485 0.600058305346286
0.49 0.590696145471375
0.495 0.581551383185598
0.5 0.572476507211133
0.505 0.563293923280107
0.51 0.553800233185305
0.515 0.543781521373741
0.52 0.533063507520285
0.525 0.521566254711062
0.53 0.509313959487317
0.535 0.496400819506339
0.54 0.482943369363496
0.545 0.469048299785303
0.55 0.454804358117032
0.555 0.440289846649872
0.56 0.425583489588254
0.565 0.410771644257896
0.57 0.395950812291248
0.575 0.381227048515738
0.58 0.366713602378983
0.585 0.352527020018551
0.59 0.338781437380338
0.595 0.325581202776111
0.6 0.313012826599636
0.605 0.301137930363013
0.61 0.28998893052361
0.615 0.279568585432994
0.62 0.269853519370675
0.625 0.260800826173275
0.63 0.25235619826391
0.635 0.244461883155795
0.64 0.237063085501019
0.645 0.230112020114516
0.65 0.223569462915682
0.655 0.217404175242648
0.66 0.211590905285989
0.665 0.206107782796998
0.67 0.200933851635925
0.675 0.19604728570937
0.68 0.191424572450542
0.685 0.187040686463892
0.69 0.182870064680001
0.695 0.178888064771805
0.7 0.175072549621457
0.705 0.171405280409886
0.71 0.167872893146608
0.715 0.164467346526728
0.72 0.161185834559035
0.725 0.158030236348808
0.73 0.155006220081722
0.735 0.152122131241113
0.74 0.149387785783819
0.745 0.146813269140009
0.75 0.14440782144993
0.755 0.142178873756478
0.76 0.140131288852554
0.765 0.138266849981738
0.77 0.136584025400125
0.775 0.135078014041113
0.78 0.133741048757712
0.785 0.132562904652122
0.79 0.131531538365762
0.795 0.13063377571739
0.8 0.129855971357034
0.805 0.129184582173803
0.81 0.128606620139977
0.815 0.128109973780338
0.82 0.12768360592339
0.825 0.127317646828479
0.83 0.12700340652163
0.835 0.126733329925724
0.84 0.126500915227678
0.845 0.126300611662415
0.85 0.126127708680483
0.855 0.125978224881435
0.86 0.125848802292361
0.865 0.125736609473238
0.87 0.125639255381861
0.875 0.125554714782859
0.88 0.125481265132884
0.885 0.125417434255349
0.89 0.125361957699734
0.895 0.125313744440518
0.9 0.125271849487081
0.905 0.125235452020281
0.91 0.125203837809891
0.915 0.125176384863169
0.92 0.12515255147411
0.925 0.12513186605675
0.93 0.125113918334066
0.935 0.125098351605385
0.94 0.125084855926945
0.945 0.125073162116278
0.95 0.125063036539648
0.955 0.125054276673823
0.96 0.125046707458714
0.965 0.125040178477778
0.97 0.125034561985197
0.975 0.125029751574714
0.98 0.125025660209438
0.985 0.125022212226635
0.99 0.125019310312244
0.995 0.12501671885813
};
\addlegendentry{FOM}
\addplot [semithick, red, mark=o,mark size=2, mark repeat=25, mark options={solid}]
table {%
0 0.999760895679759
0.005 0.999760851680844
0.01 0.999760797470731
0.015 0.999760726986592
0.02 0.999760634202013
0.025 0.999760511768295
0.03 0.999760350211052
0.035 0.999760137141386
0.04 0.999759856311489
0.045 0.9997594864208
0.05 0.999758999587014
0.055 0.999758359385316
0.06 0.99975751834002
0.065 0.999756414728244
0.07 0.999754968525773
0.075 0.999753076291286
0.08 0.999750604746799
0.085 0.999747382770229
0.09 0.999743191471376
0.095 0.999737751977115
0.1 0.999730710507782
0.105 0.999721620287933
0.11 0.999709919805669
0.115 0.999694906921063
0.12 0.999675708332906
0.125 0.999651243951663
0.13 0.999620185803635
0.135 0.999580911215244
0.14 0.999531450204506
0.145 0.999469427244953
0.15 0.999391997867752
0.155 0.999295780928574
0.16 0.99917678777859
0.165 0.999030350028865
0.17 0.998851048061305
0.175 0.998632642886997
0.18 0.998368014347335
0.185 0.998049108954126
0.19 0.9976669008303
0.195 0.997211369205461
0.2 0.99667149571117
0.205 0.99603528429424
0.21 0.995289805924162
0.215 0.994421269434689
0.22 0.993415118850799
0.225 0.992256156468984
0.23 0.990928689851751
0.235 0.989416699841983
0.24 0.987704025772366
0.245 0.985774563302139
0.25 0.983612469803068
0.255 0.981202371963766
0.26 0.978529570288291
0.265 0.975580235414127
0.27 0.972341591633106
0.275 0.96880208362481
0.28 0.964951523160958
0.285 0.960781213369607
0.29 0.956284049023491
0.295 0.951454592205381
0.3 0.946289123572831
0.305 0.940785670254691
0.31 0.934944012105768
0.315 0.928765668546039
0.32 0.922253868419701
0.325 0.915413505124007
0.33 0.908251078598436
0.335 0.900774624615425
0.34 0.892993630274175
0.345 0.88491893293017
0.35 0.876562598445683
0.355 0.867937774247201
0.36 0.859058513949998
0.365 0.849939573922002
0.37 0.840596188455743
0.375 0.831043838889471
0.38 0.821298041739997
0.385 0.811374189095531
0.39 0.801287477440989
0.395 0.791052954648059
0.4 0.780685696155044
0.405 0.770201090826175
0.41 0.759615180155827
0.415 0.748944961755757
0.42 0.738208550890919
0.425 0.727425097828532
0.43 0.716614378294661
0.435 0.705795998408027
0.44 0.694988195485123
0.445 0.684206350359944
0.45 0.673461711727061
0.455 0.662761568224042
0.46 0.652112852256648
0.465 0.641530710299299
0.47 0.631050181584805
0.475 0.620732748029827
0.48 0.610655226870101
0.485 0.600875730392182
0.49 0.591392744600075
0.495 0.582130185885733
0.5 0.572953123936908
0.505 0.563685782887917
0.51 0.55411639029354
0.515 0.544020416748259
0.52 0.533220562045958
0.525 0.521646444580508
0.53 0.509341047204445
0.535 0.496417578596848
0.54 0.483003670622555
0.545 0.469206449115345
0.55 0.455107009557864
0.555 0.440772770871734
0.56 0.426272385861248
0.565 0.411684701732669
0.57 0.397100839398715
0.575 0.382621922847698
0.58 0.368354847533703
0.585 0.354407122030914
0.59 0.340880943475802
0.595 0.327866686811195
0.6 0.315436515812782
0.605 0.303639291393874
0.61 0.292497982421681
0.615 0.282010312843701
0.62 0.272152602736278
0.625 0.262885989404059
0.63 0.254163713792584
0.635 0.24593805098716
0.64 0.238165719854067
0.645 0.230811091539637
0.65 0.223847065965373
0.655 0.217253963074347
0.66 0.211017096738863
0.665 0.205123831567485
0.67 0.199560875976747
0.675 0.194312378698639
0.68 0.189359129347358
0.685 0.184678882745078
0.69 0.180247591038008
0.695 0.176041178344575
0.7 0.172037445187326
0.705 0.168217734160593
0.71 0.164568095545718
0.715 0.16107982548543
0.72 0.157749376195892
0.725 0.154577733934068
0.73 0.151569415470327
0.735 0.148731249075148
0.74 0.146071091377888
0.745 0.14359660054363
0.75 0.141314152011149
0.755 0.139227954438339
0.76 0.137339403626989
0.765 0.135646698710632
0.77 0.134144732293453
0.775 0.132825249368378
0.78 0.131677247135903
0.785 0.130687562625713
0.79 0.129841574145688
0.795 0.12912393297997
0.8 0.128519246908503
0.805 0.128012655430691
0.81 0.12759026236858
0.815 0.127239417744213
0.82 0.126948861777316
0.825 0.126708756709368
0.83 0.12651063696808
0.835 0.126347306853931
0.84 0.126212710015468
0.845 0.126101788766555
0.85 0.126010345355565
0.855 0.125934912434913
0.86 0.125872636428346
0.865 0.125821175135015
0.87 0.125778609477627
0.875 0.12574336851376
0.88 0.125714166449377
0.885 0.125689950253501
0.89 0.125669856465837
0.895 0.125653175852953
0.9 0.125639324670331
0.905 0.125627821409779
0.91 0.125618268044381
0.915 0.125610334919376
0.92 0.125603748571127
0.925 0.125598281882641
0.93 0.125593746099059
0.935 0.125589984327631
0.94 0.125586866233433
0.945 0.125584283715405
0.95 0.125582147409843
0.955 0.125580383923944
0.96 0.125578933753462
0.965 0.125577749882576
0.97 0.125576797066214
0.975 0.125576051597688
0.98 0.125575500368943
0.985 0.125575134215665
0.99 0.12557491787781
0.995 0.12557468207066
};
\addlegendentry{POD}
\addplot  [semithick, color0, mark=pentagon,mark size=2, mark repeat=25, mark options={solid}]
table {%
0 0.999889896715913
0.005 0.999889850418435
0.01 0.999889759942836
0.015 0.999889645228969
0.02 0.999889376488678
0.025 0.999889291535343
0.03 0.9998890944651
0.035 0.999888803516347
0.04 0.999888493612792
0.045 0.999887837896033
0.05 0.999887240267534
0.055 0.999886321481999
0.06 0.999885208551529
0.065 0.999883731360503
0.07 0.999881709286074
0.075 0.999879285728038
0.08 0.999875870968167
0.085 0.999871593361815
0.09 0.999865979874727
0.095 0.999858917464642
0.1 0.999849660685453
0.105 0.999837846082557
0.11 0.999822798357541
0.115 0.999803502965805
0.12 0.999779113209765
0.125 0.999748224165443
0.13 0.999709422968741
0.135 0.999660608144012
0.14 0.999599669827735
0.145 0.99952382820114
0.15 0.999429935393943
0.155 0.999314257638441
0.16 0.999172312556718
0.165 0.998999162586952
0.17 0.998788746222169
0.175 0.998534727155538
0.18 0.998229500839777
0.185 0.997864783429282
0.19 0.997431341517477
0.195 0.996919135357515
0.2 0.996317174358021
0.205 0.995613998176877
0.21 0.994796947095656
0.215 0.993853050775972
0.22 0.992766209456507
0.225 0.991501784982295
0.23 0.990065733549219
0.235 0.988443442109162
0.24 0.986620553608313
0.245 0.984619016155526
0.25 0.98234768640484
0.255 0.979835163361959
0.26 0.977074200471638
0.265 0.974061538134736
0.27 0.970775840109386
0.275 0.967208350441081
0.28 0.963350908087454
0.285 0.959197229785271
0.29 0.954719580091709
0.295 0.949782740133654
0.3 0.944537943532506
0.305 0.938984541437023
0.31 0.933123055865446
0.315 0.926954704223535
0.32 0.920481191339042
0.325 0.913705567353083
0.33 0.906661838404829
0.335 0.89938810493045
0.34 0.891818694450306
0.345 0.883958033761461
0.35 0.875810984427427
0.355 0.867383208232725
0.36 0.858681326164423
0.365 0.849711979806145
0.37 0.840482985879556
0.375 0.831002720407316
0.38 0.821449204773231
0.385 0.811707735976238
0.39 0.801733101350297
0.395 0.791540028000689
0.4 0.78063068738797
0.405 0.769661297543677
0.41 0.759693290650922
0.415 0.749389822204438
0.42 0.738781754323772
0.425 0.727909235638344
0.43 0.716806641861194
0.435 0.70582175310789
0.44 0.694773038669693
0.445 0.683742230161597
0.45 0.672791028688885
0.455 0.66197466251573
0.46 0.651344954758161
0.465 0.640678788169702
0.47 0.63012017098022
0.475 0.619800940824599
0.48 0.609771598639321
0.485 0.600077750962264
0.49 0.590697297717803
0.495 0.581537349240608
0.5 0.572453568203658
0.505 0.563270277081462
0.51 0.553796403526757
0.515 0.543781628927921
0.52 0.5330787775733
0.525 0.521616360455908
0.53 0.509422848763397
0.535 0.496365561028901
0.54 0.48284045587851
0.545 0.469002705920641
0.55 0.454880930727704
0.555 0.440561089223657
0.56 0.426130692984493
0.565 0.410894435146579
0.57 0.395558760325827
0.575 0.381080825382805
0.58 0.36689092782442
0.585 0.352902637274513
0.59 0.339239202633625
0.595 0.326016129870033
0.6 0.313332385648234
0.605 0.301288741129902
0.61 0.289982207888689
0.615 0.279558734439664
0.62 0.26988463435324
0.625 0.260807849638292
0.63 0.252304591331387
0.635 0.244267732501766
0.64 0.236711912734269
0.645 0.229645604347919
0.65 0.223154043208132
0.655 0.21698732600748
0.66 0.21113219307073
0.665 0.205577377603698
0.67 0.200445614488505
0.675 0.195620153127841
0.68 0.191045531591086
0.685 0.186701070218788
0.69 0.182564484491662
0.695 0.178613554127804
0.7 0.174828068203604
0.705 0.171190800773833
0.71 0.167689083194891
0.715 0.164315250652181
0.72 0.161066554997869
0.725 0.157944835832833
0.73 0.154955618347614
0.735 0.152107752822982
0.74 0.149410770808133
0.745 0.14687487267321
0.75 0.144509641309553
0.755 0.142322791739595
0.76 0.14018671927908
0.765 0.13822335393251
0.77 0.13645020829545
0.775 0.134863520481422
0.78 0.133462308661517
0.785 0.132257582710262
0.79 0.131212870155279
0.795 0.130337763887685
0.8 0.129606891068827
0.805 0.128962422895357
0.81 0.128396274353069
0.815 0.127900215535456
0.82 0.127478092745031
0.825 0.127115257740429
0.83 0.126801783578263
0.835 0.126547213449261
0.84 0.126351290911366
0.845 0.126181791636522
0.85 0.126037317297622
0.855 0.125912243004845
0.86 0.125804090425255
0.865 0.125710641220131
0.87 0.125630047191775
0.875 0.12556058130397
0.88 0.125500826155407
0.885 0.125436563644409
0.89 0.125372908224385
0.895 0.125318164098698
0.9 0.125283321082617
0.905 0.125254067016856
0.91 0.125231566293427
0.915 0.12521292282749
0.92 0.125197498215334
0.925 0.125184632829839
0.93 0.125174048139886
0.935 0.125165317110197
0.94 0.12515809458427
0.945 0.125152160238648
0.95 0.125147266645996
0.955 0.125143250675617
0.96 0.12513992095863
0.965 0.125137237227276
0.97 0.125134984560961
0.975 0.125133116692057
0.98 0.125131561383366
0.985 0.125130158608596
0.99 0.125128803623808
0.995 0.125127164197584
};
\addlegendentry{FCNN}
\addplot  [semithick, green!50!black, mark=triangle,mark size=2, mark repeat=25, mark options={solid,rotate=180}, only marks]
table {%
0 1.00062537193298
0.005 1.00060212612152
0.01 0.998456299304962
0.015 1.00047671794891
0.02 1.00142884254456
0.025 0.997138798236847
0.03 0.997770369052887
0.035 0.996849179267883
0.04 0.999260425567627
0.045 1.00041127204895
0.05 0.996256053447723
0.055 0.997073233127594
0.06 0.997224569320679
0.065 0.99932336807251
0.07 1.00059998035431
0.075 0.997550785541534
0.08 0.997279524803162
0.085 0.995962381362915
0.09 0.997898519039154
0.095 0.997852385044098
0.1 0.996607422828674
0.105 0.996703267097473
0.11 0.995492577552795
0.115 0.998612463474274
0.12 0.998303353786469
0.125 0.998010277748108
0.13 0.998366117477417
0.135 1.00082731246948
0.14 0.999769389629364
0.145 0.999017059803009
0.15 1.00069200992584
0.155 1.00010633468628
0.16 1.00072956085205
0.165 0.999780893325806
0.17 0.999170899391174
0.175 0.99886280298233
0.18 0.997635722160339
0.185 0.997453272342682
0.19 0.995911180973053
0.195 0.996345937252045
0.2 0.996500849723816
0.205 0.995123744010925
0.21 0.992776453495026
0.215 0.993221402168274
0.22 0.99321585893631
0.225 0.99333256483078
0.23 0.991859078407288
0.235 0.991490125656128
0.24 0.98896461725235
0.245 0.989710450172424
0.25 0.983568549156189
0.255 0.981343150138855
0.26 0.980188488960266
0.265 0.975984513759613
0.27 0.974099397659302
0.275 0.96777206659317
0.28 0.963837206363678
0.285 0.961291193962097
0.29 0.954907536506653
0.295 0.950952172279358
0.3 0.9440957903862
0.305 0.937816560268402
0.31 0.932757973670959
0.315 0.923819363117218
0.32 0.917341709136963
0.325 0.908691942691803
0.33 0.899710059165955
0.335 0.892471373081207
0.34 0.879778981208801
0.345 0.872752845287323
0.35 0.860883474349976
0.355 0.851947426795959
0.36 0.844620287418365
0.365 0.832148551940918
0.37 0.824062943458557
0.375 0.818234264850616
0.38 0.807645082473755
0.385 0.796486020088196
0.39 0.786113500595093
0.395 0.776748061180115
0.4 0.768569827079773
0.405 0.756597459316254
0.41 0.744108140468597
0.415 0.73252284526825
0.42 0.721391022205353
0.425 0.713635683059692
0.43 0.701026618480682
0.435 0.68910950422287
0.44 0.675478041172028
0.445 0.663831889629364
0.45 0.655366539955139
0.455 0.644476771354675
0.46 0.633105397224426
0.465 0.621878743171692
0.47 0.609520494937897
0.475 0.593138337135315
0.48 0.584324419498444
0.485 0.576303958892822
0.49 0.564955055713654
0.495 0.557281255722046
0.5 0.558571398258209
0.505 0.555612981319427
0.51 0.556053340435028
0.515 0.546999096870422
0.52 0.543053925037384
0.525 0.533135175704956
0.53 0.526678025722504
0.535 0.523690104484558
0.54 0.511663615703583
0.545 0.502153158187866
0.55 0.488810986280441
0.555 0.471708387136459
0.56 0.455751329660416
0.565 0.435831308364868
0.57 0.413508415222168
0.575 0.38691321015358
0.58 0.366692394018173
0.585 0.346397340297699
0.59 0.325962960720062
0.595 0.308968037366867
0.6 0.283180296421051
0.605 0.273822039365768
0.61 0.263912588357925
0.615 0.256514102220535
0.62 0.25090679526329
0.625 0.240992724895477
0.63 0.237508565187454
0.635 0.235122993588448
0.64 0.232109189033508
0.645 0.228361904621124
0.65 0.223899900913239
0.655 0.221167996525764
0.66 0.219134718179703
0.665 0.217464298009872
0.67 0.214651271700859
0.675 0.212282910943031
0.68 0.209636181592941
0.685 0.209408164024353
0.69 0.206056088209152
0.695 0.204069748520851
0.7 0.212050512433052
0.705 0.206161141395569
0.71 0.199802026152611
0.715 0.194956034421921
0.72 0.184444606304169
0.725 0.180104449391365
0.73 0.170944944024086
0.735 0.159278213977814
0.74 0.152427136898041
0.745 0.147041335701942
0.75 0.135703563690186
0.755 0.135170713067055
0.76 0.134590059518814
0.765 0.131767526268959
0.77 0.132337734103203
0.775 0.128607735037804
0.78 0.128531217575073
0.785 0.128966301679611
0.79 0.12637934088707
0.795 0.127654209733009
0.8 0.122741937637329
0.805 0.123513452708721
0.81 0.123946636915207
0.815 0.124840132892132
0.82 0.125911086797714
0.825 0.121396273374557
0.83 0.122605308890343
0.835 0.124712459743023
0.84 0.125711098313332
0.845 0.126992344856262
0.85 0.1274583786726
0.855 0.127244770526886
0.86 0.127654552459717
0.865 0.128335610032082
0.87 0.128321126103401
0.875 0.125008150935173
0.88 0.126541778445244
0.885 0.130379155278206
0.89 0.127354696393013
0.895 0.130938321352005
0.9 0.123212061822414
0.905 0.124855846166611
0.91 0.128360271453857
0.915 0.126935511827469
0.92 0.12962219119072
0.925 0.123809270560741
0.93 0.124237112700939
0.935 0.126606017351151
0.94 0.124257110059261
0.945 0.125809282064438
0.95 0.119088351726532
0.955 0.120888791978359
0.96 0.125112175941467
0.965 0.123999275267124
0.97 0.127803355455399
0.975 0.123541682958603
0.98 0.125117644667625
0.985 0.127692326903343
0.99 0.12860606610775
0.995 0.131389111280441
};
\addlegendentry{CNN}

\nextgroupplot[
legend cell align={left},
legend style={at={(0.0,1)},anchor=north west, opacity=0.1, draw opacity=1, text opacity=1,draw=none},
xlabel={\(x\)},
ylabel={\(\rho u\)},
ytick={0,0.1,0.3,0.4},
ymin=-0.0289403948808258, ymax=0.448410852717445,
width=.37\textwidth,
height=.4\textwidth,
y label style={yshift=-.9em},
y label style={xshift=+.7em}
]
\addplot [semithick, black, dashed, mark=x,mark size=2, mark repeat=25, mark options={solid}]
table {%
0 4.54455296600542e-07
0.005 5.9985472300685e-07
0.01 7.88339375095883e-07
0.015 1.03258648852749e-06
0.02 1.34970551615619e-06
0.025 1.76218161750585e-06
0.03 2.29933729248087e-06
0.035 2.99930283451103e-06
0.04 3.91157900605425e-06
0.045 5.10032656535721e-06
0.05 6.64856007512515e-06
0.055 8.6634674442852e-06
0.06 1.12831253004778e-05
0.065 1.468493488727e-05
0.07 1.90961638762875e-05
0.075 2.48070457205261e-05
0.08 3.21869583875013e-05
0.085 4.17042759332731e-05
0.09 5.39505554185066e-05
0.095 6.96697826532025e-05
0.1 8.97934458085532e-05
0.105 0.000115482226727408
0.11 0.000148175084444427
0.115 0.000189646440684595
0.12 0.000242072048313612
0.125 0.000308103915386144
0.13 0.000390954354824754
0.135 0.000494488820231745
0.14 0.000623326663673109
0.145 0.000782948310420776
0.15 0.000979806597068939
0.155 0.00122143918446957
0.16 0.00151657807137136
0.165 0.00187525135065397
0.17 0.00230887153522797
0.175 0.00283030411612859
0.18 0.0034539095903976
0.185 0.00419555210119987
0.19 0.00507256814837626
0.195 0.00610368961503879
0.2 0.00730891664370317
0.205 0.00870933767097807
0.21 0.0103268961317184
0.215 0.0121841058611737
0.22 0.0143037199016964
0.225 0.0167083600710161
0.23 0.019420117068359
0.235 0.0224601328850072
0.24 0.0258481786771399
0.245 0.0296022419282834
0.25 0.0337381366152874
0.255 0.0382691492023285
0.26 0.0432057316947753
0.265 0.0485552508169737
0.27 0.0543217998019709
0.275 0.0605060764824809
0.28 0.0671053285351274
0.285 0.0741133640202822
0.29 0.0815206229144609
0.295 0.0893143032548566
0.3 0.0974785338790117
0.305 0.105994584596552
0.31 0.114841104009903
0.315 0.123994375135092
0.32 0.133428579482419
0.325 0.14311606134221
0.33 0.15302758564526
0.335 0.163132584823495
0.34 0.173399392375569
0.345 0.183795463015846
0.35 0.194287580911644
0.355 0.20484205808796
0.36 0.215424924140179
0.365 0.226002105694562
0.37 0.236539589774906
0.375 0.247003560212418
0.38 0.257360492132517
0.385 0.267577188789006
0.39 0.27762075033611
0.395 0.287458477611135
0.4 0.29705773556341
0.405 0.306385826775564
0.41 0.315409947230083
0.415 0.324097302489628
0.42 0.332415442078596
0.425 0.340332819416506
0.43 0.347819512527544
0.435 0.354847966262696
0.44 0.361393560352427
0.445 0.367434777728012
0.45 0.372952742570127
0.455 0.377929944441451
0.46 0.382348184917035
0.465 0.386186386143682
0.47 0.389419923077351
0.475 0.39202389014746
0.48 0.393981526145045
0.485 0.395294659731944
0.49 0.395987920090632
0.495 0.396099553510851
0.5 0.395658924537269
0.505 0.39466232310167
0.51 0.39306310382351
0.515 0.390783104604446
0.52 0.387738696177698
0.525 0.383869350009942
0.53 0.37915535793569
0.535 0.373617298320719
0.54 0.367302888216957
0.545 0.360271898211032
0.55 0.352586214718681
0.555 0.344306225940405
0.56 0.33549144247301
0.565 0.326203073298632
0.57 0.316507208304638
0.575 0.306477786978538
0.58 0.296198456559623
0.585 0.285762280876216
0.59 0.275268549641514
0.595 0.264816735902625
0.6 0.254498649957448
0.605 0.244390585790918
0.61 0.234547407439115
0.615 0.225000010230103
0.62 0.215756632540806
0.625 0.206807460279548
0.63 0.198131206765991
0.635 0.189702054908685
0.64 0.181495514906113
0.645 0.173492239017786
0.65 0.165679453456796
0.655 0.158050243972552
0.66 0.150601349509476
0.665 0.143330322405805
0.67 0.136232902283778
0.675 0.129301263579978
0.68 0.122523500768609
0.685 0.115884389975564
0.69 0.109367184684881
0.695 0.102956019193393
0.7 0.0966384282148705
0.705 0.090407535240329
0.71 0.0842635834372528
0.715 0.07821463818847
0.72 0.0722764396450493
0.725 0.0664714987964292
0.73 0.0608275996641119
0.735 0.0553758965491429
0.74 0.0501487921833653
0.745 0.0451777666759582
0.75 0.0404913111795752
0.755 0.0361131088901486
0.76 0.0320605950931976
0.765 0.0283440078275513
0.77 0.0249660026502126
0.775 0.0219218473852424
0.78 0.0192001439110504
0.785 0.0167839596465161
0.79 0.0146522079473016
0.795 0.0127811044103527
0.8 0.0111455452888791
0.805 0.00972029585256481
0.81 0.00848092702643994
0.815 0.00740448530431059
0.82 0.00646991562426667
0.825 0.00565827701947578
0.83 0.00495279813990158
0.835 0.00433881807780411
0.84 0.00380365146419993
0.845 0.00333640876386272
0.85 0.00292779503725072
0.855 0.00256990400714959
0.86 0.00225601919397449
0.865 0.00198042995414445
0.87 0.00173826718373137
0.875 0.0015253610091346
0.88 0.00133812084964607
0.885 0.00117343675199983
0.89 0.00102859984928664
0.895 0.000901239170925305
0.9 0.000789271789969033
0.905 0.000690863374632628
0.91 0.000604396527075725
0.915 0.000528444750057864
0.92 0.00046175039200118
0.925 0.000403205410113744
0.93 0.000351834208625823
0.935 0.000306778127598819
0.94 0.000267281370897916
0.945 0.000232678279182708
0.95 0.000202381893958196
0.955 0.000175873743991994
0.96 0.000152694736778497
0.965 0.000132436973771126
0.97 0.000114736250561974
0.975 9.92650028115161e-05
0.98 8.5725682258205e-05
0.985 7.38455366845868e-05
0.99 6.3377104025941e-05
0.995 5.41183395192608e-05
};
\addlegendentry{FOM}
\addplot [semithick, red, mark=o,mark size=2, mark repeat=25, mark options={solid}]
table {%
0 0.00173403546608024
0.005 0.00173411702222073
0.01 0.00173422360229542
0.015 0.00173436262656538
0.02 0.00173454405470905
0.025 0.00173478110646675
0.03 0.00173509117688572
0.035 0.00173549705825386
0.04 0.00173602854705688
0.045 0.00173672453103995
0.05 0.0017376356791182
0.055 0.00173882788947424
0.06 0.00174038668878013
0.065 0.00174242281893287
0.07 0.0017450792975303
0.075 0.00174854029477523
0.08 0.00175304223220188
0.085 0.00175888757652733
0.09 0.00176646187311294
0.095 0.00177625463484338
0.1 0.00178888476923168
0.105 0.0018051312831024
0.11 0.0018259700423657
0.115 0.00185261737425821
0.12 0.00188658126948139
0.125 0.00192972085858806
0.13 0.0019843146869673
0.135 0.00205313808190768
0.14 0.00213954958132178
0.145 0.00224758596813498
0.15 0.00238206492383614
0.155 0.00254869368434054
0.16 0.00275418136623609
0.165 0.00300635185911651
0.17 0.00331425339051844
0.175 0.00368826011723211
0.18 0.00414016044440728
0.185 0.00468322629241934
0.19 0.00533225729204907
0.195 0.00610359395520825
0.2 0.00701509428955314
0.205 0.00808606912564541
0.21 0.00933717259853975
0.215 0.0107902457314474
0.22 0.0124681128321562
0.225 0.0143943323282519
0.23 0.0165929056086363
0.235 0.0190879492714601
0.24 0.0219033377727659
0.245 0.0250623247155166
0.25 0.0285871518356336
0.255 0.0324986550887282
0.26 0.0368158771172174
0.265 0.0415556948183301
0.27 0.0467324698042296
0.275 0.0523577283299509
0.28 0.0584398758543638
0.285 0.0649839498839069
0.29 0.0719914132118095
0.295 0.0794599881846741
0.3 0.0873835312813132
0.305 0.0957519461624886
0.31 0.104551132549377
0.315 0.113762967937326
0.32 0.123365319382952
0.325 0.133332083530082
0.33 0.143633254708713
0.335 0.154235023266625
0.34 0.165099909000772
0.345 0.176186937143522
0.35 0.187451866116846
0.355 0.198847476377789
0.36 0.210323927417426
0.365 0.221829185011352
0.37 0.23330951356125
0.375 0.244710020268316
0.38 0.255975231543842
0.385 0.26704968093675
0.39 0.277878495300881
0.395 0.288407983557569
0.4 0.298586257990864
0.405 0.308363943633983
0.41 0.317695043287007
0.415 0.326538007731376
0.42 0.334857000809969
0.425 0.342623248896898
0.43 0.349816246850075
0.435 0.356424503729703
0.44 0.362445514569813
0.445 0.367884807884072
0.45 0.372754290372132
0.455 0.377070646851359
0.46 0.380854975589568
0.465 0.384134515027288
0.47 0.386945547291396
0.475 0.389333685661491
0.48 0.391346473271476
0.485 0.393017525195161
0.49 0.394350895260683
0.495 0.395319160221802
0.5 0.395878742033399
0.505 0.395987322597413
0.51 0.395605101007406
0.515 0.3946820867493
0.52 0.393148935815022
0.525 0.390920838542935
0.53 0.387911410828864
0.535 0.384050093065268
0.54 0.379294469150254
0.545 0.373633218451342
0.55 0.36708135053072
0.555 0.359672514160283
0.56 0.351452715625666
0.565 0.342477502015959
0.57 0.332812440930714
0.575 0.322535359912441
0.58 0.311738305246253
0.585 0.30052736453795
0.59 0.289019237790158
0.595 0.277334555770308
0.6 0.265589107321091
0.605 0.253884948224705
0.61 0.242303526216212
0.615 0.230902424298749
0.62 0.219716318293085
0.625 0.208761633932207
0.63 0.198043525473483
0.635 0.187563392800365
0.64 0.17732524199185
0.645 0.167339670660863
0.65 0.157624947112809
0.655 0.148205363633443
0.66 0.13910762088481
0.665 0.130356337036783
0.67 0.121969830193452
0.675 0.113957119310503
0.68 0.106316705502639
0.685 0.0990372432031019
0.69 0.0920998018557403
0.695 0.0854811392749996
0.7 0.0791572974173325
0.705 0.0731068817309427
0.71 0.0673135516806904
0.715 0.0617674708314859
0.72 0.0564656811214496
0.725 0.0514115352652861
0.73 0.0466134233515448
0.735 0.0420830650477384
0.74 0.0378336224337578
0.745 0.0338778422002028
0.75 0.0302263810911664
0.755 0.0268864202897899
0.76 0.0238606392508151
0.765 0.0211465948750556
0.77 0.0187365300029451
0.775 0.016617607820674
0.78 0.0147725326774226
0.785 0.013180477277353
0.79 0.0118182014696737
0.795 0.010661230053953
0.8 0.00968496247828392
0.805 0.00886561446941444
0.81 0.00818093206932452
0.815 0.00761066103458897
0.82 0.00713678954205623
0.825 0.00674360449148817
0.83 0.00641761092634617
0.835 0.00614736295747814
0.84 0.00592324709606331
0.845 0.00573724883767433
0.85 0.00558272338067701
0.855 0.00545418299632967
0.86 0.00534710731401293
0.865 0.00525777855225861
0.87 0.00518314113371691
0.875 0.00512068371811076
0.88 0.00506834106806817
0.885 0.00502441301869125
0.89 0.00498749793892053
0.895 0.00495643831576033
0.9 0.0049302763820346
0.905 0.00490821800168191
0.91 0.0048896033024182
0.915 0.00487388279446432
0.92 0.00486059793257463
0.925 0.0048493652663818
0.93 0.00483986348211996
0.935 0.00483182276887213
0.94 0.0048250160467077
0.945 0.00481925167452664
0.95 0.00481436731380422
0.955 0.00481022466153231
0.96 0.00480670478068123
0.965 0.00480370374455715
0.97 0.00480112825308575
0.975 0.00479889069713976
0.98 0.00479690254923892
0.985 0.00479506294027647
0.99 0.00479323280419883
0.995 0.0047911662677077
};
\addlegendentry{POD}
\addplot [semithick, color0, mark=pentagon,mark size=2, mark repeat=25, mark options={solid}]
table {%
0 8.80636134650038e-05
0.005 8.81591413825092e-05
0.01 8.83418009249265e-05
0.015 8.85639238069991e-05
0.02 8.91892524169166e-05
0.025 8.92520906671943e-05
0.03 8.9744678256899e-05
0.035 9.04301968789858e-05
0.04 9.09283870996544e-05
0.045 9.22747406218945e-05
0.05 9.35481834045386e-05
0.055 9.540865484525e-05
0.06 9.76490642572486e-05
0.065 0.000100513087812794
0.07 0.000104633218141709
0.075 0.000109749499295772
0.08 0.00011625041096241
0.085 0.000124988747032773
0.09 0.000135892936194069
0.095 0.0001504918307797
0.1 0.000168773330025527
0.105 0.000192401813292807
0.11 0.000222568229982653
0.115 0.000261167647015832
0.12 0.000309909042838148
0.125 0.000371755673394364
0.13 0.000449231324878245
0.135 0.000546965660870804
0.14 0.00066896470019355
0.145 0.000820549977759261
0.15 0.00100851261691996
0.155 0.00123991845363148
0.16 0.00152378433288655
0.165 0.00187033272919626
0.17 0.00229100690124769
0.175 0.00279850566342281
0.18 0.00340820834453713
0.185 0.00413602033135794
0.19 0.00500008216185756
0.195 0.00601998244742658
0.2 0.00721714821245075
0.205 0.00861331991483445
0.21 0.010232706323446
0.215 0.0120995279175705
0.22 0.0142444240922797
0.225 0.0167400775325646
0.23 0.019565600413681
0.235 0.0227461209635723
0.24 0.0263058042475247
0.245 0.0301106246663035
0.25 0.0342273099053165
0.255 0.038758616316681
0.26 0.0437004108350322
0.265 0.0490357269873681
0.27 0.0548137990096782
0.275 0.0610385945213321
0.28 0.0677105617863745
0.285 0.074825571405643
0.29 0.0823314873936807
0.295 0.0899508542682646
0.3 0.0979822946336325
0.305 0.106409628390413
0.31 0.115213762752392
0.315 0.124371394326659
0.32 0.133856605350594
0.325 0.143640761711734
0.33 0.153544102104007
0.335 0.163371068695528
0.34 0.173420647771693
0.345 0.183658971008097
0.35 0.194050488104736
0.355 0.20455794831722
0.36 0.215143989996624
0.365 0.225769788212369
0.37 0.236397461051555
0.375 0.246988604798882
0.38 0.25732107306726
0.385 0.267496565087277
0.39 0.277543134165329
0.395 0.28742691402825
0.4 0.297433602761005
0.405 0.307170455704289
0.41 0.316005474675939
0.415 0.324687036884668
0.42 0.333192614121243
0.425 0.341431037690095
0.43 0.348883989505967
0.435 0.355128013767135
0.44 0.361129667728224
0.445 0.366855228618427
0.45 0.37228461998416
0.455 0.37740607369834
0.46 0.382214526092496
0.465 0.386143316536228
0.47 0.38935317268317
0.475 0.391928966024905
0.48 0.393826255353167
0.485 0.395051388946199
0.49 0.395635871950964
0.495 0.395631549329923
0.5 0.395083871216471
0.505 0.394004508227423
0.51 0.392318493166792
0.515 0.390076274173918
0.52 0.387138024768254
0.525 0.383462584357431
0.53 0.379075995423838
0.535 0.373622402369064
0.54 0.367410670187221
0.545 0.360608044312733
0.55 0.353473965687815
0.555 0.346080318760333
0.56 0.338494187709869
0.565 0.328737036932544
0.57 0.317793864321442
0.575 0.307340581564508
0.58 0.297338293407433
0.585 0.2871090878098
0.59 0.276749729997887
0.595 0.266359565576865
0.6 0.256032036150035
0.605 0.245778546630211
0.61 0.235524509101143
0.615 0.225788203837985
0.62 0.216424140597909
0.625 0.207271367798085
0.63 0.198288263800142
0.635 0.189584103050795
0.64 0.181031398243966
0.645 0.172681799061345
0.65 0.16456569130242
0.655 0.156609835455703
0.66 0.148814265190556
0.665 0.141177633747801
0.67 0.134185739048126
0.675 0.127457727227843
0.68 0.120843776468059
0.685 0.114329464066213
0.69 0.10790250033009
0.695 0.101555116431334
0.7 0.0952853543805798
0.705 0.0890994120193128
0.71 0.0830115813143871
0.715 0.0770445830386809
0.72 0.071227737251895
0.725 0.0655955925005789
0.73 0.0601853897627869
0.735 0.0549135258692833
0.74 0.0499011254280168
0.745 0.0452094046204966
0.75 0.0408651030739581
0.755 0.0368870962889782
0.76 0.0325765467204066
0.765 0.0285748600334791
0.77 0.0249756302659191
0.775 0.02176776670382
0.78 0.0189370854758949
0.785 0.0164798374653772
0.79 0.0143525914530181
0.795 0.0125185219968713
0.8 0.0109470658501903
0.805 0.00957889210143677
0.81 0.00838988624677174
0.815 0.00735714806839875
0.82 0.00649634477238935
0.825 0.00576598683377509
0.83 0.00512362798652479
0.835 0.00449444616728341
0.84 0.0038383711229982
0.845 0.00327062133197685
0.85 0.00278276348006746
0.855 0.00235887123186816
0.86 0.00199053389040486
0.865 0.00167047039297391
0.87 0.00139249889467362
0.875 0.00115140728793163
0.88 0.000942502741800531
0.885 0.000798624166151212
0.89 0.000698626022874485
0.895 0.000613413735199218
0.9 0.00054975068120846
0.905 0.000496501922950456
0.91 0.000449161843071253
0.915 0.000409470892718868
0.92 0.000376437291426227
0.925 0.000349002218502848
0.93 0.000326500927922409
0.935 0.000307885498101418
0.94 0.000292650812069344
0.945 0.000280296084944529
0.95 0.000270029593042639
0.955 0.000261885924189869
0.96 0.000255369880827667
0.965 0.000249704526543264
0.97 0.000245060083344756
0.975 0.0002413987258201
0.98 0.000237970466823168
0.985 0.000235334459627416
0.99 0.000232488437967631
0.995 0.000228951784902406
};
\addlegendentry{FCNN}
\addplot [semithick, green!50!black, mark=triangle,mark size=2, mark repeat=25, mark options={solid,rotate=180}, only marks]
table {%
0 -0.00231674749215
0.005 -0.00125490701425371
0.01 0.000326827235245759
0.015 0.000803400852868427
0.02 0.000912645663810275
0.025 -0.000520638339811408
0.03 0.000172158865135407
0.035 0.000990448143080604
0.04 0.00159708900940129
0.045 0.00177589118640381
0.05 0.000450732202326935
0.055 0.00103699149364624
0.06 0.00134845711112505
0.065 0.00239400036251922
0.07 0.00260009684251834
0.075 0.00321300824675477
0.08 0.00298336332763231
0.085 0.0027895698575768
0.09 0.00247782747625746
0.095 0.00227381284815598
0.1 0.00326310348158146
0.105 0.00322935445800733
0.11 0.00312841207243308
0.115 0.00309667828149584
0.12 0.00269890153357928
0.125 0.00339060196996165
0.13 0.0034127087878769
0.135 0.0026715626544598
0.14 0.00369539684411303
0.145 0.00403914929036265
0.15 0.00350439942361796
0.155 0.00358945066627201
0.16 0.00281387315170583
0.165 0.00405503602465528
0.17 0.00445860668886477
0.175 0.00348254634671057
0.18 0.00358583265259768
0.185 0.00297546796554149
0.19 0.00409166780079825
0.195 0.00438330439126165
0.2 0.000805956877151213
0.205 0.00147079652557925
0.21 0.00166978335888612
0.215 0.00323118501813474
0.22 0.00394659750270721
0.225 0.00502909093112281
0.23 0.00603764187861
0.235 0.00681406389027538
0.24 0.00841561247136514
0.245 0.00937550686614862
0.25 0.0246578584894263
0.255 0.0277983629373758
0.26 0.0304400262798058
0.265 0.0349765608489022
0.27 0.0385504275066399
0.275 0.0473621767164738
0.28 0.0520843585916121
0.285 0.0560532245619112
0.29 0.0627697255245516
0.295 0.0678416026758741
0.3 0.0799034200353254
0.305 0.0855165037333603
0.31 0.0907135790491781
0.315 0.0979139739105606
0.32 0.103894355065036
0.325 0.115999107068481
0.33 0.124038850731758
0.335 0.131770544974057
0.34 0.141418696063193
0.345 0.149697637651641
0.35 0.161333244085034
0.355 0.17105874405465
0.36 0.181003713341161
0.365 0.191740282074441
0.37 0.200588899166916
0.375 0.23379881617062
0.38 0.245867184750607
0.385 0.258924032776551
0.39 0.270270542478796
0.395 0.282275249007449
0.4 0.294270945281089
0.405 0.305178143575594
0.41 0.317062988828322
0.415 0.326944961276411
0.42 0.337708213962697
0.425 0.3463179177314
0.43 0.35398514608261
0.435 0.362297802405306
0.44 0.368968857275793
0.445 0.376172159382119
0.45 0.37923984261008
0.455 0.383913587384197
0.46 0.388744601128831
0.465 0.393027163971995
0.47 0.396998435093546
0.475 0.404218955829532
0.48 0.404992413284095
0.485 0.406835552695419
0.49 0.40634683771753
0.495 0.406530191557552
0.5 0.422071859632892
0.505 0.42207536694103
0.51 0.420346074947807
0.515 0.422731705775851
0.52 0.423464130664283
0.525 0.426713068735705
0.53 0.421607120730998
0.535 0.415637792005785
0.54 0.411703590584047
0.545 0.404860269081974
0.55 0.404844929017498
0.555 0.386969126802654
0.56 0.368297310635322
0.565 0.350260727733196
0.57 0.327305230400845
0.575 0.308742753095176
0.58 0.28770043357046
0.585 0.264805648917613
0.59 0.244649677103537
0.595 0.223744486959003
0.6 0.20612392517513
0.605 0.195917776546495
0.61 0.185151349648375
0.615 0.175018432731937
0.62 0.167317491116021
0.625 0.149724403899027
0.63 0.147346399016946
0.635 0.14467746096542
0.64 0.142708800090247
0.645 0.139222343290901
0.65 0.131278178672863
0.655 0.128965471442937
0.66 0.12688645674207
0.665 0.124387615297682
0.67 0.120889536376425
0.675 0.116600787738963
0.68 0.112783224396565
0.685 0.1105958201585
0.69 0.105001304130155
0.695 0.0994367334595044
0.7 0.101857685721614
0.705 0.0952273582422623
0.71 0.0886905833090196
0.715 0.0813651215622733
0.72 0.0701685223413197
0.725 0.0622713678985198
0.73 0.0543862705622159
0.735 0.0437223638484846
0.74 0.0384997900106248
0.745 0.0324849652288127
0.75 0.0077974915389096
0.755 0.00773373347746441
0.76 0.00855735768459004
0.765 0.0070013183859822
0.77 0.00671420651692542
0.775 0.00557615458445624
0.78 0.00526151769422792
0.785 0.00562290905929669
0.79 0.00414595955920311
0.795 0.00373917647505368
0.8 0.00305702463609255
0.805 0.00215006904090408
0.81 0.00154809642130904
0.815 -7.49388307541624e-05
0.82 -0.000998931322177729
0.825 -0.0030648249677673
0.83 -0.003741599260815
0.835 -0.00446944211568254
0.84 -0.00516396554323807
0.845 -0.00506485763657133
0.85 -0.00629281911925232
0.855 -0.0066060991318909
0.86 -0.00673257579837498
0.865 -0.00720920329922323
0.87 -0.00724261089908618
0.875 0.00210993387540304
0.88 0.00186141856083308
0.885 0.00203440771749
0.89 0.00114165182178579
0.895 0.00135727240683087
0.9 0.00337100709005376
0.905 0.00290920335441157
0.91 0.00271632491949607
0.915 0.00184816720611912
0.92 0.0018570579383407
0.925 0.00525072978261821
0.93 0.00425194688423459
0.935 0.0034212389923345
0.94 0.00192929428040801
0.945 0.00126541032446284
0.95 0.0032378322487819
0.955 0.00297410129100165
0.96 0.00277152191609076
0.965 0.0020769328489695
0.97 0.00268280318516439
0.975 0.00182076121789137
0.98 0.00236581845584306
0.985 0.00309201248400768
0.99 0.00311876991413838
0.995 0.00385781950023266
};
\addlegendentry{CNN}

\nextgroupplot[
legend cell align={left},
legend style={at={(1,1)},anchor=north east,fill opacity=0.1, draw opacity=1, text opacity=1, draw=none},
xlabel={\(x\)},
ylabel={\(E\)},
ymin=-0.05, ymax=0.614353397172855,
y label style={yshift=-.9em},
y label style={xshift=+.7em}
]
\addplot [semithick, black, dashed, mark=x,mark size=2, mark repeat=25, mark options={solid}]
table {%
0 0.49999935680546
0.005 0.499999184516617
0.01 0.499998958044162
0.015 0.499998661709873
0.02 0.499998275147288
0.025 0.499997771866551
0.03 0.499997117461528
0.035 0.499996267334379
0.04 0.499995163800781
0.045 0.499993732414377
0.05 0.499991877316449
0.055 0.4999894753775
0.06 0.499986368851917
0.065 0.499982356215721
0.07 0.49997718080134
0.075 0.499970516783883
0.08 0.499961952012844
0.085 0.499950967124919
0.09 0.499936910322577
0.095 0.499918967165467
0.1 0.499896124705878
0.105 0.499867129315394
0.11 0.499830437609235
0.115 0.499784159990916
0.12 0.499725996526799
0.125 0.499653165131781
0.13 0.499562322416104
0.135 0.499449478017839
0.14 0.499309903829107
0.145 0.499138040211023
0.15 0.498927402066165
0.155 0.498670488467809
0.16 0.498358700386789
0.165 0.49798227184801
0.17 0.497530220513646
0.175 0.496990324141894
0.18 0.496349129517518
0.185 0.495592000206122
0.19 0.494703208776299
0.195 0.493666077917781
0.2 0.492463173153719
0.205 0.491076547643305
0.21 0.489488036992126
0.215 0.487679599178207
0.22 0.48563369185145
0.225 0.483333676590288
0.23 0.480764237426086
0.235 0.477911799279573
0.24 0.474764931059029
0.245 0.471314718149885
0.25 0.46755508990689
0.255 0.463483089491295
0.26 0.459099075851118
0.265 0.454406850638818
0.27 0.449413706176894
0.275 0.444130393984271
0.28 0.438571016642011
0.285 0.432752848712706
0.29 0.426696094883513
0.295 0.420423595376289
0.3 0.41396048990446
0.305 0.407333852041546
0.31 0.400572305819732
0.315 0.393705635744175
0.32 0.386764400258094
0.325 0.37977955711779
0.33 0.37278210725787
0.335 0.365802761703839
0.34 0.358871634118662
0.345 0.352017959878469
0.35 0.345269841390657
0.355 0.338654018885252
0.36 0.332195666215149
0.365 0.325918212210125
0.37 0.3198431895337
0.375 0.313990114233691
0.38 0.308376399499813
0.385 0.303017305735776
0.39 0.297925925329712
0.395 0.29311319447661
0.4 0.288587917090328
0.405 0.284356779508706
0.41 0.280424332635162
0.415 0.276792923722078
0.42 0.273462574861494
0.425 0.270430827458254
0.43 0.267692594949205
0.435 0.265240079944094
0.44 0.263062807795775
0.445 0.26114780346012
0.45 0.259479898018716
0.455 0.258042105589045
0.46 0.256815966903368
0.465 0.255781714613968
0.47 0.254918097933172
0.475 0.254201786016598
0.48 0.253606558386228
0.485 0.253102948359567
0.49 0.252659208169941
0.495 0.252243792059708
0.5 0.251827275102802
0.505 0.251381012497179
0.51 0.25087302853746
0.515 0.250265602223314
0.52 0.249517693945798
0.525 0.248590688140476
0.53 0.247453389289997
0.535 0.246084019377779
0.54 0.244469264456546
0.545 0.242601857795075
0.55 0.240478180516557
0.555 0.23809666448092
0.56 0.235457212698634
0.565 0.232561605229087
0.57 0.229414721585933
0.575 0.226026206372895
0.58 0.222411969184941
0.585 0.218594806862423
0.59 0.214603596555817
0.595 0.210470926205335
0.6 0.206229558715456
0.605 0.201908554426631
0.61 0.197530029228202
0.615 0.193107347902876
0.62 0.188645124924567
0.625 0.184140895173082
0.63 0.179587897877868
0.635 0.174978201200335
0.64 0.17030540968722
0.645 0.16556639879422
0.65 0.160761827791422
0.655 0.155895505486493
0.66 0.150972946975817
0.665 0.145999613634893
0.67 0.140979352315915
0.675 0.13591345321238
0.68 0.130800564862492
0.685 0.125637490651162
0.69 0.120420696893854
0.695 0.115148229071734
0.7 0.109821679976769
0.705 0.104447878554068
0.71 0.099040050356734
0.715 0.0936183101554151
0.72 0.0882094555681862
0.725 0.0828461170190982
0.73 0.0775653751442742
0.735 0.0724069844914259
0.74 0.0674113519275984
0.745 0.0626174210389261
0.75 0.0580606172134823
0.755 0.0537710115820283
0.76 0.0497718578191883
0.765 0.0460786334781048
0.77 0.042698670459725
0.775 0.0396313901640515
0.78 0.0368690808793959
0.785 0.0343980868649908
0.79 0.032200237264967
0.795 0.0302543366145871
0.8 0.028537564241122
0.805 0.0270266757298571
0.81 0.0256989512318784
0.815 0.024532880548202
0.82 0.0235086068445918
0.825 0.0226081682913537
0.83 0.0218155824541386
0.835 0.0211168160181255
0.84 0.0204996763355393
0.845 0.0199536541582351
0.85 0.0194697402960386
0.855 0.0190402333814328
0.86 0.018658551401896
0.865 0.0183190559463869
0.87 0.018016894973501
0.875 0.0177478672179364
0.88 0.017508309084488
0.885 0.017295003070728
0.89 0.017105105455802
0.895 0.0169360902085604
0.9 0.0167857057706824
0.905 0.016651941479692
0.91 0.016533000800502
0.915 0.016427279107946
0.92 0.0163333443900927
0.925 0.0162499198292296
0.93 0.0161758677015307
0.935 0.0161101743871457
0.94 0.0160519364975342
0.945 0.0160003482247449
0.95 0.0159546900284552
0.955 0.0159143187351992
0.96 0.0158786590620089
0.965 0.0158471965183368
0.97 0.0158194716003297
0.975 0.0157950751714789
0.98 0.0157736449012305
0.985 0.0157548625331191
0.99 0.0157384513721274
0.995 0.0157241722380244
};
\addlegendentry{FOM}
\addplot [semithick, red, mark=o,mark size=2, mark repeat=25, mark options={solid}]
table {%
0 0.499087239667312
0.005 0.499087147549203
0.01 0.499087025523674
0.015 0.499086864518601
0.02 0.499086652873673
0.025 0.499086375266657
0.03 0.499086011591997
0.035 0.499085535538979
0.04 0.499084912753822
0.045 0.499084098472178
0.05 0.499083034483811
0.055 0.499081645258303
0.06 0.499079833022056
0.065 0.499077471532367
0.07 0.499074398243712
0.075 0.499070404505035
0.08 0.49906522336535
0.085 0.499058514500457
0.09 0.499049845708416
0.095 0.499038670359964
0.1 0.499024300137771
0.105 0.49900587236266
0.11 0.498982311195002
0.115 0.498952282026459
0.12 0.498914138453835
0.125 0.498865861366857
0.13 0.498804989899368
0.135 0.49872854430155
0.14 0.498632941199761
0.145 0.498513902225437
0.15 0.498366357613245
0.155 0.498184347079188
0.16 0.497960921067481
0.165 0.497688046262965
0.17 0.497356520051717
0.175 0.496955899311751
0.18 0.496474449454178
0.185 0.495899119935013
0.19 0.495215552444699
0.195 0.494408127594617
0.2 0.493460055118327
0.205 0.492353511381342
0.21 0.49106982637588
0.215 0.489589720434738
0.22 0.48789358873697
0.225 0.485961829432994
0.23 0.483775209041265
0.235 0.481315256817349
0.24 0.478564678209511
0.245 0.475507776402587
0.25 0.472130870382071
0.255 0.468422697943554
0.26 0.464374792601557
0.265 0.45998182434861
0.27 0.455241895581823
0.275 0.450156785136152
0.28 0.444732135123147
0.285 0.438977577060716
0.29 0.43290679549532
0.295 0.426537528877708
0.3 0.419891508780671
0.305 0.412994339569167
0.31 0.405875321277131
0.315 0.398567218643514
0.32 0.391105978965622
0.325 0.383530400641499
0.33 0.375881753079855
0.335 0.368203347255727
0.34 0.360540054909224
0.345 0.352937773645491
0.35 0.345442835428332
0.355 0.338101357469237
0.36 0.330958537306705
0.365 0.324057897544608
0.37 0.317440489474406
0.375 0.311144067688659
0.38 0.305202249177076
0.385 0.299643670619403
0.39 0.294491158455645
0.395 0.289760931033779
0.4 0.285461864282668
0.405 0.28159487373082
0.41 0.278152493631215
0.415 0.275118759532056
0.42 0.272469509592398
0.425 0.27017319754906
0.43 0.268192248959379
0.435 0.266484898073698
0.44 0.265007334325587
0.445 0.263715889217294
0.45 0.262568931593113
0.455 0.261528154475631
0.46 0.26055912178002
0.465 0.259631415343211
0.47 0.25871941538905
0.475 0.257805046239329
0.48 0.256882588158559
0.485 0.255962555913419
0.49 0.255069259058507
0.495 0.254229697771259
0.5 0.253454407050665
0.505 0.252719673541101
0.51 0.251966550101258
0.515 0.251124215020911
0.52 0.250143857016839
0.525 0.249019945163033
0.53 0.24778336413177
0.535 0.246473387488106
0.54 0.245110118748367
0.545 0.243683536929047
0.55 0.242160108089006
0.555 0.240497557943099
0.56 0.238657822054637
0.565 0.236613321819248
0.57 0.234346837506496
0.575 0.231847725729878
0.58 0.229107366640882
0.585 0.22611578736623
0.59 0.222860312921866
0.595 0.219326219264032
0.6 0.215498792777416
0.605 0.211365949648166
0.61 0.206920624568847
0.615 0.2021624282864
0.62 0.197098451490193
0.625 0.191743387786809
0.63 0.186119247928718
0.635 0.180254840649319
0.64 0.174185004093848
0.645 0.167949423708103
0.65 0.161590864932138
0.655 0.155152795262865
0.66 0.148676603682519
0.665 0.142198840543248
0.67 0.135749003800555
0.675 0.129348342949981
0.68 0.123009955842734
0.685 0.116740179136905
0.69 0.110541003356538
0.695 0.104413050586474
0.7 0.0983585773083161
0.705 0.0923840092424795
0.71 0.0865016504160873
0.715 0.080730389336278
0.72 0.0750954043961516
0.725 0.06962701281044
0.73 0.0643588939909077
0.735 0.0593259483299041
0.74 0.0545620382568404
0.745 0.0500978193740028
0.75 0.0459588243148753
0.755 0.0421639231952072
0.76 0.0387242550793854
0.765 0.0356426988717258
0.77 0.0329139194477314
0.775 0.0305249784563946
0.78 0.0284564404904787
0.785 0.0266838456197903
0.79 0.0251793754145031
0.795 0.0239135256810374
0.8 0.0228566196652615
0.805 0.0219800433641078
0.81 0.0212571449747773
0.815 0.0206637974487358
0.82 0.0201786652604392
0.825 0.0197832395009019
0.83 0.0194617110094692
0.835 0.0192007446422737
0.84 0.0189892048337386
0.845 0.018817868146205
0.85 0.0186791455568755
0.855 0.0185668271697957
0.86 0.0184758550507492
0.865 0.0184021255053433
0.87 0.0183423196667597
0.875 0.0182937600674059
0.88 0.0182542904183201
0.885 0.0182221757586586
0.89 0.0181960202545743
0.895 0.0181747001137632
0.9 0.0181573092925049
0.905 0.0181431158914683
0.91 0.0181315273621817
0.915 0.0181220628766752
0.92 0.0181143314447212
0.925 0.0181080145900169
0.93 0.0181028526110737
0.935 0.0180986336476463
0.94 0.0180951849442568
0.945 0.0180923658461435
0.95 0.0180900621794163
0.955 0.0180881817575129
0.96 0.0180866508223222
0.965 0.0180854112737527
0.97 0.018084418573914
0.975 0.0180836402591901
0.98 0.0180830551553088
0.985 0.0180826539956485
0.99 0.0180824441417727
0.995 0.0180824668075227
};
\addlegendentry{POD}
\addplot [semithick, color0, mark=pentagon,mark size=2, mark repeat=25, mark options={solid}]
table {%
0 0.501627499124879
0.005 0.501628092359449
0.01 0.501627843990475
0.015 0.501627439517934
0.02 0.501626314176698
0.025 0.501626654779679
0.03 0.501626081145878
0.035 0.50162565536811
0.04 0.501625435048597
0.045 0.501623623948329
0.05 0.501623051057345
0.055 0.501621103917437
0.06 0.501618299505143
0.065 0.501615865544932
0.07 0.501611235295732
0.075 0.501606242235561
0.08 0.501599762560825
0.085 0.501590980687202
0.09 0.501579649234835
0.095 0.501565290057802
0.1 0.501547274976723
0.105 0.501523513583069
0.11 0.501493861624012
0.115 0.501455807119938
0.12 0.501406540437242
0.125 0.501345742144718
0.13 0.501269431882055
0.135 0.501172973084631
0.14 0.501053170288374
0.145 0.500904318271181
0.15 0.500720069378697
0.155 0.500494679995843
0.16 0.500218891829123
0.165 0.49988274969748
0.17 0.499477039105202
0.175 0.498989489797167
0.18 0.498404538841238
0.185 0.49771116211431
0.19 0.496891736899037
0.195 0.495928663678118
0.2 0.494805430727384
0.205 0.493502385056804
0.21 0.492001411244081
0.215 0.490283051710266
0.22 0.488300187370298
0.225 0.485816381185524
0.23 0.483025013663552
0.235 0.47990726265342
0.24 0.476445823087009
0.245 0.472818190750602
0.25 0.468931498475999
0.255 0.464693749267095
0.26 0.460129674287059
0.265 0.455280042116017
0.27 0.45008391745526
0.275 0.444551525444611
0.28 0.438689534014502
0.285 0.432516538657412
0.29 0.42614726360003
0.295 0.42019124369614
0.3 0.413980107057784
0.305 0.407533869536008
0.31 0.400877044560101
0.315 0.394033591152456
0.32 0.387029008004091
0.325 0.379893887008214
0.33 0.372803108747319
0.335 0.365952428037147
0.34 0.359028849108295
0.345 0.3520565387572
0.35 0.345059018568113
0.355 0.3380564722463
0.36 0.331070851852402
0.365 0.324119917845904
0.37 0.317218326002971
0.375 0.310378863097012
0.38 0.305480648130054
0.385 0.301144785294821
0.39 0.296752498561219
0.395 0.292325493599148
0.4 0.287861038979864
0.405 0.283809906524662
0.41 0.280723960773355
0.415 0.277607235387324
0.42 0.274272626188108
0.425 0.271552033234952
0.43 0.269754074708102
0.435 0.267901446348103
0.44 0.265736362005731
0.445 0.263624811894184
0.45 0.261509052091856
0.455 0.259309096604597
0.46 0.256927301068605
0.465 0.255493101631057
0.47 0.254503271574458
0.475 0.253663014440053
0.48 0.25291857682264
0.485 0.252215015994355
0.49 0.251525395004397
0.495 0.250848973773845
0.5 0.250188543726969
0.505 0.249536153367037
0.51 0.248878073232078
0.515 0.24812296936677
0.52 0.247274537042532
0.525 0.246330246300031
0.53 0.245302909491313
0.535 0.246264717025564
0.54 0.243731193216766
0.545 0.237600976681923
0.55 0.232617435677476
0.555 0.228746368242042
0.56 0.225965949179789
0.565 0.221846048861411
0.57 0.21797682057865
0.575 0.216313224997001
0.58 0.212616933301781
0.585 0.209599187572255
0.59 0.207294515496811
0.595 0.205731541597025
0.6 0.204931798408396
0.605 0.203069933013025
0.61 0.196521161078518
0.615 0.190530932322801
0.62 0.184997106786969
0.625 0.179786688008032
0.63 0.17497712431101
0.635 0.170300145026125
0.64 0.165965366487775
0.645 0.161945640991014
0.65 0.158368521255404
0.655 0.154862677677933
0.66 0.151395134901079
0.665 0.147936072101846
0.67 0.142751201864151
0.675 0.137120078281405
0.68 0.131511545171315
0.685 0.125922763569285
0.69 0.120355222497903
0.695 0.114808299776261
0.7 0.109293984554524
0.705 0.103828006846789
0.71 0.0984330251214932
0.715 0.0931382182354765
0.72 0.0879786197045325
0.725 0.0829901843316266
0.73 0.0782114618692211
0.735 0.0734668403018261
0.74 0.0689388413492565
0.745 0.0647066539711759
0.75 0.060795744721189
0.755 0.0572206781423439
0.76 0.0526906234395266
0.765 0.0483758483559015
0.77 0.0444523485972872
0.775 0.0409114522830029
0.78 0.0377370406708063
0.785 0.034919202299905
0.79 0.0324341533164702
0.795 0.0303176969818016
0.8 0.0284744956016277
0.805 0.0268612361412417
0.81 0.0254519050545832
0.815 0.0242207922596536
0.82 0.0231985877918655
0.825 0.022331447732167
0.83 0.0215843373548648
0.835 0.0209430896028447
0.84 0.0203408335258078
0.845 0.0198188010768562
0.85 0.0193715154184189
0.855 0.0189824368906918
0.86 0.0186436141205834
0.865 0.0183493251122478
0.87 0.0180931749418285
0.875 0.0178704200946653
0.88 0.0176778654501283
0.885 0.0174900455977867
0.89 0.0173132686529042
0.895 0.0171615228596116
0.9 0.0170281398930071
0.905 0.0169140830766756
0.91 0.0168138232569008
0.915 0.0167279563235565
0.92 0.0166559748654745
0.925 0.0165935160821465
0.93 0.0165406237271992
0.935 0.0164960965732652
0.94 0.0164585455150672
0.945 0.0164259284645106
0.95 0.016398789276947
0.955 0.0163762099975804
0.96 0.0163568752808636
0.965 0.0163407745399283
0.97 0.0163267431623305
0.975 0.0163155613734872
0.98 0.0163053278860422
0.985 0.0162970342477067
0.99 0.0162894278726824
0.995 0.0162820573535407
};
\addlegendentry{FCNN}
\addplot [semithick, green!50!black, mark=triangle, mark size=2, mark repeat=25, mark options={solid,rotate=180}, only marks]
table {%
0 0.575600831081838
0.005 0.566038019039606
0.01 0.534827546385292
0.015 0.546344604924404
0.02 0.569409558405495
0.025 0.501050120699161
0.03 0.504644877921139
0.035 0.496363941158682
0.04 0.519066839430853
0.045 0.548900547314719
0.05 0.486564762120493
0.055 0.49369196394251
0.06 0.505318722197614
0.065 0.521811004845874
0.07 0.558673526042673
0.075 0.506719343860391
0.08 0.503000484086945
0.085 0.491089238854602
0.09 0.514901982421548
0.095 0.536490497589817
0.1 0.551555092239797
0.105 0.545128359978601
0.11 0.524100281413601
0.115 0.557325042869148
0.12 0.577432046160461
0.125 0.523039769100695
0.13 0.493450863512429
0.135 0.496837696168638
0.14 0.459589244218705
0.145 0.442878643319826
0.15 0.570031063857647
0.155 0.527143215875088
0.16 0.510360991860418
0.165 0.466962276513675
0.17 0.447302567549295
0.175 0.54167822049842
0.18 0.494576137684801
0.185 0.473909852134704
0.19 0.421717522863192
0.195 0.414173236865915
0.2 0.523106332594164
0.205 0.483209231599087
0.21 0.432064748314411
0.215 0.418597725927238
0.22 0.397051980349473
0.225 0.469326678355653
0.23 0.447885449907274
0.235 0.45925473187899
0.24 0.403176754845347
0.245 0.40961841922853
0.25 0.366540847152664
0.255 0.362873756536012
0.26 0.377093340312145
0.265 0.359247176810458
0.27 0.367031700478197
0.275 0.378315711554614
0.28 0.355888081329631
0.285 0.350095409500552
0.29 0.318954881417911
0.295 0.3066128454491
0.3 0.385567478017579
0.305 0.343358624681207
0.31 0.317633603056012
0.315 0.262233101827285
0.32 0.219410102557922
0.325 0.283368173386805
0.33 0.234896172866046
0.335 0.216815560812197
0.34 0.123126713630139
0.345 0.106909698966267
0.35 0.0659849615852003
0.355 0.0714169470151325
0.36 0.124155302392721
0.365 0.0662258817765197
0.37 0.0673701278683742
0.375 0.213397505590004
0.38 0.196939591584731
0.385 0.179188087722196
0.39 0.185509700215973
0.395 0.205560640460302
0.4 0.270235363715956
0.405 0.241949488916404
0.41 0.218534466355241
0.415 0.205136564779519
0.42 0.208550827263202
0.425 0.277619308808529
0.43 0.244568231264124
0.435 0.240533425815837
0.44 0.191108375979093
0.445 0.17572259196233
0.45 0.21091604036801
0.455 0.19218396881266
0.46 0.180559320607086
0.465 0.149383350989819
0.47 0.0973863536863345
0.475 0.0142014759978897
0.48 -0.0144152409275429
0.485 0.00549236842950563
0.49 -0.0973961388298601
0.495 -0.122542410023585
0.5 0.224687868858147
0.505 0.250194743930554
0.51 0.345430422019171
0.515 0.263282999521557
0.52 0.279516312102743
0.525 0.239547180162426
0.53 0.254057662642836
0.535 0.329957776923944
0.54 0.258901498221282
0.545 0.273705869862189
0.55 0.164517384600925
0.555 0.179052825690945
0.56 0.225416590499439
0.565 0.210717094832615
0.57 0.241922301500645
0.575 0.187083358311972
0.58 0.199160633336981
0.585 0.223954714976145
0.59 0.242695032925844
0.595 0.291432324242122
0.6 0.246308097381921
0.605 0.260061929314319
0.61 0.269420290060799
0.615 0.33084566075868
0.62 0.382319873838594
0.625 0.297133712297775
0.63 0.292798532987234
0.635 0.324483036598932
0.64 0.328926854414806
0.645 0.319653047439437
0.65 0.366731342002179
0.655 0.345298686889958
0.66 0.344391371872303
0.665 0.347780730006407
0.67 0.332154689981301
0.675 0.384084901789914
0.68 0.358946591936971
0.685 0.35987634975769
0.69 0.350485884405711
0.695 0.343811390855978
0.7 0.486454552146683
0.705 0.43583205959542
0.71 0.363296145759755
0.715 0.354795682330565
0.72 0.28580070692508
0.725 0.325502781102004
0.73 0.285555490645537
0.735 0.269482234983104
0.74 0.198531161833975
0.745 0.189731486360332
0.75 0.24110290983973
0.755 0.224249342989962
0.76 0.210299647039784
0.765 0.158663958806988
0.77 0.162774041570655
0.775 0.0999678526770212
0.78 0.0876140142721761
0.785 0.090626618820555
0.79 0.0324349370528534
0.795 0.0476620339026511
0.8 -0.0185005472671924
0.805 -0.016979188305308
0.81 -0.0117945040212626
0.815 -0.0182641795176244
0.82 -0.00720901627795774
0.825 -0.0515141705689328
0.83 -0.0397913034796735
0.835 -0.00600708046859565
0.84 -0.00562384463035128
0.845 0.010933217084838
0.85 0.0299931513628192
0.855 0.0135469256842322
0.86 0.0175545972231806
0.865 0.00177744533437615
0.87 -0.0074334725690213
0.875 0.0165786476619237
0.88 0.0308643236731051
0.885 0.0917120396279032
0.89 0.0232131613026947
0.895 0.0778273822382296
0.9 -0.0102639508729072
0.905 0.00581435989017879
0.91 0.0621254880081043
0.915 0.0173937516499447
0.92 0.0548910728857918
0.925 -0.0136492368042178
0.93 -0.0242946125240218
0.935 0.00697074515682677
0.94 -0.0659178050622447
0.945 -0.0518022113270218
0.95 -0.160994974087425
0.955 -0.132882401808772
0.96 -0.0523590924538162
0.965 -0.0955043531651898
0.97 -0.0383546843053512
0.975 -0.0827908250124676
0.98 -0.0547301035946602
0.985 0.001695264120816
0.99 -0.00821322074572929
0.995 0.0352592214234374
};
\addlegendentry{CNN}
\end{groupplot}

\end{tikzpicture}

	\caption{Matching of macroscopic quantities \(\rho\), \(\rho u\) and \(E\) reproduced by POD, FCNN, and CNN with FOM macroscopic quantities. Top row shows results for \(\hy\), bottom row for \(\rare\). CNN is displayed with marks only because of trembles in the signal.}
	\label{Fig:ErrMacro}
\end{figure}
Loss of information described above can unfold in severe mistakes in \(\rho\), \(\rho u\) and \(E\), the macroscopic quantities, as displayed in \cref{Fig:ErrMacro}. In the following, features of the macroscopic quantities are expressed in terms of rarefaction wave, contact discontinuity and height as well as position of the shockfront. For a detailed elaboration see \cref{Ch:BGK}. Following the structure in the preceding figures, macroscopic quantities of \(\hy\) are displayed in the top row and for \(\rare\) in the bottom row of \cref{Fig:ErrMacro}. First the reproduction of the macroscopic quantities \(\rho\), \(\rho u\) and \(E\), obtained by the FCNN is exact for both cases \(\hy\) and \(\rare\). In particular the density \(\rho\) matches with results from the FOM exactly for the neural networks in both cases \(\hy\) and \(\rare\). Second the CNN produces trembles in \(\rho u\) and especially in \(E\) which is why it's shown with marks only. Results from the CNN are similar for \(\hy\) and \(\rare\). The momentum \(\rho u\) holds errors for the tail of the rarefaction wave as well as the contact discontinuity and the position and height of the shockwave. Third POD performs better on \(\rare\), holding only small deviations in the contact discontinuity and position and height of the shockwave for the momentum \(\rho u\) and the total energy \(E\). The density \(\rho\) matches with the FOM solution exact. Distinct deviations from the FOM solution occur using POD on \(\hy\). The density \(\rho\) holds errors in the  height of the shockwave. The momentum \(\rho u\) holds errors in the tail of the rarefaction wave, the contact discontinuity and the height of the shockwave. Hence in the total energy \(E\) errors occur in the rarefaction wave, especially the tail, the contact discontinuity and the height of the shockwave.
\begin{figure}[H]
	% This file was created by tikzplotlib v0.9.8.
\begin{tikzpicture}

\begin{groupplot}[
group style={group size=3 by 2,
	horizontal sep=.8cm,
	vertical sep=1.1cm
},
tick align=outside,
tick pos=left,
x grid style={white!69.0196078431373!black},
xlabel={\(t\)},
xmin=-0.006, xmax=0.126,
xtick style={color=black},
y grid style={white!69.0196078431373!black},
ymin=-0.031347233897592, ymax=0.112481710620415,
ytick style={color=black},
x tick label style={/pgf/number format/fixed},
y tick label style={/pgf/number format/fixed},
width=.55\textwidth,
height=.6\textwidth
]
\nextgroupplot[
legend cell align={left},
legend style={fill opacity=0,
	at={(1,1)},
	anchor=north east,
	draw opacity=1,
	text opacity=1,
	draw=none,
	nodes={
		scale=0.7,
		transform shape
	}
},
ylabel={\(\bar{\dot{\rho}}\)},
ymin=-0.0614423751831055, ymax=0.158164024353027,
y label style={yshift=-1em},
y label style={xshift=-.7em}
]
\addplot [semithick, black, mark=x, mark size=2.5, mark repeat=5, mark options={solid}]
table {%
0 -1.06764214677924e-05
0.005 -1.05043313283204e-05
0.01 -1.01943976815733e-05
0.015 -9.92951054712421e-06
0.02 -9.68121178601677e-06
0.025 -9.44233560318253e-06
0.03 -9.20958690642237e-06
0.035 -8.98114045355669e-06
0.04 -8.7558689685352e-06
0.045 -8.53302895365005e-06
0.05 -8.31210804363991e-06
0.055 -8.09274173718677e-06
0.06 -7.87466429130745e-06
0.065 -7.6576779619586e-06
0.07 -7.44163272514697e-06
0.075 -7.22641276951208e-06
0.08 -7.01192718111088e-06
0.085 -6.79810317905094e-06
0.09 -6.58488170302007e-06
0.095 -6.37221425847656e-06
0.1 -6.16006092712951e-06
0.105 -5.94838910217277e-06
0.11 -5.73717198193435e-06
0.115 -5.52638756801116e-06
0.12 -5.42110106493965e-06
};
\addlegendentry{FOM}
\addplot [semithick, red, mark=o, mark size=2.5, mark repeat=5, mark options={solid}]
table {%
0 0.00320090036619547
0.005 -0.00437489515648082
0.01 -0.0113141286949912
0.015 -0.00983423356615987
0.02 -0.00832201478583983
0.025 -0.00715067115115886
0.03 -0.00626926456104826
0.035 -0.00559929078165311
0.04 -0.00507985756838281
0.045 -0.00466825950748984
0.05 -0.00433514926056944
0.055 -0.00406028361365429
0.06 -0.00382951651894814
0.065 -0.00363279606843747
0.07 -0.00346284994708412
0.075 -0.00331432081637928
0.08 -0.0031831922008152
0.085 -0.00306640243334755
0.09 -0.00296158137682312
0.095 -0.00286686807133663
0.1 -0.00278078218987332
0.105 -0.00270213161850563
0.11 -0.00262994470215716
0.115 -0.0025634198784843
0.12 -0.00253148207930565
};
\addlegendentry{POD}
\addplot [semithick, color0, dashed, mark=pentagon, mark size=2.5, mark repeat=5, mark options={solid}]
table {%
0 -0.00746633686219411
0.005 -0.00191205168188446
0.01 0.00320702240208703
0.015 0.00214250930242343
0.02 0.00111506648273263
0.025 0.000475115820727012
0.03 0.00010403093992295
0.035 -7.12205054895776e-05
0.04 -0.000147954261507266
0.045 -0.00017581447766446
0.05 -0.000168761500205505
0.055 -0.000137564761239162
0.06 -0.000105405954101911
0.065 -7.944014357264e-05
0.07 -4.56097915417786e-05
0.075 -1.80734487642553e-05
0.08 1.01643112486727e-05
0.085 3.72892004847358e-05
0.09 5.28989428474347e-05
0.095 7.05332969062056e-05
0.1 9.31240780275289e-05
0.105 0.00010478244558243
0.11 0.000116572332629516
0.115 0.000132521768925642
0.12 0.000141214530785305
};
\addlegendentry{FCNN}
\addplot  [semithick, green!50!black, mark=triangle, mark size=2.5, mark repeat=5, mark options={solid,rotate=180}, only marks]
table {%
0 0.0842819213867188
0.005 0.0232086181640625
0.01 0.0284805297851562
0.015 0.148181915283203
0.02 0.046661376953125
0.025 -0.0418891906738281
0.03 0.0140495300292969
0.035 0.000995635986328125
0.04 0.027679443359375
0.045 0.0475044250488281
0.05 0.0190849304199219
0.055 -0.0116539001464844
0.06 -0.0280494689941406
0.065 -0.00231170654296875
0.07 0.0535316467285156
0.075 0.0304374694824219
0.08 -0.0279808044433594
0.085 -0.0463638305664062
0.09 -0.0435104370117188
0.095 0.0211029052734375
0.1 0.0492706298828125
0.105 -0.00998687744140625
0.11 -0.0488853454589844
0.115 -0.0514602661132812
0.12 -0.0494155883789062
};
\addlegendentry{CNN}

\nextgroupplot[
legend cell align={left},
legend style={fill opacity=0,
	draw opacity=1,
	text opacity=1, 
	at={(1,0)}, 
	anchor=south east, 
	draw=none,
	nodes={
		scale=0.7,
		transform shape
	}
},
ylabel={\(\bar{\dot{\rho u}}\)},
ymin=0.721508781709842, ymax=1.06553556938162,
width=.55\textwidth,
height=.6\textwidth,
y label style={yshift=-1.6em},
y label style={xshift=-1.4em}
]
\addplot  [semithick, black, mark=x, mark size=2.5, mark repeat=5, mark options={solid}]
table {%
0 0.968749498492742
0.005 0.968748999531741
0.01 0.968747998630038
0.015 0.968746993741499
0.02 0.968745987523224
0.025 0.968744980913048
0.03 0.968743974314931
0.035 0.968742967983464
0.04 0.968741962093837
0.045 0.968740956764444
0.05 0.968739952072257
0.055 0.968738948064989
0.06 0.968737944770258
0.065 0.968736942202105
0.07 0.968735940365403
0.075 0.968734939258829
0.08 0.968733938877292
0.085 0.968732939213847
0.09 0.968731940261284
0.095 0.968730942013192
0.1 0.968729944464235
0.105 0.968728947609632
0.11 0.968727951444045
0.115 0.968726955960737
0.12 0.968726458388709
};
\addlegendentry{FOM}
\addplot [semithick, red, mark=o, mark size=2.5, mark repeat=5, mark options={solid}]
table {%
0 0.776747752282815
0.005 0.799229546504149
0.01 0.83735394829241
0.015 0.861900922251396
0.02 0.876455586960931
0.025 0.885983570693113
0.03 0.892681143287458
0.035 0.897645432745145
0.04 0.901479309509238
0.045 0.904538271974328
0.05 0.90704386353466
0.055 0.909140538750041
0.06 0.910926147319329
0.065 0.912469178230276
0.07 0.91381896011011
0.075 0.91501192813239
0.08 0.916075602263358
0.085 0.917031185955863
0.09 0.917895307382834
0.095 0.918681213322456
0.1 0.919399605471469
0.105 0.920059238319471
0.11 0.920667354864786
0.115 0.921230009678775
0.12 0.921500662838827
};
\addlegendentry{POD}
\addplot  [semithick, color0, dashed, mark=pentagon, mark size=2.5, mark repeat=5, mark options={solid}]
table {%
0 0.981022530264175
0.005 0.969845267497955
0.01 0.959354321437879
0.015 0.961170023556659
0.02 0.963262696338745
0.025 0.964992540696261
0.03 0.966216851008718
0.035 0.967038764346892
0.04 0.967648145098694
0.045 0.96805996934129
0.05 0.96834140075498
0.055 0.968603158959088
0.06 0.968791679955521
0.065 0.968910933009226
0.07 0.969025615081555
0.075 0.969102313505783
0.08 0.969166812752425
0.085 0.969199221553223
0.09 0.969228125357358
0.095 0.969266452281406
0.1 0.969295363019938
0.105 0.96931896078058
0.11 0.96933353091193
0.115 0.969335049760666
0.12 0.969333696878397
};
\addlegendentry{FCNN}
\addplot [semithick, green!50!black, mark=triangle, mark size=2.5, mark repeat=5, mark options={solid,rotate=180}, only marks]
table {%
0 0.97954326764293
0.005 0.951945287849175
0.01 0.924409057007322
0.015 0.95474182480782
0.02 0.881638584425079
0.025 0.907850569358606
0.03 1.04989798812381
0.035 1.02313119627251
0.04 0.96617057868159
0.045 0.96673523863701
0.05 1.00865441136285
0.055 1.02940238470562
0.06 0.984113614670854
0.065 0.915812590632678
0.07 0.949019600881284
0.075 0.962200688930006
0.08 0.923812266239562
0.085 0.895938572593428
0.09 0.830964156112351
0.095 0.933549032946724
0.1 0.968627714754298
0.105 0.869239411782875
0.11 0.833597986548245
0.115 0.768703975596079
0.12 0.73714636296765
};
\addlegendentry{CNN}

\nextgroupplot[
legend cell align={left},
legend style={fill opacity=0,
	draw opacity=1, 
	text opacity=1, 
	at={(1,0)}, 
	anchor=south east, 
	draw=none,
	nodes={
		scale=0.7,
		transform shape
	}
},
ylabel={\(\bar{\dot{E}}\)},
ymin=-2.11101520609519, ymax=2.42930439772177,
width=.55\textwidth,
height=.6\textwidth,
y label style={yshift=-2em},
y label style={xshift=-3em}
]
\addplot [semithick, black, mark=x, mark size=2.5, mark repeat=5, mark options={solid}]
table {%
0 4.87748394917276e-05
0.005 4.79943104636504e-05
0.01 4.65877911999257e-05
0.015 4.53842673948657e-05
0.02 4.42552087491777e-05
0.025 4.31684471564608e-05
0.03 4.21090988496076e-05
0.035 4.1068895487939e-05
0.04 4.00427243398838e-05
0.045 3.90272160544214e-05
0.05 3.80200565288646e-05
0.055 3.70196084311658e-05
0.06 3.60246853468027e-05
0.065 3.50344095743083e-05
0.07 3.40481191756226e-05
0.075 3.30653053168817e-05
0.08 3.20855694617705e-05
0.085 3.11085940438716e-05
0.09 3.01341225821261e-05
0.095 2.91619465144777e-05
0.1 2.81918961952954e-05
0.105 2.72238339640296e-05
0.11 2.62576476686149e-05
0.115 2.52932443203235e-05
0.12 2.48114777576802e-05
};
\addlegendentry{FOM}
\addplot  [semithick, red, mark=o, mark size=2.5, mark repeat=5, mark options={solid}]
table {%
0 0.0143189698871922
0.005 -0.00810455140208077
0.01 -0.0298314956071479
0.015 -0.0275728368836106
0.02 -0.0246849481216564
0.025 -0.022325029104703
0.03 -0.0204887749707865
0.035 -0.0190569897727855
0.04 -0.0179250309049905
0.045 -0.0170152912259454
0.05 -0.0162722242749638
0.055 -0.015656108470349
0.06 -0.0151382259954005
0.065 -0.0146975051119611
0.07 -0.0143182441718253
0.075 -0.0139885690766448
0.08 -0.0136993767897806
0.085 -0.0134436012520851
0.09 -0.013215695339742
0.095 -0.0130112593925737
0.1 -0.0128267703619791
0.105 -0.0126593809300424
0.11 -0.0125067682475297
0.115 -0.0123670189805338
0.12 -0.0123001435222179
};
\addlegendentry{POD}
\addplot [semithick, color0, dashed, mark=pentagon, mark size=2.5, mark repeat=5, mark options={solid}]
table {%
0 -0.0144530996476604
0.005 -0.013301816060352
0.01 -0.002997673423355
0.015 0.00688496029257024
0.02 0.00809063479628591
0.025 0.00799179979590647
0.03 0.00721117010957428
0.035 0.0069053952409277
0.04 0.00646577613554911
0.045 0.00591729384263573
0.05 0.00574836602891438
0.055 0.00569547312151641
0.06 0.00551190072786412
0.065 0.00528559237558568
0.07 0.00511527266292688
0.075 0.00491798123422171
0.08 0.00473666708664311
0.085 0.00470337758398642
0.09 0.00442858501268262
0.095 0.0041957603795808
0.1 0.00419847093976955
0.105 0.00404308396729647
0.11 0.0038131760915121
0.115 0.00370871306318676
0.12 0.00370363698505116
};
\addlegendentry{FCNN}
\addplot [semithick, green!50!black, mark=triangle, mark size=2.5, mark repeat=5, mark options={solid,rotate=180}, only marks]
table {%
0 2.22292623391191
0.005 1.77012520595716
0.01 1.42936709093432
0.015 1.82681980799178
0.02 -1.3737616654558
0.025 -1.90463704228533
0.03 0.617198420515578
0.035 0.257909501820961
0.04 0.769456099482134
0.045 -0.134924309656782
0.05 -0.541436050787752
0.055 -0.213320312091664
0.06 -0.815704385410292
0.065 -0.232429535139634
0.07 1.1558903965101
0.075 0.825456903239349
0.08 -0.689314545307141
0.085 -1.07111149054275
0.09 -0.773864974842621
0.095 0.995079248783416
0.1 1.47938049120881
0.105 -0.469208151587615
0.11 -1.08921418156228
0.115 -0.206365288559116
0.12 0.327869451377538
};
\addlegendentry{CNN}

\nextgroupplot[
legend cell align={left},
legend style={fill opacity=0,
	text opacity=1, 
	at={(1,1)}, 
	anchor=north east, 
	draw=none,
	nodes={
		scale=0.7,
		transform shape
	}
},
ylabel={\(\bar{\dot{\rho}}\)},
ymin=-0.0869522094726562, ymax=0.153236389160156,
width=.55\textwidth,
height=.6\textwidth,
y label style={yshift=-1em},
y label style={xshift=-.7em}
]
\addplot [semithick, black, mark=x, mark size=2.5, mark repeat=5, mark options={solid}]
table {%
0 -1.0566054697847e-08
0.005 -1.02966097870194e-08
0.01 -9.79558478775289e-09
0.015 -9.34416988229714e-09
0.02 -8.91122198254379e-09
0.025 -8.48964276656261e-09
0.03 -8.07637690058982e-09
0.035 -7.66986119060675e-09
0.04 -7.26916482562956e-09
0.045 -6.87370516061492e-09
0.05 -6.48320508389588e-09
0.055 -6.1007909835098e-09
0.06 -5.7680509257807e-09
0.065 -5.8217608511768e-09
0.07 -8.19480305835896e-09
0.075 -2.12261852539086e-08
0.08 -7.32128810909671e-08
0.085 -2.42511106307575e-07
0.09 -7.11676761966373e-07
0.095 -1.84790528123813e-06
0.1 -4.3030998000404e-06
0.105 -9.11806750991673e-06
0.11 -1.78100265486592e-05
0.115 -3.24263887492293e-05
0.12 -4.15057958917942e-05
};
\addlegendentry{FOM}
\addplot [semithick, red, mark=o, mark size=2.5, mark repeat=5, mark options={solid}]
table {%
0 -0.00850277287727863
0.005 -0.00772076083175932
0.01 -0.00627892573997002
0.015 -0.00505141716242008
0.02 -0.00399637070279368
0.025 -0.00309606729447864
0.03 -0.00233937069619117
0.035 -0.00171506938696098
0.04 -0.00121061889818463
0.045 -0.000812734404206594
0.05 -0.000508152395383377
0.055 -0.000284182881259198
0.06 -0.000129037084739991
0.065 -3.19895342002496e-05
0.07 1.65679465879975e-05
0.075 2.51385344043342e-05
0.08 1.17137958000058e-06
0.085 -4.88665757245599e-05
0.09 -0.000119410074333359
0.095 -0.000205715696424136
0.1 -0.000303790813717342
0.105 -0.000410338307908376
0.11 -0.000522721868449594
0.115 -0.000638953255311492
0.12 -0.000697822820029614
};
\addlegendentry{POD}
\addplot[semithick, color0, dashed, mark=pentagon, mark size=2.5, mark repeat=5, mark options={solid}]
table {%
0 0.000326463574765512
0.005 -0.00131532022280823
0.01 -0.00240714307327039
0.015 -0.00113891668360822
0.02 -0.000588241155391245
0.025 -0.000590722846460778
0.03 -0.000656246120463777
0.035 -0.000607481205086913
0.04 -6.6451880215368e-05
0.045 4.51196531940923e-05
0.05 -8.27034739785404e-05
0.055 -0.000183065868625931
0.06 -0.000287167389060983
0.065 -0.000487041868908022
0.07 -0.000788421822448981
0.075 -0.000905088287311173
0.08 -0.00105589884350366
0.085 -0.00115068076524949
0.09 -0.0012397377247737
0.095 -0.00139431988364436
0.1 -0.00165710132255015
0.105 -0.0019471158124631
0.11 -0.00238798052932054
0.115 -0.00286884322223102
0.12 -0.0029792309312171
};
\addlegendentry{FCNN}
\addplot [semithick, green!50!black, mark=triangle, mark size=2.5, mark repeat=5, mark options={solid,rotate=180}, only marks]
table {%
0 0.0805206298828125
0.005 0.0176162719726562
0.01 0.0207061767578125
0.015 0.142318725585938
0.02 0.0116500854492188
0.025 -0.0760345458984375
0.03 0.00912857055664062
0.035 -0.0053253173828125
0.04 0.0281791687011719
0.045 0.009979248046875
0.05 -0.0237503051757812
0.055 -0.0180511474609375
0.06 -0.0328407287597656
0.065 0.0026397705078125
0.07 0.0280723571777344
0.075 -0.00083160400390625
0.08 -0.0287628173828125
0.085 -0.0436363220214844
0.09 -0.0279083251953125
0.095 0.0157318115234375
0.1 0.0327720642089844
0.105 -0.00761795043945312
0.11 -0.0422439575195312
0.115 -0.0342559814453125
0.12 -0.0254592895507812
};
\addlegendentry{CNN}

\nextgroupplot[
legend cell align={left},
legend style={fill opacity=0,
	text opacity=1, 
	at={(1,0)}, 
	anchor=south east, 
	draw=none,
	nodes={
		scale=0.7,
		transform shape
	}
},
ylabel={\(\bar{\dot{\rho u}}\)},
ytick={0.8,0.9,1},
ymin=0.701558316660603, ymax=1.04570506859874,
width=.55\textwidth,
height=.6\textwidth,
y label style={yshift=-1.6em},
y label style={xshift=-.9em}
]
\addplot [semithick, black, mark=x, mark size=2.5, mark repeat=5, mark options={solid}]
table {%
0 0.968749977995855
0.005 0.968749977493689
0.01 0.968749976488528
0.015 0.968749975482364
0.02 0.968749974475969
0.025 0.968749973469496
0.03 0.968749972463012
0.035 0.968749971456551
0.04 0.968749970450135
0.045 0.968749969443747
0.05 0.968749968436312
0.055 0.96874996740731
0.06 0.968749966119898
0.065 0.968749962757677
0.07 0.968749947403054
0.075 0.968749879330216
0.08 0.968749627063247
0.085 0.968748844536202
0.09 0.968746766180359
0.095 0.968741931728314
0.1 0.968731885809511
0.105 0.96871291830678
0.11 0.968679910009525
0.115 0.968626314303693
0.12 0.96859346363712
};
\addlegendentry{FOM}
\addplot  [semithick, red, mark=o, mark size=2.5, mark repeat=5, mark options={solid}]
table {%
0 0.874450523447081
0.005 0.882012675145708
0.01 0.896373906746789
0.015 0.908519731859784
0.02 0.918013001095483
0.025 0.92537536910019
0.03 0.931091478463895
0.035 0.935538828032017
0.04 0.939001516793107
0.045 0.941693218607178
0.05 0.943776372159542
0.055 0.945376223453617
0.06 0.946590699787999
0.065 0.947497302172493
0.07 0.948157958205335
0.075 0.948622499149348
0.08 0.948931205690461
0.085 0.949116712127438
0.09 0.949205454769583
0.095 0.949218783052661
0.1 0.949173810629732
0.105 0.949084060185012
0.11 0.948959943099316
0.115 0.948809107113092
0.12 0.948727801667509
};
\addlegendentry{POD}
\addplot [semithick, color0, dashed, mark=pentagon, mark size=2.5, mark repeat=5, mark options={solid}]
table {%
0 0.970399317976372
0.005 0.963503522566775
0.01 0.957508562394808
0.015 0.961566075034872
0.02 0.966404637762006
0.025 0.968473543644464
0.03 0.969027915221748
0.035 0.970405037818537
0.04 0.971588561357327
0.045 0.971385233690603
0.05 0.97084447858429
0.055 0.970142786797226
0.06 0.970058801045607
0.065 0.969591850053609
0.07 0.968916380492354
0.075 0.968742961697338
0.08 0.968382479535162
0.085 0.968240874778057
0.09 0.967734768725052
0.095 0.967435535078145
0.1 0.967173210304168
0.105 0.966540158655194
0.11 0.965694011500871
0.115 0.964573306518972
0.12 0.96413484084443
};
\addlegendentry{FCNN}
\addplot  [semithick, green!50!black, mark=triangle, mark size=2.5, mark repeat=5, mark options={solid,rotate=180}, only marks]
table {%
0 0.961906278822966
0.005 0.933425723879134
0.01 0.905878353170239
0.015 0.940320663551927
0.02 0.857029422900776
0.025 0.879651316447493
0.03 1.03006203441973
0.035 1.00402846909231
0.04 0.953349925570686
0.045 0.935897540926501
0.05 0.970320240998085
0.055 1.00498821285898
0.06 0.961259257132621
0.065 0.898568595199741
0.07 0.907571697214435
0.075 0.911803760208734
0.08 0.892125015958701
0.085 0.867297804450649
0.09 0.809100371138463
0.095 0.88698271808509
0.1 0.912923489640189
0.105 0.833686946039219
0.11 0.802155935069418
0.115 0.744677807304853
0.12 0.717201350839609
};
\addlegendentry{CNN}

\nextgroupplot[
legend cell align={left},
legend style={fill opacity=0, 
	text opacity=1, 
	at={(1,0)},
	 anchor=south east, 
	 draw=none,
	 nodes={
	 	scale=0.7,
	 	transform shape
	 }
},
ylabel={\(\bar{\dot{E}}\)},
ymin=-2.1075135100065, ymax=2.44711060492608,
width=.55\textwidth,
height=.6\textwidth,
y label style={yshift=-2em},
y label style={xshift=-3em}
]
\addplot [semithick, black, mark=x, mark size=2.5, mark repeat=5, mark options={solid}]
table {%
0 4.82813575786167e-08
0.005 4.7063597463648e-08
0.01 4.47984191964679e-08
0.015 4.27564792460089e-08
0.02 4.07976692429202e-08
0.025 3.88903416137509e-08
0.03 3.70209036759661e-08
0.035 3.51821611843661e-08
0.04 3.33698650933911e-08
0.045 3.15811483631023e-08
0.05 2.98109235075117e-08
0.055 2.79997891539097e-08
0.06 2.55037626573085e-08
0.065 1.76024457232415e-08
0.07 -2.03018117872489e-08
0.075 -1.84349591592081e-07
0.08 -7.69065788830403e-07
0.085 -2.50761453557402e-06
0.09 -6.92790824885492e-06
0.095 -1.676259263661e-05
0.1 -3.6295084708371e-05
0.105 -7.1516663975757e-05
0.11 -0.000130010286365945
0.115 -0.000220553638243359
0.12 -0.000274993156637038
};
\addlegendentry{FOM}
\addplot [semithick, red, mark=o, mark size=2.5, mark repeat=5, mark options={solid}]
table {%
0 -0.0369185874576559
0.005 -0.0334847636942506
0.01 -0.0271861464861836
0.015 -0.0218886278141817
0.02 -0.0173890616645664
0.025 -0.0135858142109271
0.03 -0.0104139688794262
0.035 -0.00781383933774293
0.04 -0.00572432983522475
0.045 -0.00408429540355471
0.05 -0.00283483794892092
0.055 -0.00192094953581545
0.06 -0.00129239866497599
0.065 -0.000904061088800034
0.07 -0.000715899060683256
0.075 -0.000692736728687748
0.08 -0.000803927287726935
0.085 -0.00102297077507174
0.09 -0.00132711814463704
0.095 -0.00169698339989566
0.1 -0.00211617764014349
0.105 -0.00257097444239918
0.11 -0.00305001321775222
0.115 -0.00354404418175491
0.12 -0.00379377610802578
};
\addlegendentry{POD}
\addplot  [semithick, color0, dashed, mark=pentagon, mark size=2.5, mark repeat=5, mark options={solid}]
table {%
0 -0.0220373594494561
0.005 -0.0187375871121702
0.01 -0.0137267475558431
0.015 -0.00708892810284567
0.02 -0.000805795336006554
0.025 0.00385607701109691
0.03 0.00688734023708193
0.035 0.00506515072139635
0.04 -0.0011157740152008
0.045 -0.00625783112414524
0.05 -0.00777156337428764
0.055 -0.0101094557751722
0.06 -0.0125639515699199
0.065 -0.0137910901982785
0.07 -0.0135298759123259
0.075 -0.0147397060042884
0.08 -0.0167050544650884
0.085 -0.0165941921786654
0.09 -0.0155688464916217
0.095 -0.0160095281063555
0.1 -0.0163660479894041
0.105 -0.0146851150221678
0.11 -0.0137112109201389
0.115 -0.0133200208008226
0.12 -0.0128320391498704
};
\addlegendentry{FCNN}
\addplot [semithick, green!50!black, mark=triangle, mark size=2.5, mark repeat=5, mark options={solid,rotate=180}, only marks]
table {%
0 2.2400822360655
0.005 1.78518401371828
0.01 1.44243569371781
0.015 1.84259146740263
0.02 -1.36916045778383
0.025 -1.90048514114593
0.03 0.63336327700209
0.035 0.272824821472579
0.04 0.786352591134989
0.045 -0.131427471443889
0.05 -0.540336303919108
0.055 -0.200548516735186
0.06 -0.803961333714966
0.065 -0.218569256769491
0.07 1.1578414467834
0.075 0.82358957236357
0.08 -0.681481865767363
0.085 -1.06250963505425
0.09 -0.761910306073361
0.095 0.997457794438301
0.1 1.47727023886346
0.105 -0.464125495002811
0.11 -1.08283991448095
0.115 -0.195796045692205
0.12 0.341143847513344
};
\addlegendentry{CNN}
\end{groupplot}

\end{tikzpicture}

	\caption{Comparison of the conservative properties of reconstructions obatined from POD, the FCNN and the CNN against the conservative properties of the FOM solution using the temporal mean.}
	\label{Fig:Conservation}
\end{figure}
Conservative properties of the FOM are discussed in \cref{Ch:BGK}. Now we want to estimate if the conservation of mass, momentum and total energy could be sustained using POD, the FCNN, and the CNN. To do so, the temporal mean over the time derivative of the macroscopic quantities is employed. \Cref{Fig:Conservation} shows the conservation of mass, momentum and total energy over time for \(\hy\) in the top row and for \(\rare\) in the bottom row.\\
Conservation of mass is met using the FCNN, except for small deviations at the outset for both cases \(\hy\) and \(\rare\). Similarly, does POD meet conservation of mass for \(\rare\). The erroneous \(\hy\) case shows deviations from conservation with POD. Conservation of momentum meets the FOM solution after \(t=0.03s\) using the FCNN for both cases \(\hy\) and \(\rare\). POD conserves momentum close to the FOM solution, but deviations are similarly present for \(\hy\) and \(\rare\). Next conservation of total energy is met for \(\hy\) and \(\rare\) using POD and the FCNN. Finally the reconstructions of the CNN do not conserve mass, momentum nor total energy. All conservative properties behave comparable to a sawtooth wave. A gain and loss of either of the quantities can be observed.\\
In conclusion the error over time for the CNN is disordered, showing gain and subsequent loss of information from one timestep to another. The CNN performs slighly better with \(\hy\) than with \(\rare\). What is hidden when looking at reconstructions becomes visible when verifying over the macroscopic quantities. Reconstructions obtained from the CNN show oscillations in the momentum \(\rho u\) and the total energy \(E\). On top of that the CNN does not meet conservation in any of the conservative properties. All of this together makes the CNN with this setup, especially the access to only 40 samples, unsuited for building a ROM. Next POD shows a noticeable increase in loss of information over time for both cases \(\hy\) and \(\rare\). Reconstructions of the last timestep as well as the macroscopic quantities at that time reveal that the POD is unsuited for building a ROM with the \(\hy\)case. However, with \(\rare\) POD shows only slight deviations from the FOM solution. Taking conservative properties of the reconstructions obtained from POD into consideration only underlines aforementioned findings. Ultimately POD could be taken for building a ROM with \(\rare\). Finally the the FCNN is the best performing model out of the three for both cases \(\hy\) and \(\rare\), while the performance for \(\hy\) is slightly better than that for \(\rare\). The error over time reveals a constant low loss. Only at the first time steps a noticeable loss of information is observed. Reconstructions of the last time step and the macroscopic quantities at that time are close to exact to the FOM solution for both cases \(\hy\) and \(\rare\). The conservation of the macroscopic quantites only emphasize the proximity to the FOM solution. In total the FCNN is suited for building a ROM with both cases \(\hy\) and \(\rare\) and will be taken further into the online phase.\\
We now reached the online phase where we want to be independent from the FOM solution. A first ROM relying on pure interpolation in the intrinsic variables is performed for \(\hy\). The intrinsic variables of the FCNN for \(\hy\) are shown in ... . 

With POD one usually exploits the intrinsic variables within a Galerkin framework as in \cite{Bernard} to produce new states. The same can be done with the intrinsic variables obtained from autoencoders as in \cite{Carlberg}. Both won't be discussed in this contribution. Rather new states are obtained by interpolating \(\idhy\) and \(\idrare\) in time \(t\). This approach tests a different kind of generalization about the FOM solution. Therefore this kind of generalization ability of the proposed autoencoder architectures will be analyzed.

\begin{figure}[H]
	% This file was created by tikzplotlib v0.9.6.
\begin{tikzpicture}

\begin{groupplot}[group style={group size=1 by 3,vertical sep=1cm}]
\nextgroupplot[
colorbar,
colorbar style={ylabel={}},
colormap/blackwhite,
point meta max=0.532396674156189,
point meta min=-0.00969430990517139,
tick align=outside,
tick pos=left,
x grid style={white!69.0196078431373!black},
xmin=0.0025, xmax=0.9975,
xtick style={color=black},
xlabel={x},
y grid style={white!69.0196078431373!black},
ymin=0, ymax=0.12,
ytick style={color=black},
ylabel={t},
ytick={0,0.06,0.12},
width=.9\textwidth,
height=.25\textwidth,
y tick label style={/pgf/number format/fixed}
]
\addplot graphics [includegraphics cmd=\pgfimage,xmin=0.0025, xmax=0.9975, ymin=0, ymax=0.12] {Figures/Results/Code2D_hy_FCNN-000.png};
\node [draw,fill=white] at (0.95,0.1) {\(\alpha_1\)};

\nextgroupplot[
colorbar,
colorbar style={ylabel={}},
colormap/blackwhite,
point meta max=-0.197054535150528,
point meta min=-0.552837491035461,
tick align=outside,
tick pos=left,
x grid style={white!69.0196078431373!black},
xmin=0.0025, xmax=0.9975,
xtick style={color=black},
xlabel={x},
y grid style={white!69.0196078431373!black},
ymin=0, ymax=0.12,
ytick={0,0.06,0.12},
ytick style={color=black},
ylabel={t},
width=.9\textwidth,
height=.25\textwidth,
y tick label style={/pgf/number format/fixed}
]
\addplot graphics [includegraphics cmd=\pgfimage,xmin=0.0025, xmax=0.9975, ymin=0, ymax=0.12] {Figures/Results/Code2D_hy_FCNN-001.png};
\node [draw,fill=white] at (0.95,0.1) {\(\alpha_2\)};

\nextgroupplot[
colorbar,
colorbar style={ylabel={}},
colormap/blackwhite,
point meta max=-0.136639997363091,
point meta min=-0.320842951536179,
tick align=outside,
tick pos=left,
x grid style={white!69.0196078431373!black},
xmin=0.0025, xmax=0.9975,
xlabel={x},
xtick style={color=black},
y grid style={white!69.0196078431373!black},
ymin=0, ymax=0.12,
ytick={0,0.06,0.12},
ytick style={color=black},
ylabel={t},
width=.9\textwidth,
height=.25\textwidth,
y tick label style={/pgf/number format/fixed}
]
\addplot graphics [includegraphics cmd=\pgfimage,xmin=0.0025, xmax=0.9975, ymin=0, ymax=0.12] {Figures/Results/Code2D_hy_FCNN-002.png};
\node [draw,fill=white] at (0.95,0.1) {\(\alpha_3\)};
\end{groupplot}

\end{tikzpicture}

	\caption{\(\alpha_1\), \(\alpha_2\) and \(\alpha_3\), the reduced basis \(\idhy\) obtained from the FCNN.}
\end{figure}
\begin{figure}[H]
	% This file was created by tikzplotlib v0.9.8.
\begin{tikzpicture}

\begin{groupplot}[group style={group size=3 by 1},
legend cell align={left},
legend style={fill opacity=0.1, draw opacity=1, text opacity=1, at={(1,1)}, anchor=north east, draw=none},
tick align=outside,
tick pos=left,
x grid style={white!69.0196078431373!black},
xlabel={\(x\)},
xmin=-0.04725, xmax=1.04725,
xtick style={color=black},
y grid style={white!69.0196078431373!black},
ytick style={color=black},
width=.35\textwidth,
height=.4\textwidth,
y label style={yshift=-2em}
]
\nextgroupplot[
ylabel={\(\rho\)},
ytick={0.2,0.4,0.8,1},
ymin=0.0812408233619941, ymax=1.04383396967043,
]
\addplot [semithick, color0, mark=pentagon, mark size=2, mark options={solid},mark repeat=5]
table {%
0.0025 1.00007958960132
0.0075 1.00007958960132
0.0125 1.00007958960132
0.0175 1.00007958960132
0.0225 1.00007958960132
0.0275 1.00007958960132
0.0325 1.00007958960132
0.0375 1.00007958960132
0.0425 1.00007958960132
0.0475 1.00007958960132
0.0525 1.00007958960132
0.0575 1.00007958960132
0.0625 1.00007958960132
0.0675 1.00007958960132
0.0725 1.00007958960132
0.0775 1.00007958960132
0.0825 1.00007958960132
0.0875 1.00007958960132
0.0925 1.00007958960132
0.0975 1.00007960560096
0.1025 1.00007967294275
0.1075 1.00007964882388
0.1125 1.00007973574732
0.1175 1.00007942697807
0.1225 1.00007952906535
0.1275 1.00007935247227
0.1325 1.00007910507478
0.1375 1.0000786045486
0.1425 1.00007773758281
0.1475 1.00007644805925
0.1525 1.00007424166057
0.1575 1.00007047887858
0.1625 1.00006458671113
0.1675 1.00005566690952
0.1725 1.00004139821195
0.1775 1.00001937338839
0.1825 0.999986432031501
0.1875 0.999937488403667
0.1925 0.999865611912762
0.1975 0.999762540533591
0.2025 0.999616785088856
0.2075 0.999413966793886
0.2125 0.999135611671512
0.2175 0.998761416889719
0.2225 0.998265151151335
0.2275 0.997618268225603
0.2325 0.996788949596336
0.2375 0.99574209013388
0.2425 0.994440960625299
0.2475 0.992848293362174
0.2525 0.990926713533989
0.2575 0.988643694819403
0.2625 0.98597447156029
0.2675 0.982894724536353
0.2725 0.979384886899208
0.2775 0.97543120528311
0.2825 0.971025352350884
0.2875 0.966164945468956
0.2925 0.960852690965192
0.2975 0.955096454813896
0.3025 0.948908228748352
0.3075 0.942303409679839
0.3125 0.935299842942792
0.3175 0.927917993992475
0.3225 0.920179602285666
0.3275 0.912107081055203
0.3325 0.903720666985319
0.3375 0.895043348699038
0.3425 0.886099242316696
0.3475 0.876913020961165
0.3525 0.867506658352671
0.3575 0.8578984900834
0.3625 0.848106184178299
0.3675 0.838090457212586
0.3725 0.827832277131951
0.3775 0.817345481359971
0.3825 0.806643968094992
0.3875 0.795742298087786
0.3925 0.784654691798187
0.3975 0.773418127866418
0.4025 0.76206761590187
0.4075 0.750619441273673
0.4125 0.739091075712631
0.4175 0.727500537444857
0.4225 0.715867658268065
0.4275 0.704210523511352
0.4325 0.692526782779235
0.4375 0.680846175469963
0.4425 0.66921524980196
0.4475 0.657692996940784
0.4525 0.646356617796227
0.4575 0.635305164135372
0.4625 0.624665874322266
0.4675 0.61460028374918
0.4725 0.605298140195823
0.4775 0.596974775589291
0.4825 0.589842921078701
0.4875 0.584063832062555
0.4925 0.579672229440244
0.4975 0.576530995545067
0.5025 0.574369383563039
0.5075 0.572842073294033
0.5125 0.571619951938634
0.5175 0.570435925138936
0.5225 0.56906307830009
0.5275 0.567255114691168
0.5325 0.564675626033231
0.5375 0.560842670806613
0.5425 0.555107028704735
0.5475 0.546692667435545
0.5525 0.534817340636247
0.5575 0.5188380042211
0.5625 0.498502658421844
0.5675 0.473995488276239
0.5725 0.446001658485437
0.5775 0.415641278464789
0.5825 0.38431216452763
0.5875 0.353492774178785
0.5925 0.324502245123317
0.5975 0.298354483407366
0.6025 0.275683994082713
0.6075 0.256797161618465
0.6125 0.241638212449143
0.6175 0.229901758262591
0.6225 0.221134202819148
0.6275 0.214821467045285
0.6325 0.210452181777347
0.6375 0.207558450317622
0.6425 0.205741919376864
0.6475 0.204680016666904
0.6525 0.204124898744949
0.6575 0.203894491451661
0.6625 0.203860315972972
0.6675 0.203935182367526
0.6725 0.204061199146803
0.6775 0.204198292814041
0.6825 0.204315014755273
0.6875 0.204378813575093
0.6925 0.204344825913122
0.6975 0.204142378236774
0.7025 0.203654290704554
0.7075 0.202686372769895
0.7125 0.200921816195532
0.7175 0.197857485664529
0.7225 0.19278071511726
0.7275 0.184850776625979
0.7325 0.173437997270236
0.7375 0.159227317397964
0.7425 0.144888745564241
0.7475 0.134016824746112
0.7525 0.128206411531143
0.7575 0.125959403239509
0.7625 0.125261302826069
0.7675 0.125066164599088
0.7725 0.125013926119164
0.7775 0.125000055502295
0.7825 0.124996353614482
0.7875 0.124995377277988
0.7925 0.124995147432354
0.7975 0.124995057285105
0.8025 0.124995140387734
0.8075 0.124995080209968
0.8125 0.124995080209968
0.8175 0.124995080209968
0.8225 0.124995080209968
0.8275 0.124995080209968
0.8325 0.124995080209968
0.8375 0.124995080209968
0.8425 0.124995080209968
0.8475 0.124995080209968
0.8525 0.124995080209968
0.8575 0.124995080209968
0.8625 0.124995080209968
0.8675 0.124995080209968
0.8725 0.124995080209968
0.8775 0.124995080209968
0.8825 0.124995080209968
0.8875 0.124995080209968
0.8925 0.124995080209968
0.8975 0.124995080209968
0.9025 0.124995080209968
0.9075 0.124995080209968
0.9125 0.124995080209968
0.9175 0.124995080209968
0.9225 0.124995080209968
0.9275 0.124995080209968
0.9325 0.124995080209968
0.9375 0.124995080209968
0.9425 0.124995080209968
0.9475 0.124995080209968
0.9525 0.124995080209968
0.9575 0.124995080209968
0.9625 0.124995080209968
0.9675 0.124995080209968
0.9725 0.124995080209968
0.9775 0.124995080209968
0.9825 0.124995080209968
0.9875 0.124995080209968
0.9925 0.124995080209968
0.9975 0.124995080209968
};
\addlegendentry{prediction}
\addplot [semithick, black, mark=+, mark size=2, mark options={solid},
dashed,%only marks,
mark repeat=5]
table {%
0.0025 0.999999994499998
0.0075 0.999999994499989
0.0125 0.999999994499973
0.0175 0.999999994499938
0.0225 0.999999994499854
0.0275 0.999999994499657
0.0325 0.999999994499225
0.0375 0.999999994498307
0.0425 0.999999994496346
0.0475 0.999999994492153
0.0525 0.999999994483302
0.0575 0.999999994464812
0.0625 0.999999994426618
0.0675 0.999999994348593
0.0725 0.999999994191032
0.0775 0.999999993876523
0.0825 0.999999993255951
0.0875 0.999999992045765
0.0925 0.999999989713722
0.0975 0.999999985273699
0.1025 0.999999976922728
0.1075 0.999999961408873
0.1125 0.999999932947061
0.1175 0.999999881389466
0.1225 0.999999789188686
0.1275 0.999999626442624
0.1325 0.99999934295137
0.1375 0.999998855715861
0.1425 0.99999802963589
0.1475 0.99999664829786
0.1525 0.999994370682796
0.1575 0.999990668414943
0.1625 0.999984736919399
0.1675 0.999975372768066
0.1725 0.999960808885077
0.1775 0.999938499600449
0.1825 0.999904849332842
0.1875 0.999854882524753
0.1925 0.99978185882593
0.1975 0.999676846595887
0.2025 0.999528279209094
0.2075 0.999321531266986
0.2125 0.999038563648431
0.2175 0.998657694619451
0.2225 0.99815355591569
0.2275 0.997497285204109
0.2325 0.996656988391704
0.2375 0.995598477836268
0.2425 0.994286259099913
0.2475 0.992684705039226
0.2525 0.990759328214853
0.2575 0.988478046531036
0.2625 0.985812336037212
0.2675 0.982738179020057
0.2725 0.979236741747293
0.2775 0.975294749126552
0.2825 0.970904557028253
0.2875 0.96606395178504
0.2925 0.960775726924111
0.2975 0.955047098184107
0.3025 0.948889019907117
0.3075 0.942315460872332
0.3125 0.9353426880196
0.3175 0.927988594730014
0.3225 0.920272098375855
0.3275 0.912212621103049
0.3325 0.903829659028516
0.3375 0.895142438489912
0.3425 0.886169653587621
0.3475 0.876929276719067
0.3525 0.867438432750232
0.3575 0.857713327521522
0.3625 0.847769222220737
0.3675 0.837620446519981
0.3725 0.82728044509819
0.3775 0.816761854164244
0.3825 0.806076606850691
0.3875 0.795236068929107
0.3925 0.784251209358396
0.3975 0.773132813964529
0.4025 0.761891755435828
0.4075 0.750539339336509
0.4125 0.739087754739222
0.4175 0.727550670369378
0.4225 0.715944034148932
0.4275 0.704287157239119
0.4325 0.692604194420389
0.4375 0.680926170788684
0.4425 0.669293745622052
0.4475 0.657760931521731
0.4525 0.646399957544768
0.4575 0.635307281677215
0.4625 0.624610228710984
0.4675 0.614472544087769
0.4725 0.605094996557836
0.4775 0.596704251904308
0.4825 0.589521611837022
0.4875 0.583707948506557
0.4925 0.57929759656908
0.4975 0.576156313265513
0.5025 0.574000965899689
0.5075 0.57248017604117
0.5125 0.571264200514682
0.5175 0.570086379629996
0.5225 0.568721456981151
0.5275 0.566925162056573
0.5325 0.564365789892603
0.5375 0.560570505963129
0.5425 0.554906052745071
0.5475 0.546613496573865
0.5525 0.534908968811207
0.5575 0.519142277507728
0.5625 0.49897861904467
0.5675 0.474548678169257
0.5725 0.446512146926428
0.5775 0.41600335371414
0.5825 0.384466860249091
0.5875 0.353428287708994
0.5925 0.324264601694908
0.5975 0.298031573575595
0.6025 0.275379773169268
0.6075 0.256558296419699
0.6125 0.241481616612008
0.6175 0.229826468110973
0.6225 0.221130573161533
0.6275 0.214876144173828
0.6325 0.210551636151645
0.6375 0.207691966758486
0.6425 0.205900478048754
0.6475 0.20485677362768
0.6525 0.204314507250902
0.6575 0.204092812694663
0.6625 0.204064483070227
0.6675 0.204143251128216
0.6725 0.204271659470887
0.6775 0.204410148437548
0.6825 0.204527208694704
0.6875 0.204589750704457
0.6925 0.204552164518092
0.6975 0.204341776813738
0.7025 0.203837533803555
0.7075 0.202838125078778
0.7125 0.201017075159804
0.7175 0.197870633082619
0.7225 0.192691646081915
0.7275 0.18466735979763
0.7325 0.173289805104828
0.7375 0.159203830364094
0.7425 0.144954928763386
0.7475 0.134090935533034
0.7525 0.128237750573878
0.7575 0.125970038823052
0.7625 0.125265821320929
0.7675 0.125069121636156
0.7725 0.125016331920844
0.7775 0.125002354274959
0.7825 0.124998671121747
0.7875 0.124997702861611
0.7925 0.124997448751868
0.7975 0.124997382171878
0.8025 0.124997364756704
0.8075 0.124997360209753
0.8125 0.124997359024918
0.8175 0.124997358716831
0.8225 0.124997358636905
0.8275 0.124997358616222
0.8325 0.124997358610884
0.8375 0.12499735860951
0.8425 0.124997358609158
0.8475 0.124997358609067
0.8525 0.124997358609045
0.8575 0.124997358609039
0.8625 0.124997358609038
0.8675 0.124997358609038
0.8725 0.124997358609037
0.8775 0.124997358609037
0.8825 0.124997358609037
0.8875 0.124997358609037
0.8925 0.124997358609037
0.8975 0.124997358609037
0.9025 0.124997358609037
0.9075 0.124997358609037
0.9125 0.124997358609037
0.9175 0.124997358609037
0.9225 0.124997358609037
0.9275 0.124997358609037
0.9325 0.124997358609037
0.9375 0.124997358609037
0.9425 0.124997358609037
0.9475 0.124997358609037
0.9525 0.124997358609037
0.9575 0.124997358609037
0.9625 0.124997358609037
0.9675 0.124997358609037
0.9725 0.124997358609037
0.9775 0.124997358609037
0.9825 0.124997358609037
0.9875 0.124997358609037
0.9925 0.124997358609037
0.9975 0.124997358609037
};
\addlegendentry{truth}

\nextgroupplot[
ylabel={\(\rho u\)},
ytick = {0,0.1,0.3,0.4},
ymin=-0.0210742313599136, ymax=0.437621533883013,
]
\addplot [semithick, color0, mark=pentagon, mark size=2, mark options={solid},mark repeat=5]
table {%
0.0025 -0.000224018744984643
0.0075 -0.000224018744984643
0.0125 -0.000224018744984643
0.0175 -0.000224018744984643
0.0225 -0.000224018744984643
0.0275 -0.000224018744984643
0.0325 -0.000224018744984643
0.0375 -0.000224018744984643
0.0425 -0.000224018744984643
0.0475 -0.000224018744984643
0.0525 -0.000224018744984643
0.0575 -0.000224018744984643
0.0625 -0.000224018744984643
0.0675 -0.000224018744984643
0.0725 -0.000224018744984643
0.0775 -0.000224018744984643
0.0825 -0.000224018744984643
0.0875 -0.000224018744984643
0.0925 -0.000224018744984643
0.0975 -0.000224114204017176
0.1025 -0.000224420971017424
0.1075 -0.000224423848871476
0.1125 -0.000224399111572556
0.1175 -0.000224021377914952
0.1225 -0.000223996242614871
0.1275 -0.000223399761430633
0.1325 -0.000222882727388758
0.1375 -0.000222388961103418
0.1425 -0.000220583597507155
0.1475 -0.000218405245650912
0.1525 -0.000214517724002303
0.1575 -0.000207951256962543
0.1625 -0.000198811069898842
0.1675 -0.000182532549448376
0.1725 -0.000157972483380186
0.1775 -0.000120629754814146
0.1825 -6.45468077751703e-05
0.1875 1.94128097132156e-05
0.1925 0.000142050020835177
0.1975 0.000317399813851215
0.2025 0.000566233682376917
0.2075 0.000911136198722704
0.2125 0.00138317272628902
0.2175 0.00201805094390059
0.2225 0.00285827773621694
0.2275 0.00395250206330534
0.2325 0.00535272705479511
0.2375 0.00711846857027321
0.2425 0.00930975956459987
0.2475 0.0119877343535197
0.2525 0.0152145700462701
0.2575 0.0190384658616776
0.2625 0.023486006956628
0.2675 0.0285888621560336
0.2725 0.0343668356271487
0.2775 0.0408301942981457
0.2825 0.0479735901058079
0.2875 0.0557848471250038
0.2925 0.0642390732036719
0.2975 0.0733041575213734
0.3025 0.0829408320345379
0.3075 0.093101483810572
0.3125 0.103735478966086
0.3175 0.114789414264818
0.3225 0.126204462695237
0.3275 0.137922470816743
0.3325 0.149856288696278
0.3375 0.161922343022342
0.3425 0.17405417015264
0.3475 0.186189567965098
0.3525 0.198268231517027
0.3575 0.210235498357924
0.3625 0.222050954078593
0.3675 0.233822526473728
0.3725 0.245508581475077
0.3775 0.257068285670135
0.3825 0.268466084457423
0.3875 0.279669139654188
0.3925 0.290645325132257
0.3975 0.30128351815843
0.4025 0.311509890508708
0.4075 0.321307295897787
0.4125 0.330665317075539
0.4175 0.339573258160456
0.4225 0.348027166971018
0.4275 0.356030595024101
0.4325 0.363641410975208
0.4375 0.370859526743636
0.4425 0.3776673689889
0.4475 0.384045174629705
0.4525 0.389965828769702
0.4575 0.395394642704727
0.4625 0.400289485355001
0.4675 0.404598553234369
0.4725 0.408279565124794
0.4775 0.4112957652881
0.4825 0.413636596928851
0.4875 0.415293668469434
0.4925 0.416305254573296
0.4975 0.416742638645411
0.5025 0.416771726371971
0.5075 0.416527856893603
0.5125 0.416087128785646
0.5175 0.415465345538933
0.5225 0.414619796982088
0.5275 0.413435572513614
0.5325 0.411703304513104
0.5375 0.409091789631392
0.5425 0.405106248364163
0.5475 0.399101061056479
0.5525 0.39036861576975
0.5575 0.378290466986318
0.5625 0.363146681986418
0.5675 0.345323436214576
0.5725 0.32549162805932
0.5775 0.304513083434557
0.5825 0.283249913988369
0.5875 0.262475023127174
0.5925 0.242767803704885
0.5975 0.224524474631891
0.6025 0.208051999031094
0.6075 0.193581449267863
0.6125 0.181341328977864
0.6175 0.171418746395887
0.6225 0.163721182217289
0.6275 0.15801678880118
0.6325 0.153986948144191
0.6375 0.151285444173278
0.6425 0.149584317095146
0.6475 0.148598559150083
0.6525 0.148099745476024
0.6575 0.147915445005698
0.6625 0.14792264607018
0.6675 0.148034800207918
0.6725 0.148191477993727
0.6775 0.148346908903325
0.6825 0.148456643561773
0.6875 0.148466094832618
0.6925 0.148289537902271
0.6975 0.147783888314429
0.7025 0.146705594838524
0.7075 0.14462977813771
0.7125 0.140840847912772
0.7175 0.134184715481149
0.7225 0.123067795069719
0.7275 0.105610533503771
0.7325 0.0802160459178515
0.7375 0.049383167051561
0.7425 0.0224540554733486
0.7475 0.00768127561627308
0.7525 0.00231781407386183
0.7575 0.000701356647177434
0.7625 0.000246654138973484
0.7675 0.00012330573050479
0.7725 9.08364029228138e-05
0.7775 8.20810216900367e-05
0.7825 7.98726666741706e-05
0.7875 7.89232503624513e-05
0.7925 7.88321693419298e-05
0.7975 7.90195053972418e-05
0.8025 7.88600600342628e-05
0.8075 7.87138405539246e-05
0.8125 7.87138405539246e-05
0.8175 7.87138405539246e-05
0.8225 7.87138405539246e-05
0.8275 7.87138405539246e-05
0.8325 7.87138405539246e-05
0.8375 7.87138405539246e-05
0.8425 7.87138405539246e-05
0.8475 7.87138405539246e-05
0.8525 7.87138405539246e-05
0.8575 7.87138405539246e-05
0.8625 7.87138405539246e-05
0.8675 7.87138405539246e-05
0.8725 7.87138405539246e-05
0.8775 7.87138405539246e-05
0.8825 7.87138405539246e-05
0.8875 7.87138405539246e-05
0.8925 7.87138405539246e-05
0.8975 7.87138405539246e-05
0.9025 7.87138405539246e-05
0.9075 7.87138405539246e-05
0.9125 7.87138405539246e-05
0.9175 7.87138405539246e-05
0.9225 7.87138405539246e-05
0.9275 7.87138405539246e-05
0.9325 7.87138405539246e-05
0.9375 7.87138405539246e-05
0.9425 7.87138405539246e-05
0.9475 7.87138405539246e-05
0.9525 7.87138405539246e-05
0.9575 7.87138405539246e-05
0.9625 7.87138405539246e-05
0.9675 7.87138405539246e-05
0.9725 7.87138405539246e-05
0.9775 7.87138405539246e-05
0.9825 7.87138405539246e-05
0.9875 7.87138405539246e-05
0.9925 7.87138405539246e-05
0.9975 7.87138405539246e-05
};
\addlegendentry{pred.}
\addplot [semithick, black, mark=+, mark size=2, mark options={solid},dashed,
mark repeat=5]
table {%
0.0025 6.30909939231123e-15
0.0075 2.52743809394443e-14
0.0125 6.71665948905054e-14
0.0175 1.47797315006362e-13
0.0225 3.07924882400649e-13
0.0275 6.6172472758689e-13
0.0325 1.42564590009118e-12
0.0375 3.07415834705836e-12
0.0425 6.61513596957766e-12
0.0475 1.41586575371753e-11
0.0525 3.00685203765747e-11
0.0575 6.32804197847002e-11
0.0625 1.31840368981028e-10
0.0675 2.71816268178485e-10
0.0725 5.54376848420904e-10
0.0775 1.11825149293422e-09
0.0825 2.23051508115254e-09
0.0875 4.3988436607807e-09
0.0925 8.57587718345818e-09
0.0975 1.6525966137083e-08
0.1025 3.14735316685944e-08
0.1075 5.92319220162265e-08
0.1125 1.1013832102345e-07
0.1175 2.02317177002482e-07
0.1225 3.67093736874415e-07
0.1275 6.57821440259511e-07
0.1325 1.16402287545304e-06
0.1375 2.03362895936117e-06
0.1425 3.50728899710936e-06
0.1475 5.97025139178314e-06
0.1525 1.00291758876392e-05
0.1575 1.66233526126904e-05
0.1625 2.71819698904176e-05
0.1675 4.38409222097891e-05
0.1725 6.97336015392367e-05
0.1775 0.000109369362893115
0.1825 0.000169109904238073
0.1875 0.000257746595168556
0.1925 0.000387169949609523
0.1975 0.000573105626212421
0.2025 0.000835870200910254
0.2075 0.00120107654392011
0.2125 0.0017001966437239
0.2175 0.00237087431188884
0.2225 0.00325687722538598
0.2275 0.00440759236330102
0.2325 0.00587700372294412
0.2375 0.00772214475261246
0.2425 0.0100010837355784
0.2475 0.0127705675659017
0.2525 0.0160835049890066
0.2575 0.0199865026149812
0.2625 0.024517668517973
0.2675 0.0297048683221924
0.2725 0.0355645634604718
0.2775 0.0421012918433731
0.2825 0.0493077803908931
0.2875 0.0571656182273979
0.2925 0.0656463765219179
0.2975 0.0747130389685672
0.3025 0.0843216044817968
0.3075 0.0944227366293171
0.3125 0.104963357117122
0.3175 0.115888107826333
0.3225 0.127140633030916
0.3275 0.138664657456851
0.3325 0.150404855140034
0.3375 0.162307518113764
0.3425 0.174321043116104
0.3475 0.186396259514057
0.3525 0.198486623435526
0.3575 0.210548302610127
0.3625 0.222540174456852
0.3675 0.234423757164425
0.3725 0.2461630903626
0.3775 0.257724578799727
0.3825 0.269076809419216
0.3875 0.280190349466021
0.3925 0.291037530789031
0.3975 0.301592223328621
0.4025 0.311829598863364
0.4075 0.321725884407147
0.4125 0.33125810319655
0.4175 0.340403800052193
0.4225 0.34914074723873
0.4275 0.357446627257301
0.4325 0.36529869126666
0.4375 0.37267339798658
0.4425 0.379546051649685
0.4475 0.385890485406314
0.4525 0.391678889528999
0.4575 0.396881977899246
0.4625 0.401469838716726
0.4675 0.405414024841958
0.4725 0.408691630863368
0.4775 0.411292024184161
0.4825 0.413226029159821
0.4875 0.414535185870344
0.4925 0.415295776454039
0.4975 0.415611531428894
0.5025 0.41559377314098
0.5075 0.415336270392398
0.5125 0.414895446220246
0.5175 0.41427973531504
0.5225 0.413441913351203
0.5275 0.412265556853831
0.5325 0.410543069713507
0.5375 0.407952684100831
0.5425 0.404050220477402
0.5475 0.39829414322411
0.5525 0.390116259686097
0.5575 0.379034706827778
0.5625 0.364785475907473
0.5675 0.347433283903082
0.5725 0.327421672581876
0.5775 0.305538966019578
0.5825 0.282804899447141
0.5875 0.260309643125283
0.5925 0.239050561287535
0.5975 0.219807541815168
0.6025 0.20307934526991
0.6075 0.189081107060286
0.6125 0.177786616825983
0.6175 0.168992999905332
0.6225 0.162388541129918
0.6275 0.15761173233599
0.6325 0.154296586879928
0.6375 0.152103711892583
0.6425 0.150738704005009
0.6475 0.149960167756696
0.6525 0.149579858547135
0.6575 0.149457421963541
0.6625 0.149491943424773
0.6675 0.1496120131206
0.6725 0.149765299386181
0.6775 0.149907790936039
0.6825 0.149991956492704
0.6875 0.149952035270804
0.6925 0.149683390951819
0.6975 0.149011228649705
0.7025 0.147642219487674
0.7075 0.145092233973356
0.7125 0.140589880124687
0.7175 0.132983873988083
0.7225 0.120763965024808
0.7275 0.10247138733323
0.7325 0.0779153045242086
0.7375 0.0500927744393442
0.7425 0.0255869633865413
0.7475 0.0101029770077908
0.7525 0.00323067177688417
0.7575 0.000916685479303123
0.7625 0.000247535672051547
0.7675 6.57015394421608e-05
0.7725 1.733476779243e-05
0.7775 4.56166541654529e-06
0.7825 1.19832317168437e-06
0.7875 3.14296210643595e-07
0.7925 8.22998342934205e-08
0.7975 2.15132336229347e-08
0.8025 5.61307170845078e-09
0.8075 1.46156535792315e-09
0.8125 3.79741250295264e-10
0.8175 9.8431384383556e-11
0.8225 2.54493088990962e-11
0.8275 6.56196969727746e-12
0.8325 1.68704996773948e-12
0.8375 4.3243891523643e-13
0.8425 1.10497770422186e-13
0.8475 2.8177175461474e-14
0.8525 7.14463267733202e-15
0.8575 1.78835622885588e-15
0.8625 4.08881035414389e-16
0.8675 5.23823250656242e-17
0.8725 2.23070011976512e-18
0.8775 -1.70270269849063e-18
0.8825 1.91820139524602e-19
0.8875 -2.40410875285755e-18
0.8925 -2.60371400131214e-18
0.8975 1.34734437976866e-19
0.9025 1.3517563413719e-19
0.9075 1.3561001001166e-19
0.9125 1.3561001001166e-19
0.9175 1.3561001001166e-19
0.9225 1.3561001001166e-19
0.9275 1.3561001001166e-19
0.9325 1.3561001001166e-19
0.9375 1.3561001001166e-19
0.9425 1.3561001001166e-19
0.9475 1.3561001001166e-19
0.9525 1.3561001001166e-19
0.9575 1.3561001001166e-19
0.9625 1.3561001001166e-19
0.9675 1.3561001001166e-19
0.9725 1.3561001001166e-19
0.9775 1.3561001001166e-19
0.9825 1.3561001001166e-19
0.9875 1.3561001001166e-19
0.9925 1.3561001001166e-19
0.9975 1.3561001001166e-19
};
\addlegendentry{truth}

\nextgroupplot[
ytick={0,0.1,0.2,0.4,0.5},
ylabel={\(E\)},
ymin=-0.0100275328732878, ymax=0.52428702249394,
y label style={xshift=.8em}
]
\addplot [semithick, color0, mark=pentagon, mark size=2, mark options={solid},mark repeat=5]
table {%
0.0025 0.499566434696068
0.0075 0.499566434696068
0.0125 0.499566434696068
0.0175 0.499566434696068
0.0225 0.499566434696068
0.0275 0.499566434696068
0.0325 0.499566434696068
0.0375 0.499566434696068
0.0425 0.499566434696068
0.0475 0.499566434696068
0.0525 0.499566434696068
0.0575 0.499566434696068
0.0625 0.499566434696068
0.0675 0.499566434696068
0.0725 0.499566434696068
0.0775 0.499566434696068
0.0825 0.499566434696068
0.0875 0.499566434696068
0.0925 0.499566434696068
0.0975 0.499566370348632
0.1025 0.499567510892607
0.1075 0.499567527370009
0.1125 0.499568461188917
0.1175 0.499565495813832
0.1225 0.499567820889932
0.1275 0.499567439440812
0.1325 0.499567908092993
0.1375 0.499566768757936
0.1425 0.499564687891228
0.1475 0.499561228374623
0.1525 0.499559679675419
0.1575 0.499553906343711
0.1625 0.499542880865482
0.1675 0.499527347451757
0.1725 0.499503439223567
0.1775 0.499467262862731
0.1825 0.499412237656599
0.1875 0.499332065979628
0.1925 0.499212074995381
0.1975 0.499043806748767
0.2025 0.498802532356096
0.2075 0.498472301059849
0.2125 0.498020814721587
0.2175 0.497416882395611
0.2225 0.49662143605863
0.2275 0.495597837974689
0.2325 0.494300632658181
0.2375 0.492685228301581
0.2425 0.490716071969207
0.2475 0.488356221513139
0.2525 0.485586777843386
0.2575 0.482366202590156
0.2625 0.478636349469023
0.2675 0.474370902309825
0.2725 0.469563017550409
0.2775 0.464211038835019
0.2825 0.458326442635222
0.2875 0.451927760843783
0.2925 0.445051297040587
0.2975 0.437740810869383
0.3025 0.430049988082055
0.3075 0.422041759481815
0.3125 0.41377940822452
0.3175 0.405341573274708
0.3225 0.396805402694221
0.3275 0.388252174850362
0.3325 0.379800179743692
0.3375 0.371559981380593
0.3425 0.363620787652613
0.3475 0.356180819505493
0.3525 0.349377926001172
0.3575 0.3432583843994
0.3625 0.33782942289075
0.3675 0.332347496805064
0.3725 0.326394980320559
0.3775 0.320012823852642
0.3825 0.313236933914264
0.3875 0.306129853124111
0.3925 0.298745550819665
0.3975 0.291790480170699
0.4025 0.285852767651042
0.4075 0.281006252061158
0.4125 0.277299036681199
0.4175 0.274759348070896
0.4225 0.273400023151207
0.4275 0.272964188700684
0.4325 0.271815780753738
0.4375 0.269828438346021
0.4425 0.267249841352473
0.4475 0.264350149833369
0.4525 0.261423391158425
0.4575 0.258766695461318
0.4625 0.256654295660602
0.4675 0.25529988899042
0.4725 0.254513976549939
0.4775 0.254076380414675
0.4825 0.253857447711592
0.4875 0.253797470865467
0.4925 0.253808752077065
0.4975 0.253763714009828
0.5025 0.253699368179262
0.5075 0.25362855235414
0.5125 0.253532433385479
0.5175 0.253393969678747
0.5225 0.253173501656991
0.5275 0.252807950506907
0.5325 0.252185353063773
0.5375 0.251113538453846
0.5425 0.249354667481387
0.5475 0.246752242224898
0.5525 0.243418878017469
0.5575 0.239465106067682
0.5625 0.234279419366452
0.5675 0.227783804517479
0.5725 0.220192690876746
0.5775 0.211972309824358
0.5825 0.203765609306105
0.5875 0.196264027025048
0.5925 0.190010807978054
0.5975 0.184577986106031
0.6025 0.179033526307619
0.6075 0.173685600452327
0.6125 0.168782162839598
0.6175 0.164524354824771
0.6225 0.16104175884175
0.6275 0.158359583934681
0.6325 0.156417219529006
0.6375 0.15509608100915
0.6425 0.154262423100925
0.6475 0.153786435607038
0.6525 0.153558933343705
0.6575 0.153494577775065
0.6625 0.15352806838995
0.6675 0.153617849069447
0.6725 0.153725001225156
0.6775 0.15382214803754
0.6825 0.1538752986271
0.6875 0.153841953133464
0.6925 0.153640439458696
0.6975 0.153146652727694
0.7025 0.152149559024472
0.7075 0.150290444535956
0.7125 0.146908177847255
0.7175 0.140638036539332
0.7225 0.129557863401295
0.7275 0.11102893867151
0.7325 0.0820449516669314
0.7375 0.046488003712627
0.7425 0.0207998444859785
0.7475 0.0142594923706771
0.7525 0.014927622493576
0.7575 0.0156025802899868
0.7625 0.0158366322815483
0.7675 0.0158999766958207
0.7725 0.0159181611108628
0.7775 0.0159240230115524
0.7825 0.0159260359236937
0.7875 0.0159249571562401
0.7925 0.0159248001028088
0.7975 0.0159241176643776
0.8025 0.0159246474218944
0.8075 0.0159247248240804
0.8125 0.0159247248240804
0.8175 0.0159247248240804
0.8225 0.0159247248240804
0.8275 0.0159247248240804
0.8325 0.0159247248240804
0.8375 0.0159247248240804
0.8425 0.0159247248240804
0.8475 0.0159247248240804
0.8525 0.0159247248240804
0.8575 0.0159247248240804
0.8625 0.0159247248240804
0.8675 0.0159247248240804
0.8725 0.0159247248240804
0.8775 0.0159247248240804
0.8825 0.0159247248240804
0.8875 0.0159247248240804
0.8925 0.0159247248240804
0.8975 0.0159247248240804
0.9025 0.0159247248240804
0.9075 0.0159247248240804
0.9125 0.0159247248240804
0.9175 0.0159247248240804
0.9225 0.0159247248240804
0.9275 0.0159247248240804
0.9325 0.0159247248240804
0.9375 0.0159247248240804
0.9425 0.0159247248240804
0.9475 0.0159247248240804
0.9525 0.0159247248240804
0.9575 0.0159247248240804
0.9625 0.0159247248240804
0.9675 0.0159247248240804
0.9725 0.0159247248240804
0.9775 0.0159247248240804
0.9825 0.0159247248240804
0.9875 0.0159247248240804
0.9925 0.0159247248240804
0.9975 0.0159247248240804
};
\addlegendentry{prediction}
\addplot [semithick, black, mark=+, mark size=2, mark options={solid},dashed,mark repeat=5]
table {%
0.0025 0.499999997249975
0.0075 0.499999997249952
0.0125 0.49999999724991
0.0175 0.49999999724983
0.0225 0.499999997249673
0.0275 0.499999997249343
0.0325 0.499999997248633
0.0375 0.499999997247108
0.0425 0.499999997243849
0.0475 0.499999997236918
0.0525 0.49999999722231
0.0575 0.499999997191854
0.0625 0.499999997129034
0.0675 0.499999997000896
0.0725 0.499999996742452
0.0775 0.499999996227163
0.0825 0.499999995211631
0.0875 0.499999993233655
0.0925 0.499999989426775
0.0975 0.499999982187882
0.1025 0.499999968590229
0.1075 0.499999943362711
0.1125 0.49999989714242
0.1175 0.499999813531151
0.1225 0.499999664219035
0.1275 0.499999401043895
0.1325 0.499998943288962
0.1375 0.499998157735899
0.1425 0.499996827939107
0.1475 0.499994607841606
0.1525 0.499990953221392
0.1575 0.499985022618125
0.1625 0.499975537529369
0.1675 0.499960590121808
0.1725 0.499937386002681
0.1775 0.499901910468537
0.1825 0.499848509989294
0.1875 0.499769387444734
0.1925 0.499654020597409
0.1975 0.499488528760043
0.2025 0.499255032002629
0.2075 0.498931068658278
0.2125 0.498489156918177
0.2175 0.497896600156329
0.2225 0.497115637784343
0.2275 0.496104028964957
0.2325 0.494816122733982
0.2375 0.493204416224345
0.2425 0.491221538805241
0.2475 0.488822534483274
0.2525 0.48596726066466
0.2575 0.482622690300756
0.2625 0.478764904226945
0.2675 0.474380592224794
0.2725 0.469467939022164
0.2775 0.46403684380784
0.2825 0.458108495428486
0.2875 0.451714388097888
0.2925 0.444894905918178
0.2975 0.437697625542276
0.3025 0.430175486289908
0.3075 0.422384960603958
0.3125 0.414384331017942
0.3175 0.406232148833711
0.3225 0.397985919383409
0.3275 0.389701032374154
0.3325 0.381429935098179
0.3375 0.373221531647741
0.3425 0.365120782228682
0.3475 0.357168472254166
0.3525 0.349401119983176
0.3575 0.341850992947928
0.3625 0.334546206351425
0.3675 0.327510880277021
0.3725 0.320765336408751
0.3775 0.314326318665062
0.3825 0.308207225493483
0.3875 0.302418344457086
0.3925 0.296967082133645
0.3975 0.291858184256781
0.4025 0.287093942489105
0.4075 0.28267438527183
0.4125 0.278597450881622
0.4175 0.27485914117029
0.4225 0.271453654478241
0.4275 0.268373495894285
0.4325 0.265609562373013
0.4375 0.263151199234399
0.4425 0.260986223393296
0.4475 0.25910090775627
0.4525 0.257479921790285
0.4575 0.256106228028066
0.4625 0.254960948347607
0.4675 0.254023244450158
0.4725 0.253270307401023
0.4775 0.252677601550096
0.4825 0.252219485718782
0.4875 0.251870132202037
0.4925 0.251604330194165
0.4975 0.251397729084703
0.5025 0.25122669129834
0.5075 0.251068332781992
0.5125 0.250900333775037
0.5175 0.250698706378175
0.5225 0.250431897599141
0.5275 0.250051148444404
0.5325 0.249478662542547
0.5375 0.248596912673306
0.5425 0.247244381726547
0.5475 0.245223917214396
0.5525 0.242327893303772
0.5575 0.238378924611544
0.5625 0.233277606118821
0.5675 0.227043160220145
0.5725 0.219832439446087
0.5775 0.211928643264648
0.5825 0.203701164573315
0.5875 0.195547680233291
0.5925 0.187834659876278
0.5975 0.18085111996356
0.6025 0.174784098494021
0.6075 0.169716410533201
0.6125 0.165641200169253
0.6175 0.162485367499619
0.6225 0.160134752274621
0.6275 0.158456399070025
0.6325 0.157315740975606
0.6375 0.156588304507358
0.6425 0.156166477645325
0.6475 0.155962258709567
0.6525 0.155906982680582
0.6575 0.155948945374948
0.6625 0.156049656989133
0.6675 0.156179160619016
0.6725 0.156310449279297
0.6775 0.156412496184095
0.6825 0.156440733162282
0.6875 0.156322878793174
0.6925 0.155936716265043
0.6975 0.155074755703434
0.7025 0.15338930096658
0.7075 0.150312941539415
0.7125 0.144962247510296
0.7175 0.136077279136682
0.7225 0.122162286306937
0.7275 0.102173607088272
0.7325 0.0770700772857907
0.7375 0.0513960997098126
0.7425 0.0317715372289992
0.7475 0.0212761317050686
0.7525 0.0172929078157419
0.7575 0.0160879851700508
0.7625 0.0157570212643047
0.7675 0.0156687593965459
0.7725 0.0156454185249256
0.7775 0.0156392644262287
0.7825 0.015637644580833
0.7875 0.0156372188316351
0.7925 0.0156371070941755
0.7975 0.0156370778143173
0.8025 0.0156370701545981
0.8075 0.0156370681544042
0.8125 0.0156370676331101
0.8175 0.0156370674975363
0.8225 0.0156370674623578
0.8275 0.0156370674532522
0.8325 0.0156370674509016
0.8375 0.0156370674502965
0.8425 0.0156370674501412
0.8475 0.0156370674501014
0.8525 0.0156370674500913
0.8575 0.0156370674500887
0.8625 0.015637067450088
0.8675 0.0156370674500878
0.8725 0.0156370674500877
0.8775 0.0156370674500877
0.8825 0.0156370674500877
0.8875 0.0156370674500877
0.8925 0.0156370674500877
0.8975 0.0156370674500877
0.9025 0.0156370674500877
0.9075 0.0156370674500877
0.9125 0.0156370674500877
0.9175 0.0156370674500877
0.9225 0.0156370674500877
0.9275 0.0156370674500877
0.9325 0.0156370674500877
0.9375 0.0156370674500877
0.9425 0.0156370674500877
0.9475 0.0156370674500877
0.9525 0.0156370674500877
0.9575 0.0156370674500877
0.9625 0.0156370674500877
0.9675 0.0156370674500877
0.9725 0.0156370674500877
0.9775 0.0156370674500877
0.9825 0.0156370674500877
0.9875 0.0156370674500877
0.9925 0.0156370674500877
0.9975 0.0156370674500877
};
\addlegendentry{truth}
\end{groupplot}

\end{tikzpicture}

	\caption{Interpoaltionin in time for \(\hy\) from 25 snapshots to 241 snapshots with cubic splines using the FCNN.}
\end{figure}
\begin{figure}[H]
	\scalebox{1}{% This file was created by tikzplotlib v0.9.6.
\begin{tikzpicture}

\begin{groupplot}[group style={group size=3 by 1,horizontal sep=2cm, vertical sep=2cm}]
\nextgroupplot[
legend cell align={left},
legend style={draw=none},
tick align=outside,
tick pos=left,
x grid style={white!69.0196078431373!black},
xlabel={\(x\)},
xmin=-0.04725, xmax=1.04725,
xtick style={color=black},
y grid style={white!69.0196078431373!black},
ylabel={\(c_0\) and \(\rho\)},
ymin=-0.0600705937195856, ymax=1.05041666166284,
ytick style={color=black},
axis lines=left
]
\addplot [semithick, black, dashed]
table {%
0.0025 0.532396674156189
0.0075 0.532396674156189
0.0125 0.532396674156189
0.0175 0.532396674156189
0.0225 0.532396674156189
0.0275 0.532396674156189
0.0325 0.532396674156189
0.0375 0.532396674156189
0.0425 0.532396674156189
0.0475 0.532396674156189
0.0525 0.532396674156189
0.0575 0.532396674156189
0.0625 0.532396674156189
0.0675 0.532396674156189
0.0725 0.532396674156189
0.0775 0.532396674156189
0.0825 0.532396674156189
0.0875 0.532396674156189
0.0925 0.532396674156189
0.0975 0.532396674156189
0.1025 0.532396674156189
0.1075 0.532396674156189
0.1125 0.532396674156189
0.1175 0.532396674156189
0.1225 0.532396674156189
0.1275 0.532396674156189
0.1325 0.532396674156189
0.1375 0.532396674156189
0.1425 0.532396674156189
0.1475 0.532396674156189
0.1525 0.532396674156189
0.1575 0.532396674156189
0.1625 0.532396674156189
0.1675 0.532396674156189
0.1725 0.532396674156189
0.1775 0.532396674156189
0.1825 0.532396674156189
0.1875 0.532396674156189
0.1925 0.532396674156189
0.1975 0.532396674156189
0.2025 0.532396674156189
0.2075 0.532396674156189
0.2125 0.532396674156189
0.2175 0.532396674156189
0.2225 0.532396674156189
0.2275 0.532396674156189
0.2325 0.532396674156189
0.2375 0.532396674156189
0.2425 0.532396674156189
0.2475 0.532396674156189
0.2525 0.532396674156189
0.2575 0.532396674156189
0.2625 0.532396674156189
0.2675 0.532396674156189
0.2725 0.532396674156189
0.2775 0.532396674156189
0.2825 0.532396614551544
0.2875 0.532396554946899
0.2925 0.532396554946899
0.2975 0.532396376132965
0.3025 0.532396078109741
0.3075 0.532395303249359
0.3125 0.53239369392395
0.3175 0.532390356063843
0.3225 0.532383680343628
0.3275 0.532370090484619
0.3325 0.532343864440918
0.3375 0.532294631004333
0.3425 0.532204687595367
0.3475 0.532045602798462
0.3525 0.531773149967194
0.3575 0.531321406364441
0.3625 0.530597150325775
0.3675 0.529474556446075
0.3725 0.527792632579803
0.3775 0.525356471538544
0.3825 0.521943271160126
0.3875 0.517312824726105
0.3925 0.511222898960114
0.3975 0.503444910049438
0.4025 0.493780314922333
0.4075 0.482075095176697
0.4125 0.468230247497559
0.4175 0.452209562063217
0.4225 0.43382865190506
0.4275 0.413315415382385
0.4325 0.3905388712883
0.4375 0.365876168012619
0.4425 0.339789062738419
0.4475 0.312636196613312
0.4525 0.284825265407562
0.4575 0.256570726633072
0.4625 0.228063091635704
0.4675 0.200090900063515
0.4725 0.173177376389503
0.4775 0.147900074720383
0.4825 0.124915175139904
0.4875 0.106026835739613
0.4925 0.0917501226067543
0.4975 0.0818334966897964
0.5025 0.0766019970178604
0.5075 0.0748537853360176
0.5125 0.0742902159690857
0.5175 0.073918804526329
0.5225 0.0725244954228401
0.5275 0.0700906664133072
0.5325 0.066360130906105
0.5375 0.0614439621567726
0.5425 0.0554116740822792
0.5475 0.0496930554509163
0.5525 0.0449490025639534
0.5575 0.041465163230896
0.5625 0.0391549542546272
0.5675 0.0376920029520988
0.5725 0.0366387143731117
0.5775 0.0355018638074398
0.5825 0.0337106138467789
0.5875 0.0305485334247351
0.5925 0.0251204036176205
0.5975 0.016600364819169
0.6025 0.00535319838672876
0.6075 -0.00505704572424293
0.6125 -0.0092976912856102
0.6175 -0.0095939002931118
0.6225 -0.00928737316280603
0.6275 -0.0091561870649457
0.6325 -0.00911904219537973
0.6375 -0.00910959485918283
0.6425 -0.00910726375877857
0.6475 -0.0091067124158144
0.6525 -0.00910657085478306
0.6575 -0.00910652615129948
0.6625 -0.00910654105246067
0.6675 -0.00910652615129948
0.6725 -0.00910652615129948
0.6775 -0.00910652615129948
0.6825 -0.00910652615129948
0.6875 -0.00910652615129948
0.6925 -0.00910652615129948
0.6975 -0.00910652615129948
0.7025 -0.00910652615129948
0.7075 -0.00910652615129948
0.7125 -0.00910652615129948
0.7175 -0.00910652615129948
0.7225 -0.00910652615129948
0.7275 -0.00910652615129948
0.7325 -0.00910652615129948
0.7375 -0.00910652615129948
0.7425 -0.00910652615129948
0.7475 -0.00910652615129948
0.7525 -0.00910652615129948
0.7575 -0.00910652615129948
0.7625 -0.00910652615129948
0.7675 -0.00910652615129948
0.7725 -0.00910652615129948
0.7775 -0.00910652615129948
0.7825 -0.00910652615129948
0.7875 -0.00910652615129948
0.7925 -0.00910652615129948
0.7975 -0.00910652615129948
0.8025 -0.00910652615129948
0.8075 -0.00910652615129948
0.8125 -0.00910652615129948
0.8175 -0.00910652615129948
0.8225 -0.00910652615129948
0.8275 -0.00910652615129948
0.8325 -0.00910652615129948
0.8375 -0.00910652615129948
0.8425 -0.00910652615129948
0.8475 -0.00910652615129948
0.8525 -0.00910652615129948
0.8575 -0.00910652615129948
0.8625 -0.00910652615129948
0.8675 -0.00910652615129948
0.8725 -0.00910652615129948
0.8775 -0.00910652615129948
0.8825 -0.00910652615129948
0.8875 -0.00910652615129948
0.8925 -0.00910652615129948
0.8975 -0.00910652615129948
0.9025 -0.00910652615129948
0.9075 -0.00910652615129948
0.9125 -0.00910652615129948
0.9175 -0.00910652615129948
0.9225 -0.00910652615129948
0.9275 -0.00910652615129948
0.9325 -0.00910652615129948
0.9375 -0.00910652615129948
0.9425 -0.00910652615129948
0.9475 -0.00910652615129948
0.9525 -0.00910652615129948
0.9575 -0.00910652615129948
0.9625 -0.00910652615129948
0.9675 -0.00910652615129948
0.9725 -0.00910652615129948
0.9775 -0.00910652615129948
0.9825 -0.00910652615129948
0.9875 -0.00910652615129948
0.9925 -0.00910652615129948
0.9975 -0.00910652615129948
};
\addlegendentry{\(c_0\)}
\addplot [semithick, black]
table {%
0.0025 0.999939968236364
0.0075 0.999939968236364
0.0125 0.999939968236364
0.0175 0.999939968236364
0.0225 0.999939968236364
0.0275 0.999939968236364
0.0325 0.999939968236364
0.0375 0.999939968236364
0.0425 0.999939968236364
0.0475 0.999939968236364
0.0525 0.999939968236364
0.0575 0.999939968236364
0.0625 0.999939968236364
0.0675 0.999939968236364
0.0725 0.999939968236364
0.0775 0.999939968236364
0.0825 0.999939968236364
0.0875 0.999939968236364
0.0925 0.999939968236364
0.0975 0.999939968236364
0.1025 0.999939968236364
0.1075 0.999939968236364
0.1125 0.999939968236364
0.1175 0.999939968236364
0.1225 0.999939968236364
0.1275 0.999939968236364
0.1325 0.999939968236364
0.1375 0.999939968236364
0.1425 0.999939968236364
0.1475 0.999939968236364
0.1525 0.999939968236364
0.1575 0.999939968236364
0.1625 0.999939968236364
0.1675 0.999939968236364
0.1725 0.999939968236364
0.1775 0.999939968236364
0.1825 0.999939968236364
0.1875 0.999939968236364
0.1925 0.999939968236364
0.1975 0.999939968236364
0.2025 0.999939968236364
0.2075 0.999939968236364
0.2125 0.999939968236364
0.2175 0.999939968236364
0.2225 0.999939968236364
0.2275 0.999939968236364
0.2325 0.999939968236364
0.2375 0.999939968236364
0.2425 0.999939968236364
0.2475 0.999939968236364
0.2525 0.999939968236364
0.2575 0.999939968236364
0.2625 0.999939968236364
0.2675 0.999939968236364
0.2725 0.999939968236364
0.2775 0.999939968236364
0.2825 0.999939891820153
0.2875 0.999939795822287
0.2925 0.999939859343263
0.2975 0.999939768599012
0.3025 0.999939568006457
0.3075 0.999939004317499
0.3125 0.999937914670087
0.3175 0.999935610124316
0.3225 0.999930894050078
0.3275 0.999921475513241
0.3325 0.999903148159576
0.3375 0.999868639553778
0.3425 0.999805401198757
0.3475 0.999693513179246
0.3525 0.999501534403326
0.3575 0.999182681433665
0.3625 0.998670157785408
0.3675 0.997873304220728
0.3725 0.996675169787919
0.3775 0.994931922341959
0.3825 0.992476384585293
0.3875 0.989122819633056
0.3925 0.984677037057013
0.3975 0.978946031716007
0.4025 0.971748285974638
0.4075 0.962925586276329
0.4125 0.952351180573878
0.4175 0.939938538134671
0.4225 0.926414078388076
0.4275 0.911102869118062
0.4325 0.894411588565279
0.4375 0.87616816521264
0.4425 0.856281408252051
0.4475 0.834931061627009
0.4525 0.812339105476171
0.4575 0.788944653975658
0.4625 0.764412948240836
0.4675 0.73869173713506
0.4725 0.711952378758444
0.4775 0.684410636790861
0.4825 0.65641625963438
0.4875 0.62931059835813
0.4925 0.604090741477334
0.4975 0.582040463789151
0.5025 0.565329562180126
0.5075 0.552076098556893
0.5125 0.535170405697173
0.5175 0.509202869561238
0.5225 0.474026685771652
0.5275 0.429425106670421
0.5325 0.382033329671965
0.5375 0.335415347168843
0.5425 0.294759331318812
0.5475 0.262096491642296
0.5525 0.238131975086454
0.5575 0.221985514060809
0.5625 0.211974344073007
0.5675 0.206237247404762
0.5725 0.203080492524
0.5775 0.201084886868604
0.5825 0.199046708667316
0.5875 0.195821415609083
0.5925 0.190153305227749
0.5975 0.180685935088266
0.6025 0.166637054405724
0.6075 0.149738479596682
0.6125 0.136586853470176
0.6175 0.12911991096842
0.6225 0.126236281118905
0.6275 0.125364558771253
0.6325 0.125054053198069
0.6375 0.124968802198195
0.6425 0.124947830246618
0.6475 0.12494275226998
0.6525 0.124941546326646
0.6575 0.124941325913637
0.6625 0.124941193856872
0.6675 0.124941221796549
0.6725 0.124941221796549
0.6775 0.124941221796549
0.6825 0.124941221796549
0.6875 0.124941221796549
0.6925 0.124941221796549
0.6975 0.124941221796549
0.7025 0.124941221796549
0.7075 0.124941221796549
0.7125 0.124941221796549
0.7175 0.124941221796549
0.7225 0.124941221796549
0.7275 0.124941221796549
0.7325 0.124941221796549
0.7375 0.124941221796549
0.7425 0.124941221796549
0.7475 0.124941221796549
0.7525 0.124941221796549
0.7575 0.124941221796549
0.7625 0.124941221796549
0.7675 0.124941221796549
0.7725 0.124941221796549
0.7775 0.124941221796549
0.7825 0.124941221796549
0.7875 0.124941221796549
0.7925 0.124941221796549
0.7975 0.124941221796549
0.8025 0.124941221796549
0.8075 0.124941221796549
0.8125 0.124941221796549
0.8175 0.124941221796549
0.8225 0.124941221796549
0.8275 0.124941221796549
0.8325 0.124941221796549
0.8375 0.124941221796549
0.8425 0.124941221796549
0.8475 0.124941221796549
0.8525 0.124941221796549
0.8575 0.124941221796549
0.8625 0.124941221796549
0.8675 0.124941221796549
0.8725 0.124941221796549
0.8775 0.124941221796549
0.8825 0.124941221796549
0.8875 0.124941221796549
0.8925 0.124941221796549
0.8975 0.124941221796549
0.9025 0.124941221796549
0.9075 0.124941221796549
0.9125 0.124941221796549
0.9175 0.124941221796549
0.9225 0.124941221796549
0.9275 0.124941221796549
0.9325 0.124941221796549
0.9375 0.124941221796549
0.9425 0.124941221796549
0.9475 0.124941221796549
0.9525 0.124941221796549
0.9575 0.124941221796549
0.9625 0.124941221796549
0.9675 0.124941221796549
0.9725 0.124941221796549
0.9775 0.124941221796549
0.9825 0.124941221796549
0.9875 0.124941221796549
0.9925 0.124941221796549
0.9975 0.124941221796549
};
\addlegendentry{\(\rho\)}

\nextgroupplot[
legend cell align={left},
legend style={draw=none},
tick align=outside,
tick pos=left,
x grid style={white!69.0196078431373!black},
xlabel={\(x\)},
xmin=-0.04725, xmax=1.04725,
xtick style={color=black},
y grid style={white!69.0196078431373!black},
ylabel={\(c_1\) and \(\rho u\)},
ymin=-0.597515286427337, ymax=0.452524393651436,
ytick style={color=black},
axis lines=left
]
\addplot [semithick, black, dashed]
table {%
0.0025 -0.432732731103897
0.0075 -0.432732731103897
0.0125 -0.432732731103897
0.0175 -0.432732731103897
0.0225 -0.432732731103897
0.0275 -0.432732731103897
0.0325 -0.432732731103897
0.0375 -0.432732731103897
0.0425 -0.432732731103897
0.0475 -0.432732731103897
0.0525 -0.432732731103897
0.0575 -0.432732731103897
0.0625 -0.432732731103897
0.0675 -0.432732731103897
0.0725 -0.432732731103897
0.0775 -0.432732731103897
0.0825 -0.432732731103897
0.0875 -0.432732731103897
0.0925 -0.432732731103897
0.0975 -0.432732731103897
0.1025 -0.432732731103897
0.1075 -0.432732731103897
0.1125 -0.432732731103897
0.1175 -0.432732731103897
0.1225 -0.432732731103897
0.1275 -0.432732731103897
0.1325 -0.432732731103897
0.1375 -0.432732731103897
0.1425 -0.432732731103897
0.1475 -0.432732731103897
0.1525 -0.432732731103897
0.1575 -0.432732731103897
0.1625 -0.432732731103897
0.1675 -0.432732731103897
0.1725 -0.432732731103897
0.1775 -0.432732731103897
0.1825 -0.432732731103897
0.1875 -0.432732731103897
0.1925 -0.432732731103897
0.1975 -0.432732731103897
0.2025 -0.432732731103897
0.2075 -0.432732731103897
0.2125 -0.432732731103897
0.2175 -0.432732731103897
0.2225 -0.432732731103897
0.2275 -0.432732731103897
0.2325 -0.432732731103897
0.2375 -0.432732731103897
0.2425 -0.432732731103897
0.2475 -0.432732731103897
0.2525 -0.432732731103897
0.2575 -0.432732731103897
0.2625 -0.432732731103897
0.2675 -0.432732731103897
0.2725 -0.432732731103897
0.2775 -0.432732731103897
0.2825 -0.432732731103897
0.2875 -0.432732731103897
0.2925 -0.432732790708542
0.2975 -0.432732880115509
0.3025 -0.432733088731766
0.3075 -0.432733476161957
0.3125 -0.432734400033951
0.3175 -0.432736217975616
0.3225 -0.43274000287056
0.3275 -0.432747662067413
0.3325 -0.432762205600739
0.3375 -0.432789623737335
0.3425 -0.432839512825012
0.3475 -0.432927429676056
0.3525 -0.4330775141716
0.3575 -0.433325350284576
0.3625 -0.433720856904984
0.3675 -0.434330135583878
0.3725 -0.435236155986786
0.3775 -0.436536222696304
0.3825 -0.438336074352264
0.3875 -0.440741539001465
0.3925 -0.443847119808197
0.3975 -0.447724401950836
0.4025 -0.452411115169525
0.4075 -0.457903206348419
0.4125 -0.464150786399841
0.4175 -0.471056699752808
0.4225 -0.479671806097031
0.4275 -0.489005982875824
0.4325 -0.499341070652008
0.4375 -0.509970843791962
0.4425 -0.52016693353653
0.4475 -0.529586613178253
0.4525 -0.537876844406128
0.4575 -0.544366419315338
0.4625 -0.548331916332245
0.4675 -0.54978621006012
0.4725 -0.548344135284424
0.4775 -0.543678402900696
0.4825 -0.535618782043457
0.4875 -0.523471593856812
0.4925 -0.508366823196411
0.4975 -0.492929041385651
0.5025 -0.479555636644363
0.5075 -0.467656821012497
0.5125 -0.452263534069061
0.5175 -0.429323792457581
0.5225 -0.400096535682678
0.5275 -0.363904923200607
0.5325 -0.32466995716095
0.5375 -0.287306934595108
0.5425 -0.257477670907974
0.5475 -0.234640568494797
0.5525 -0.218610465526581
0.5575 -0.208207130432129
0.5625 -0.201991692185402
0.5675 -0.198689639568329
0.5725 -0.197360694408417
0.5775 -0.197444394230843
0.5825 -0.198761835694313
0.5875 -0.201499730348587
0.5925 -0.206139147281647
0.5975 -0.213125392794609
0.6025 -0.221724510192871
0.6075 -0.228136301040649
0.6125 -0.228484690189362
0.6175 -0.225195422768593
0.6225 -0.223013624548912
0.6275 -0.22223174571991
0.6325 -0.22201269865036
0.6375 -0.221956744790077
0.6425 -0.221942976117134
0.6475 -0.221939638257027
0.6525 -0.221938803792
0.6575 -0.221938639879227
0.6625 -0.221938580274582
0.6675 -0.221938580274582
0.6725 -0.221938580274582
0.6775 -0.221938580274582
0.6825 -0.221938580274582
0.6875 -0.221938580274582
0.6925 -0.221938580274582
0.6975 -0.221938580274582
0.7025 -0.221938580274582
0.7075 -0.221938580274582
0.7125 -0.221938580274582
0.7175 -0.221938580274582
0.7225 -0.221938580274582
0.7275 -0.221938580274582
0.7325 -0.221938580274582
0.7375 -0.221938580274582
0.7425 -0.221938580274582
0.7475 -0.221938580274582
0.7525 -0.221938580274582
0.7575 -0.221938580274582
0.7625 -0.221938580274582
0.7675 -0.221938580274582
0.7725 -0.221938580274582
0.7775 -0.221938580274582
0.7825 -0.221938580274582
0.7875 -0.221938580274582
0.7925 -0.221938580274582
0.7975 -0.221938580274582
0.8025 -0.221938580274582
0.8075 -0.221938580274582
0.8125 -0.221938580274582
0.8175 -0.221938580274582
0.8225 -0.221938580274582
0.8275 -0.221938580274582
0.8325 -0.221938580274582
0.8375 -0.221938580274582
0.8425 -0.221938580274582
0.8475 -0.221938580274582
0.8525 -0.221938580274582
0.8575 -0.221938580274582
0.8625 -0.221938580274582
0.8675 -0.221938580274582
0.8725 -0.221938580274582
0.8775 -0.221938580274582
0.8825 -0.221938580274582
0.8875 -0.221938580274582
0.8925 -0.221938580274582
0.8975 -0.221938580274582
0.9025 -0.221938580274582
0.9075 -0.221938580274582
0.9125 -0.221938580274582
0.9175 -0.221938580274582
0.9225 -0.221938580274582
0.9275 -0.221938580274582
0.9325 -0.221938580274582
0.9375 -0.221938580274582
0.9425 -0.221938580274582
0.9475 -0.221938580274582
0.9525 -0.221938580274582
0.9575 -0.221938580274582
0.9625 -0.221938580274582
0.9675 -0.221938580274582
0.9725 -0.221938580274582
0.9775 -0.221938580274582
0.9825 -0.221938580274582
0.9875 -0.221938580274582
0.9925 -0.221938580274582
0.9975 -0.221938580274582
};
\addlegendentry{\(c_1\)}
\addplot [semithick, black]
table {%
0.0025 -0.000632143126926788
0.0075 -0.000632143126926788
0.0125 -0.000632143126926788
0.0175 -0.000632143126926788
0.0225 -0.000632143126926788
0.0275 -0.000632143126926788
0.0325 -0.000632143126926788
0.0375 -0.000632143126926788
0.0425 -0.000632143126926788
0.0475 -0.000632143126926788
0.0525 -0.000632143126926788
0.0575 -0.000632143126926788
0.0625 -0.000632143126926788
0.0675 -0.000632143126926788
0.0725 -0.000632143126926788
0.0775 -0.000632143126926788
0.0825 -0.000632143126926788
0.0875 -0.000632143126926788
0.0925 -0.000632143126926788
0.0975 -0.000632143126926788
0.1025 -0.000632143126926788
0.1075 -0.000632143126926788
0.1125 -0.000632143126926788
0.1175 -0.000632143126926788
0.1225 -0.000632143126926788
0.1275 -0.000632143126926788
0.1325 -0.000632143126926788
0.1375 -0.000632143126926788
0.1425 -0.000632143126926788
0.1475 -0.000632143126926788
0.1525 -0.000632143126926788
0.1575 -0.000632143126926788
0.1625 -0.000632143126926788
0.1675 -0.000632143126926788
0.1725 -0.000632143126926788
0.1775 -0.000632143126926788
0.1825 -0.000632143126926788
0.1875 -0.000632143126926788
0.1925 -0.000632143126926788
0.1975 -0.000632143126926788
0.2025 -0.000632143126926788
0.2075 -0.000632143126926788
0.2125 -0.000632143126926788
0.2175 -0.000632143126926788
0.2225 -0.000632143126926788
0.2275 -0.000632143126926788
0.2325 -0.000632143126926788
0.2375 -0.000632143126926788
0.2425 -0.000632143126926788
0.2475 -0.000632143126926788
0.2525 -0.000632143126926788
0.2575 -0.000632143126926788
0.2625 -0.000632143126926788
0.2675 -0.000632143126926788
0.2725 -0.000632143126926788
0.2775 -0.000632143126926788
0.2825 -0.000632173375010415
0.2875 -0.000632007194243283
0.2925 -0.00063202397152044
0.2975 -0.000631875302801798
0.3025 -0.00063135079858255
0.3075 -0.000630262632958971
0.3125 -0.000628540512813578
0.3175 -0.000624436478545078
0.3225 -0.000615503068310432
0.3275 -0.000598422146936829
0.3325 -0.000565411660141781
0.3375 -0.000503633072125836
0.3425 -0.000390692391750638
0.3475 -0.000190644761955293
0.3525 0.000152100743401347
0.3575 0.000719298081402627
0.3625 0.0016283939171884
0.3675 0.0030357892123553
0.3725 0.0051411870195923
0.3775 0.00818680967574724
0.3825 0.0124447812790995
0.3875 0.0182065823746699
0.3925 0.0257615754068545
0.3975 0.0353748145966958
0.4025 0.0472672618276523
0.4075 0.0615958609348218
0.4125 0.0784439155078793
0.4175 0.0978072289313771
0.4225 0.119432971373759
0.4275 0.139301568471792
0.4325 0.161420454177454
0.4375 0.185197899467082
0.4425 0.209965857901879
0.4475 0.235314556396885
0.4525 0.26079012848863
0.4575 0.284682962747094
0.4625 0.306761641493759
0.4675 0.328102108626274
0.4725 0.348191657657738
0.4775 0.366437134843008
0.4825 0.382161402785851
0.4875 0.393995357938184
0.4925 0.401444049376963
0.4975 0.404795317284219
0.5025 0.404355673481913
0.5075 0.40062380593973
0.5125 0.392499756843175
0.5175 0.374914746655593
0.5225 0.349303765711053
0.5275 0.316143827329168
0.5325 0.284308368550769
0.5375 0.252408394559405
0.5425 0.221984836173089
0.5475 0.196588067901195
0.5525 0.17733245072819
0.5575 0.164008134440871
0.5625 0.155525898957658
0.5675 0.150401894747721
0.5725 0.147078615502454
0.5775 0.144024359437047
0.5825 0.139635926012114
0.5875 0.132013712461639
0.5925 0.118766937871451
0.5975 0.0973178915995406
0.6025 0.0669648065945081
0.6075 0.033727921251597
0.6125 0.0115989859263063
0.6175 0.00239796569126362
0.6225 -0.000184599186473184
0.6275 -0.000822004102253741
0.6325 -0.000605732814839292
0.6375 -0.000516643299476872
0.6425 -0.000494866148813
0.6475 -0.00048947271344146
0.6525 -0.000488321571797622
0.6575 -0.000488073157880118
0.6625 -0.000487950267386545
0.6675 -0.000487994537355084
0.6725 -0.000487994537355084
0.6775 -0.000487994537355084
0.6825 -0.000487994537355084
0.6875 -0.000487994537355084
0.6925 -0.000487994537355084
0.6975 -0.000487994537355084
0.7025 -0.000487994537355084
0.7075 -0.000487994537355084
0.7125 -0.000487994537355084
0.7175 -0.000487994537355084
0.7225 -0.000487994537355084
0.7275 -0.000487994537355084
0.7325 -0.000487994537355084
0.7375 -0.000487994537355084
0.7425 -0.000487994537355084
0.7475 -0.000487994537355084
0.7525 -0.000487994537355084
0.7575 -0.000487994537355084
0.7625 -0.000487994537355084
0.7675 -0.000487994537355084
0.7725 -0.000487994537355084
0.7775 -0.000487994537355084
0.7825 -0.000487994537355084
0.7875 -0.000487994537355084
0.7925 -0.000487994537355084
0.7975 -0.000487994537355084
0.8025 -0.000487994537355084
0.8075 -0.000487994537355084
0.8125 -0.000487994537355084
0.8175 -0.000487994537355084
0.8225 -0.000487994537355084
0.8275 -0.000487994537355084
0.8325 -0.000487994537355084
0.8375 -0.000487994537355084
0.8425 -0.000487994537355084
0.8475 -0.000487994537355084
0.8525 -0.000487994537355084
0.8575 -0.000487994537355084
0.8625 -0.000487994537355084
0.8675 -0.000487994537355084
0.8725 -0.000487994537355084
0.8775 -0.000487994537355084
0.8825 -0.000487994537355084
0.8875 -0.000487994537355084
0.8925 -0.000487994537355084
0.8975 -0.000487994537355084
0.9025 -0.000487994537355084
0.9075 -0.000487994537355084
0.9125 -0.000487994537355084
0.9175 -0.000487994537355084
0.9225 -0.000487994537355084
0.9275 -0.000487994537355084
0.9325 -0.000487994537355084
0.9375 -0.000487994537355084
0.9425 -0.000487994537355084
0.9475 -0.000487994537355084
0.9525 -0.000487994537355084
0.9575 -0.000487994537355084
0.9625 -0.000487994537355084
0.9675 -0.000487994537355084
0.9725 -0.000487994537355084
0.9775 -0.000487994537355084
0.9825 -0.000487994537355084
0.9875 -0.000487994537355084
0.9925 -0.000487994537355084
0.9975 -0.000487994537355084
};
\addlegendentry{\(rho u\)}

\nextgroupplot[
legend cell align={left},
legend style={draw=none},
tick align=outside,
tick pos=left,
x grid style={white!69.0196078431373!black},
xlabel={\(x\)},
xmin=-0.04725, xmax=1.04725,
xtick style={color=black},
y grid style={white!69.0196078431373!black},
ylabel={\(c_2\) and \(E\)},
ymin=-0.380284859894522, ymax=1.03982424471276,
ytick style={color=black},
axis lines=left
]
\addplot [semithick, black, dashed]
table {%
0.0025 -0.177741408348083
0.0075 -0.177741408348083
0.0125 -0.177741408348083
0.0175 -0.177741408348083
0.0225 -0.177741408348083
0.0275 -0.177741408348083
0.0325 -0.177741408348083
0.0375 -0.177741408348083
0.0425 -0.177741408348083
0.0475 -0.177741408348083
0.0525 -0.177741408348083
0.0575 -0.177741408348083
0.0625 -0.177741408348083
0.0675 -0.177741408348083
0.0725 -0.177741408348083
0.0775 -0.177741408348083
0.0825 -0.177741408348083
0.0875 -0.177741408348083
0.0925 -0.177741408348083
0.0975 -0.177741408348083
0.1025 -0.177741408348083
0.1075 -0.177741408348083
0.1125 -0.177741408348083
0.1175 -0.177741408348083
0.1225 -0.177741408348083
0.1275 -0.177741408348083
0.1325 -0.177741408348083
0.1375 -0.177741408348083
0.1425 -0.177741408348083
0.1475 -0.177741408348083
0.1525 -0.177741408348083
0.1575 -0.177741408348083
0.1625 -0.177741408348083
0.1675 -0.177741408348083
0.1725 -0.177741408348083
0.1775 -0.177741408348083
0.1825 -0.177741408348083
0.1875 -0.177741408348083
0.1925 -0.177741408348083
0.1975 -0.177741408348083
0.2025 -0.177741408348083
0.2075 -0.177741408348083
0.2125 -0.177741408348083
0.2175 -0.177741408348083
0.2225 -0.177741408348083
0.2275 -0.177741408348083
0.2325 -0.177741408348083
0.2375 -0.177741408348083
0.2425 -0.177741408348083
0.2475 -0.177741408348083
0.2525 -0.177741408348083
0.2575 -0.177741408348083
0.2625 -0.177741408348083
0.2675 -0.177741408348083
0.2725 -0.177741408348083
0.2775 -0.177741408348083
0.2825 -0.177741408348083
0.2875 -0.177741378545761
0.2925 -0.177741348743439
0.2975 -0.177741348743439
0.3025 -0.177741199731827
0.3075 -0.177740976214409
0.3125 -0.177740469574928
0.3175 -0.177739366889
0.3225 -0.177737087011337
0.3275 -0.17773263156414
0.3325 -0.177724033594131
0.3375 -0.177707955241203
0.3425 -0.177678778767586
0.3475 -0.177627399563789
0.3525 -0.17754003405571
0.3575 -0.177396222949028
0.3625 -0.177167773246765
0.3675 -0.176817715167999
0.3725 -0.176300853490829
0.3775 -0.17556619644165
0.3825 -0.174561873078346
0.3875 -0.173242062330246
0.3925 -0.171576172113419
0.3975 -0.169558435678482
0.4025 -0.167216286063194
0.4075 -0.164617672562599
0.4125 -0.161874487996101
0.4175 -0.159143015742302
0.4225 -0.156095817685127
0.4275 -0.153310760855675
0.4325 -0.151156857609749
0.4375 -0.149965643882751
0.4425 -0.150083541870117
0.4475 -0.151832059025764
0.4525 -0.155518546700478
0.4575 -0.161904186010361
0.4625 -0.171773359179497
0.4675 -0.18463471531868
0.4725 -0.200466051697731
0.4775 -0.218969643115997
0.4825 -0.239427492022514
0.4875 -0.262117713689804
0.4925 -0.284339308738708
0.4975 -0.301868826150894
0.5025 -0.312276989221573
0.5075 -0.315734446048737
0.5125 -0.314571976661682
0.5175 -0.310171991586685
0.5225 -0.30202442407608
0.5275 -0.291808098554611
0.5325 -0.280926644802094
0.5375 -0.270824879407883
0.5425 -0.261575758457184
0.5475 -0.254636913537979
0.5525 -0.249808222055435
0.5575 -0.246643751859665
0.5625 -0.244650661945343
0.5675 -0.243358820676804
0.5725 -0.242312699556351
0.5775 -0.241020604968071
0.5825 -0.238868311047554
0.5875 -0.23499296605587
0.5925 -0.228125914931297
0.5975 -0.216508731245995
0.6025 -0.19837874174118
0.6075 -0.174530372023582
0.6125 -0.1527980864048
0.6175 -0.141425430774689
0.6225 -0.137841701507568
0.6275 -0.136943116784096
0.6325 -0.136725559830666
0.6375 -0.136672586202621
0.6425 -0.136659666895866
0.6475 -0.136656552553177
0.6525 -0.136655792593956
0.6575 -0.136655628681183
0.6625 -0.136655569076538
0.6675 -0.136655569076538
0.6725 -0.136655569076538
0.6775 -0.136655569076538
0.6825 -0.136655569076538
0.6875 -0.136655569076538
0.6925 -0.136655569076538
0.6975 -0.136655569076538
0.7025 -0.136655569076538
0.7075 -0.136655569076538
0.7125 -0.136655569076538
0.7175 -0.136655569076538
0.7225 -0.136655569076538
0.7275 -0.136655569076538
0.7325 -0.136655569076538
0.7375 -0.136655569076538
0.7425 -0.136655569076538
0.7475 -0.136655569076538
0.7525 -0.136655569076538
0.7575 -0.136655569076538
0.7625 -0.136655569076538
0.7675 -0.136655569076538
0.7725 -0.136655569076538
0.7775 -0.136655569076538
0.7825 -0.136655569076538
0.7875 -0.136655569076538
0.7925 -0.136655569076538
0.7975 -0.136655569076538
0.8025 -0.136655569076538
0.8075 -0.136655569076538
0.8125 -0.136655569076538
0.8175 -0.136655569076538
0.8225 -0.136655569076538
0.8275 -0.136655569076538
0.8325 -0.136655569076538
0.8375 -0.136655569076538
0.8425 -0.136655569076538
0.8475 -0.136655569076538
0.8525 -0.136655569076538
0.8575 -0.136655569076538
0.8625 -0.136655569076538
0.8675 -0.136655569076538
0.8725 -0.136655569076538
0.8775 -0.136655569076538
0.8825 -0.136655569076538
0.8875 -0.136655569076538
0.8925 -0.136655569076538
0.8975 -0.136655569076538
0.9025 -0.136655569076538
0.9075 -0.136655569076538
0.9125 -0.136655569076538
0.9175 -0.136655569076538
0.9225 -0.136655569076538
0.9275 -0.136655569076538
0.9325 -0.136655569076538
0.9375 -0.136655569076538
0.9425 -0.136655569076538
0.9475 -0.136655569076538
0.9525 -0.136655569076538
0.9575 -0.136655569076538
0.9625 -0.136655569076538
0.9675 -0.136655569076538
0.9725 -0.136655569076538
0.9775 -0.136655569076538
0.9825 -0.136655569076538
0.9875 -0.136655569076538
0.9925 -0.136655569076538
0.9975 -0.136655569076538
};
\addlegendentry{\(c_2\)}
\addplot [semithick, black]
table {%
0.0025 0.975271997551442
0.0075 0.975271997551442
0.0125 0.975271997551442
0.0175 0.975271997551442
0.0225 0.975271997551442
0.0275 0.975271997551442
0.0325 0.975271997551442
0.0375 0.975271997551442
0.0425 0.975271997551442
0.0475 0.975271997551442
0.0525 0.975271997551442
0.0575 0.975271997551442
0.0625 0.975271997551442
0.0675 0.975271997551442
0.0725 0.975271997551442
0.0775 0.975271997551442
0.0825 0.975271997551442
0.0875 0.975271997551442
0.0925 0.975271997551442
0.0975 0.975271997551442
0.1025 0.975271997551442
0.1075 0.975271997551442
0.1125 0.975271997551442
0.1175 0.975271997551442
0.1225 0.975271997551442
0.1275 0.975271997551442
0.1325 0.975271997551442
0.1375 0.975271997551442
0.1425 0.975271997551442
0.1475 0.975271997551442
0.1525 0.975271997551442
0.1575 0.975271997551442
0.1625 0.975271997551442
0.1675 0.975271997551442
0.1725 0.975271997551442
0.1775 0.975271997551442
0.1825 0.975271997551442
0.1875 0.975271997551442
0.1925 0.975271997551442
0.1975 0.975271997551442
0.2025 0.975271997551442
0.2075 0.975271997551442
0.2125 0.975271997551442
0.2175 0.975271997551442
0.2225 0.975271997551442
0.2275 0.975271997551442
0.2325 0.975271997551442
0.2375 0.975271997551442
0.2425 0.975271997551442
0.2475 0.975271997551442
0.2525 0.975271997551442
0.2575 0.975271997551442
0.2625 0.975271997551442
0.2675 0.975271997551442
0.2725 0.975271997551442
0.2775 0.975271997551442
0.2825 0.975271708786336
0.2875 0.975273103351576
0.2925 0.975273260010932
0.2975 0.975273294116564
0.3025 0.975273830866972
0.3075 0.975269472624808
0.3125 0.975268942441419
0.3175 0.975262575181548
0.3225 0.975244515529545
0.3275 0.975219685511442
0.3325 0.975167416095829
0.3375 0.975063422755727
0.3425 0.974878929338046
0.3475 0.974552268136347
0.3525 0.973989893546062
0.3575 0.973066470642708
0.3625 0.971583337353127
0.3675 0.969285850920136
0.3725 0.965851531886018
0.3775 0.960886826437765
0.3825 0.953959686095954
0.3875 0.944591803509279
0.3925 0.932335515084378
0.3975 0.916774241351091
0.4025 0.897585975508623
0.4075 0.874560561862619
0.4125 0.847628385948788
0.4175 0.816861435642097
0.4225 0.783828633560638
0.4275 0.756724094544929
0.4325 0.72756249062352
0.4375 0.697009463673292
0.4425 0.665840944423745
0.4475 0.634763694939587
0.4525 0.604537001735654
0.4575 0.581086561681408
0.4625 0.564537958543868
0.4675 0.548732327810139
0.4725 0.533966945104955
0.4775 0.520505679868967
0.4825 0.508598048605189
0.4875 0.499451522060831
0.4925 0.492900883748927
0.4975 0.488098938523851
0.5025 0.485027230867931
0.5075 0.482804296370767
0.5125 0.478804101033184
0.5175 0.467466163240192
0.5225 0.448261610113506
0.5275 0.42189707957976
0.5325 0.406458525314381
0.5375 0.389607609105759
0.5425 0.367413519530812
0.5475 0.34647091804807
0.5525 0.329103946418212
0.5575 0.316313260132637
0.5625 0.307739278224139
0.5675 0.30212845296013
0.5725 0.29773094619696
0.5775 0.29247439288694
0.5825 0.283796024497446
0.5875 0.268315380084941
0.5925 0.241581481407818
0.5975 0.199022157756669
0.6025 0.140761272645911
0.6075 0.0815676370014747
0.6125 0.0484020940752493
0.6175 0.039030297553408
0.6225 0.037845432472689
0.6275 0.0378070142454444
0.6325 0.0332612882905262
0.6375 0.031708747956456
0.6425 0.0313277619849232
0.6475 0.0312349964318984
0.6525 0.0312143014462262
0.6575 0.0312102315022826
0.6625 0.0312069399566448
0.6675 0.031207593612221
0.6725 0.031207593612221
0.6775 0.031207593612221
0.6825 0.031207593612221
0.6875 0.031207593612221
0.6925 0.031207593612221
0.6975 0.031207593612221
0.7025 0.031207593612221
0.7075 0.031207593612221
0.7125 0.031207593612221
0.7175 0.031207593612221
0.7225 0.031207593612221
0.7275 0.031207593612221
0.7325 0.031207593612221
0.7375 0.031207593612221
0.7425 0.031207593612221
0.7475 0.031207593612221
0.7525 0.031207593612221
0.7575 0.031207593612221
0.7625 0.031207593612221
0.7675 0.031207593612221
0.7725 0.031207593612221
0.7775 0.031207593612221
0.7825 0.031207593612221
0.7875 0.031207593612221
0.7925 0.031207593612221
0.7975 0.031207593612221
0.8025 0.031207593612221
0.8075 0.031207593612221
0.8125 0.031207593612221
0.8175 0.031207593612221
0.8225 0.031207593612221
0.8275 0.031207593612221
0.8325 0.031207593612221
0.8375 0.031207593612221
0.8425 0.031207593612221
0.8475 0.031207593612221
0.8525 0.031207593612221
0.8575 0.031207593612221
0.8625 0.031207593612221
0.8675 0.031207593612221
0.8725 0.031207593612221
0.8775 0.031207593612221
0.8825 0.031207593612221
0.8875 0.031207593612221
0.8925 0.031207593612221
0.8975 0.031207593612221
0.9025 0.031207593612221
0.9075 0.031207593612221
0.9125 0.031207593612221
0.9175 0.031207593612221
0.9225 0.031207593612221
0.9275 0.031207593612221
0.9325 0.031207593612221
0.9375 0.031207593612221
0.9425 0.031207593612221
0.9475 0.031207593612221
0.9525 0.031207593612221
0.9575 0.031207593612221
0.9625 0.031207593612221
0.9675 0.031207593612221
0.9725 0.031207593612221
0.9775 0.031207593612221
0.9825 0.031207593612221
0.9875 0.031207593612221
0.9925 0.031207593612221
0.9975 0.031207593612221
};
\addlegendentry{\(E\)}
\end{groupplot}

\end{tikzpicture}
}
	\caption{Code variables \(c_1\), \(c_2\) and \(c_3\) (dashed lines - -) and macroscopic quantities \(\rho\), \(E\), \(\rho u\) (full lines --) for \(t=0.05s\).}
\end{figure}
\begin{figure}[H]
	\scalebox{1}{% This file was created by tikzplotlib v0.9.6.
\begin{tikzpicture}

\begin{groupplot}[group style={group size=3 by 1,horizontal sep=2cm, vertical sep=2cm}]
\nextgroupplot[
legend cell align={left},
legend style={draw=none},
tick align=outside,
tick pos=left,
x grid style={white!69.0196078431373!black},
xlabel={\(x\)},
xmin=-0.04725, xmax=1.04725,
xtick style={color=black},
y grid style={white!69.0196078431373!black},
ylabel={\(c_0\) and \(\rho\)},
ymin=-0.0601167461548287, ymax=1.05041885939785,
ytick style={color=black},
axis lines=left,
width=0.3\textwidth,
height =.4\textwidth
]
\addplot [semithick, black, dashed]
table {%
0.0025 0.532396674156189
0.0075 0.532396674156189
0.0125 0.532396674156189
0.0175 0.532396674156189
0.0225 0.532396674156189
0.0275 0.532396674156189
0.0325 0.532396674156189
0.0375 0.532396674156189
0.0425 0.532396674156189
0.0475 0.532396674156189
0.0525 0.532396674156189
0.0575 0.532396674156189
0.0625 0.532396674156189
0.0675 0.532396674156189
0.0725 0.532396674156189
0.0775 0.532396674156189
0.0825 0.532396674156189
0.0875 0.532396674156189
0.0925 0.532396674156189
0.0975 0.532396674156189
0.1025 0.532396674156189
0.1075 0.532396674156189
0.1125 0.532396674156189
0.1175 0.532396674156189
0.1225 0.532396674156189
0.1275 0.532396674156189
0.1325 0.532396674156189
0.1375 0.532396674156189
0.1425 0.532396674156189
0.1475 0.532396614551544
0.1525 0.532396614551544
0.1575 0.532396554946899
0.1625 0.532396554946899
0.1675 0.53239643573761
0.1725 0.532396256923676
0.1775 0.532395958900452
0.1825 0.532395362854004
0.1875 0.532394349575043
0.1925 0.532392561435699
0.1975 0.532389581203461
0.2025 0.53238433599472
0.2075 0.532375633716583
0.2125 0.532361447811127
0.2175 0.532338559627533
0.2225 0.53230232000351
0.2275 0.532245934009552
0.2325 0.532159745693207
0.2375 0.532030642032623
0.2425 0.531840622425079
0.2475 0.531566202640533
0.2525 0.531177699565887
0.2575 0.530638039112091
0.2625 0.529902994632721
0.2675 0.528921067714691
0.2725 0.527634024620056
0.2775 0.525978863239288
0.2825 0.523889541625977
0.2875 0.521298944950104
0.2925 0.518142342567444
0.2975 0.514359056949615
0.3025 0.509895980358124
0.3075 0.504708409309387
0.3125 0.498762756586075
0.3175 0.492036581039429
0.3225 0.484519153833389
0.3275 0.476211398839951
0.3325 0.46712526679039
0.3375 0.457282871007919
0.3425 0.446695625782013
0.3475 0.435274511575699
0.3525 0.423205614089966
0.3575 0.410542845726013
0.3625 0.397137641906738
0.3675 0.383145272731781
0.3725 0.368730843067169
0.3775 0.353964537382126
0.3825 0.338918328285217
0.3875 0.323665529489517
0.3925 0.308280169963837
0.3975 0.292836308479309
0.4025 0.277407735586166
0.4075 0.261997550725937
0.4125 0.246466621756554
0.4175 0.231102213263512
0.4225 0.21597096323967
0.4275 0.201139256358147
0.4325 0.186673760414124
0.4375 0.172642007470131
0.4425 0.159113585948944
0.4475 0.146161213517189
0.4525 0.133862689137459
0.4575 0.122302934527397
0.4625 0.11172292381525
0.4675 0.102868385612965
0.4725 0.0950575843453407
0.4775 0.0883814021945
0.4825 0.0829150080680847
0.4875 0.0786925405263901
0.4925 0.0756731256842613
0.4975 0.0737144201993942
0.5025 0.0727607756853104
0.5075 0.0723534971475601
0.5125 0.0722023397684097
0.5175 0.0721774846315384
0.5225 0.0722092762589455
0.5275 0.0722649097442627
0.5325 0.0723256096243858
0.5375 0.0723692774772644
0.5425 0.0723552107810974
0.5475 0.0722099766135216
0.5525 0.0714246407151222
0.5575 0.0701148957014084
0.5625 0.0682081282138824
0.5675 0.0656430199742317
0.5725 0.0624613836407661
0.5775 0.0586692057549953
0.5825 0.0544614866375923
0.5875 0.0504112876951694
0.5925 0.0467777512967587
0.5975 0.0437299720942974
0.6025 0.0413325913250446
0.6075 0.0395610518753529
0.6125 0.0383322313427925
0.6175 0.0375366881489754
0.6225 0.0370632223784924
0.6275 0.036813847720623
0.6325 0.036709651350975
0.6375 0.0366904102265835
0.6425 0.0367102921009064
0.6475 0.0367310643196106
0.6525 0.0367135368287563
0.6575 0.0366066433489323
0.6625 0.0363319627940655
0.6675 0.0357608906924725
0.6725 0.0346800498664379
0.6775 0.0327421240508556
0.6825 0.0294098276644945
0.6875 0.0239496249705553
0.6925 0.015686022117734
0.6975 0.00500450562685728
0.7025 -0.00498133525252342
0.7075 -0.00924072973430157
0.7125 -0.00963785499334335
0.7175 -0.00931992195546627
0.7225 -0.00916687026619911
0.7275 -0.00912009179592133
0.7325 -0.00910733081400394
0.7375 -0.00910396315157413
0.7425 -0.00910309888422489
0.7475 -0.00910286046564579
0.7525 -0.00910280831158161
0.7575 -0.00910280086100101
0.7625 -0.00910278595983982
0.7675 -0.00910278595983982
0.7725 -0.00910279341042042
0.7775 -0.00910279341042042
0.7825 -0.00910279341042042
0.7875 -0.00910279341042042
0.7925 -0.00910279341042042
0.7975 -0.00910279341042042
0.8025 -0.00910279341042042
0.8075 -0.00910279341042042
0.8125 -0.00910279341042042
0.8175 -0.00910279341042042
0.8225 -0.00910279341042042
0.8275 -0.00910279341042042
0.8325 -0.00910279341042042
0.8375 -0.00910279341042042
0.8425 -0.00910279341042042
0.8475 -0.00910279341042042
0.8525 -0.00910279341042042
0.8575 -0.00910279341042042
0.8625 -0.00910279341042042
0.8675 -0.00910279341042042
0.8725 -0.00910279341042042
0.8775 -0.00910279341042042
0.8825 -0.00910279341042042
0.8875 -0.00910279341042042
0.8925 -0.00910279341042042
0.8975 -0.00910279341042042
0.9025 -0.00910279341042042
0.9075 -0.00910279341042042
0.9125 -0.00910279341042042
0.9175 -0.00910279341042042
0.9225 -0.00910279341042042
0.9275 -0.00910279341042042
0.9325 -0.00910279341042042
0.9375 -0.00910279341042042
0.9425 -0.00910279341042042
0.9475 -0.00910279341042042
0.9525 -0.00910279341042042
0.9575 -0.00910279341042042
0.9625 -0.00910279341042042
0.9675 -0.00910279341042042
0.9725 -0.00910279341042042
0.9775 -0.00910279341042042
0.9825 -0.00910279341042042
0.9875 -0.00910279341042042
0.9925 -0.00910279341042042
0.9975 -0.00910279341042042
};
\addlegendentry{\(c_0\)}
\addplot [semithick, black]
table {%
0.0025 0.999939968236364
0.0075 0.999939968236364
0.0125 0.999939968236364
0.0175 0.999939968236364
0.0225 0.999939968236364
0.0275 0.999939968236364
0.0325 0.999939968236364
0.0375 0.999939968236364
0.0425 0.999939968236364
0.0475 0.999939968236364
0.0525 0.999939968236364
0.0575 0.999939968236364
0.0625 0.999939968236364
0.0675 0.999939968236364
0.0725 0.999939968236364
0.0775 0.999939968236364
0.0825 0.999939968236364
0.0875 0.999939968236364
0.0925 0.999939968236364
0.0975 0.999939968236364
0.1025 0.999939968236364
0.1075 0.999939968236364
0.1125 0.999939968236364
0.1175 0.999939968236364
0.1225 0.999939968236364
0.1275 0.999939968236364
0.1325 0.999939968236364
0.1375 0.999939968236364
0.1425 0.999939968236364
0.1475 0.999939891820153
0.1525 0.999939896596166
0.1575 0.999939795822287
0.1625 0.999939859343263
0.1675 0.999939805851915
0.1725 0.999939694093206
0.1775 0.999939501619874
0.1825 0.999939005750303
0.1875 0.999938344153074
0.1925 0.999937119822089
0.1975 0.999935072942231
0.2025 0.999931231236611
0.2075 0.999925225877609
0.2125 0.999915257502252
0.2175 0.999899133323477
0.2225 0.999873507624635
0.2275 0.999833609765539
0.2325 0.99977253900411
0.2375 0.9996809445035
0.2425 0.999546136993628
0.2475 0.999351227775407
0.2525 0.999074761325923
0.2575 0.998690082954291
0.2625 0.99816518680503
0.2675 0.997462545831998
0.2725 0.996539198005429
0.2775 0.995348231890836
0.2825 0.993840270795119
0.2875 0.991963298800282
0.2925 0.989666673092124
0.2975 0.986901124437841
0.3025 0.983621211221012
0.3075 0.979786665202715
0.3125 0.975364012858615
0.3175 0.97032644200879
0.3225 0.964655201189602
0.3275 0.958339407419165
0.3325 0.951376274013175
0.3375 0.943770287319636
0.3425 0.935610410017081
0.3475 0.927371961566118
0.3525 0.918455868840027
0.3575 0.908970900763495
0.3625 0.899218359890466
0.3675 0.889084333768831
0.3725 0.87847134958093
0.3775 0.867419432824812
0.3825 0.855971563320893
0.3875 0.844173005973108
0.3925 0.832070847256825
0.3975 0.819712711665302
0.4025 0.807146266556512
0.4075 0.794695000976133
0.4125 0.78190466747261
0.4175 0.768850565625307
0.4225 0.755560808600141
0.4275 0.742063220853034
0.4325 0.728385018614622
0.4375 0.714554377019596
0.4425 0.700602839963558
0.4475 0.686569614335894
0.4525 0.672506648235214
0.4575 0.658487629814026
0.4625 0.644717437979311
0.4675 0.631765780660013
0.4725 0.619325883423862
0.4775 0.607687344965644
0.4825 0.597212820743712
0.4875 0.588285788201178
0.4925 0.581195238285149
0.4975 0.575985645349973
0.5025 0.572734052936236
0.5075 0.570536559113325
0.5125 0.568678801497206
0.5175 0.566643481142819
0.5225 0.563857749773142
0.5275 0.559670780785382
0.5325 0.553210844190266
0.5375 0.543409478611862
0.5425 0.529145993387852
0.5475 0.50948945315889
0.5525 0.485234439181976
0.5575 0.455965806419651
0.5625 0.42263749986887
0.5675 0.388745872471004
0.5725 0.354970117004063
0.5775 0.32313509641263
0.5825 0.294892554147503
0.5875 0.270800437921515
0.5925 0.25120485562067
0.5975 0.235968408031532
0.6025 0.224628301743322
0.6075 0.216551893390715
0.6125 0.21106138980637
0.6175 0.207518658791788
0.6225 0.20537306029254
0.6275 0.204180987336888
0.6325 0.203605286466579
0.6375 0.203402295637016
0.6425 0.203401903048731
0.6475 0.203486189890939
0.6525 0.203567316445212
0.6575 0.203565239954071
0.6625 0.20338402965512
0.6675 0.202881565126471
0.6725 0.201828048492853
0.6775 0.199844798980615
0.6825 0.19632382867619
0.6875 0.190356519097128
0.6925 0.180826629631412
0.6975 0.16706815120788
0.7025 0.150496277671594
0.7075 0.137438638470112
0.7125 0.129664560111287
0.7175 0.126477104778855
0.7225 0.125452522307825
0.7275 0.125098642893136
0.7325 0.124985097477642
0.7375 0.124955146979445
0.7425 0.12494730631797
0.7475 0.124945280333169
0.7525 0.124944723211229
0.7575 0.124944570378806
0.7625 0.124944525484282
0.7675 0.124944538857119
0.7725 0.124944498738608
0.7775 0.124944498738608
0.7825 0.124944498738608
0.7875 0.124944498738608
0.7925 0.124944498738608
0.7975 0.124944498738608
0.8025 0.124944498738608
0.8075 0.124944498738608
0.8125 0.124944498738608
0.8175 0.124944498738608
0.8225 0.124944498738608
0.8275 0.124944498738608
0.8325 0.124944498738608
0.8375 0.124944498738608
0.8425 0.124944498738608
0.8475 0.124944498738608
0.8525 0.124944498738608
0.8575 0.124944498738608
0.8625 0.124944498738608
0.8675 0.124944498738608
0.8725 0.124944498738608
0.8775 0.124944498738608
0.8825 0.124944498738608
0.8875 0.124944498738608
0.8925 0.124944498738608
0.8975 0.124944498738608
0.9025 0.124944498738608
0.9075 0.124944498738608
0.9125 0.124944498738608
0.9175 0.124944498738608
0.9225 0.124944498738608
0.9275 0.124944498738608
0.9325 0.124944498738608
0.9375 0.124944498738608
0.9425 0.124944498738608
0.9475 0.124944498738608
0.9525 0.124944498738608
0.9575 0.124944498738608
0.9625 0.124944498738608
0.9675 0.124944498738608
0.9725 0.124944498738608
0.9775 0.124944498738608
0.9825 0.124944498738608
0.9875 0.124944498738608
0.9925 0.124944498738608
0.9975 0.124944498738608
};
\addlegendentry{\(\rho\)}

\nextgroupplot[
legend cell align={left},
legend style={draw=none},
tick align=outside,
tick pos=left,
x grid style={white!69.0196078431373!black},
xlabel={\(x\)},
xmin=-0.04725, xmax=1.04725,
xtick style={color=black},
y grid style={white!69.0196078431373!black},
ylabel={\(c_1\) and \(\rho u\)},
ymin=-0.600631769639265, ymax=0.461477020365247,
ytick style={color=black},
axis lines=left,
width=0.3\textwidth,
height =.4\textwidth
]
\addplot [semithick, black, dashed]
table {%
0.0025 -0.432732731103897
0.0075 -0.432732731103897
0.0125 -0.432732731103897
0.0175 -0.432732731103897
0.0225 -0.432732731103897
0.0275 -0.432732731103897
0.0325 -0.432732731103897
0.0375 -0.432732731103897
0.0425 -0.432732731103897
0.0475 -0.432732731103897
0.0525 -0.432732731103897
0.0575 -0.432732731103897
0.0625 -0.432732731103897
0.0675 -0.432732731103897
0.0725 -0.432732731103897
0.0775 -0.432732731103897
0.0825 -0.432732731103897
0.0875 -0.432732731103897
0.0925 -0.432732731103897
0.0975 -0.432732731103897
0.1025 -0.432732731103897
0.1075 -0.432732731103897
0.1125 -0.432732731103897
0.1175 -0.432732731103897
0.1225 -0.432732731103897
0.1275 -0.432732731103897
0.1325 -0.432732731103897
0.1375 -0.432732731103897
0.1425 -0.432732731103897
0.1475 -0.432732731103897
0.1525 -0.432732731103897
0.1575 -0.432732731103897
0.1625 -0.432732790708542
0.1675 -0.432732850313187
0.1725 -0.432732939720154
0.1775 -0.432733118534088
0.1825 -0.432733416557312
0.1875 -0.43273401260376
0.1925 -0.432734996080399
0.1975 -0.432736665010452
0.2025 -0.432739526033401
0.2075 -0.432744324207306
0.2125 -0.432752132415771
0.2175 -0.432764679193497
0.2225 -0.432784557342529
0.2275 -0.432815372943878
0.2325 -0.432862430810928
0.2375 -0.432932764291763
0.2425 -0.433035999536514
0.2475 -0.433184742927551
0.2525 -0.433394700288773
0.2575 -0.433685302734375
0.2625 -0.434079647064209
0.2675 -0.434604167938232
0.2725 -0.435287922620773
0.2775 -0.43616184592247
0.2825 -0.43725711107254
0.2875 -0.438603729009628
0.2925 -0.440228641033173
0.2975 -0.442154169082642
0.3025 -0.444396555423737
0.3075 -0.446964740753174
0.3125 -0.449859857559204
0.3175 -0.453074663877487
0.3225 -0.456594347953796
0.3275 -0.46039617061615
0.3325 -0.464451283216476
0.3375 -0.468724995851517
0.3425 -0.473291903734207
0.3475 -0.478819519281387
0.3525 -0.484451830387115
0.3575 -0.490132510662079
0.3625 -0.496311396360397
0.3675 -0.502679526805878
0.3725 -0.508933007717133
0.3775 -0.515013039112091
0.3825 -0.520862698554993
0.3875 -0.52642685174942
0.3925 -0.531652331352234
0.3975 -0.536488115787506
0.4025 -0.540884792804718
0.4075 -0.544702112674713
0.4125 -0.547621369361877
0.4175 -0.54989105463028
0.4225 -0.551464855670929
0.4275 -0.552299320697784
0.4325 -0.552354097366333
0.4375 -0.55159318447113
0.4425 -0.549987435340881
0.4475 -0.547518193721771
0.4525 -0.544182181358337
0.4575 -0.539999961853027
0.4625 -0.534905731678009
0.4675 -0.528467237949371
0.4725 -0.521527111530304
0.4775 -0.514382779598236
0.4825 -0.50743043422699
0.4875 -0.50112122297287
0.4925 -0.495852291584015
0.4975 -0.49181717634201
0.5025 -0.489068746566772
0.5075 -0.487084209918976
0.5125 -0.485352098941803
0.5175 -0.483432203531265
0.5225 -0.480851918458939
0.5275 -0.477020800113678
0.5325 -0.471169352531433
0.5375 -0.462372064590454
0.5425 -0.449680656194687
0.5475 -0.43235045671463
0.5525 -0.412041693925858
0.5575 -0.388003587722778
0.5625 -0.361093729734421
0.5675 -0.332924395799637
0.5725 -0.305303752422333
0.5775 -0.280411273241043
0.5825 -0.259629011154175
0.5875 -0.242484986782074
0.5925 -0.228977903723717
0.5975 -0.218766763806343
0.6025 -0.211337506771088
0.6075 -0.206131130456924
0.6125 -0.202622607350349
0.6175 -0.200358927249908
0.6225 -0.198971793055534
0.6275 -0.19817641377449
0.6325 -0.197763219475746
0.6375 -0.197586163878441
0.6425 -0.197550982236862
0.6475 -0.197604939341545
0.6525 -0.197729483246803
0.6575 -0.197938069701195
0.6625 -0.198280200362206
0.6675 -0.198854014277458
0.6725 -0.19982998073101
0.6775 -0.201485872268677
0.6825 -0.204242646694183
0.6875 -0.208646968007088
0.6925 -0.215106576681137
0.6975 -0.222911700606346
0.7025 -0.228644728660583
0.7075 -0.228919893503189
0.7125 -0.225593611598015
0.7175 -0.223220273852348
0.7225 -0.222305253148079
0.7275 -0.222030952572823
0.7325 -0.221956059336662
0.7375 -0.221936255693436
0.7425 -0.22193107008934
0.7475 -0.221929714083672
0.7525 -0.221929371356964
0.7575 -0.221929296851158
0.7625 -0.221929267048836
0.7675 -0.221929267048836
0.7725 -0.221929267048836
0.7775 -0.221929267048836
0.7825 -0.221929267048836
0.7875 -0.221929267048836
0.7925 -0.221929267048836
0.7975 -0.221929267048836
0.8025 -0.221929267048836
0.8075 -0.221929267048836
0.8125 -0.221929267048836
0.8175 -0.221929267048836
0.8225 -0.221929267048836
0.8275 -0.221929267048836
0.8325 -0.221929267048836
0.8375 -0.221929267048836
0.8425 -0.221929267048836
0.8475 -0.221929267048836
0.8525 -0.221929267048836
0.8575 -0.221929267048836
0.8625 -0.221929267048836
0.8675 -0.221929267048836
0.8725 -0.221929267048836
0.8775 -0.221929267048836
0.8825 -0.221929267048836
0.8875 -0.221929267048836
0.8925 -0.221929267048836
0.8975 -0.221929267048836
0.9025 -0.221929267048836
0.9075 -0.221929267048836
0.9125 -0.221929267048836
0.9175 -0.221929267048836
0.9225 -0.221929267048836
0.9275 -0.221929267048836
0.9325 -0.221929267048836
0.9375 -0.221929267048836
0.9425 -0.221929267048836
0.9475 -0.221929267048836
0.9525 -0.221929267048836
0.9575 -0.221929267048836
0.9625 -0.221929267048836
0.9675 -0.221929267048836
0.9725 -0.221929267048836
0.9775 -0.221929267048836
0.9825 -0.221929267048836
0.9875 -0.221929267048836
0.9925 -0.221929267048836
0.9975 -0.221929267048836
};
\addlegendentry{\(c_1\)}
\addplot [semithick, black]
table {%
0.0025 -0.000632143126926788
0.0075 -0.000632143126926788
0.0125 -0.000632143126926788
0.0175 -0.000632143126926788
0.0225 -0.000632143126926788
0.0275 -0.000632143126926788
0.0325 -0.000632143126926788
0.0375 -0.000632143126926788
0.0425 -0.000632143126926788
0.0475 -0.000632143126926788
0.0525 -0.000632143126926788
0.0575 -0.000632143126926788
0.0625 -0.000632143126926788
0.0675 -0.000632143126926788
0.0725 -0.000632143126926788
0.0775 -0.000632143126926788
0.0825 -0.000632143126926788
0.0875 -0.000632143126926788
0.0925 -0.000632143126926788
0.0975 -0.000632143126926788
0.1025 -0.000632143126926788
0.1075 -0.000632143126926788
0.1125 -0.000632143126926788
0.1175 -0.000632143126926788
0.1225 -0.000632143126926788
0.1275 -0.000632143126926788
0.1325 -0.000632143126926788
0.1375 -0.000632143126926788
0.1425 -0.000632143126926788
0.1475 -0.000632173375010415
0.1525 -0.000632104061587989
0.1575 -0.000632007194243283
0.1625 -0.00063202397152044
0.1675 -0.000631778068071429
0.1725 -0.000631768516045024
0.1775 -0.00063129115964851
0.1825 -0.000630613241312872
0.1875 -0.000629283825791557
0.1925 -0.000627327558537402
0.1975 -0.000623569540302226
0.2025 -0.000616716053202141
0.2075 -0.00060563643734209
0.2125 -0.000587832776585883
0.2175 -0.000559192954181093
0.2225 -0.000513500835866577
0.2275 -0.000442384529728912
0.2325 -0.000334737570141997
0.2375 -0.000173054613481597
0.2425 6.53383098567214e-05
0.2475 0.000409476114908478
0.2525 0.000895869769376475
0.2575 0.00157195114623076
0.2625 0.00249156578371079
0.2675 0.00371920516779354
0.2725 0.00532691352887426
0.2775 0.00739231173568737
0.2825 0.00999584380123348
0.2875 0.0132203304786222
0.2925 0.0171427676217835
0.2975 0.0218344117863193
0.3025 0.0273575593902926
0.3075 0.0337626894030973
0.3125 0.0410829401617279
0.3175 0.0493410337020757
0.3225 0.0585408233177164
0.3275 0.0686713618046942
0.3325 0.0797101026149877
0.3375 0.091618603067246
0.3425 0.10441242142248
0.3475 0.117959440694993
0.3525 0.129703089887525
0.3575 0.1419404308807
0.3625 0.154997461140499
0.3675 0.168616103466397
0.3725 0.182533790133918
0.3775 0.196672578352668
0.3825 0.210955368206738
0.3875 0.225302622115324
0.3925 0.239637028732087
0.3975 0.253881643912623
0.4025 0.267957338762031
0.4075 0.280961195646218
0.4125 0.293150727063378
0.4175 0.305144263114254
0.4225 0.316881644064995
0.4275 0.328302127814299
0.4325 0.33933957799887
0.4375 0.349926546168251
0.4425 0.359992724938231
0.4475 0.369462272596841
0.4525 0.378256193242776
0.4575 0.386292554491997
0.4625 0.393398801651021
0.4675 0.399102463756175
0.4725 0.40380815175692
0.4775 0.407486924329708
0.4825 0.41015787594471
0.4875 0.411898922695632
0.4925 0.412851790967964
0.4975 0.413199348092314
0.5025 0.413082789122706
0.5075 0.41266932194461
0.5125 0.412017518665487
0.5175 0.411084868389817
0.5225 0.409360494731603
0.5275 0.406620506213134
0.5325 0.402334269979953
0.5375 0.395758255469973
0.5425 0.386095798790533
0.5475 0.372655703278992
0.5525 0.355083416023198
0.5575 0.333510192122156
0.5625 0.308616539214803
0.5675 0.28590732584735
0.5725 0.263293272246809
0.5775 0.240949501315919
0.5825 0.219911543798145
0.5875 0.201477424806764
0.5925 0.186114280988547
0.5975 0.173920234187673
0.6025 0.164698581942979
0.6075 0.158058163201109
0.6125 0.153517864067732
0.6175 0.150590227613131
0.6225 0.148833462910912
0.6275 0.147882022023603
0.6325 0.147451653820631
0.6375 0.147331032974255
0.6425 0.147365106277065
0.6475 0.147433794470336
0.6525 0.147424972133024
0.6575 0.14720636093591
0.6625 0.146584147752742
0.6675 0.145247168924701
0.6725 0.142677445368771
0.6775 0.138024700209155
0.6825 0.129942269169191
0.6875 0.116484360353021
0.6925 0.0954700091989671
0.6975 0.0663669489355933
0.7025 0.0343868951022351
0.7075 0.0125485931000844
0.7125 0.00289424899953278
0.7175 1.92283516177756e-06
0.7225 -0.000752186045089806
0.7275 -0.000643932991058006
0.7325 -0.000525182076312849
0.7375 -0.000493739530160031
0.7425 -0.000485546555049242
0.7475 -0.000483613861706367
0.7525 -0.000483165345081839
0.7575 -0.000482871681500782
0.7625 -0.000482868742415733
0.7675 -0.000482851352829197
0.7725 -0.000482760731040217
0.7775 -0.000482760731040217
0.7825 -0.000482760731040217
0.7875 -0.000482760731040217
0.7925 -0.000482760731040217
0.7975 -0.000482760731040217
0.8025 -0.000482760731040217
0.8075 -0.000482760731040217
0.8125 -0.000482760731040217
0.8175 -0.000482760731040217
0.8225 -0.000482760731040217
0.8275 -0.000482760731040217
0.8325 -0.000482760731040217
0.8375 -0.000482760731040217
0.8425 -0.000482760731040217
0.8475 -0.000482760731040217
0.8525 -0.000482760731040217
0.8575 -0.000482760731040217
0.8625 -0.000482760731040217
0.8675 -0.000482760731040217
0.8725 -0.000482760731040217
0.8775 -0.000482760731040217
0.8825 -0.000482760731040217
0.8875 -0.000482760731040217
0.8925 -0.000482760731040217
0.8975 -0.000482760731040217
0.9025 -0.000482760731040217
0.9075 -0.000482760731040217
0.9125 -0.000482760731040217
0.9175 -0.000482760731040217
0.9225 -0.000482760731040217
0.9275 -0.000482760731040217
0.9325 -0.000482760731040217
0.9375 -0.000482760731040217
0.9425 -0.000482760731040217
0.9475 -0.000482760731040217
0.9525 -0.000482760731040217
0.9575 -0.000482760731040217
0.9625 -0.000482760731040217
0.9675 -0.000482760731040217
0.9725 -0.000482760731040217
0.9775 -0.000482760731040217
0.9825 -0.000482760731040217
0.9875 -0.000482760731040217
0.9925 -0.000482760731040217
0.9975 -0.000482760731040217
};
\addlegendentry{\(\rho u\)}

\nextgroupplot[
legend cell align={left},
legend style={draw=none},
tick align=outside,
tick pos=left,
x grid style={white!69.0196078431373!black},
xlabel={\(x\)},
xmin=-0.04725, xmax=1.04725,
xtick style={color=black},
y grid style={white!69.0196078431373!black},
ylabel={\(c_2\) amd \(E\)},
ymin=-0.385158878988706, ymax=1.04005574282044,
ytick style={color=black},
axis lines=left,
width=0.3\textwidth,
height =.4\textwidth
]
\addplot [semithick, black, dashed]
table {%
0.0025 -0.177741408348083
0.0075 -0.177741408348083
0.0125 -0.177741408348083
0.0175 -0.177741408348083
0.0225 -0.177741408348083
0.0275 -0.177741408348083
0.0325 -0.177741408348083
0.0375 -0.177741408348083
0.0425 -0.177741408348083
0.0475 -0.177741408348083
0.0525 -0.177741408348083
0.0575 -0.177741408348083
0.0625 -0.177741408348083
0.0675 -0.177741408348083
0.0725 -0.177741408348083
0.0775 -0.177741408348083
0.0825 -0.177741408348083
0.0875 -0.177741408348083
0.0925 -0.177741408348083
0.0975 -0.177741408348083
0.1025 -0.177741408348083
0.1075 -0.177741408348083
0.1125 -0.177741408348083
0.1175 -0.177741408348083
0.1225 -0.177741408348083
0.1275 -0.177741408348083
0.1325 -0.177741408348083
0.1375 -0.177741408348083
0.1425 -0.177741408348083
0.1475 -0.177741408348083
0.1525 -0.177741393446922
0.1575 -0.177741378545761
0.1625 -0.177741348743439
0.1675 -0.1777413636446
0.1725 -0.177741289138794
0.1775 -0.177741169929504
0.1825 -0.177741035819054
0.1875 -0.177740678191185
0.1925 -0.177740082144737
0.1975 -0.177739098668098
0.2025 -0.177737444639206
0.2075 -0.177734658122063
0.2125 -0.17773012816906
0.2175 -0.177722796797752
0.2225 -0.177711308002472
0.2275 -0.177693411707878
0.2325 -0.177666246891022
0.2375 -0.177625611424446
0.2425 -0.177566096186638
0.2475 -0.177480578422546
0.2525 -0.177360102534294
0.2575 -0.177193835377693
0.2625 -0.176969036459923
0.2675 -0.17667131125927
0.2725 -0.176285192370415
0.2775 -0.175794810056686
0.2825 -0.175185024738312
0.2875 -0.174442440271378
0.2925 -0.173556804656982
0.2975 -0.172522187232971
0.3025 -0.171337842941284
0.3075 -0.170009329915047
0.3125 -0.168548956513405
0.3175 -0.166975513100624
0.3225 -0.16531465947628
0.3275 -0.163597971200943
0.3325 -0.161862626671791
0.3375 -0.160150215029716
0.3425 -0.158456206321716
0.3475 -0.156514629721642
0.3525 -0.154727399349213
0.3575 -0.153149142861366
0.3625 -0.151801973581314
0.3675 -0.15075595676899
0.3725 -0.150080397725105
0.3775 -0.149827197194099
0.3825 -0.150045812129974
0.3875 -0.150782898068428
0.3925 -0.152081668376923
0.3975 -0.153981238603592
0.4025 -0.156516149640083
0.4075 -0.159861445426941
0.4125 -0.164472028613091
0.4175 -0.169904589653015
0.4225 -0.176169723272324
0.4275 -0.183266013860703
0.4325 -0.191177740693092
0.4375 -0.199871897697449
0.4425 -0.209294691681862
0.4475 -0.219366699457169
0.4525 -0.229977145791054
0.4575 -0.240976229310036
0.4625 -0.252393841743469
0.4675 -0.264954209327698
0.4725 -0.277084320783615
0.4775 -0.288353323936462
0.4825 -0.298288077116013
0.4875 -0.306449711322784
0.4925 -0.312558472156525
0.4975 -0.31661930680275
0.5025 -0.318940103054047
0.5075 -0.320032507181168
0.5125 -0.320376396179199
0.5175 -0.320279002189636
0.5225 -0.319854766130447
0.5275 -0.319063723087311
0.5325 -0.317756950855255
0.5375 -0.315721541643143
0.5425 -0.312739461660385
0.5475 -0.308664947748184
0.5525 -0.302765876054764
0.5575 -0.295719653367996
0.5625 -0.288034588098526
0.5675 -0.280246078968048
0.5725 -0.272880136966705
0.5775 -0.266055703163147
0.5825 -0.259876877069473
0.5875 -0.254898935556412
0.5925 -0.251044660806656
0.5975 -0.248163357377052
0.6025 -0.246080696582794
0.6075 -0.24462828040123
0.6125 -0.243656873703003
0.6175 -0.243040844798088
0.6225 -0.242678299546242
0.6275 -0.242489129304886
0.6325 -0.242411822080612
0.6375 -0.242399305105209
0.6425 -0.242413878440857
0.6475 -0.242420628666878
0.6525 -0.242379456758499
0.6575 -0.24223330616951
0.6625 -0.241891264915466
0.6675 -0.241201236844063
0.6725 -0.239906370639801
0.6775 -0.237575203180313
0.6825 -0.233497560024261
0.6875 -0.226554483175278
0.6925 -0.215164035558701
0.6975 -0.197748988866806
0.7025 -0.174994006752968
0.7075 -0.153826326131821
0.7125 -0.142082512378693
0.7175 -0.138100951910019
0.7225 -0.137038215994835
0.7275 -0.136766105890274
0.7325 -0.136695802211761
0.7375 -0.136677488684654
0.7425 -0.136672720313072
0.7475 -0.136671483516693
0.7525 -0.136671155691147
0.7575 -0.13667106628418
0.7625 -0.13667106628418
0.7675 -0.136671036481857
0.7725 -0.136671036481857
0.7775 -0.136671036481857
0.7825 -0.136671036481857
0.7875 -0.136671036481857
0.7925 -0.136671036481857
0.7975 -0.136671036481857
0.8025 -0.136671036481857
0.8075 -0.136671036481857
0.8125 -0.136671036481857
0.8175 -0.136671036481857
0.8225 -0.136671036481857
0.8275 -0.136671036481857
0.8325 -0.136671036481857
0.8375 -0.136671036481857
0.8425 -0.136671036481857
0.8475 -0.136671036481857
0.8525 -0.136671036481857
0.8575 -0.136671036481857
0.8625 -0.136671036481857
0.8675 -0.136671036481857
0.8725 -0.136671036481857
0.8775 -0.136671036481857
0.8825 -0.136671036481857
0.8875 -0.136671036481857
0.8925 -0.136671036481857
0.8975 -0.136671036481857
0.9025 -0.136671036481857
0.9075 -0.136671036481857
0.9125 -0.136671036481857
0.9175 -0.136671036481857
0.9225 -0.136671036481857
0.9275 -0.136671036481857
0.9325 -0.136671036481857
0.9375 -0.136671036481857
0.9425 -0.136671036481857
0.9475 -0.136671036481857
0.9525 -0.136671036481857
0.9575 -0.136671036481857
0.9625 -0.136671036481857
0.9675 -0.136671036481857
0.9725 -0.136671036481857
0.9775 -0.136671036481857
0.9825 -0.136671036481857
0.9875 -0.136671036481857
0.9925 -0.136671036481857
0.9975 -0.136671036481857
};
\addlegendentry{\(c_2\)}
\addplot [semithick, black]
table {%
0.0025 0.975271997551442
0.0075 0.975271997551442
0.0125 0.975271997551442
0.0175 0.975271997551442
0.0225 0.975271997551442
0.0275 0.975271997551442
0.0325 0.975271997551442
0.0375 0.975271997551442
0.0425 0.975271997551442
0.0475 0.975271997551442
0.0525 0.975271997551442
0.0575 0.975271997551442
0.0625 0.975271997551442
0.0675 0.975271997551442
0.0725 0.975271997551442
0.0775 0.975271997551442
0.0825 0.975271997551442
0.0875 0.975271997551442
0.0925 0.975271997551442
0.0975 0.975271997551442
0.1025 0.975271997551442
0.1075 0.975271997551442
0.1125 0.975271997551442
0.1175 0.975271997551442
0.1225 0.975271997551442
0.1275 0.975271997551442
0.1325 0.975271997551442
0.1375 0.975271997551442
0.1425 0.975271997551442
0.1475 0.975271708786336
0.1525 0.975272345894251
0.1575 0.975273103351576
0.1625 0.975273260010932
0.1675 0.975272412574743
0.1725 0.97527230670645
0.1775 0.975273202300772
0.1825 0.975268970378035
0.1875 0.975267240504522
0.1925 0.975263804729409
0.1975 0.975259757861875
0.2025 0.975245787479831
0.2075 0.975232480833507
0.2125 0.975201277983516
0.2175 0.975155565574799
0.2225 0.975082412268415
0.2275 0.97496418065326
0.2325 0.974788535550117
0.2375 0.974521618308885
0.2425 0.974134176621737
0.2475 0.973571806141579
0.2525 0.972779885457644
0.2575 0.97167609724229
0.2625 0.970175578606417
0.2675 0.968174981828938
0.2725 0.965556486713027
0.2775 0.962188162170118
0.2825 0.957952551255927
0.2875 0.952702259567103
0.2925 0.946329352458147
0.2975 0.938713770210657
0.3025 0.92976104825849
0.3075 0.919395214513698
0.3125 0.907574256447026
0.3175 0.894272008570637
0.3225 0.87948846452417
0.3275 0.863259905697576
0.3325 0.845644010019958
0.3375 0.826716533556512
0.3425 0.806526487257226
0.3475 0.78592180169722
0.3525 0.769855939811628
0.3575 0.753248773797193
0.3625 0.736026455965726
0.3675 0.718395550066959
0.3725 0.700563203113584
0.3775 0.682669806970439
0.3825 0.664834364425736
0.3875 0.647192430597676
0.3925 0.629871662976777
0.3975 0.612988455086655
0.4025 0.596668836356059
0.4075 0.584371643008347
0.4125 0.575277458384687
0.4175 0.566407026206238
0.4225 0.557804660198697
0.4275 0.549505581767932
0.4325 0.541548341644939
0.4375 0.533961068373786
0.4425 0.526777477745141
0.4475 0.520021691847957
0.4525 0.513720220578792
0.4575 0.507894285356309
0.4625 0.502712153720949
0.4675 0.498802386253967
0.4725 0.495425657542786
0.4775 0.492577143898264
0.4825 0.490251562093066
0.4875 0.488427958586226
0.4925 0.487078322863964
0.4975 0.486127402784057
0.5025 0.485625410333697
0.5075 0.485307492902361
0.5125 0.485025667879341
0.5175 0.484664115696879
0.5225 0.483770123030807
0.5275 0.482261706109022
0.5325 0.479816120351499
0.5375 0.47592702610259
0.5425 0.470015472339994
0.5475 0.461493012010696
0.5525 0.448663490545801
0.5575 0.432088647088566
0.5625 0.412086561692047
0.5675 0.401183319139373
0.5725 0.390479542315925
0.5775 0.377266867220081
0.5825 0.361998244598973
0.5875 0.347356304644505
0.5925 0.334243157982305
0.5975 0.323241188539588
0.6025 0.314581149233533
0.6075 0.308180152033538
0.6125 0.303743960941119
0.6175 0.300877062332272
0.6225 0.299185743285398
0.6275 0.298316712773795
0.6325 0.297972671667703
0.6375 0.297937113346249
0.6425 0.298031047055465
0.6475 0.298098991706142
0.6525 0.297969610809626
0.6575 0.297401101943191
0.6625 0.296029213290566
0.6675 0.293220715021089
0.6725 0.287944708909793
0.6775 0.278493335848964
0.6825 0.262217338283881
0.6875 0.23536113753485
0.6925 0.194066197961633
0.6975 0.13866791976028
0.7025 0.0820898085362342
0.7075 0.049327634382726
0.7125 0.0392961122167837
0.7175 0.0378800558053657
0.7225 0.0378259570083111
0.7275 0.0341393846395762
0.7325 0.0320726948273463
0.7375 0.031528125344399
0.7425 0.0313845556091583
0.7475 0.0313482451115482
0.7525 0.0313390442292665
0.7575 0.0313372785739238
0.7625 0.0313353665154072
0.7675 0.0313353054069306
0.7725 0.0313345729624442
0.7775 0.0313345729624442
0.7825 0.0313345729624442
0.7875 0.0313345729624442
0.7925 0.0313345729624442
0.7975 0.0313345729624442
0.8025 0.0313345729624442
0.8075 0.0313345729624442
0.8125 0.0313345729624442
0.8175 0.0313345729624442
0.8225 0.0313345729624442
0.8275 0.0313345729624442
0.8325 0.0313345729624442
0.8375 0.0313345729624442
0.8425 0.0313345729624442
0.8475 0.0313345729624442
0.8525 0.0313345729624442
0.8575 0.0313345729624442
0.8625 0.0313345729624442
0.8675 0.0313345729624442
0.8725 0.0313345729624442
0.8775 0.0313345729624442
0.8825 0.0313345729624442
0.8875 0.0313345729624442
0.8925 0.0313345729624442
0.8975 0.0313345729624442
0.9025 0.0313345729624442
0.9075 0.0313345729624442
0.9125 0.0313345729624442
0.9175 0.0313345729624442
0.9225 0.0313345729624442
0.9275 0.0313345729624442
0.9325 0.0313345729624442
0.9375 0.0313345729624442
0.9425 0.0313345729624442
0.9475 0.0313345729624442
0.9525 0.0313345729624442
0.9575 0.0313345729624442
0.9625 0.0313345729624442
0.9675 0.0313345729624442
0.9725 0.0313345729624442
0.9775 0.0313345729624442
0.9825 0.0313345729624442
0.9875 0.0313345729624442
0.9925 0.0313345729624442
0.9975 0.0313345729624442
};
\addlegendentry{\(E\)}
\end{groupplot}

\end{tikzpicture}
}
	\caption{Code variables \(c_1\), \(c_2\) and \(c_3\) (dashed lines - -) and macroscopic quantities \(\rho\), \(E\), \(\rho u\) (full lines --) for \(t=0.099s\).}
\end{figure}
\begin{figure}[H]
	\scalebox{.6}{% This file was created by tikzplotlib v0.9.6.
\begin{tikzpicture}

\begin{groupplot}[group style={group size=3 by 2,horizontal sep=2cm, vertical sep=2cm}]
\nextgroupplot[
tick align=outside,
tick pos=left,
x grid style={white!69.0196078431373!black},
xlabel={\(v\)},
xmin=-11, xmax=11,
xtick style={color=black},
y grid style={white!69.0196078431373!black},
ylabel={\(\beta_0\)},
ymin=-6.88802815675735, ymax=1.90020395517349,
ytick style={color=black},
axis lines=left
]
\addplot [semithick, black]
table {%
-10 1.50073885917664
-9.48717948717949 1.50073885917664
-8.97435897435897 1.50073885917664
-8.46153846153846 1.50073885917664
-7.94871794871795 1.50073885917664
-7.43589743589744 1.50073885917664
-6.92307692307692 1.50073885917664
-6.41025641025641 1.50073885917664
-5.8974358974359 1.50073885917664
-5.38461538461538 1.50073635578156
-4.87179487179487 1.500697016716
-4.35897435897436 1.50029397010803
-3.84615384615385 1.49704301357269
-3.33333333333333 1.47711193561554
-2.82051282051282 1.38468647003174
-2.30769230769231 1.06730175018311
-1.79487179487179 0.304895222187042
-1.28205128205128 -0.922769010066986
-0.769230769230769 -2.52028822898865
-0.256410256410256 -4.36609077453613
0.256410256410256 -6.18956422805786
0.769230769230769 -6.4885630607605
1.28205128205128 -4.33615875244141
1.79487179487179 -1.32077956199646
2.30769230769231 0.601730704307556
2.82051282051282 1.27828693389893
3.33333333333333 1.45472884178162
3.84615384615385 1.49290776252747
4.35897435897436 1.49966502189636
4.87179487179487 1.50062108039856
5.38461538461538 1.50072908401489
5.8974358974359 1.50073838233948
6.41025641025641 1.50073885917664
6.92307692307692 1.50073885917664
7.43589743589744 1.50073885917664
7.94871794871795 1.50073885917664
8.46153846153846 1.50073885917664
8.97435897435897 1.50073885917664
9.48717948717949 1.50073885917664
10 1.50073885917664
};

\nextgroupplot[
tick align=outside,
tick pos=left,
x grid style={white!69.0196078431373!black},
xlabel={\(v\)},
xmin=-11, xmax=11,
xtick style={color=black},
y grid style={white!69.0196078431373!black},
ylabel={\(\beta_1\)},
ymin=-3.78430556356907, ymax=0.961265376210213,
ytick style={color=black},
axis lines=left
]
\addplot [semithick, black]
table {%
-10 0.745557606220245
-9.48717948717949 0.745557606220245
-8.97435897435897 0.745557606220245
-8.46153846153846 0.745557606220245
-7.94871794871795 0.745557606220245
-7.43589743589744 0.745557606220245
-6.92307692307692 0.745557606220245
-6.41025641025641 0.745557606220245
-5.8974358974359 0.745557606220245
-5.38461538461538 0.745556771755219
-4.87179487179487 0.745545029640198
-4.35897435897436 0.745418906211853
-3.84615384615385 0.744408249855042
-3.33333333333333 0.738228678703308
-2.82051282051282 0.709602534770966
-2.30769230769231 0.610597670078278
-1.79487179487179 0.365436613559723
-1.28205128205128 -0.159861743450165
-0.769230769230769 -1.51125168800354
-0.256410256410256 -3.5189425945282
0.256410256410256 -3.5685977935791
0.769230769230769 -1.69052612781525
1.28205128205128 -0.361645400524139
1.79487179487179 0.245922803878784
2.30769230769231 0.561992526054382
2.82051282051282 0.694838762283325
3.33333333333333 0.734742224216461
3.84615384615385 0.743755459785461
4.35897435897436 0.745321333408356
4.87179487179487 0.745533525943756
5.38461538461538 0.745555460453033
5.8974358974359 0.745557606220245
6.41025641025641 0.745557546615601
6.92307692307692 0.745557606220245
7.43589743589744 0.745557606220245
7.94871794871795 0.745557606220245
8.46153846153846 0.745557606220245
8.97435897435897 0.745557606220245
9.48717948717949 0.745557606220245
10 0.745557606220245
};

\nextgroupplot[
tick align=outside,
tick pos=left,
x grid style={white!69.0196078431373!black},
xlabel={\(v\)},
xmin=-11, xmax=11,
xtick style={color=black},
y grid style={white!69.0196078431373!black},
ylabel={\(\beta_2\)},
ymin=-0.610432699322701, ymax=4.70868660509586,
ytick style={color=black},
axis lines=left
]
\addplot [semithick, black]
table {%
-10 -0.368653476238251
-9.48717948717949 -0.368653476238251
-8.97435897435897 -0.368653476238251
-8.46153846153846 -0.368653476238251
-7.94871794871795 -0.368653476238251
-7.43589743589744 -0.368653476238251
-6.92307692307692 -0.368653476238251
-6.41025641025641 -0.368653476238251
-5.8974358974359 -0.368653357028961
-5.38461538461538 -0.368648827075958
-4.87179487179487 -0.368583858013153
-4.35897435897436 -0.367914021015167
-3.84615384615385 -0.362561643123627
-3.33333333333333 -0.330129861831665
-2.82051282051282 -0.182376563549042
-2.30769230769231 0.309535503387451
-1.79487179487179 1.43555212020874
-1.28205128205128 3.11136317253113
-0.769230769230769 4.46307420730591
-0.256410256410256 4.46690845489502
0.256410256410256 3.28958964347839
0.769230769230769 1.9076247215271
1.28205128205128 0.956885397434235
1.79487179487179 0.434787750244141
2.30769230769231 0.0257612727582455
2.82051282051282 -0.250134825706482
3.33333333333333 -0.345528900623322
3.84615384615385 -0.365595459938049
4.35897435897436 -0.368394196033478
4.87179487179487 -0.368643701076508
5.38461538461538 -0.368654549121857
5.8974358974359 -0.368653655052185
6.41025641025641 -0.368653476238251
6.92307692307692 -0.368653476238251
7.43589743589744 -0.368653476238251
7.94871794871795 -0.368653476238251
8.46153846153846 -0.368653476238251
8.97435897435897 -0.368653476238251
9.48717948717949 -0.368653476238251
10 -0.368653476238251
};

\nextgroupplot[
tick align=outside,
tick pos=left,
x grid style={white!69.0196078431373!black},
xlabel={\(v\)},
xmin=-11, xmax=11,
xtick style={color=black},
y grid style={white!69.0196078431373!black},
ylabel={\(\gamma_0\)},
ymin=-0.568114146636858, ymax=0.0270530546017552,
ytick style={color=black},
axis lines=left
]
\addplot [semithick, black]
table {%
-10 1.11022302462516e-16
-9.48717948717949 -1.49756037906857e-20
-8.97435897435897 -1.27810118294275e-18
-8.46153846153846 -1.12744683074863e-16
-7.94871794871795 -7.61031278983007e-15
-7.43589743589744 -3.95050479625332e-13
-6.92307692307692 -1.57714583487996e-11
-6.41025641025641 -4.84275102585206e-10
-5.8974358974359 -1.14380972444433e-08
-5.38461538461538 -2.07830187863026e-07
-4.87179487179487 -2.90552413910108e-06
-4.35897435897436 -3.12601969992907e-05
-3.84615384615385 -0.000258900668414548
-3.33333333333333 -0.00165127621583845
-2.82051282051282 -0.00811512031118912
-2.30769230769231 -0.0307554404646581
-1.79487179487179 -0.0900252664219377
-1.28205128205128 -0.204408297707841
-0.769230769230769 -0.363493236763491
-0.256410256410256 -0.506703365379294
0.256410256410256 -0.541061092035103
0.769230769230769 -0.433340373945296
1.28205128205128 -0.256535943244692
1.79487179487179 -0.111602690436555
2.30769230769231 -0.0365475607273425
2.82051282051282 -0.00929664348004875
3.33333333333333 -0.00185629538558574
3.84615384615385 -0.000289820815853585
4.35897435897436 -3.5225241343174e-05
4.87179487179487 -3.32774351739608e-06
5.38461538461538 -2.4444710128489e-07
5.8974358974359 -1.39837998972549e-08
6.41025641025641 -6.24428204040681e-10
6.92307692307692 -2.18275076466274e-11
7.43589743589744 -5.9915325787844e-13
7.94871794871795 -1.2952475837196e-14
8.46153846153846 -2.21019702020631e-16
8.97435897435897 -2.9804272176168e-18
9.48717948717949 -3.1755125641467e-20
10 -2.66916408904971e-22
};

\nextgroupplot[
tick align=outside,
tick pos=left,
x grid style={white!69.0196078431373!black},
xlabel={\(v\)},
xmin=-11, xmax=11,
xtick style={color=black},
y grid style={white!69.0196078431373!black},
ylabel={\(\gamma_1\)},
ymin=-0.534209545581082, ymax=0.600709312993124,
ytick style={color=black},
axis lines=left
]
\addplot [semithick, black]
table {%
-10 -3.46944695195361e-16
-9.48717948717949 1.1102130508315e-16
-8.97435897435897 -6.00279985198224e-18
-8.46153846153846 -2.43089976887143e-16
-7.94871794871795 -1.63779640628311e-14
-7.43589743589744 -8.49092822076895e-13
-6.92307692307692 -3.38441280150103e-11
-6.41025641025641 -1.03712137046538e-09
-5.8974358974359 -2.44324195657029e-08
-5.38461538461538 -4.42430689096351e-07
-4.87179487179487 -6.15715736041038e-06
-4.35897435897436 -6.58295067224784e-05
-3.84615384615385 -0.000540376231108169
-3.33333333333333 -0.00340181835101554
-2.82051282051282 -0.0163877027864573
-2.30769230769231 -0.0601457264406097
-1.79487179487179 -0.166593776856988
-1.28205128205128 -0.340778444520313
-0.769230769230769 -0.4826223247368
-0.256410256410256 -0.350573153470886
0.256410256410256 0.16305672323786
0.769230769230769 0.549122092148842
1.28205128205128 0.395734276311465
1.79487179487179 0.107797605642259
2.30769230769231 0.00113955872203916
2.82051282051282 -0.00628828145589854
3.33333333333333 -0.00183037661651124
3.84615384615385 -0.000293980239536051
4.35897435897436 -3.06998183837196e-05
4.87179487179487 -1.99051118798304e-06
5.38461538461538 -4.83099134786566e-08
5.8974358974359 4.76124359665474e-09
6.41025641025641 6.42627485384193e-10
6.92307692307692 4.09436697221996e-11
7.43589743589744 1.72243753098337e-12
7.94871794871795 5.18312293205685e-14
8.46153846153846 1.15181996987467e-15
8.97435897435897 1.92044935175516e-17
9.48717948717949 2.42388524543346e-19
10 2.32839870106148e-21
};

\nextgroupplot[
tick align=outside,
tick pos=left,
x grid style={white!69.0196078431373!black},
xlabel={\(v\)},
xmin=-11, xmax=11,
xtick style={color=black},
y grid style={white!69.0196078431373!black},
ylabel={\(\gamma_2\)},
ymin=-0.502335108871341, ymax=0.505441492195811,
ytick style={color=black},
axis lines=left
]
\addplot [semithick, black]
table {%
-10 -1.27675647831893e-15
-9.48717948717949 -2.93023219381888e-22
-8.97435897435897 2.20235372015017e-16
-8.46153846153846 -4.11302163870099e-16
-7.94871794871795 -2.73855527116848e-14
-7.43589743589744 -1.41767600983015e-12
-6.92307692307692 -5.64079853813417e-11
-6.41025641025641 -1.724830854549e-09
-5.8974358974359 -4.05240621292349e-08
-5.38461538461538 -7.31333772813112e-07
-4.87179487179487 -1.01335627152193e-05
-4.35897435897436 -0.000107733076686539
-3.84615384615385 -0.000877781684076761
-3.33333333333333 -0.00547097471652983
-2.82051282051282 -0.0260021653540801
-2.30769230769231 -0.0936925974540094
-1.79487179487179 -0.251576815346058
-1.28205128205128 -0.456527081550107
-0.769230769230769 -0.317891765968506
-0.256410256410256 0.401768320918203
0.256410256410256 0.459633464874577
0.769230769230769 -0.213550725589106
1.28205128205128 -0.380666111280626
1.79487179487179 -0.215754658190286
2.30769230769231 -0.0852674972409338
2.82051282051282 -0.0264365709515214
3.33333333333333 -0.00630008262332984
3.84615384615385 -0.00113692600891485
4.35897435897436 -0.00015652926761505
4.87179487179487 -1.66465605259822e-05
5.38461538461538 -1.38045539282732e-06
5.8974358974359 -8.97854083246136e-08
6.41025641025641 -4.59329613409237e-09
6.92307692307692 -1.84947282959922e-10
7.43589743589744 -5.85480723152104e-12
7.94871794871795 -1.45400512715473e-13
8.46153846153846 -2.82487700068086e-15
8.97435897435897 -4.28096738270601e-17
9.48717948717949 -5.04652899526609e-19
10 -4.61634328769642e-21
};
\end{groupplot}

\end{tikzpicture}
}
	\caption{Comparison of the intrinsic variables generated by POD \(\gamma_1\), \(\gamma_2\) and \(\gamma_3\) with the intrinsic variables of the convolutional autoencoder \(\beta_1\), \(\beta_2\) and \(\beta_3\).}
\end{figure}
\begin{figure}[H]
	\scalebox{0.9}{% This file was created by tikzplotlib v0.9.6.
\begin{tikzpicture}

\begin{groupplot}[group style={group size=1 by 5,horizontal sep=2cm, vertical sep=2cm}]
\nextgroupplot[
colorbar,
colorbar style={ylabel={}},
colormap/blackwhite,
point meta max=-0.0167456492781639,
point meta min=-0.31139874458313,
tick align=outside,
tick pos=left,
x grid style={white!69.0196078431373!black},
xlabel={x},
xmin=0.0025, xmax=0.9975,
xtick style={color=black},
y grid style={white!69.0196078431373!black},
ylabel={t},
ymin=0, ymax=0.12,
ytick style={color=black},
ytick={0,0.06,0.12},
width=\textwidth,
height=.25\textwidth
]
\addplot graphics [includegraphics cmd=\pgfimage,xmin=0.0025, xmax=0.9975, ymin=0, ymax=0.12] {Figures/Results/Rare/Code-000.png};
\node [draw,fill=white] at (0.9,0.1) {\(c_1\)};

\nextgroupplot[
colorbar,
colorbar style={ylabel={}},
colormap/blackwhite,
point meta max=0.188924923539162,
point meta min=-0.245918944478035,
tick align=outside,
tick pos=left,
x grid style={white!69.0196078431373!black},
xlabel={x},
xmin=0.0025, xmax=0.9975,
xtick style={color=black},
y grid style={white!69.0196078431373!black},
ylabel={t},
ymin=0, ymax=0.12,
ytick style={color=black},
ytick={0,0.06,0.12},
width=\textwidth,
height=.25\textwidth
]
\addplot graphics [includegraphics cmd=\pgfimage,xmin=0.0025, xmax=0.9975, ymin=0, ymax=0.12] {Figures/Results/Rare/Code-001.png};
\node [draw,fill=white] at (0.9,0.1) {\(c_2\)};

\nextgroupplot[
colorbar,
colorbar style={ylabel={}},
colormap/blackwhite,
point meta max=0.397124499082565,
point meta min=0.0549342148005962,
tick align=outside,
tick pos=left,
x grid style={white!69.0196078431373!black},
xlabel={x},
xmin=0.0025, xmax=0.9975,
xtick style={color=black},
y grid style={white!69.0196078431373!black},
ylabel={t},
ymin=0, ymax=0.12,
ytick style={color=black},
ytick={0,0.06,0.12},
width=\textwidth,
height=.25\textwidth
]
\addplot graphics [includegraphics cmd=\pgfimage,xmin=0.0025, xmax=0.9975, ymin=0, ymax=0.12] {Figures/Results/Rare/Code-002.png};
\node [draw,fill=white] at (0.9,0.1) {\(c_3\)};

\nextgroupplot[
colorbar,
colorbar style={ylabel={}},
colormap/blackwhite,
point meta max=0.00300436303950846,
point meta min=-0.165321439504623,
tick align=outside,
tick pos=left,
x grid style={white!69.0196078431373!black},
xlabel={x},
xmin=0.0025, xmax=0.9975,
xtick style={color=black},
y grid style={white!69.0196078431373!black},
ylabel={t},
ymin=0, ymax=0.12,
ytick style={color=black},
ytick={0,0.06,0.12},
width=\textwidth,
height=.25\textwidth
]
\addplot graphics [includegraphics cmd=\pgfimage,xmin=0.0025, xmax=0.9975, ymin=0, ymax=0.12] {Figures/Results/Rare/Code-003.png};
\node [draw,fill=white] at (0.9,0.1) {\(c_4\)};

\nextgroupplot[
colorbar,
colorbar style={ylabel={}},
colormap/blackwhite,
point meta max=-0.157852962613106,
point meta min=-0.423109173774719,
tick align=outside,
tick pos=left,
x grid style={white!69.0196078431373!black},
xlabel={x},
xmin=0.0025, xmax=0.9975,
xtick style={color=black},
y grid style={white!69.0196078431373!black},
ylabel={t},
ymin=0, ymax=0.12,
ytick style={color=black},
ytick={0,0.06,0.12},
width=\textwidth,
height=.25\textwidth
]
\addplot graphics [includegraphics cmd=\pgfimage,xmin=0.0025, xmax=0.9975, ymin=0, ymax=0.12] {Figures/Results/Rare/Code-004.png};
\node [draw,fill=white] at (0.9,0.1) {\(c_5\)};
\end{groupplot}

\end{tikzpicture}
}
	\caption{Intrinsic Variables of $\rare$}
\end{figure}
\subsection{Discussion and Outlook}
 One reason of the lacking ability of the CNN is the small number of samples, as described in \cref{Ch:ApB} and \cref{Ch:ROM}. The resulting trembles of the signal when calculating the macroscopic quantities is due the kernel approach of the CNN. During reconstruction the resolution of the output image is bounded by the size of the kernel, which leads to pixelation.\\
  What we see here is the known drawback of POD. Sharp fronts and especially advection dominated problems lead to a fast decaying kogolomorov n-width. These problems need a nonlienar ansatz, as described in \cref{Ch:ROM}.