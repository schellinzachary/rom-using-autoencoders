% !TEX root = master.tex

\chapter{Results}
\label{Ch:Results}
%\pagenumbering{arabic}
This chapter covers the evaluation of reconstructions \(\tilde{f}\) obtained from the FCNN and the CNN through a comparison against reconstructions obtained from POD. Additionally, an analysis of the interpretability of the intrinsic variables \(\idhy\) and \(\idrare\) is provided. This chapter concludes with the attempt to create new states of the FOM with the FCNN and the CNN, which can be viewed as an online phase of MOR.\\ 
To begin with the benchmarking of POD and neural networks, the number of parameters to obtain \(\tilde{f}\) is contrasted. Beforehand solely the number of trainable parameters that compose both neural networks were called \(\frepar\). For this comparison, \(\frepar\) is extended to also include all elements of the left and right singular vectors as well as the singular values of POD. Additionally the amount of intrinsic variables used for reconstruction is set to \(p=3\) and \(p=5\) for \(\hy\) and \(\rare\) respectively. An exception is the CNN, that uses \(p=5\) independent from rarefaction level. A summary is provided in \cref{Tab: Parameters}.   
\begin{table}[htp]
	\centering
	\caption{Amount of parameters \(\frepar\) used to reconstruct \(f\), number  of intrinsic variables \(p\) and the corresponding $\L2$-Error for POD, FCNN and CNN.}
	\begin{tabular*}{16cm}{ @{\extracolsep{\fill}} c c c c c c c @{} }
		\toprule
		Algorithm & \multicolumn{2}{c}{Parameters \(\frepar\)} & \multicolumn{2}{c}{Int. variables \(p\)}& \multicolumn{2}{c}{$\L2$-Error} \\ [.5ex]
		 & \(\hy\)&\(\rare\)&\(\hy\)&\(\rare\)&\(\hy\)&\(\rare\)\\   
		\hline
		POD     & 15129 & 25225 & 3 & 5 & 0.0205 & 0.0087 \\
		FCNN 	& 2683 & 3725 & 3 & 5 & 0.0008 & 0.0009 \\
		CNN   	& 8246 & 8246 & 5 & 5 &	0.025 & 0.027\\
		\bottomrule
	\end{tabular*} \label{Tab: Parameters}
\end{table}
POD uses with 15129 and 25225 parameters to reconstruct \(\hy\) and \(\rare\) respectively the largest amount of parameters of all three algorithms. These yield \(\L2\)-errors of 0.0205 and 0.0087 respectively. Interestingly, the elevation of \(p\) amounts to an increase of paramters by approximately 1.7 which is comparable to the FCNN with an approximate increase of 1.4. The FCNN, which holds the best \(\L2\)-errors of 0.0008 and 0.0009 for \(\hy\) and \(\rare\) respectively, does so with the least amount of parameters. For reconstructing \(\hy\) solely 2683 and for the reconstruction of \(\rare\) solely 3725 parameters are used, which is a fraction of the need for POD. The second most populous alhorithm is the CNN which uses 8246 parameters for both rarefaction levels. The resulting \(\L2\)-errors with 0.025 for \(\hy\) and 0.027 for \(\rare\) are the largest of all three algorithms. \\

Next a qualitative analysis with actual reconstructions is presented. For this purpose the \(\L2\)-error over time \(t\), seen in \cref{Fig:ErrTime}, is used to localize the most challenging snapshot for each algorithm.\\ 
\begin{figure}[tp!]
	% This file was created by tikzplotlib v0.9.6.
\begin{tikzpicture}
\definecolor{color0}{rgb}{0.12156862745098,0.466666666666667,0.705882352941177}

\begin{groupplot}[group style={group size=2 by 1,horizontal sep=2cm}]
\nextgroupplot[
legend cell align={left},
legend style={draw=none, at={(0,1)},anchor= north west},
tick align=outside,
tick pos=left,
x grid style={white!69.0196078431373!black},
xmin=-0.006, xmax=0.126,
xtick style={color=black},
y grid style={white!69.0196078431373!black},
ymin=-0.000841513772702749, ymax=0.0304345512636816,
ytick style={color=black},
ylabel={Relative Error},
xlabel={\(t\)},
axis lines=left,
width=0.47\textwidth,
height =.45\textwidth,
x tick label style={/pgf/number format/fixed}
]
\addplot [semithick, red, mark=o, mark size=2, mark options={solid}]
table {%
0 0.0041745608742312
0.005 0.00772826090461208
0.01 0.00998129651030254
0.015 0.0114814636830063
0.02 0.0127128633875819
0.025 0.0138212914705674
0.03 0.0148590810264679
0.035 0.0158480900427887
0.04 0.0167989301972853
0.045 0.0177175305866807
0.05 0.0186076563966316
0.055 0.0194719780627452
0.06 0.0203125628459335
0.065 0.021131115773516
0.07 0.0219291044243208
0.075 0.0227078270777979
0.08 0.0234684520909674
0.085 0.0242120420959511
0.09 0.0249395698621951
0.095 0.0256519293912289
0.1 0.0263499442110898
0.105 0.0270343740437529
0.11 0.0277059205812353
0.115 0.0283652327959761
0.12 0.029012911943846
};
\addlegendentry{POD}
\addplot [semithick, color0, mark=pentagon, mark size=2, mark options={solid}]
table {%
0 0.000580125547132904
0.005 0.0016071267939448
0.01 0.00188504649223781
0.015 0.00133036247337402
0.02 0.00132628189845874
0.025 0.00132716953321487
0.03 0.00138970165580601
0.035 0.00142017716840674
0.04 0.00145638808517116
0.045 0.00148681063630736
0.05 0.00153865172400739
0.055 0.0015731085384213
0.06 0.00160068094345799
0.065 0.00164077357612924
0.07 0.00168465107889107
0.075 0.00170978225630885
0.08 0.00174559426233016
0.085 0.00178917826233095
0.09 0.00182042226136015
0.095 0.0018497979559777
0.1 0.00189400391034215
0.105 0.00192810267450896
0.11 0.00196043475749194
0.115 0.00199731114052969
0.12 0.00204309455217968
};
\addlegendentry{FCNN}
\addplot [semithick, green!50!black, mark=triangle, mark size=2, mark options={solid,rotate=180}]
table {%
0 0.00649221194908023
0.005 0.00637732213363051
0.01 0.00686583481729031
0.015 0.00666404515504837
0.02 0.00614608032628894
0.025 0.0067147696390748
0.03 0.00703709293156862
0.035 0.00886726658791304
0.04 0.00913859903812408
0.045 0.00808194372802973
0.05 0.00891359616070986
0.055 0.00899100676178932
0.06 0.00881004240363836
0.065 0.009485456161201
0.07 0.00957572367042303
0.075 0.00939757097512484
0.08 0.0101738022640347
0.085 0.0107985194772482
0.09 0.0102939195930958
0.095 0.0112069239839911
0.1 0.0109974481165409
0.105 0.0116838105022907
0.11 0.0127648552879691
0.115 0.0127696730196476
0.12 0.0131859742105007
};
\addlegendentry{CNN}

\nextgroupplot[
legend cell align={left},
legend style={draw=none,at={(0,1)},anchor= north west},
tick align=outside,
tick pos=left,
x grid style={white!69.0196078431373!black},
xmin=-0.006, xmax=0.126,
xtick style={color=black},
y grid style={white!69.0196078431373!black},
ymin=-0.00230221234608083, ymax=0.0611092213046212,
ytick style={color=black},
ylabel={Relative Error},
xlabel={\(t\)},
axis lines=left,
width=0.47\textwidth,
height =.45\textwidth,
x tick label style={/pgf/number format/fixed}
]
\addplot [semithick, red, mark=o, mark size=2, mark options={solid}]
table {%
0 0.00740144721634158
0.005 0.0106893808997395
0.01 0.0149150434602834
0.015 0.0187372384735488
0.02 0.0219134756297143
0.025 0.0245262733618508
0.03 0.0266961103723662
0.035 0.0285235128498648
0.04 0.0300844182519668
0.045 0.0314352845718027
0.05 0.0326184730907696
0.055 0.0336663066570384
0.06 0.0346038953801234
0.065 0.0354510718869172
0.07 0.0362237260413102
0.075 0.0369347412336944
0.08 0.0375946650603757
0.085 0.038212200327165
0.09 0.0387945721638091
0.095 0.0393478079345977
0.1 0.0398769544960087
0.105 0.0403862495460136
0.11 0.0408792586959919
0.115 0.0413589864772983
0.12 0.0418279671630029
};
\addlegendentry{POD}
\addplot [semithick, color0, mark=pentagon, mark size=2, mark options={solid}]
table {%
0 0.000580125547132904
0.005 0.00316589483113868
0.01 0.00644069989189273
0.015 0.0094435622115595
0.02 0.0117462910277023
0.025 0.0135083335876847
0.03 0.0147541401649302
0.035 0.0156776008463723
0.04 0.016359257000976
0.045 0.0168387643606902
0.05 0.0171780580682302
0.055 0.0174023878750501
0.06 0.0175410884132009
0.065 0.0176139846919089
0.07 0.0176235927054341
0.075 0.0175937947736521
0.08 0.0175227229486254
0.085 0.0174309194625182
0.09 0.0173139350118248
0.095 0.0171816191398723
0.1 0.0170365979594161
0.105 0.0168830016501064
0.11 0.0167221999919594
0.115 0.0165578630537945
0.12 0.0163915040253931
};
\addlegendentry{FCNN}
\addplot [semithick, green!50!black, mark=triangle, mark size=2, mark options={solid,rotate=180}]
table {%
0 0.00956460367888212
0.005 0.0124108670279384
0.01 0.0172932539135218
0.015 0.0221939664334059
0.02 0.0249469373375177
0.025 0.0283007752150297
0.03 0.0311273168772459
0.035 0.0343553461134434
0.04 0.036633376032114
0.045 0.0377688743174076
0.05 0.0399965308606625
0.055 0.041291106492281
0.06 0.0428216233849525
0.065 0.0447116009891033
0.07 0.0460423566401005
0.075 0.0471654385328293
0.08 0.048737995326519
0.085 0.049955528229475
0.09 0.0508080013096333
0.095 0.0526490993797779
0.1 0.053377740085125
0.105 0.0546315461397171
0.11 0.0561474151909351
0.115 0.0573081187903881
0.12 0.0582268834114075
};
\addlegendentry{CNN}
\end{groupplot}

\end{tikzpicture}

	\caption{\(\L2\)-error over time for POD, FCNN and CNN. Results for $\hy$ are displayed on the left, the results for $\rare$ are displayed on the right.}
	\label{Fig:ErrTime}
\end{figure}
With POD and the CNN the last timestep at $t=0.12s$ for both rarefaction levels is the most rich in the $L2$-Error. In contrast, the FCNN does not show a distinct time dependence of the \(\L2\)-error. Nonetheless, struggles at the onset at around $t=0.005s$ with $\hy$ and around \(t=0.005s\) and \(t=0.0115\) (in the beginning and at the end) with $\rare$ can be observed for the FCNN. Examples of reconstructions $\tilde{f}(x,v,t_i)$ with $t_i=0.12s$ and $x \in [0.375,0.75]$ are given in \cref{Fig: ErrWorst}.\\
\begin{figure}[tp!]
	% This file was created by tikzplotlib v0.9.6.
\begin{tikzpicture}

\begin{groupplot}[
group style={group size=4 by 2,
	horizontal sep= 1.1cm,
	vertical sep = 1.5cm},
tick align=outside,
tick pos=left,
x grid style={white!69.0196078431373!black},
xmin=0.375, xmax=0.75,
xtick style={color=black},
y grid style={white!69.0196078431373!black},
ymin=-10, ymax=10,
ytick style={color=black},
height=.26\textwidth,
width=.26\textwidth,
xlabel={\(x\)},
ylabel={\(v\)},
y label style={yshift=-1.4em}
]
\nextgroupplot[
]
\addplot graphics [
includegraphics cmd=\pgfimage,
xmin=0.375, xmax=0.75,
ymin=-10, ymax=10
] {Figures/Chapter_5/ErrWorst-000.png};
\node[fill=white] at (axis cs:0.65,15) {FOM};
\nextgroupplot[
]
\addplot graphics [
includegraphics cmd=\pgfimage,
xmin=0.375, xmax=0.75,
ymin=-10, ymax=10
] {Figures/Chapter_5/ErrWorst-001.png};
\node[fill=white] at (axis cs:0.65,15) {POD};
\nextgroupplot[
]
\addplot graphics [
includegraphics cmd=\pgfimage,
xmin=0.375, xmax=0.75,
ymin=-10, ymax=10
] {Figures/Chapter_5/ErrWorst-002.png};
\node[fill=white] at (axis cs:0.65,15) {FCNN};
\nextgroupplot[
colorbar,
colorbar style={
	ylabel={$\tilde{f}$},
	ytick={0,0.1,.3,.39},
	yticklabels={0,0.1,0.3,0.39},
	y label style={yshift=1.3cm},
	ticklabel style={font=\footnotesize},
	tick align=outside,
	tick pos=right,
	width=0.1*\pgfkeysvalueof{/pgfplots/parent axis width},
	xshift=-0.2cm
},
colormap/blackwhite,
point meta max=0.397007430832041,
point meta min=8.50982895819395e-73,
]
\addplot graphics [
includegraphics cmd=\pgfimage,
xmin=0.375, xmax=0.75,
ymin=-10, ymax=10
] {Figures/Chapter_5/ErrWorst-003.png};
\node[fill=white] at (axis cs:0.65,15) {CNN};
\nextgroupplot[
]
\addplot graphics [
includegraphics cmd=\pgfimage,
xmin=0.375, xmax=0.75,
ymin=-10, ymax=10
] {Figures/Chapter_5/ErrWorst-004.png};
\node[fill=white] at (axis cs:0.65,15) {FOM};
\nextgroupplot[
]
\addplot graphics [
includegraphics cmd=\pgfimage,
xmin=0.375, xmax=0.75,
ymin=-10, ymax=10
] {Figures/Chapter_5/ErrWorst-005.png};
\node[fill=white] at (axis cs:0.65,15) {POD};
\nextgroupplot[
]
\addplot graphics [
includegraphics cmd=\pgfimage,
xmin=0.375, xmax=0.75,
ymin=-10, ymax=10
] {Figures/Chapter_5/ErrWorst-006.png};
\node[fill=white] at (axis cs:0.65,15) {FCNN};
\nextgroupplot[
colorbar,
colorbar style={
	ylabel={$\tilde{f}$},
	ytick={0,0.1,.3,.4},
	yticklabels={0,0.1,0.3,0.4},
	y label style={yshift=1.3cm},
	ticklabel style={font=\footnotesize},
	tick align=outside,
	tick pos=right,
	width=0.1*\pgfkeysvalueof{/pgfplots/parent axis width},
	xshift=-0.2cm
},
colormap/blackwhite,
point meta max=0.406101604565777,
point meta min=1.26406789996295e-34
]
\addplot graphics [
includegraphics cmd=\pgfimage,
xmin=0.375, xmax=0.75,
ymin=-10, ymax=10
] {Figures/Chapter_5/ErrWorst-007.png};
\node[fill=white] at (axis cs:0.65,15) {CNN};
\end{groupplot}

\end{tikzpicture}

	\caption{Comparison of the FOM solution \(f\) with three reconstructions \(\tilde{f}\) obtained from POD, the FCNN and the CNN. Reconstrucions are shown at \(t=0.12s\) for \(x\in [0.375,0.75]\). Case $\hy$ is displayed in the top row, $\rare$ in the bottom row. The colobars reference \(f\) and \(\tilde{f}\).}
	\label{Fig: ErrWorst}
\end{figure}
The FOM solution viewed as $f(x,v,t_i)$ has been introduced in \cref{Ch:BGK}. There, $f(x_j,v,t_i)$ is the probability distribution of the microscopic velocities $v$ at point $x_j$ in space at one moment $t_i$ in time for a gas.
With this in mind, a qualitative comparison between the three algorithms can be made considering the rendition of the velocity probabilities. Starting with \(\hy\), seen in the top row of \cref{Fig: ErrWorst} one can observe that \(\tilde{f}(x,v,t_i)\) starting around \(x=0.6\) gets defective for POD and the CNN. Noteworthy, here the probability distribution is thinner as the original with POD. This in turn leads to errors in the temperature \(T\) once passing \(x\approx 0.6\). Prominent qualitative deviations using the CNN are especially blurriness/pixelation of \(\tilde{f}(x,v,t_i)\) after \(x\approx 0.6\). In contrast the FCNN seems to reproduce the FOM solution almost exactly.\\
Continuing with a row further down in \cref{Fig: ErrWorst} and therefore with \(\rare\). The FCNN seems to reproduce the FOM solution without any visible drawback. Also POD seems to reproduce all important structures, except after \(x\approx 0.7\) around the contact discontinuity, some values for velocities with \(v>0\) appear to be missing. Again the CNN struggles with blurriness making \(\tilde{f}\) for both rarefaction levels look largely similar.    
\begin{figure}[H]
	% This file was created by tikzplotlib v0.9.6.
\begin{tikzpicture}

\definecolor{color0}{rgb}{0.12156862745098,0.466666666666667,0.705882352941177}

\begin{groupplot}[group style={group size=3 by 2,horizontal sep=1.5cm,vertical sep=1.1cm}]
\nextgroupplot[
legend cell align={left},
legend style={at={(1,1)},anchor=north east,fill opacity=0.1, draw opacity=1, text opacity=1,draw=none},
tick align=outside,
tick pos=left,
x grid style={white!69.0196078431373!black},
xmajorgrids,
xlabel={\(x\)},
xmin=-0.04725, xmax=1.04725,
xtick style={color=black},
y grid style={white!69.0196078431373!black},
ymajorgrids,
ylabel={\(\rho\)},
ymin=0.0787652333577474, ymax=1.04575335673797,
ytick style={color=black},
width=.35\textwidth,
height=.4\textwidth,
x label style={xshift=-.10em}
]
\addplot [semithick, black, dashed, mark=x,mark size=2,mark repeat=25, mark options={solid}]
table {%
0.0025 0.999999963320219
0.0075 0.999999963320219
0.0125 0.999999963320219
0.0175 0.999999963320219
0.0225 0.999999963320219
0.0275 0.999999963320219
0.0325 0.999999963320219
0.0375 0.999999963320219
0.0425 0.999999963320219
0.0475 0.999999963320219
0.0525 0.999999963320219
0.0575 0.999999963320219
0.0625 0.999999963320219
0.0675 0.999999963320219
0.0725 0.999999963320219
0.0775 0.999999963320219
0.0825 0.999999963320219
0.0875 0.999999963320219
0.0925 0.999999963320219
0.0975 0.999999963320219
0.1025 0.999999963320219
0.1075 0.999999963320219
0.1125 0.999999963320219
0.1175 0.99999990218725
0.1225 0.999999779921312
0.1275 0.999999535389436
0.1325 0.999999413123497
0.1375 0.999998862926777
0.1425 0.99999800706521
0.1475 0.999996539873955
0.1525 0.999994400220039
0.1575 0.99999067110893
0.1625 0.999984741210937
0.1675 0.99997538786668
0.1725 0.999960654821151
0.1775 0.999938524686373
0.1825 0.999904779287485
0.1875 0.999854833651812
0.1925 0.999782024285732
0.1975 0.999676814446082
0.2025 0.999528261331411
0.2075 0.999321570763221
0.2125 0.99903846398378
0.2175 0.998657727852846
0.2225 0.998153625390468
0.2275 0.997497362968249
0.2325 0.99665702917637
0.2375 0.995598572951097
0.2425 0.994286231505565
0.2475 0.992684731116662
0.2525 0.990759225992056
0.2575 0.988477988120837
0.2625 0.985812468406482
0.2675 0.982738213661389
0.2725 0.979236639462984
0.2775 0.975294846754808
0.2825 0.970904521453075
0.2875 0.966063951834654
0.2925 0.960775827750181
0.2975 0.955047118358123
0.3025 0.948889010991806
0.3075 0.94231550510113
0.3125 0.935342617523976
0.3175 0.927988627018073
0.3225 0.920272118005997
0.3275 0.912212653037829
0.3325 0.903829550131773
0.3375 0.895142555236816
0.3425 0.886169641445845
0.3475 0.876929209782527
0.3525 0.867438438611153
0.3575 0.857713405902569
0.3625 0.84776915036715
0.3675 0.837620466183393
0.3725 0.827280374673697
0.3775 0.816761958293426
0.3825 0.806076526641846
0.3875 0.79523612291385
0.3925 0.784251078581199
0.3975 0.773132764376127
0.4025 0.761891756302271
0.4075 0.750539363958897
0.4125 0.739087813939804
0.4175 0.72755067776411
0.4225 0.715944155668601
0.4275 0.704287198873667
0.4325 0.692604199433938
0.4375 0.680926212897667
0.4425 0.669293831556271
0.4475 0.657760913555439
0.4525 0.646399901463435
0.4575 0.635307385371282
0.4625 0.624610216189653
0.4675 0.614472535940317
0.4725 0.605094983027532
0.4775 0.59670429963332
0.4825 0.589521603706555
0.4875 0.583707980620555
0.4925 0.579297664837959
0.4975 0.576156347225874
0.5025 0.574000982137827
0.5075 0.572480116135035
0.5125 0.571264242514586
0.5175 0.570086332467886
0.5225 0.568721416669014
0.5275 0.566925146640875
0.5325 0.564365814893674
0.5375 0.560570557912191
0.5425 0.554906038137583
0.5475 0.546613473158616
0.5525 0.534909016046769
0.5575 0.519142212011875
0.5625 0.498978602580535
0.5675 0.474548706641564
0.5725 0.446512118363992
0.5775 0.416003312820043
0.5825 0.384466831500714
0.5875 0.353428247647408
0.5925 0.324264611953344
0.5975 0.298031599093706
0.6025 0.275379786124596
0.6075 0.256558289894691
0.6125 0.241481600663601
0.6175 0.229826477857736
0.6225 0.221130557549305
0.6275 0.21487615047357
0.6325 0.210551619529724
0.6375 0.207691956789066
0.6425 0.205900455132509
0.6475 0.204856793085734
0.6525 0.204314497800974
0.6575 0.204092799088894
0.6625 0.204064494524247
0.6675 0.204143249071561
0.6725 0.204271674156189
0.6775 0.204410170897459
0.6825 0.204527209966611
0.6875 0.204589748993898
0.6925 0.204552182784447
0.6975 0.20434177838839
0.7025 0.203837538376833
0.7075 0.202838136599614
0.7125 0.201017092435788
0.7175 0.197870639654306
0.7225 0.192691668485984
0.7275 0.184667370258233
0.7325 0.173289806414873
0.7375 0.159203853362646
0.7425 0.144954935098306
0.7475 0.134090918761033
0.7525 0.12823774264409
0.7575 0.125970038083883
0.7625 0.125265816847483
0.7675 0.125069113878103
0.7725 0.125016333200993
0.7775 0.125002356675955
0.7825 0.124998665772952
0.7875 0.12499770292869
0.7925 0.124997443113572
0.7975 0.124997389622224
0.8025 0.124997374338981
0.8075 0.12499736669736
0.8125 0.12499736669736
0.8175 0.12499736669736
0.8225 0.12499736669736
0.8275 0.12499736669736
0.8325 0.12499736669736
0.8375 0.12499736669736
0.8425 0.12499736669736
0.8475 0.12499736669736
0.8525 0.12499736669736
0.8575 0.12499736669736
0.8625 0.12499736669736
0.8675 0.12499736669736
0.8725 0.12499736669736
0.8775 0.12499736669736
0.8825 0.12499736669736
0.8875 0.12499736669736
0.8925 0.12499736669736
0.8975 0.12499736669736
0.9025 0.12499736669736
0.9075 0.12499736669736
0.9125 0.12499736669736
0.9175 0.12499736669736
0.9225 0.12499736669736
0.9275 0.12499736669736
0.9325 0.12499736669736
0.9375 0.12499736669736
0.9425 0.12499736669736
0.9475 0.12499736669736
0.9525 0.12499736669736
0.9575 0.12499736669736
0.9625 0.12499736669736
0.9675 0.12499736669736
0.9725 0.12499736669736
0.9775 0.12499736669736
0.9825 0.12499736669736
0.9875 0.12499736669736
0.9925 0.12499736669736
0.9975 0.12499736669736
};
\addlegendentry{FOM}
\addplot [semithick, red, mark=o,mark size=2, mark repeat=25, mark options={solid}]
table {%
0.0025 0.999548771594793
0.0075 0.999548771594787
0.0125 0.999548771594777
0.0175 0.999548771594753
0.0225 0.999548771594688
0.0275 0.99954877159453
0.0325 0.999548771594179
0.0375 0.999548771593437
0.0425 0.999548771591852
0.0475 0.999548771588456
0.0525 0.999548771581279
0.0575 0.999548771566277
0.0625 0.999548771535268
0.0675 0.999548771471879
0.0725 0.999548771343803
0.0775 0.999548771088022
0.0825 0.999548770583062
0.0875 0.999548769597802
0.0925 0.999548767698169
0.0975 0.99954876407943
0.1025 0.999548757269338
0.1075 0.999548744610781
0.1125 0.999548721373772
0.1175 0.999548679255706
0.1225 0.999548603890035
0.1275 0.999548470777807
0.1325 0.999548238760097
0.1375 0.999547839734983
0.1425 0.999547162765054
0.1475 0.999546030000005
0.1525 0.999544160954158
0.1575 0.999541120663193
0.1625 0.999536246184591
0.1675 0.999528544966733
0.1725 0.999516558049316
0.1775 0.999498181231678
0.1825 0.999470438699194
0.1875 0.999429206591679
0.1925 0.9993688890128
0.1975 0.999282056142071
0.2025 0.99915906311425
0.2075 0.998987678260307
0.2125 0.998752758572666
0.2175 0.998436016733616
0.2225 0.998015925419097
0.2275 0.997467798986495
0.2325 0.996764079322538
0.2375 0.99587483249879
0.2425 0.994768438710398
0.2475 0.993412433898535
0.2525 0.991774442038987
0.2575 0.989823126091609
0.2625 0.98752908500575
0.2675 0.984865633633234
0.2725 0.981809419515683
0.2775 0.978340851622089
0.2825 0.9744443373183
0.2875 0.970108341936711
0.2925 0.965325298291781
0.2975 0.960091400703041
0.3025 0.954406319994776
0.3075 0.94827287369326
0.3125 0.941696680663758
0.3175 0.934685823088954
0.3225 0.927250532091985
0.3275 0.91940290722078
0.3325 0.911156674872146
0.3375 0.902526986722904
0.3425 0.893530256338715
0.3475 0.884184030229781
0.3525 0.874506888547891
0.3575 0.864518370193878
0.3625 0.854238917174887
0.3675 0.843689833494142
0.3725 0.832893254596429
0.3775 0.821872124393158
0.3825 0.810650178169559
0.3875 0.799251931294829
0.3925 0.78770267574189
0.3975 0.77602848917673
0.4025 0.764256265096905
0.4075 0.752413777619085
0.4125 0.740529801647178
0.4175 0.728634319137727
0.4225 0.716758856123944
0.4275 0.70493701438836
0.4325 0.693205287449573
0.4375 0.681604283152755
0.4425 0.670180511628393
0.4475 0.65898892591229
0.4525 0.648096390752349
0.4575 0.637586127827967
0.4625 0.627562790490795
0.4675 0.618156902786947
0.4725 0.609525670234001
0.4775 0.6018447263329
0.4825 0.595283734240775
0.4875 0.589962041804688
0.4925 0.585893694609532
0.4975 0.582950451517276
0.5025 0.580875840796191
0.5075 0.579352251627723
0.5125 0.578077723544553
0.5175 0.576800623627844
0.5225 0.575296690289324
0.5275 0.573308767286148
0.5325 0.570477432918538
0.5375 0.566285778369634
0.5425 0.560040704121852
0.5475 0.550913667192523
0.5525 0.538054899617929
0.5575 0.52077187876436
0.5625 0.49873191332678
0.5675 0.472125596353103
0.5725 0.441728225170466
0.5775 0.408825407012185
0.5825 0.3750163990988
0.5875 0.341952318250269
0.5925 0.311085403139529
0.5975 0.28349274631892
0.6025 0.259803155850325
0.6075 0.240218452263519
0.6125 0.224596872533648
0.6175 0.21256136281911
0.6225 0.203604094956206
0.6275 0.197171596305905
0.6325 0.192725777382
0.6375 0.189782476887888
0.6425 0.187931638026962
0.6475 0.18684363567553
0.6525 0.186265936683102
0.6575 0.186013763883529
0.6625 0.185957830374123
0.6675 0.186011486873066
0.6725 0.186118822997236
0.6775 0.186244486244602
0.6825 0.186365323201625
0.6875 0.186463437237978
0.6925 0.18651985312962
0.6975 0.186507573177517
0.7025 0.186382227399486
0.7075 0.186067492116135
0.7125 0.185430535378378
0.7175 0.18423927878628
0.7225 0.18208804768065
0.7275 0.178277990908776
0.7325 0.171696753937413
0.7375 0.161099168034382
0.7425 0.147018646050487
0.7475 0.134243307515287
0.7525 0.127109461113498
0.7575 0.124412936576949
0.7625 0.123592848531648
0.7675 0.123365803728222
0.7725 0.123305048460296
0.7775 0.123288975392124
0.7825 0.123284741006781
0.7875 0.123283627872577
0.7925 0.123283335737192
0.7975 0.123283259191473
0.8025 0.123283239168794
0.8075 0.123283233940817
0.8125 0.123283232578456
0.8175 0.123283232224191
0.8225 0.123283232132279
0.8275 0.123283232108493
0.8325 0.123283232102354
0.8375 0.123283232100774
0.8425 0.123283232100368
0.8475 0.123283232100264
0.8525 0.123283232100238
0.8575 0.123283232100232
0.8625 0.12328323210023
0.8675 0.12328323210023
0.8725 0.12328323210023
0.8775 0.12328323210023
0.8825 0.12328323210023
0.8875 0.123283232100229
0.8925 0.123283232100229
0.8975 0.123283232100229
0.9025 0.123283232100229
0.9075 0.123283232100229
0.9125 0.123283232100229
0.9175 0.123283232100229
0.9225 0.123283232100229
0.9275 0.123283232100229
0.9325 0.123283232100229
0.9375 0.123283232100229
0.9425 0.123283232100229
0.9475 0.123283232100229
0.9525 0.123283232100229
0.9575 0.123283232100229
0.9625 0.123283232100229
0.9675 0.123283232100229
0.9725 0.123283232100229
0.9775 0.123283232100229
0.9825 0.123283232100229
0.9875 0.123283232100229
0.9925 0.123283232100229
0.9975 0.123283232100229
};
\addlegendentry{POD}
\addplot [semithick, color0, mark=pentagon,mark size=2, mark repeat=25, mark options={solid}]
table {%
0.0025 0.999939968236364
0.0075 0.999939968236364
0.0125 0.999939968236364
0.0175 0.999939968236364
0.0225 0.999939968236364
0.0275 0.999939968236364
0.0325 0.999939968236364
0.0375 0.999939968236364
0.0425 0.999939968236364
0.0475 0.999939968236364
0.0525 0.999939968236364
0.0575 0.999939968236364
0.0625 0.999939968236364
0.0675 0.999939968236364
0.0725 0.999939968236364
0.0775 0.999939968236364
0.0825 0.999939968236364
0.0875 0.999939968236364
0.0925 0.999939968236364
0.0975 0.999939891820153
0.1025 0.999939896596166
0.1075 0.999939795822287
0.1125 0.999939862208871
0.1175 0.999939878924917
0.1225 0.999939654452296
0.1275 0.999939597617739
0.1325 0.999939325504387
0.1375 0.999938859007297
0.1425 0.999938065592104
0.1475 0.999936676130463
0.1525 0.999934615400166
0.1575 0.99993094897423
0.1625 0.999925315666657
0.1675 0.999916495683675
0.1725 0.999902459816673
0.1775 0.999881169782617
0.1825 0.999849121898222
0.1875 0.999801299439218
0.1925 0.999731557825819
0.1975 0.999631308472882
0.2025 0.999489218020477
0.2075 0.99929166957736
0.2125 0.999021155353731
0.2175 0.998656816828327
0.2225 0.998174267319532
0.2275 0.997545471391044
0.2325 0.996739041681091
0.2375 0.995721569260917
0.2425 0.994457296358469
0.2475 0.992910082046038
0.2525 0.991043596862791
0.2575 0.988823653509219
0.2625 0.986217566264363
0.2675 0.983196732779153
0.2725 0.979736232055494
0.2775 0.97581622369874
0.2825 0.971420934805885
0.2875 0.966540744456534
0.2925 0.961170660761686
0.2975 0.955310777331201
0.3025 0.948965688092777
0.3075 0.942144474396721
0.3125 0.934979659743989
0.3175 0.927785223421569
0.3225 0.920081615734559
0.3275 0.911964373543667
0.3325 0.903547288229068
0.3375 0.89501510720509
0.3425 0.886130499390837
0.3475 0.876919190662029
0.3525 0.867408619692119
0.3575 0.857626938571532
0.3625 0.847603971950519
0.3675 0.83736945851109
0.3725 0.826953578955279
0.3775 0.816385962594396
0.3825 0.805695764362239
0.3875 0.795205382224268
0.3925 0.784419378480659
0.3975 0.773457267727607
0.4025 0.762336483129706
0.4075 0.751073205819688
0.4125 0.739683521887622
0.4175 0.728183065732129
0.4225 0.716587809296564
0.4275 0.70491533439893
0.4325 0.693185986616673
0.4375 0.681425008612374
0.4425 0.669666346855079
0.4475 0.657956207720324
0.4525 0.646394694056839
0.4575 0.635514160952507
0.4625 0.624949089251459
0.4675 0.61485644382162
0.4725 0.605438914961922
0.4775 0.596935878722713
0.4825 0.589592156525797
0.4875 0.583598525979771
0.4925 0.579017400860977
0.4975 0.575732599036434
0.5025 0.573749198482778
0.5075 0.572390212462499
0.5125 0.571257876685988
0.5175 0.570111481043009
0.5225 0.568769076743569
0.5275 0.567027286339838
0.5325 0.564539783920806
0.5375 0.560837504334557
0.5425 0.555283192258615
0.5475 0.547092540117984
0.5525 0.535416085249147
0.5575 0.519480185392193
0.5625 0.499157757044603
0.5675 0.475012445105956
0.5725 0.446823057360374
0.5775 0.41563895650399
0.5825 0.384757045704203
0.5875 0.353900500788138
0.5925 0.324724445549341
0.5975 0.298561543727723
0.6025 0.275809307439396
0.6075 0.256807394086932
0.6125 0.241533189128416
0.6175 0.22969940975786
0.6225 0.220858842397156
0.6275 0.214495856362658
0.6325 0.210094252266945
0.6375 0.207182188661626
0.6425 0.205356042282895
0.6475 0.204289610354373
0.6525 0.203732133437044
0.6575 0.203499824692232
0.6625 0.203463445441463
0.6675 0.203535807772707
0.6725 0.203659435710273
0.6775 0.203795783603803
0.6825 0.20391557747737
0.6875 0.203990367575525
0.6925 0.203982203339155
0.6975 0.203831830444053
0.7025 0.203439521794327
0.7075 0.202635776681396
0.7125 0.201135637978904
0.7175 0.198469977133358
0.7225 0.193903454674933
0.7275 0.186410662837518
0.7325 0.174961739386886
0.7375 0.159579780645286
0.7425 0.144058139158938
0.7475 0.133231449872255
0.7525 0.127834568922527
0.7575 0.125873483574161
0.7625 0.125272463028056
0.7675 0.125032803998926
0.7725 0.124968871569786
0.7775 0.124951929856951
0.7825 0.124947448762564
0.7875 0.12494626478889
0.7925 0.124945930467966
0.7975 0.124945836858107
0.8025 0.124945856678562
0.8075 0.124945856678562
0.8125 0.124945856678562
0.8175 0.124945822052467
0.8225 0.124945822052467
0.8275 0.124945822052467
0.8325 0.124945822052467
0.8375 0.124945822052467
0.8425 0.124945822052467
0.8475 0.124945822052467
0.8525 0.124945822052467
0.8575 0.124945822052467
0.8625 0.124945822052467
0.8675 0.124945822052467
0.8725 0.124945822052467
0.8775 0.124945822052467
0.8825 0.124945822052467
0.8875 0.124945822052467
0.8925 0.124945822052467
0.8975 0.124945822052467
0.9025 0.124945822052467
0.9075 0.124945822052467
0.9125 0.124945822052467
0.9175 0.124945822052467
0.9225 0.124945822052467
0.9275 0.124945822052467
0.9325 0.124945822052467
0.9375 0.124945822052467
0.9425 0.124945822052467
0.9475 0.124945822052467
0.9525 0.124945822052467
0.9575 0.124945822052467
0.9625 0.124945822052467
0.9675 0.124945822052467
0.9725 0.124945822052467
0.9775 0.124945822052467
0.9825 0.124945822052467
0.9875 0.124945822052467
0.9925 0.124945857633765
0.9975 0.124945857633765
};
\addlegendentry{FCNN}
\addplot [semithick, green!50!black, mark=triangle,mark size=2, mark repeat=25, mark options={solid,rotate=180}, only marks]
table {%
0.0025 0.997826564006316
0.0075 0.999634082500751
0.0125 0.999670150952461
0.0175 1.00089586698092
0.0225 1.00048700968424
0.0275 0.999750051742945
0.0325 1.00179935112978
0.0375 0.998835991590451
0.0425 1.00054997664232
0.0475 1.00057424643101
0.0525 0.997986365587283
0.0575 0.998990902533898
0.0625 1.00076326957116
0.0675 1.00147754718096
0.0725 0.999688368577223
0.0775 0.999325605539175
0.0825 1.00170001005515
0.0875 1.00006812658065
0.0925 1.00029890353863
0.0975 0.999496227655655
0.1025 0.997886718847813
0.1075 0.999347613408015
0.1125 1.00085362409934
0.1175 1.00004006654788
0.1225 0.999976182595277
0.1275 1.00009961005969
0.1325 1.00124609776032
0.1375 1.00066771874061
0.1425 1.00034903257321
0.1475 0.999733912639129
0.1525 0.99842383311345
0.1575 0.999465783437093
0.1625 1.00011764428554
0.1675 1.00016074302869
0.1725 0.999795045608129
0.1775 1.00020836561154
0.1825 1.00125098839784
0.1875 1.00000894986666
0.1925 0.999992321699093
0.1975 0.999634877229348
0.2025 0.997431094829853
0.2075 0.999444631429819
0.2125 0.999194780985514
0.2175 0.999104304191394
0.2225 0.998374804472312
0.2275 0.99752358901195
0.2325 0.997723982884334
0.2375 0.995986400506435
0.2425 0.994922197782076
0.2475 0.993364713130853
0.2525 0.98899089373075
0.2575 0.988139922802265
0.2625 0.985307326683631
0.2675 0.983291161365998
0.2725 0.979989614242162
0.2775 0.975429767217391
0.2825 0.973076147910876
0.2875 0.968115024077587
0.2925 0.964070100050706
0.2975 0.958142158312675
0.3025 0.947710306216509
0.3075 0.941470892001421
0.3125 0.936301427009778
0.3175 0.929134442256047
0.3225 0.921500340486184
0.3275 0.913304365598238
0.3325 0.905694655883006
0.3375 0.898112517136794
0.3425 0.892012180426182
0.3475 0.880463001055595
0.3525 0.868978867164025
0.3575 0.857521142715063
0.3625 0.851284296084673
0.3675 0.839678996648544
0.3725 0.828442084483611
0.3775 0.818615632179456
0.3825 0.806652643741705
0.3875 0.796960194905599
0.3925 0.789229319645808
0.3975 0.774383300389999
0.4025 0.764610706231533
0.4075 0.752236109513503
0.4125 0.741357069749099
0.4175 0.728611640441112
0.4225 0.71491308701344
0.4275 0.703751185001471
0.4325 0.691442856421837
0.4375 0.681317219367394
0.4425 0.670965023529835
0.4475 0.654357885703062
0.4525 0.648130331283961
0.4575 0.637230200645251
0.4625 0.626791073725774
0.4675 0.613789008213923
0.4725 0.603192830697084
0.4775 0.596600556984926
0.4825 0.591698243067815
0.4875 0.585167469122471
0.4925 0.58006512813079
0.4975 0.574537179408929
0.5025 0.572046438852946
0.5075 0.571879484714606
0.5125 0.573569872440436
0.5175 0.571612089108198
0.5225 0.568160643944373
0.5275 0.568164800986265
0.5325 0.566617892338679
0.5375 0.561307026789739
0.5425 0.557518311035939
0.5475 0.549704967400967
0.5525 0.531966747381748
0.5575 0.516557510082538
0.5625 0.499046888106909
0.5675 0.47705916258005
0.5725 0.444945372067965
0.5775 0.41721414297055
0.5825 0.382595031689375
0.5875 0.351334412892659
0.5925 0.324990046329987
0.5975 0.29638042816749
0.6025 0.269737029686952
0.6075 0.255421980833396
0.6125 0.242452422777812
0.6175 0.226303415420728
0.6225 0.218887711182619
0.6275 0.211986211630014
0.6325 0.204669053737934
0.6375 0.205674156164512
0.6425 0.20386951091962
0.6475 0.203292201726865
0.6525 0.199740452644152
0.6575 0.199890916164105
0.6625 0.201896459628374
0.6675 0.199385866140708
0.6725 0.200673861381335
0.6775 0.201314122248919
0.6825 0.201220909754435
0.6875 0.202678839365641
0.6925 0.200926050161704
0.6975 0.201315543590448
0.7025 0.198644674741305
0.7075 0.20112170622899
0.7125 0.200702196512467
0.7175 0.195979078610738
0.7225 0.191836402966426
0.7275 0.183470860505715
0.7325 0.173079035221002
0.7375 0.160602942491189
0.7425 0.146515476397979
0.7475 0.133639100270394
0.7525 0.126855396307432
0.7575 0.12492203559631
0.7625 0.127516190210978
0.7675 0.123466971593025
0.7725 0.124990153007018
0.7775 0.125107138584822
0.7825 0.12285223374
0.7875 0.125900140175453
0.7925 0.125363919979487
0.7975 0.124922822683285
0.8025 0.123820465344649
0.8075 0.123094549545875
0.8125 0.127414900522966
0.8175 0.12337829822149
0.8225 0.125258610798762
0.8275 0.124610868784098
0.8325 0.12319470063234
0.8375 0.126051153892126
0.8425 0.124572874643864
0.8475 0.124192971449632
0.8525 0.123434219604883
0.8575 0.123454859623542
0.8625 0.127833386262258
0.8675 0.123612704949501
0.8725 0.125545607163356
0.8775 0.124685840728955
0.8825 0.123431055973738
0.8875 0.125218278322464
0.8925 0.124563834606073
0.8975 0.124385952949524
0.9025 0.122719238965939
0.9075 0.12347678343455
0.9125 0.12764199421956
0.9175 0.124149498267051
0.9225 0.125206250410814
0.9275 0.124200116365384
0.9325 0.123076607019473
0.9375 0.125165398304279
0.9425 0.124789606302212
0.9475 0.125109224747389
0.9525 0.123349282986079
0.9575 0.123584935298333
0.9625 0.127726541115687
0.9675 0.123196954910572
0.9725 0.12529276120357
0.9775 0.125617744066776
0.9825 0.123182252431527
0.9875 0.125945210456848
0.9925 0.124717201942053
0.9975 0.12518231685345
};
\addlegendentry{CNN}

\nextgroupplot[
legend cell align={left},
legend style={at={(0.0,1)},anchor=north west,fill opacity=0.1, draw opacity=1, text opacity=1, draw=none},
tick align=outside,
tick pos=left,
x grid style={white!69.0196078431373!black},
xmajorgrids,
xlabel={\(x\)},
xmin=-0.04725, xmax=1.04725,
xtick style={color=black},
y grid style={white!69.0196078431373!black},
ymajorgrids,
ylabel={\(\rho u\)},
ymin=-0.0237175295924032, ymax=0.479965021862334,
ytick style={color=black},
width=.37\textwidth,
height=.4\textwidth
]
\addplot [semithick, black, dashed, mark=x,mark size=2, mark repeat=25, mark options={solid}]
table {%
0.0025 -2.9701575942379e-17
0.0075 -2.9701575942379e-17
0.0125 -2.9701575942379e-17
0.0175 -2.9701575942379e-17
0.0225 -2.9701575942379e-17
0.0275 -2.9701575942379e-17
0.0325 -2.9701575942379e-17
0.0375 -2.9701575942379e-17
0.0425 -2.9701575942379e-17
0.0475 -2.9701575942379e-17
0.0525 -2.9701575942379e-17
0.0575 -2.9701575942379e-17
0.0625 -2.9701575942379e-17
0.0675 -3.58603575113451e-17
0.0725 -3.58638102426239e-17
0.0775 5.6805661435368e-13
0.0825 1.99042476361002e-10
0.0875 2.20837991649488e-09
0.0925 2.93997368028267e-09
0.0975 1.31357376888643e-08
0.1025 2.38145813217986e-08
0.1075 4.87195462126188e-08
0.1125 1.18482312201542e-07
0.1175 2.0095587206295e-07
0.1225 3.83396059768099e-07
0.1275 6.58121348779825e-07
0.1325 1.16128545818365e-06
0.1375 2.03225357455255e-06
0.1425 3.48646389712185e-06
0.1475 5.97480063111906e-06
0.1525 1.00168031352998e-05
0.1575 1.66154888208619e-05
0.1625 2.71792475439631e-05
0.1675 4.38360783100733e-05
0.1725 6.9738147530809e-05
0.1775 0.000109360034940665
0.1825 0.000169110916869244
0.1875 0.000257759063581096
0.1925 0.000387167364876908
0.1975 0.000573117748608961
0.2025 0.000835869786797033
0.2075 0.00120107973443098
0.2125 0.00170018895588471
0.2175 0.00237086844167696
0.2225 0.00325687357947661
0.2275 0.00440760492585291
0.2325 0.0058770045036118
0.2375 0.00772213506918395
0.2425 0.0100010852238978
0.2475 0.0127705763429908
0.2525 0.0160835079558624
0.2575 0.0199864997521351
0.2625 0.0245176736126449
0.2675 0.029704885273648
0.2725 0.0355645653482579
0.2775 0.0421012893858932
0.2825 0.0493077793042677
0.2875 0.0571655985840627
0.2925 0.0656463751672976
0.2975 0.07471303783688
0.3025 0.0843216061489801
0.3075 0.0944227378619742
0.3125 0.104963347960292
0.3175 0.115888113820293
0.3225 0.127140629164304
0.3275 0.138664664012591
0.3325 0.150404853532333
0.3375 0.162307518067251
0.3425 0.174321041457807
0.3475 0.186396266022466
0.3525 0.198486618269269
0.3575 0.210548312786717
0.3625 0.222540185550889
0.3675 0.234423749038626
0.3725 0.24616309786956
0.3775 0.257724580624995
0.3825 0.269076805027718
0.3875 0.280190349490242
0.3925 0.291037541508123
0.3975 0.301592220966071
0.4025 0.311829593524858
0.4075 0.321725891157009
0.4125 0.331258108006691
0.4175 0.34040380705866
0.4225 0.349140751931193
0.4275 0.357446640915221
0.4325 0.365298701484088
0.4375 0.372673393425291
0.4425 0.379546061984702
0.4475 0.3858904907899
0.4525 0.391678889356953
0.4575 0.396881980119472
0.4625 0.401469837523235
0.4675 0.405414023679009
0.4725 0.408691643888321
0.4775 0.411292021579195
0.4825 0.413226034663786
0.4875 0.41453519470013
0.4925 0.415295784669826
0.4975 0.415611525216486
0.5025 0.415593772491148
0.5075 0.415336270698027
0.5125 0.41489544485938
0.5175 0.414279735123743
0.5225 0.413441921116945
0.5275 0.412265562656191
0.5325 0.41054307470679
0.5375 0.407952698513731
0.5425 0.404050230313847
0.5475 0.398294142046233
0.5525 0.390116252725716
0.5575 0.379034700041386
0.5625 0.364785466199088
0.5675 0.347433290718267
0.5725 0.327421677785053
0.5775 0.305538970309093
0.5825 0.282804902037727
0.5875 0.260309643297353
0.5925 0.239050560230286
0.5975 0.219807538955069
0.6025 0.20307934623608
0.6075 0.189081108725716
0.6125 0.17778662249403
0.6175 0.168993003883264
0.6225 0.162388542113512
0.6275 0.157611733837737
0.6325 0.154296588877
0.6375 0.152103708794801
0.6425 0.150738703044433
0.6475 0.149960169583879
0.6525 0.149579864110399
0.6575 0.149457424423213
0.6625 0.149491946494713
0.6675 0.149612014228558
0.6725 0.149765301786734
0.6775 0.149907796060004
0.6825 0.149991959449217
0.6875 0.149952034921847
0.6925 0.149683391104505
0.6975 0.149011228633714
0.7025 0.147642222218352
0.7075 0.145092235684392
0.7125 0.140589882050874
0.7175 0.132983875095863
0.7225 0.120763964502009
0.7275 0.10247138701595
0.7325 0.0779153039253849
0.7375 0.0500927752607743
0.7425 0.0255869634889079
0.7475 0.010102976420237
0.7525 0.00323067095348623
0.7575 0.000916686735148521
0.7625 0.000247536233262668
0.7675 6.57017337512221e-05
0.7725 1.73342190031844e-05
0.7775 4.56262885444554e-06
0.7825 1.19902286403957e-06
0.7875 3.14993485349219e-07
0.7925 8.20702035475298e-08
0.7975 2.12618186030591e-08
0.8025 5.83523126442528e-09
0.8075 1.06852817499928e-09
0.8125 1.53347718367724e-10
0.8175 1.33942684000716e-11
0.8225 1.28520434948373e-15
0.8275 5.28135721761187e-19
0.8325 5.28135721758471e-19
0.8375 5.28135721758267e-19
0.8425 5.2813572175825e-19
0.8475 5.28135721758248e-19
0.8525 5.28135721758248e-19
0.8575 5.28135721758248e-19
0.8625 5.28135721758248e-19
0.8675 5.28135721758248e-19
0.8725 5.28135721758248e-19
0.8775 5.28135721758248e-19
0.8825 5.28135721758248e-19
0.8875 5.28135721758248e-19
0.8925 5.28135721758248e-19
0.8975 5.28135721758248e-19
0.9025 5.28135721758248e-19
0.9075 5.28135721758248e-19
0.9125 5.28135721758248e-19
0.9175 5.28135721758248e-19
0.9225 5.28135721758248e-19
0.9275 5.28135721758248e-19
0.9325 5.28135721758248e-19
0.9375 5.28135721758248e-19
0.9425 5.28135721758248e-19
0.9475 5.28135721758248e-19
0.9525 5.28135721758248e-19
0.9575 5.28135721758248e-19
0.9625 5.28135721758248e-19
0.9675 5.28135721758248e-19
0.9725 5.28135721758248e-19
0.9775 5.28135721758248e-19
0.9825 5.28135721758248e-19
0.9875 5.28135721758248e-19
0.9925 5.28135721758248e-19
0.9975 5.28135721758248e-19
};
\addlegendentry{FOM}
\addplot [semithick, red, mark=o,mark size=2, mark repeat=25, mark options={solid}]
table {%
0.0025 0.00393237574162307
0.0075 0.00393237574163614
0.0125 0.00393237574166477
0.0175 0.00393237574171964
0.0225 0.00393237574182809
0.0275 0.00393237574206728
0.0325 0.00393237574258346
0.0375 0.00393237574369718
0.0425 0.00393237574608892
0.0475 0.00393237575118406
0.0525 0.00393237576192968
0.0575 0.00393237578436057
0.0625 0.00393237583066469
0.0675 0.00393237592520066
0.0725 0.00393237611603382
0.0775 0.00393237649685655
0.0825 0.00393237724804053
0.0875 0.00393237871244595
0.0925 0.00393238153343673
0.0975 0.00393238690255335
0.1025 0.00393239699736137
0.1075 0.0039324157437772
0.1125 0.00393245012270637
0.1175 0.00393251237378752
0.1225 0.00393262365100663
0.1275 0.00393281998348956
0.1325 0.00393316182416166
0.1375 0.0039337490664989
0.1425 0.00393474421179744
0.1475 0.00393640740084916
0.1525 0.0039391482816405
0.1575 0.00394360111542185
0.1625 0.00395073099413663
0.1675 0.00396198030874983
0.1725 0.00397946528925886
0.1775 0.00400623200805109
0.1825 0.00404657906024217
0.1875 0.00410644954553326
0.1925 0.00419388745402029
0.1975 0.00431954295975657
0.2025 0.00449719796363842
0.2075 0.00474426891353296
0.2125 0.00508223086272865
0.2175 0.00553689809257864
0.2225 0.00613849581716569
0.2275 0.006921467195858
0.2325 0.00792398103448396
0.2375 0.00918713651402677
0.2425 0.0107538976560125
0.2475 0.012667825597595
0.2525 0.0149717041880112
0.2575 0.0177061683886042
0.2625 0.0209084428487252
0.2675 0.024611280815639
0.2725 0.0288421653284213
0.2775 0.0336228013599424
0.2825 0.0389688952699348
0.2875 0.0448901914153248
0.2925 0.0513907177843819
0.2975 0.0584691837024481
0.3025 0.0661194719396857
0.3075 0.0743311728872895
0.3125 0.0830901175039003
0.3175 0.0923788763200986
0.3225 0.102177202267955
0.3275 0.112462404393532
0.3325 0.12320964705733
0.3375 0.134392174881238
0.3425 0.145981467579303
0.3475 0.157947331179376
0.3525 0.170257933324466
0.3575 0.182879790651119
0.3625 0.195777715953219
0.3675 0.208914732174675
0.3725 0.222251959400929
0.3775 0.235748480055173
0.3825 0.249361186532075
0.3875 0.263044614573805
0.3925 0.276750764854394
0.3975 0.290428914537441
0.4025 0.304025420087843
0.4075 0.317483512493661
0.4125 0.330743086559333
0.4175 0.343740487563639
0.4225 0.356408302237943
0.4275 0.368675168322387
0.4325 0.380465630717854
0.4375 0.391700097277873
0.4425 0.402294991460353
0.4475 0.41216327443322
0.4525 0.421215631784808
0.4575 0.429362804071025
0.4625 0.436519778058944
0.4675 0.442612762846729
0.4725 0.447589790767143
0.4775 0.451434865822219
0.4825 0.454183160773785
0.4875 0.455930953641405
0.4925 0.45683136196469
0.4975 0.457070360432574
0.5025 0.456828526427732
0.5075 0.456244022962827
0.5125 0.455389804273891
0.5175 0.454263808217002
0.5225 0.45277891087297
0.5275 0.450740151095367
0.5325 0.44780822825044
0.5375 0.443463394119888
0.5425 0.436996432917189
0.5475 0.427557383232156
0.5525 0.414281774058628
0.5575 0.396486578936554
0.5625 0.373891731431095
0.5675 0.346795295447884
0.5725 0.316129707754751
0.5775 0.283360071350005
0.5825 0.250241893082791
0.5875 0.218509256206492
0.5925 0.189589086473467
0.5975 0.164421155244688
0.6025 0.143416227206516
0.6075 0.126531603798344
0.6125 0.113410083472511
0.6175 0.10352585862365
0.6225 0.0963008073731597
0.6275 0.0911796275383113
0.6325 0.0876693019577585
0.6375 0.0853544433776073
0.6425 0.0838988479913904
0.6475 0.0830399641168731
0.6525 0.0825799093956901
0.6575 0.0823749451883289
0.6625 0.0823245853726994
0.6675 0.0823612015926955
0.6725 0.08244072511115
0.6775 0.0825347283366435
0.6825 0.0826238225116309
0.6875 0.082691972635448
0.6925 0.0827209975330994
0.6975 0.0826841077247918
0.7025 0.0825366468183002
0.7075 0.0822008578347515
0.7125 0.0815386693538691
0.7175 0.080300412577288
0.7225 0.0780249726353877
0.7275 0.0738497032187504
0.7325 0.0662278274931491
0.7375 0.053036060513252
0.7425 0.0341561499016392
0.7475 0.0160919951404636
0.7525 0.00586903038490919
0.7575 0.0020760331315755
0.7625 0.000946018726904214
0.7675 0.00063641683158109
0.7725 0.000553883596154878
0.7775 0.000532074986526063
0.7825 0.000526331645229695
0.7875 0.000524822013836168
0.7925 0.000524425840860136
0.7975 0.00052432203885198
0.8025 0.000524294887453232
0.8075 0.000524287798399671
0.8125 0.000524285951129817
0.8175 0.00052428547078955
0.8225 0.000524285346174678
0.8275 0.000524285313925921
0.8325 0.00052428530560256
0.8375 0.000524285303460556
0.8425 0.000524285302910899
0.8475 0.000524285302770344
0.8525 0.000524285302734448
0.8575 0.000524285302725269
0.8625 0.000524285302722924
0.8675 0.000524285302722314
0.8725 0.000524285302722221
0.8775 0.000524285302722232
0.8825 0.000524285302722232
0.8875 0.000524285302722219
0.8925 0.000524285302722228
0.8975 0.000524285302722223
0.9025 0.000524285302722223
0.9075 0.000524285302722223
0.9125 0.000524285302722223
0.9175 0.000524285302722223
0.9225 0.000524285302722223
0.9275 0.000524285302722223
0.9325 0.000524285302722223
0.9375 0.000524285302722223
0.9425 0.000524285302722223
0.9475 0.000524285302722223
0.9525 0.000524285302722223
0.9575 0.000524285302722223
0.9625 0.000524285302722223
0.9675 0.000524285302722223
0.9725 0.000524285302722223
0.9775 0.000524285302722223
0.9825 0.000524285302722223
0.9875 0.000524285302722223
0.9925 0.000524285302722223
0.9975 0.000524285302722223
};
\addlegendentry{POD}
\addplot [semithick, color0, mark=pentagon,mark size=2, mark repeat=25, mark options={solid}]
table {%
0.0025 -0.000632143126926788
0.0075 -0.000632143126926788
0.0125 -0.000632143126926788
0.0175 -0.000632143126926788
0.0225 -0.000632143126926788
0.0275 -0.000632143126926788
0.0325 -0.000632143126926788
0.0375 -0.000632143126926788
0.0425 -0.000632143126926788
0.0475 -0.000632143126926788
0.0525 -0.000632143126926788
0.0575 -0.000632143126926788
0.0625 -0.000632143126926788
0.0675 -0.000632143126926788
0.0725 -0.000632143126926788
0.0775 -0.000632143126926788
0.0825 -0.000632143126926788
0.0875 -0.000632143126926788
0.0925 -0.000632143126926788
0.0975 -0.000632173375010415
0.1025 -0.000632104061587989
0.1075 -0.000632007194243283
0.1125 -0.000632120716403293
0.1175 -0.000631910449360469
0.1225 -0.000631615438698764
0.1275 -0.00063152885815176
0.1325 -0.000630726702241898
0.1375 -0.000630169347624127
0.1425 -0.000628958475199835
0.1475 -0.000626601114683022
0.1525 -0.000622838565358478
0.1575 -0.000616276415063646
0.1625 -0.000605655051547406
0.1675 -0.000589919465739111
0.1725 -0.00056527024728935
0.1775 -0.000527699678195745
0.1825 -0.00047049709252235
0.1875 -0.00038642168686745
0.1925 -0.000263097280248266
0.1975 -8.63857408109713e-05
0.2025 0.000164091843011961
0.2075 0.00051175814212581
0.2125 0.000986357952451564
0.2175 0.00162556869107044
0.2225 0.00246904134005801
0.2275 0.00356540961324542
0.2325 0.0049675836850553
0.2375 0.00673055279581158
0.2425 0.00891387756524476
0.2475 0.0115732176818399
0.2525 0.0147664071753239
0.2575 0.0185438069924565
0.2625 0.0229509243035026
0.2675 0.0280261224345165
0.2725 0.0337981312479983
0.2775 0.0402849387098288
0.2825 0.0474968248516576
0.2875 0.055432582168762
0.2925 0.0640818716947143
0.2975 0.0734254435246062
0.3025 0.0834375517326484
0.3075 0.0940848816061278
0.3125 0.105425631713164
0.3175 0.11735601301068
0.3225 0.127535347167322
0.3275 0.138073698338917
0.3325 0.149057239274458
0.3375 0.160685125254658
0.3425 0.1725432862216
0.3475 0.184582720861515
0.3525 0.196754146963701
0.3575 0.209007790479944
0.3625 0.221294751494907
0.3675 0.233566555893202
0.3725 0.245777175796661
0.3775 0.257878534356224
0.3825 0.269825895697143
0.3875 0.280762751134239
0.3925 0.291125823998974
0.3975 0.301344687318739
0.4025 0.311383799901434
0.4075 0.321208499280159
0.4125 0.330782891511378
0.4175 0.340069206688678
0.4225 0.349030239424282
0.4275 0.357625064030437
0.4325 0.365811650550028
0.4375 0.373547694037799
0.4425 0.380786886392192
0.4475 0.387482958791299
0.4525 0.393554584425063
0.4575 0.398544508187415
0.4625 0.40284485512099
0.4675 0.406430877456141
0.4725 0.409295340053331
0.4775 0.411457697175722
0.4825 0.412966894784319
0.4875 0.413909365308522
0.4925 0.414393553363354
0.4975 0.414537071733491
0.5025 0.414414817643677
0.5075 0.414082601808506
0.5125 0.413605133145019
0.5175 0.41292593676663
0.5225 0.412146663042196
0.5275 0.411037759265123
0.5325 0.409423356341843
0.5375 0.406986588819159
0.5425 0.403290774788338
0.5475 0.397791048844959
0.5525 0.389889244113869
0.5575 0.379022289640628
0.5625 0.364790908884524
0.5675 0.34711405654474
0.5725 0.326258519922801
0.5775 0.30293366629912
0.5825 0.282746635654171
0.5875 0.262080746005429
0.5925 0.241685011604755
0.5975 0.222285537913755
0.6025 0.20499204194821
0.6075 0.190213260490063
0.6125 0.178092157708225
0.6175 0.168543554417529
0.6225 0.161318627034879
0.6275 0.156073291148601
0.6325 0.152428916739367
0.6375 0.150018538710252
0.6425 0.148518155509561
0.6475 0.147659144652695
0.6525 0.147232993087688
0.6575 0.147083805344955
0.6625 0.147102255604422
0.6675 0.147211749258498
0.6725 0.147359006818357
0.6775 0.14750284267091
0.6825 0.14760310588347
0.6875 0.14760750756133
0.6925 0.147436108173194
0.6975 0.146956216540239
0.7025 0.145939461804833
0.7075 0.144000666584626
0.7125 0.140496822843116
0.7175 0.134379596080212
0.7225 0.124047144379205
0.7275 0.107373947854461
0.7325 0.0825959413308089
0.7375 0.0512412213766304
0.7425 0.0228344296569949
0.7475 0.0069091441862917
0.7525 0.00114053280204829
0.7575 -0.000450641031938506
0.7625 -0.00082286816264241
0.7675 -0.000571324762488414
0.7725 -0.000504635391973649
0.7775 -0.000486962000039126
0.7825 -0.000482494039687571
0.7875 -0.000480878645068014
0.7925 -0.000480640824102871
0.7975 -0.000480611188328638
0.8025 -0.000480510096049167
0.8075 -0.000480510096049167
0.8125 -0.000480510096049167
0.8175 -0.000480612045561773
0.8225 -0.000480612045561773
0.8275 -0.000480612045561773
0.8325 -0.000480612045561773
0.8375 -0.000480612045561773
0.8425 -0.000480612045561773
0.8475 -0.000480612045561773
0.8525 -0.000480612045561773
0.8575 -0.000480612045561773
0.8625 -0.000480612045561773
0.8675 -0.000480612045561773
0.8725 -0.000480612045561773
0.8775 -0.000480612045561773
0.8825 -0.000480612045561773
0.8875 -0.000480612045561773
0.8925 -0.000480612045561773
0.8975 -0.000480612045561773
0.9025 -0.000480612045561773
0.9075 -0.000480612045561773
0.9125 -0.000480612045561773
0.9175 -0.000480612045561773
0.9225 -0.000480612045561773
0.9275 -0.000480612045561773
0.9325 -0.000480612045561773
0.9375 -0.000480612045561773
0.9425 -0.000480612045561773
0.9475 -0.000480612045561773
0.9525 -0.000480612045561773
0.9575 -0.000480612045561773
0.9625 -0.000480612045561773
0.9675 -0.000480612045561773
0.9725 -0.000480612045561773
0.9775 -0.000480612045561773
0.9825 -0.000480612045561773
0.9875 -0.000480612045561773
0.9925 -0.000480903566059982
0.9975 -0.000480903566059982
};
\addlegendentry{FCNN}
\addplot [semithick, green!50!black, mark=triangle,mark size=2, mark repeat=25, mark options={solid,rotate=180}, only marks]
table {%
0.0025 0.000175403527955961
0.0075 0.000809469424986566
0.0125 -4.28012590885427e-05
0.0175 -2.13725949829784e-05
0.0225 0.000106659253842119
0.0275 -0.000207309840107127
0.0325 -0.000406330196533797
0.0375 0.000719278590769813
0.0425 0.000708382661566267
0.0475 0.00151094687892279
0.0525 0.000782135159085221
0.0575 0.000420549123567561
0.0625 0.000575461592621713
0.0675 -0.000334871587445141
0.0725 -4.22705439951491e-05
0.0775 -0.000391575842255389
0.0825 -0.000329514631361994
0.0875 8.62757567707992e-05
0.0925 0.000426264078291846
0.0975 0.00107902767427385
0.1025 0.00041452302526441
0.1075 -0.000274161092785773
0.1125 0.000149601210385744
0.1175 -0.000348618166103473
0.1225 0.000170061512221714
0.1275 0.000212061725450335
0.1325 -6.43727563173199e-05
0.1375 0.000390991689056366
0.1425 0.000945584923951971
0.1475 0.000425616888798382
0.1525 0.000315607140189136
0.1575 0.000298669304038044
0.1625 0.000260548123862224
0.1675 0.000238298448783892
0.1725 -4.78869264758811e-05
0.1775 -1.71785794584437e-05
0.1825 -0.000111327265237153
0.1875 0.000412756977481636
0.1925 0.000654780177145764
0.1975 0.00103517798150457
0.2025 0.000809445303823731
0.2075 0.00126831066111352
0.2125 0.00108979657995524
0.2175 0.00149758069507804
0.2225 0.00188770460444465
0.2275 0.00444920901053215
0.2325 0.00401811724628406
0.2375 0.0064961348410123
0.2425 0.00803347663949747
0.2475 0.00943346706116915
0.2525 0.0151807164319011
0.2575 0.0190670830415002
0.2625 0.0236958872570769
0.2675 0.0266541807716849
0.2725 0.0277527706143023
0.2775 0.038101595551501
0.2825 0.0394963452629499
0.2875 0.0450273758159197
0.2925 0.0525917736430445
0.2975 0.0573718901889903
0.3025 0.0772370961353912
0.3075 0.0874065494114768
0.3125 0.0990030909915725
0.3175 0.105361230914033
0.3225 0.111720061622446
0.3275 0.125981263173547
0.3325 0.135245487874055
0.3375 0.143755156614223
0.3425 0.153268859817011
0.3475 0.163206992293405
0.3525 0.179091370359411
0.3575 0.196997957939161
0.3625 0.214657940757866
0.3675 0.226622963454056
0.3725 0.232712489083021
0.3775 0.250367358968631
0.3825 0.262262702309606
0.3875 0.269055868021418
0.3925 0.283560391332872
0.3975 0.297825562408221
0.4025 0.294209449247869
0.4075 0.310386180107936
0.4125 0.333372264686348
0.4175 0.345845313835254
0.4225 0.350517336271742
0.4275 0.367641765355586
0.4325 0.377294950610159
0.4375 0.383325661719335
0.4425 0.398678684031367
0.4475 0.409361635625675
0.4525 0.393592872821909
0.4575 0.399633152323373
0.4625 0.413758055019732
0.4675 0.423186575706933
0.4725 0.425620536613638
0.4775 0.429338757215925
0.4825 0.432656046321236
0.4875 0.436466574663106
0.4925 0.441513350108103
0.4975 0.43907859374908
0.5025 0.434735403743101
0.5075 0.435016330394402
0.5125 0.435365325276201
0.5175 0.437391802621699
0.5225 0.435080408365847
0.5275 0.431875747774584
0.5325 0.428959858596801
0.5375 0.43189843558262
0.5425 0.427101737762629
0.5475 0.410456835376721
0.5525 0.415074614376898
0.5575 0.421190103317747
0.5625 0.376319102817588
0.5675 0.362212307262518
0.5725 0.343881160147155
0.5775 0.309516060181335
0.5825 0.27695907691785
0.5875 0.259431246163464
0.5925 0.226963440562187
0.5975 0.190669043669428
0.6025 0.179309015919869
0.6075 0.168748523285006
0.6125 0.157087029550389
0.6175 0.147946574849586
0.6225 0.136950639572348
0.6275 0.136942970613523
0.6325 0.120703155931207
0.6375 0.123694305986908
0.6425 0.119249708032055
0.6475 0.127434117045532
0.6525 0.113774696028229
0.6575 0.118041082733402
0.6625 0.117105936236927
0.6675 0.122547788002218
0.6725 0.121198179592782
0.6775 0.120762819193125
0.6825 0.126633200782824
0.6875 0.129049905226949
0.6925 0.121611179383643
0.6975 0.120297568589375
0.7025 0.1050542872494
0.7075 0.103155173282314
0.7125 0.11100755319202
0.7175 0.110079937771864
0.7225 0.0992148621297882
0.7275 0.0908871932615529
0.7325 0.0798180565013511
0.7375 0.0538235905708119
0.7425 0.0435336821651016
0.7475 0.0161004991377832
0.7525 0.00517258741972474
0.7575 0.00365303297016947
0.7625 0.00160971696834973
0.7675 0.000554130291658048
0.7725 0.000803267893108249
0.7775 0.00128989553153956
0.7825 0.00133569223327627
0.7875 0.000352735052620432
0.7925 0.000427837958454728
0.7975 -3.35467911604981e-05
0.8025 0.00117219282540369
0.8075 0.00136400393893383
0.8125 0.00135754588506148
0.8175 0.00118592402961761
0.8225 0.00127105991004131
0.8275 0.00130141694120046
0.8325 0.00127627738790659
0.8375 0.00078850483399089
0.8425 0.00131854202754108
0.8475 0.001184829884171
0.8525 0.000649235891942487
0.8575 0.000342584167095695
0.8625 0.000277490639874809
0.8675 -0.00046968056025785
0.8725 0.00147115502694091
0.8775 0.00119081598171926
0.8825 0.000625006123360205
0.8875 0.00139212050039445
0.8925 0.00112974630514878
0.8975 0.00114711828360282
0.9025 0.000652579252109391
0.9075 0.00199950188595975
0.9125 0.000841037697121741
0.9175 -0.000248485644763418
0.9225 0.00121683763230818
0.9275 0.0019918280838506
0.9325 0.000553619813658961
0.9375 0.000615889055948389
0.9425 0.000935519871257249
0.9475 0.000247785696848638
0.9525 0.000895704595459115
0.9575 -2.83251234849078e-05
0.9625 0.000475915676648202
0.9675 0.000102767374729777
0.9725 0.000779575344928168
0.9775 0.00105107507539391
0.9825 -0.000154752364770014
0.9875 0.00275949208665127
0.9925 0.00177268677665771
0.9975 0.00200581502443838
};
\addlegendentry{CNN}

\nextgroupplot[
legend cell align={left},
legend style={at={(1,1)},anchor=north east,fill opacity=0.1, draw opacity=1, text opacity=1, draw=none},
tick align=outside,
tick pos=left,
x grid style={white!69.0196078431373!black},
xmajorgrids,
xlabel={\(x\)},
xmin=-0.04725, xmax=1.04725,
xtick style={color=black},
y grid style={white!69.0196078431373!black},
ymajorgrids,
ylabel={\(E\)},
ymin=-0.120552432712838, ymax=0.605980670973315,
ytick style={color=black},
width=.35\textwidth,
height=.4\textwidth
]
\addplot [semithick, black, dashed, mark=x,mark size=2, mark repeat=25, mark options={solid}]
table {%
0.0025 0.499999997457854
0.0075 0.499999997457854
0.0125 0.499999997457854
0.0175 0.499999997457854
0.0225 0.499999997457854
0.0275 0.499999997457854
0.0325 0.499999997457854
0.0375 0.499999997457854
0.0425 0.499999997457854
0.0475 0.499999997457854
0.0525 0.499999997457854
0.0575 0.499999997457854
0.0625 0.499999997457854
0.0675 0.499999997457854
0.0725 0.499999997457854
0.0775 0.49999999745647
0.0825 0.499999997123304
0.0875 0.499999994239222
0.0925 0.49999999315962
0.0975 0.499999982706901
0.1025 0.499999967212159
0.1075 0.499999945766997
0.1125 0.499999908329267
0.1175 0.4999998219352
0.1225 0.499999664287051
0.1275 0.499999408709247
0.1325 0.499998942510754
0.1375 0.4999981652197
0.1425 0.499996830561658
0.1475 0.499994612526022
0.1525 0.499990956182364
0.1575 0.499985022918745
0.1625 0.499975537745301
0.1675 0.499960592267752
0.1725 0.499937393293539
0.1775 0.499901909902479
0.1825 0.499848511035655
0.1875 0.499769387782158
0.1925 0.499654031258562
0.1975 0.499488526210481
0.2025 0.499255030732484
0.2075 0.498931073732302
0.2125 0.498489154452836
0.2175 0.497896602633831
0.2225 0.49711563857505
0.2275 0.496104031920521
0.2325 0.494816125131012
0.2375 0.493204409942022
0.2425 0.491221536834207
0.2475 0.488822538745084
0.2525 0.485967268521428
0.2575 0.482622689956056
0.2625 0.478764910116803
0.2675 0.474380595345601
0.2725 0.469467933202883
0.2775 0.464036847019814
0.2825 0.458108501333654
0.2875 0.451714388147804
0.2925 0.444894901525337
0.2975 0.437697625718639
0.3025 0.43017548415983
0.3075 0.422384967214943
0.3125 0.41438433189132
0.3175 0.406232146680525
0.3225 0.397985925585357
0.3275 0.389701040596312
0.3325 0.381429940465515
0.3375 0.373221536863317
0.3425 0.365120781658711
0.3475 0.35716847155665
0.3525 0.349401123148409
0.3575 0.341850998944951
0.3625 0.334546205617342
0.3675 0.327510880106934
0.3725 0.320765342451106
0.3775 0.314326317516064
0.3825 0.308207224996573
0.3875 0.302418348265026
0.3925 0.296967089557285
0.3975 0.291858183808528
0.4025 0.287093943575464
0.4075 0.282674389872564
0.4125 0.27859745148855
0.4175 0.274859147252174
0.4225 0.271453659257701
0.4275 0.268373503265531
0.4325 0.265609568116938
0.4375 0.263151195986163
0.4425 0.260986230966502
0.4475 0.259100912260108
0.4525 0.257479921599843
0.4575 0.2561062289821
0.4625 0.254960947439195
0.4675 0.254023244845512
0.4725 0.253270314529259
0.4775 0.252677600104063
0.4825 0.252219487436491
0.4875 0.251870137005716
0.4925 0.251604335682046
0.4975 0.251397725545779
0.5025 0.251226691525528
0.5075 0.251068334142746
0.5125 0.250900335380158
0.5175 0.250698705687687
0.5225 0.250431901804233
0.5275 0.250051150241851
0.5325 0.249478665887373
0.5375 0.248596919839037
0.5425 0.247244384869381
0.5475 0.245223917976736
0.5525 0.242327889855065
0.5575 0.238378920422901
0.5625 0.233277600007316
0.5675 0.227043163796654
0.5725 0.219832444821558
0.5775 0.211928646571401
0.5825 0.203701167142895
0.5875 0.195547681866796
0.5925 0.187834659992751
0.5975 0.18085111835914
0.6025 0.174784099823094
0.6075 0.169716411704117
0.6125 0.165641205077031
0.6175 0.162485369842056
0.6225 0.160134753015422
0.6275 0.158456400881594
0.6325 0.157315743947614
0.6375 0.156588301770535
0.6425 0.156166478079637
0.6475 0.15596226117267
0.6525 0.155906986609067
0.6575 0.155948948709906
0.6625 0.156049660237618
0.6675 0.156179161378539
0.6725 0.156310450227518
0.6775 0.156412499238597
0.6825 0.156440735266932
0.6875 0.156322878209411
0.6925 0.155936716342982
0.6975 0.155074757600502
0.7025 0.153389302709368
0.7075 0.150312942138363
0.7125 0.144962248699927
0.7175 0.136077279393745
0.7225 0.122162284914289
0.7275 0.102173606704146
0.7325 0.0770700776468148
0.7375 0.051396100443933
0.7425 0.031771537528569
0.7475 0.0212761318479062
0.7525 0.017292907803068
0.7575 0.0160879855050049
0.7625 0.0157570211704135
0.7675 0.0156687592971979
0.7725 0.0156454184213528
0.7775 0.0156392644073019
0.7825 0.0156376445647867
0.7875 0.0156372187972693
0.7925 0.0156371070022085
0.7975 0.0156370779062883
0.8025 0.0156370704822387
0.8075 0.015637068192549
0.8125 0.0156370675656021
0.8175 0.0156370674551392
0.8225 0.0156370674431169
0.8275 0.0156370674431151
0.8325 0.0156370674431151
0.8375 0.0156370674431151
0.8425 0.0156370674431151
0.8475 0.0156370674431151
0.8525 0.0156370674431151
0.8575 0.0156370674431151
0.8625 0.0156370674431151
0.8675 0.0156370674431151
0.8725 0.0156370674431151
0.8775 0.0156370674431151
0.8825 0.0156370674431151
0.8875 0.0156370674431151
0.8925 0.0156370674431151
0.8975 0.0156370674431151
0.9025 0.0156370674431151
0.9075 0.0156370674431151
0.9125 0.0156370674431151
0.9175 0.0156370674431151
0.9225 0.0156370674431151
0.9275 0.0156370674431151
0.9325 0.0156370674431151
0.9375 0.0156370674431151
0.9425 0.0156370674431151
0.9475 0.0156370674431151
0.9525 0.0156370674431151
0.9575 0.0156370674431151
0.9625 0.0156370674431151
0.9675 0.0156370674431151
0.9725 0.0156370674431151
0.9775 0.0156370674431151
0.9825 0.0156370674431151
0.9875 0.0156370674431151
0.9925 0.0156370674431151
0.9975 0.0156370674431151
};
\addlegendentry{FOM}
\addplot [semithick, red, mark=o,mark size=2, mark repeat=25, mark options={solid}]
table {%
0.0025 0.49862507099537
0.0075 0.498625070995358
0.0125 0.498625070995335
0.0175 0.498625070995289
0.0225 0.498625070995195
0.0275 0.498625070994987
0.0325 0.498625070994536
0.0375 0.498625070993572
0.0425 0.498625070991509
0.0475 0.498625070987111
0.0525 0.498625070977834
0.0575 0.498625070958476
0.0625 0.49862507091852
0.0675 0.498625070836961
0.0725 0.498625070672361
0.0775 0.498625070343984
0.0825 0.498625069696427
0.0875 0.498625068434377
0.0925 0.498625066003869
0.0975 0.498625061379234
0.1025 0.49862505268657
0.1075 0.498625036548459
0.1125 0.498625006961122
0.1175 0.498624953401292
0.1225 0.498624857686967
0.1275 0.498624688860349
0.1325 0.498624394993753
0.1375 0.498623890307746
0.1425 0.498623035305027
0.1475 0.49862160674214
0.1525 0.498619253187858
0.1575 0.498615430698208
0.1625 0.498609311888697
0.1675 0.498599660615519
0.1725 0.498584663918093
0.1775 0.498561713278738
0.1825 0.498527129175792
0.1875 0.498475826919541
0.1925 0.498400928328475
0.1975 0.498293333066813
0.2025 0.498141275003092
0.2075 0.497929901566226
0.2125 0.497640925705907
0.2175 0.497252407933146
0.2225 0.496738727011619
0.2275 0.496070789730308
0.2325 0.495216511824596
0.2375 0.494141574769251
0.2425 0.492810430540245
0.2475 0.49118749413781
0.2525 0.489238437906705
0.2575 0.486931487672566
0.2625 0.484238621141534
0.2675 0.481136583397302
0.2725 0.477607659268651
0.2775 0.473640172608262
0.2825 0.469228712554903
0.2875 0.464374112063574
0.2925 0.45908322158397
0.2975 0.453368529902798
0.3025 0.447247685609862
0.3075 0.440742968179374
0.3125 0.433880749428725
0.3175 0.426690976158106
0.3225 0.419206694698188
0.3275 0.41146362898685
0.3325 0.403499816247336
0.3375 0.395355298541466
0.3425 0.387071864341846
0.3475 0.378692831569695
0.3525 0.370262861977174
0.3575 0.361827796005775
0.3625 0.353434497046703
0.3675 0.345130694129389
0.3725 0.336964812281608
0.3775 0.328985779986757
0.3825 0.321242803186622
0.3875 0.313785095026547
0.3925 0.306661549894885
0.3975 0.299920349123918
0.4025 0.293608483800653
0.4075 0.287771177213868
0.4125 0.282451185164978
0.4175 0.277687946194899
0.4225 0.273516545106461
0.4275 0.269966441344469
0.4325 0.267059898486333
0.4375 0.264810033124293
0.4425 0.263218384890518
0.4475 0.262271906267568
0.4525 0.261939310151276
0.4575 0.262166856616607
0.4625 0.262874021462233
0.4675 0.263950230587403
0.4725 0.265255071304541
0.4775 0.266625778589816
0.4825 0.267895891198504
0.4875 0.268925021887292
0.4925 0.269630339497512
0.4975 0.270001703213151
0.5025 0.270087324494209
0.5075 0.269957249654707
0.5125 0.269667082626097
0.5175 0.269236878201456
0.5225 0.268641994106717
0.5275 0.267804764421337
0.5325 0.266581191454045
0.5375 0.264746631547403
0.5425 0.261992020488676
0.5475 0.25794461121141
0.5525 0.252222458497304
0.5575 0.244519462099447
0.5625 0.234701241014227
0.5675 0.222879634844955
0.5725 0.209434201254302
0.5775 0.194965968684945
0.5825 0.180195231913223
0.5875 0.165837357957321
0.5925 0.152496015577265
0.5975 0.140600034540706
0.6025 0.130388159134685
0.6075 0.121928998813938
0.6125 0.115158489735012
0.6175 0.10992129549136
0.6225 0.106008879051785
0.6275 0.1031912453944
0.6325 0.101241319233524
0.6375 0.0999518175373785
0.6425 0.099145116658001
0.6475 0.0986771620948313
0.6525 0.098436837862418
0.6575 0.0983423287272417
0.6625 0.0983358767613919
0.6675 0.098378019269526
0.6725 0.0984419881410418
0.6775 0.0985085288225847
0.6825 0.098560999625045
0.6875 0.0985802271650378
0.6925 0.0985381545880715
0.6975 0.0983886978519858
0.7025 0.098053213392447
0.7075 0.0973962545569519
0.7125 0.0961844117902861
0.7175 0.0940167796372562
0.7225 0.0902125241779154
0.7275 0.083658631856229
0.7325 0.0727603303030725
0.7375 0.0562332992635219
0.7425 0.0362830771816603
0.7475 0.0206913795284433
0.7525 0.0135920223442743
0.7575 0.0113884739044625
0.7625 0.0107936392393487
0.7675 0.0106369031078263
0.7725 0.0105956420995906
0.7775 0.0105847780160647
0.7825 0.0105819194090026
0.7875 0.0105811681010425
0.7925 0.0105809709113385
0.7975 0.0105809192354426
0.8025 0.0105809057156192
0.8075 0.0105809021848178
0.8125 0.010580901264515
0.8175 0.0105809010251426
0.8225 0.0105809009630227
0.8275 0.0105809009469414
0.8325 0.0105809009427893
0.8375 0.0105809009417203
0.8425 0.0105809009414458
0.8475 0.0105809009413756
0.8525 0.0105809009413575
0.8575 0.0105809009413528
0.8625 0.0105809009413515
0.8675 0.0105809009413511
0.8725 0.010580900941351
0.8775 0.010580900941351
0.8825 0.0105809009413509
0.8875 0.0105809009413509
0.8925 0.010580900941351
0.8975 0.010580900941351
0.9025 0.010580900941351
0.9075 0.010580900941351
0.9125 0.010580900941351
0.9175 0.010580900941351
0.9225 0.010580900941351
0.9275 0.010580900941351
0.9325 0.010580900941351
0.9375 0.010580900941351
0.9425 0.010580900941351
0.9475 0.010580900941351
0.9525 0.010580900941351
0.9575 0.010580900941351
0.9625 0.010580900941351
0.9675 0.010580900941351
0.9725 0.010580900941351
0.9775 0.010580900941351
0.9825 0.010580900941351
0.9875 0.010580900941351
0.9925 0.010580900941351
0.9975 0.010580900941351
};
\addlegendentry{POD}
\addplot [semithick, color0, mark=pentagon,mark size=2, mark repeat=25, mark options={solid}]
table {%
0.0025 0.500139485923816
0.0075 0.500139485923816
0.0125 0.500139485923816
0.0175 0.500139485923816
0.0225 0.500139485923816
0.0275 0.500139485923816
0.0325 0.500139485923816
0.0375 0.500139485923816
0.0425 0.500139485923816
0.0475 0.500139485923816
0.0525 0.500139485923816
0.0575 0.500139485923816
0.0625 0.500139485923816
0.0675 0.500139485923816
0.0725 0.500139485923816
0.0775 0.500139485923816
0.0825 0.500139485923816
0.0875 0.500139485923816
0.0925 0.500139485923816
0.0975 0.500139337839146
0.1025 0.500139664561154
0.1075 0.500140053000808
0.1125 0.500140032449193
0.1175 0.50014009944212
0.1225 0.500139170521682
0.1275 0.500139534782965
0.1325 0.500139105542311
0.1375 0.500138185339432
0.1425 0.500137333958558
0.1475 0.500134796623042
0.1525 0.500132729140778
0.1575 0.500127458652421
0.1625 0.500117775701708
0.1675 0.5001058112195
0.1725 0.500085240854488
0.1775 0.500052333134403
0.1825 0.500006052722577
0.1875 0.499933946564938
0.1925 0.49983115302702
0.1975 0.49968495226449
0.2025 0.499474429488304
0.2075 0.499182689120001
0.2125 0.498787126899044
0.2175 0.49825185877148
0.2225 0.497545480005185
0.2275 0.496627887781643
0.2325 0.495456468370328
0.2375 0.493983405188992
0.2425 0.492159893074038
0.2475 0.489941042749161
0.2525 0.487277898752358
0.2575 0.484130465234777
0.2625 0.480462314949465
0.2675 0.476245331724927
0.2725 0.471456610676344
0.2775 0.466082120062974
0.2825 0.460123431679082
0.2875 0.453579679275667
0.2925 0.446465684068298
0.2975 0.438804387185797
0.3025 0.430622934544413
0.3075 0.42195525025856
0.3125 0.412802485018499
0.3175 0.403466877668696
0.3225 0.396321453837362
0.3275 0.388971444589845
0.3325 0.381429485228611
0.3375 0.373672873009438
0.3425 0.365829630159456
0.3475 0.357941948292521
0.3525 0.350050858715831
0.3575 0.342199158695723
0.3625 0.334427537992437
0.3675 0.326778298152913
0.3725 0.319286767579964
0.3775 0.311991369028979
0.3825 0.304923709227454
0.3875 0.299794630676976
0.3925 0.295833990796102
0.3975 0.291956744780501
0.4025 0.288179524986253
0.4075 0.28451025739724
0.4125 0.280960201739344
0.4175 0.277541744580041
0.4225 0.274261937117206
0.4275 0.271129019445513
0.4325 0.268151836758685
0.4375 0.265335838249142
0.4425 0.262687329086426
0.4475 0.26021625488235
0.4525 0.257950575928568
0.4575 0.256232802037641
0.4625 0.254710041212527
0.4675 0.253377003987205
0.4725 0.252237557617474
0.4775 0.251288316860622
0.4825 0.250525685925861
0.4875 0.249938056392815
0.4925 0.249507245306465
0.4975 0.24920629573436
0.5025 0.249061294540865
0.5075 0.249070943810637
0.5125 0.248972070308803
0.5175 0.248725492797423
0.5225 0.248417009484471
0.5275 0.248128541729278
0.5325 0.247693876861861
0.5375 0.247010142148361
0.5425 0.245931538450181
0.5475 0.244273021341647
0.5525 0.241818063622992
0.5575 0.238336672633911
0.5625 0.233419846570831
0.5675 0.226627119072287
0.5725 0.218305180222336
0.5775 0.20864871476336
0.5825 0.20447698054689
0.5875 0.199437463367592
0.5925 0.193334002919517
0.5975 0.18621671764822
0.6025 0.179304742818442
0.6075 0.172967307544202
0.6125 0.167468696923938
0.6175 0.162950056093543
0.6225 0.159424681331687
0.6275 0.15681329327236
0.6325 0.154977903434728
0.6375 0.153765165370411
0.6425 0.153018411512027
0.6475 0.152607068606321
0.6525 0.152421798085408
0.6575 0.152384467101201
0.6625 0.152435486528614
0.6675 0.152533989084787
0.6725 0.152645844095092
0.6775 0.152747666541726
0.6825 0.152797543317399
0.6875 0.152754939289622
0.6925 0.152531504325453
0.6975 0.151994031577948
0.7025 0.150906772666807
0.7075 0.148869610413333
0.7125 0.145221857209477
0.7175 0.138899891244964
0.7225 0.128290490040301
0.7275 0.111354335000316
0.7325 0.086708262148705
0.7375 0.0569398717137683
0.7425 0.03263693492199
0.7475 0.0219526916475295
0.7525 0.0196229218111859
0.7575 0.0194009471028487
0.7625 0.0191529856545062
0.7675 0.0169050878237854
0.7725 0.0163095745326533
0.7775 0.0161520862273404
0.7825 0.0161104863280783
0.7875 0.0160987477134576
0.7925 0.0160964296671283
0.7975 0.0160953086071456
0.8025 0.0160959472536843
0.8075 0.0160959472536843
0.8125 0.0160959472536843
0.8175 0.0160953520811119
0.8225 0.0160953520811119
0.8275 0.0160953520811119
0.8325 0.0160953520811119
0.8375 0.0160953520811119
0.8425 0.0160953520811119
0.8475 0.0160953520811119
0.8525 0.0160953520811119
0.8575 0.0160953520811119
0.8625 0.0160953520811119
0.8675 0.0160953520811119
0.8725 0.0160953520811119
0.8775 0.0160953520811119
0.8825 0.0160953520811119
0.8875 0.0160953520811119
0.8925 0.0160953520811119
0.8975 0.0160953520811119
0.9025 0.0160953520811119
0.9075 0.0160953520811119
0.9125 0.0160953520811119
0.9175 0.0160953520811119
0.9225 0.0160953520811119
0.9275 0.0160953520811119
0.9325 0.0160953520811119
0.9375 0.0160953520811119
0.9425 0.0160953520811119
0.9475 0.0160953520811119
0.9525 0.0160953520811119
0.9575 0.0160953520811119
0.9625 0.0160953520811119
0.9675 0.0160953520811119
0.9725 0.0160953520811119
0.9775 0.0160953520811119
0.9825 0.0160953520811119
0.9875 0.0160953520811119
0.9925 0.0160963666699128
0.9975 0.0160963666699128
};
\addlegendentry{FCNN}
\addplot [semithick, green!50!black, mark=triangle,mark size=2, mark repeat=25, mark options={solid,rotate=180}, only marks]
table {%
0.0025 0.543558293894175
0.0075 0.564719463902347
0.0125 0.486404641787632
0.0175 0.4601241306863
0.0225 0.517028663357154
0.0275 0.476395964376408
0.0325 0.536590539303784
0.0375 0.494025085020027
0.0425 0.518386156252738
0.0475 0.563932195575686
0.0525 0.544471356407682
0.0575 0.529741751859604
0.0625 0.524852769033439
0.0675 0.48054148857693
0.0725 0.509762738907053
0.0775 0.46260325847772
0.0825 0.518876710893644
0.0875 0.49531200811389
0.0925 0.513481317519518
0.0975 0.517420025262313
0.1025 0.538449224338755
0.1075 0.539262363166237
0.1125 0.518216694746861
0.1175 0.435467282418118
0.1225 0.52076085186127
0.1275 0.47820874969494
0.1325 0.517141735350883
0.1375 0.516765930194936
0.1425 0.52243399627489
0.1475 0.527320419769502
0.1525 0.550114241409219
0.1575 0.531695089030882
0.1625 0.51769304017175
0.1675 0.464866874134334
0.1725 0.511729709721504
0.1775 0.479527185106051
0.1825 0.518568462647271
0.1875 0.495456249252165
0.1925 0.516289168802967
0.1975 0.528494285579435
0.2025 0.536252548514671
0.2075 0.572956438987581
0.2125 0.481976767684064
0.2175 0.451413984658676
0.2225 0.513133040687513
0.2275 0.483977520622519
0.2325 0.515088378362418
0.2375 0.490332705301279
0.2425 0.494783952540496
0.2475 0.511628825333278
0.2525 0.547262596371474
0.2575 0.544387907382801
0.2625 0.48148458858107
0.2675 0.421920586146238
0.2725 0.449918034460684
0.2775 0.420957393266648
0.2825 0.441483571354365
0.2875 0.394088438775616
0.2925 0.403846208936138
0.2975 0.383613786105359
0.3025 0.490328932948911
0.3075 0.485678217644804
0.3125 0.412746727264713
0.3175 0.325566006093673
0.3225 0.353423677089089
0.3275 0.305282501316173
0.3325 0.324139350748391
0.3375 0.292916635336728
0.3425 0.26059866911807
0.3475 0.247887060706375
0.3525 0.354876640329637
0.3575 0.376811035625196
0.3625 0.321815097377393
0.3675 0.254753381668985
0.3725 0.252858442030585
0.3775 0.237525447295275
0.3825 0.24259423141391
0.3875 0.210036117012141
0.3925 0.194386506348762
0.3975 0.222260084215196
0.4025 0.296486751089792
0.4075 0.308673791705261
0.4125 0.26453340961612
0.4175 0.251622232447928
0.4225 0.226146376265577
0.4275 0.21788043668769
0.4325 0.234018365552338
0.4375 0.200193079577118
0.4425 0.201033990521449
0.4475 0.186992294735369
0.4525 0.32156011729418
0.4575 0.271817326626354
0.4625 0.267915981140165
0.4675 0.236352221863691
0.4725 0.204120746454436
0.4775 0.208420357613006
0.4825 0.282898340619552
0.4875 0.199971060732828
0.4925 0.18401794121326
0.4975 0.184593350978308
0.5025 0.192440951769475
0.5075 0.187925647335276
0.5125 0.240410123786211
0.5175 0.212681354709498
0.5225 0.191993619425629
0.5275 0.193111172736345
0.5325 0.255530934199263
0.5375 0.212191427819647
0.5425 0.221085225203747
0.5475 0.303845017639265
0.5525 0.107611224802143
0.5575 0.0325796382493863
0.5625 0.249607360868059
0.5675 0.255060824832714
0.5725 0.14047186234441
0.5775 0.210359369342283
0.5825 0.206080529708012
0.5875 0.183253303998363
0.5925 0.255374057500228
0.5975 0.299364379671689
0.6025 0.215090411894404
0.6075 0.233807654925717
0.6125 0.226816462507188
0.6175 0.184526397879567
0.6225 0.229623984801968
0.6275 0.191642649041512
0.6325 0.265899799760188
0.6375 0.260917350698222
0.6425 0.244879923956296
0.6475 0.228028651938125
0.6525 0.249847368340966
0.6575 0.243229273804138
0.6625 0.296909367638542
0.6675 0.232077167832147
0.6725 0.264724841277657
0.6775 0.223377018923797
0.6825 0.21815616185635
0.6875 0.248097024084459
0.6925 0.239170664698377
0.6975 0.218609684796004
0.7025 0.272515116035421
0.7075 0.327595080283301
0.7125 0.284486908898688
0.7175 0.167183262653453
0.7225 0.164428335091317
0.7275 0.0991260432094094
0.7325 0.0273982976520321
0.7375 0.014485525287322
0.7425 -0.0875282007271037
0.7475 -0.0199902619341918
0.7525 0.00654694414887855
0.7575 -0.00144983336186735
0.7625 0.0627840533342506
0.7675 0.0114020638858208
0.7725 0.0286057376945837
0.7775 0.0286284919462966
0.7825 -0.00474399228678573
0.7875 0.0332732077510281
0.7925 0.0195773043842302
0.7975 0.014646074613457
0.8025 0.0139425280513923
0.8075 0.00552939850931217
0.8125 0.083673117384618
0.8175 0.00489534613105954
0.8225 0.048903442754067
0.8275 0.0224436954248665
0.8325 -0.00475785971724294
0.8375 0.0476093448454647
0.8425 0.0281479499324392
0.8475 0.0246747690691391
0.8525 0.00463177649102247
0.8575 -0.000518744724103201
0.8625 0.0730632628028802
0.8675 0.00764964093203847
0.8725 0.0373482973088953
0.8775 0.014372019074874
0.8825 -0.0106741840211157
0.8875 0.0454520033668718
0.8925 0.0324108889687051
0.8975 0.0281011226368596
0.9025 -0.0127392373305129
0.9075 0.0110874287660044
0.9125 0.0791579655843602
0.9175 0.0226732046121916
0.9225 0.0326002872797183
0.9275 0.0200573920925657
0.9325 -0.00491472365550866
0.9375 0.030061259064328
0.9425 0.0292240598674576
0.9475 0.0162076003648492
0.9525 0.0122862958556062
0.9575 0.000873731331321777
0.9625 0.073119140238547
0.9675 -0.00348471523310695
0.9725 0.0338053387048922
0.9775 0.0386164095534954
0.9825 -0.00653518769833601
0.9875 0.0529919584102714
0.9925 0.0254556437456522
0.9975 0.0325447631810872
};
\addlegendentry{CNN}

\nextgroupplot[
legend cell align={left},
legend style={at={(1,1)},anchor=north east,fill opacity=0.1, draw opacity=1, text opacity=1,draw=none},
tick align=outside,
tick pos=left,
x grid style={white!69.0196078431373!black},
xmajorgrids,
xlabel={\(x\)},
xmin=-0.04725, xmax=1.04725,
xtick style={color=black},
y grid style={white!69.0196078431373!black},
ymajorgrids,
ylabel={\(\rho\)},
ymin=0.0798726185710705, ymax=1.04764448950172,
ytick style={color=black},
width=.35\textwidth,
height=.4\textwidth
]
\addplot [semithick, black, dashed, mark=x,mark size=2, mark repeat=25, mark options={solid}]
table {%
0.0025 0.999999779921312
0.0075 0.999999779921312
0.0125 0.999999657655373
0.0175 0.999999535389436
0.0225 0.999999413123497
0.0275 0.999999290857559
0.0325 0.999999046325683
0.0375 0.999998740660838
0.0425 0.999998373863024
0.0475 0.999997823666304
0.0525 0.999997273469582
0.0575 0.999996295342078
0.0625 0.999995194948637
0.0675 0.99999378889035
0.0725 0.999991954901279
0.0775 0.999989570715488
0.0825 0.9999863918011
0.0875 0.999982295892177
0.0925 0.999976855057936
0.0975 0.999969947032439
0.1025 0.999961082751934
0.1075 0.999949650886731
0.1125 0.999934978974171
0.1175 0.999916394551595
0.1225 0.999892797225561
0.1275 0.999862780937782
0.1325 0.999825061895908
0.1375 0.999777378180088
0.1425 0.999717895801251
0.1475 0.999643741509853
0.1525 0.999551736391508
0.1575 0.999437967936198
0.1625 0.999298279102032
0.1675 0.999127473586645
0.1725 0.998919682625012
0.1775 0.998667937058669
0.1825 0.998364900931334
0.1875 0.998001954494378
0.1925 0.997569927802453
0.1975 0.99705818371895
0.2025 0.996455473777575
0.2075 0.995749815916404
0.2125 0.9949280665471
0.2175 0.993976654150547
0.2225 0.992881823808719
0.2275 0.991628597944211
0.2325 0.990202182378524
0.2375 0.988587660667224
0.2425 0.986770177498842
0.2475 0.984735733423477
0.2525 0.982469656528571
0.2575 0.9799591700236
0.2625 0.977191680516952
0.2675 0.974155389345609
0.2725 0.970840392968593
0.2775 0.967237154642741
0.2825 0.963338277278802
0.2875 0.959137158516126
0.2925 0.954628418653439
0.2975 0.94980857311151
0.3025 0.944675054305639
0.3075 0.939226700709416
0.3125 0.933463695721748
0.3175 0.927387078603109
0.3225 0.920999539204133
0.3275 0.914303950774364
0.3325 0.907304959419446
0.3375 0.900007516909868
0.3425 0.892417308611748
0.3475 0.884541609348395
0.3525 0.8763871437464
0.3575 0.867962531554393
0.3625 0.859276147989126
0.3675 0.85033716299595
0.3725 0.841155174450997
0.3775 0.831740636091966
0.3825 0.822103879390619
0.3875 0.812256152813251
0.3925 0.802208949358035
0.3975 0.791975351480337
0.4025 0.78156923636412
0.4075 0.771006681980231
0.4125 0.760306517283122
0.4175 0.749489466349284
0.4225 0.738579676701472
0.4275 0.727603680048233
0.4325 0.716589719821245
0.4375 0.705566895313752
0.4425 0.694562043899145
0.4475 0.683599679897993
0.4525 0.672698754530687
0.4575 0.661872717050406
0.4625 0.651134038582826
0.4675 0.640501914880215
0.4725 0.630014064984444
0.4775 0.619732783390925
0.4825 0.609731490795429
0.4875 0.600058298844558
0.4925 0.59069615143996
0.4975 0.581551392873128
0.5025 0.572476509289864
0.5075 0.56329397054819
0.5125 0.553800264994303
0.5175 0.543781549502642
0.5225 0.533063533978584
0.5275 0.521566317631648
0.5325 0.509313956285134
0.5375 0.496400778110211
0.5425 0.482943394245245
0.5475 0.469048328888722
0.5525 0.454804377678113
0.5575 0.440289851946708
0.5625 0.425583521525065
0.5675 0.410771675598927
0.5725 0.395950781993377
0.5775 0.381227089808537
0.5825 0.366713603337606
0.5875 0.352526994851919
0.5925 0.338781399604602
0.5975 0.325581202140221
0.6025 0.31301281391046
0.6075 0.301137948647524
0.6125 0.289988915125529
0.6175 0.279568556027535
0.6225 0.269853487992898
0.6275 0.260800795677381
0.6325 0.25235619300451
0.6375 0.244461878752097
0.6425 0.237063108346401
0.6475 0.230112029955937
0.6525 0.223569457347576
0.6575 0.217404182140644
0.6625 0.211590895285973
0.6675 0.206107787596874
0.6725 0.200933844615252
0.6775 0.196047272437658
0.6825 0.191424565437512
0.6875 0.187040689664009
0.6925 0.182870030403137
0.6975 0.178888042767843
0.7025 0.175072520207136
0.7075 0.171405245096256
0.7125 0.167872875164717
0.7175 0.164467356143854
0.7225 0.161185830067366
0.7275 0.158030222623776
0.7325 0.155006219179202
0.7375 0.15212214910067
0.7425 0.149387808946463
0.7475 0.146813270373222
0.7525 0.144407795025752
0.7575 0.142178871692755
0.7625 0.140131299312298
0.7675 0.138266850740482
0.7725 0.136584028219565
0.7775 0.135078032811483
0.7825 0.133741039496202
0.7875 0.132562930767353
0.7925 0.131531541164105
0.7975 0.130633803514334
0.8025 0.129855947616773
0.8075 0.12918458535121
0.8125 0.128606618979038
0.8175 0.128109967097258
0.8225 0.127683602846586
0.8275 0.127317637969286
0.8325 0.127003406867003
0.8375 0.12673332905158
0.8425 0.126500924428304
0.8475 0.126300606972132
0.8525 0.126127707652557
0.8575 0.125978222260108
0.8625 0.125848811406356
0.8675 0.125736601841755
0.8725 0.125639255230243
0.8775 0.125554723617358
0.8825 0.125481257071862
0.8875 0.125417434252225
0.8925 0.125361956082858
0.8975 0.125313752736801
0.9025 0.125271861369793
0.9075 0.125235441403511
0.9125 0.125203828016917
0.9175 0.125176386955457
0.9225 0.125152552739168
0.9275 0.12513185158754
0.9325 0.125113924344381
0.9375 0.125098358362149
0.9425 0.125084847976
0.9475 0.1250731639373
0.9525 0.12506304643093
0.9575 0.125054266208257
0.9625 0.125046716286586
0.9675 0.125040175058903
0.9725 0.125034558467376
0.9775 0.125029751887688
0.9825 0.125025663620386
0.9875 0.125022209607638
0.9925 0.125019313433231
0.9975 0.12501672292367
};
\addlegendentry{FOM}
\addplot [semithick, red, mark=o,mark size=2, mark repeat=25, mark options={solid}]
table {%
0.0025 0.998698302494394
0.0075 0.998698273326595
0.0125 0.998698239455942
0.0175 0.998698196204499
0.0225 0.998698139392356
0.0275 0.998698064184519
0.0325 0.998697964451367
0.0375 0.998697832197717
0.0425 0.998697656911356
0.0475 0.998697424751913
0.0525 0.998697117513723
0.0575 0.998696711291434
0.0625 0.998696174764184
0.0675 0.998695466996326
0.0725 0.998694534630346
0.0775 0.998693308321378
0.0825 0.998691698232338
0.0875 0.998689588374572
0.0925 0.99868682954154
0.0975 0.998683230543227
0.1025 0.998678547408363
0.1075 0.998672470182162
0.1125 0.998664606912185
0.1175 0.998654464388049
0.1225 0.998641425186778
0.1275 0.9986247205804
0.1325 0.998603398892314
0.1375 0.99857628895073
0.1425 0.998541958387887
0.1475 0.99849866667837
0.1525 0.998444313002865
0.1575 0.998376379266358
0.1625 0.998291868889386
0.1675 0.998187242320423
0.1725 0.998058350573667
0.1775 0.997900368461072
0.1825 0.997707729536365
0.1875 0.9974740650741
0.1925 0.997192149638342
0.1975 0.996853855924052
0.2025 0.996450121554054
0.2075 0.99597093036793
0.2125 0.995405310438635
0.2175 0.99474135060363
0.2225 0.993966236718855
0.2275 0.993066308167807
0.2325 0.992027134426715
0.2375 0.990833610749002
0.2425 0.989470071338078
0.2475 0.987920417773415
0.2525 0.986168259978534
0.2575 0.9841970666972
0.2625 0.981990322287675
0.2675 0.979531686654096
0.2725 0.976805155297111
0.2775 0.973795216764851
0.2825 0.970487005198974
0.2875 0.966866446178896
0.2925 0.962920394652705
0.2975 0.958636764388681
0.3025 0.954004649065567
0.3075 0.949014435808484
0.3125 0.943657912613614
0.3175 0.937928371599843
0.3225 0.93182071025845
0.3275 0.925331532698594
0.3325 0.918459252169867
0.3375 0.911204194800864
0.3425 0.903568702564973
0.3475 0.895557231209375
0.3525 0.887176436759925
0.3575 0.878435243021843
0.3625 0.869344883220239
0.3675 0.859918912545435
0.3725 0.850173195462653
0.3775 0.840125881810801
0.3825 0.829797396918779
0.3875 0.819210479104472
0.3925 0.808390297036259
0.3975 0.797364663162253
0.4025 0.786164323909006
0.4075 0.774823255248402
0.4125 0.763378835781004
0.4175 0.751871729969605
0.4225 0.740345314837129
0.4275 0.72884453647866
0.4325 0.717414177076701
0.4375 0.706096617458577
0.4425 0.694929275129902
0.4475 0.683942030409673
0.4525 0.673155258667752
0.4575 0.662579656940736
0.4625 0.652219569245489
0.4675 0.642080857227587
0.4725 0.632181024694021
0.4775 0.622553591742255
0.4825 0.613235751217255
0.4875 0.604236632686526
0.4925 0.595503679651282
0.4975 0.58691746306255
0.5025 0.578316110388185
0.5075 0.569516845631515
0.5125 0.560315441909217
0.5175 0.550494569498016
0.5225 0.53986720883965
0.5275 0.528329501861953
0.5325 0.515876607865928
0.5375 0.502578645462612
0.5425 0.488540690791141
0.5475 0.47387225333178
0.5525 0.45867571445608
0.5575 0.443048559957467
0.5625 0.427090122284292
0.5675 0.410907036722874
0.5725 0.394616081166999
0.5775 0.378344992711069
0.5825 0.362231527641442
0.5875 0.346420198056971
0.5925 0.331056022544265
0.5975 0.316275462181985
0.6025 0.302195939753015
0.6075 0.28890623510095
0.6125 0.276460125959793
0.6175 0.26487484970709
0.6225 0.254134642782134
0.6275 0.244198300488583
0.6325 0.235008843217355
0.6375 0.226503176654411
0.6425 0.218620019900139
0.6475 0.211305100244882
0.6525 0.204513397474426
0.6575 0.198208855190955
0.6625 0.192362360243628
0.6675 0.186948915262344
0.6725 0.181944842188357
0.6775 0.177325630078609
0.6825 0.173064755079178
0.6875 0.169133521968747
0.6925 0.165501755966607
0.6975 0.162139039797925
0.7025 0.159016150284471
0.7075 0.156106386657481
0.7125 0.153386571843381
0.7175 0.150837616519517
0.7225 0.14844463628612
0.7275 0.146196686747079
0.7325 0.14408622243703
0.7375 0.142108395532208
0.7425 0.140260297366477
0.7475 0.138540220581907
0.7525 0.136946992218088
0.7575 0.135479405324883
0.7625 0.134135762100721
0.7675 0.132913534690799
0.7725 0.131809147455078
0.7775 0.130817882532615
0.7825 0.129933905610679
0.7875 0.129150400077716
0.7925 0.128459787035304
0.7975 0.127853999478442
0.8025 0.12732477467171
0.8075 0.126863930901653
0.8125 0.126463602559474
0.8175 0.126116418355617
0.8225 0.125815618386506
0.8275 0.125555114458726
0.8325 0.125329503464708
0.8375 0.125134045782418
0.8425 0.124964620423029
0.8475 0.124817666963882
0.8525 0.124690122035919
0.8575 0.12457935586753
0.8625 0.124483112430963
0.8675 0.124399455202832
0.8725 0.124326719429532
0.8775 0.124263471020959
0.8825 0.124208471708138
0.8875 0.124160649823498
0.8925 0.124119075938947
0.8975 0.124082942580037
0.9025 0.124051547286842
0.9075 0.124024278383951
0.9125 0.124000602930333
0.9175 0.12398005642796
0.9225 0.123962233965332
0.9275 0.123946782552754
0.9325 0.123933394468968
0.9375 0.123921801485361
0.9425 0.123911769868239
0.9475 0.123903096086674
0.9525 0.123895603178997
0.9575 0.123889137761161
0.9625 0.123883567699913
0.9675 0.123878780521018
0.9725 0.123874682643982
0.9775 0.12387119937193
0.9825 0.123868274619189
0.9875 0.123865865617798
0.9925 0.123863915343346
0.9975 0.123862249067918
};
\addlegendentry{POD}
\addplot [semithick, color0, mark=pentagon,mark size=2, mark repeat=25, mark options={solid}]
table {%
0.0025 0.999939748539756
0.0075 0.999939691227598
0.0125 0.999939575170477
0.0175 0.999939472486193
0.0225 0.999939326817791
0.0275 0.999939119658218
0.0325 0.999938851962678
0.0375 0.999938460807197
0.0425 0.999938105830015
0.0475 0.999937524469808
0.0525 0.999936729741211
0.0575 0.999936003190202
0.0625 0.999934709965227
0.0675 0.999932965765206
0.0725 0.999930800320819
0.0775 0.999928111067185
0.0825 0.999924475565935
0.0875 0.999919676030867
0.0925 0.999913643926191
0.0975 0.999905828458185
0.1025 0.999895610297337
0.1075 0.999882567363481
0.1125 0.999865992209659
0.1175 0.99984476951739
0.1225 0.99981783853414
0.1275 0.999783773619968
0.1325 0.999740784844527
0.1375 0.999687083949072
0.1425 0.999619722103652
0.1475 0.999535844685175
0.1525 0.999432137259879
0.1575 0.999304587976673
0.1625 0.999147707763582
0.1675 0.998956547238124
0.1725 0.998724333297175
0.1775 0.998444300837433
0.1825 0.998107602055638
0.1875 0.99770568526135
0.1925 0.997228300055632
0.1975 0.9966642386877
0.2025 0.996001928399962
0.2075 0.995228307154507
0.2125 0.994329543975301
0.2175 0.993291718813662
0.2225 0.992099572904408
0.2275 0.990737577995811
0.2325 0.989190108524874
0.2375 0.987441065745094
0.2425 0.985474502070783
0.2475 0.98327452615381
0.2525 0.980825564489724
0.2575 0.978112902778845
0.2625 0.975121886541064
0.2675 0.971839267115753
0.2725 0.968252209564432
0.2775 0.964349209665297
0.2825 0.960120758017859
0.2875 0.95555768861698
0.2925 0.950652893322209
0.2975 0.945400731781354
0.3025 0.939797690878503
0.3075 0.934260087445951
0.3125 0.928395335825208
0.3175 0.922133065401935
0.3225 0.915521407284989
0.3275 0.908563889276523
0.3325 0.901286943624608
0.3375 0.893947889264195
0.3425 0.886281654238701
0.3475 0.878299870528281
0.3525 0.870015412115325
0.3575 0.861444172138969
0.3625 0.852603819460059
0.3675 0.843514560435254
0.3725 0.834197588145542
0.3775 0.824746730832908
0.3825 0.815773046551607
0.3875 0.806606280593536
0.3925 0.797267998807514
0.3975 0.787780041424319
0.4025 0.778165214981597
0.4075 0.768446971256381
0.4125 0.758649606544238
0.4175 0.748798172586621
0.4225 0.738154956115744
0.4275 0.727441433506707
0.4325 0.716651244184528
0.4375 0.705778407625472
0.4425 0.694845443925796
0.4475 0.683870809510923
0.4525 0.672867709579758
0.4575 0.661843689994361
0.4625 0.650858930192697
0.4675 0.639988203795674
0.4725 0.629177449438243
0.4775 0.61877694995835
0.4825 0.608734680124773
0.4875 0.598905063950672
0.4925 0.589268625331804
0.4975 0.579727338746381
0.5025 0.570124700569954
0.5075 0.561270500557163
0.5125 0.552479314068571
0.5175 0.543034219612869
0.5225 0.53274636144917
0.5275 0.521713407375874
0.5325 0.510640766304464
0.5375 0.498740428413909
0.5425 0.48603227457557
0.5475 0.47260600894403
0.5525 0.458931265733181
0.5575 0.445353894924315
0.5625 0.431195471244745
0.5675 0.416837310752807
0.5725 0.402233228445626
0.5775 0.387502070158147
0.5825 0.372775582333979
0.5875 0.358234208960755
0.5925 0.344228067268164
0.5975 0.330641503708485
0.6025 0.317592416197444
0.6075 0.3051739003366
0.6125 0.293445707394336
0.6175 0.282433894701684
0.6225 0.272132290097383
0.6275 0.262509962209524
0.6325 0.253519554717992
0.6375 0.245107069301109
0.6425 0.237219013775197
0.6475 0.229807604725162
0.6525 0.222833372939091
0.6575 0.216264377754086
0.6625 0.210074918607298
0.6675 0.204242818559018
0.6725 0.198746776590363
0.6775 0.193564373020751
0.6825 0.188671609339042
0.6875 0.184042406125137
0.6925 0.179650091494505
0.6975 0.175468535281909
0.7025 0.171473949717788
0.7075 0.168071820114094
0.7125 0.164805877452286
0.7175 0.161660612823489
0.7225 0.158631950497436
0.7275 0.155719912443788
0.7325 0.152928386217891
0.7375 0.150263438908718
0.7425 0.147733457076053
0.7475 0.145346669910046
0.7525 0.143111342301544
0.7575 0.141034556839329
0.7625 0.139121096103619
0.7675 0.137373151329274
0.7725 0.135790394762388
0.7775 0.134369332749301
0.7825 0.133103761248864
0.7875 0.131985244985956
0.7925 0.131003591829003
0.7975 0.130147187946699
0.8025 0.129404256168084
0.8075 0.128762269846331
0.8125 0.128209807384664
0.8175 0.127735711538639
0.8225 0.127329558778841
0.8275 0.126982382140481
0.8325 0.126685762109283
0.8375 0.126432687378465
0.8425 0.126216820297906
0.8475 0.126032876328398
0.8525 0.125876132112283
0.8575 0.125742860926458
0.8625 0.125629545476001
0.8675 0.125533380091954
0.8725 0.125452004229793
0.8775 0.125383300682864
0.8825 0.125325494804061
0.8875 0.125277080597022
0.8925 0.125236646272242
0.8975 0.125203171434502
0.9025 0.125166063364118
0.9075 0.125128739537337
0.9125 0.125097310027251
0.9175 0.125071190966245
0.9225 0.125049355989083
0.9275 0.125031324748236
0.9325 0.125016331648788
0.9375 0.125003891686598
0.9425 0.124993549468808
0.9475 0.124984951570439
0.9525 0.124977916861192
0.9575 0.124972011440266
0.9625 0.124967053580361
0.9675 0.124962959462442
0.9725 0.124959461987974
0.9775 0.124956497635979
0.9825 0.124954251118768
0.9875 0.124952227641375
0.9925 0.124950619318928
0.9975 0.124949141142842
};
\addlegendentry{FCNN}
\addplot [semithick, green!50!black, mark=triangle,mark size=2, mark repeat=25, mark options={solid,rotate=180}, only marks]
table {%
0.0025 0.999841690063476
0.0075 1.00157768298418
0.0125 1.00145187133398
0.0175 1.0025885166266
0.0225 1.00229238852476
0.0275 1.00162512216813
0.0325 1.00365485900488
0.0375 1.00085276823777
0.0425 1.00248098373413
0.0475 1.00255330403646
0.0525 1.00001897567358
0.0575 1.0008953167842
0.0625 1.00257946894719
0.0675 1.00317575992682
0.0725 1.00149668180026
0.0775 1.00117781223395
0.0825 1.00353552744939
0.0875 1.00201038213877
0.0925 1.00221505531898
0.0975 1.00140865032489
0.1025 0.999895792741042
0.1075 1.00124542529766
0.1125 1.00264225250635
0.1175 1.0017083852719
0.1225 1.00179916773087
0.1275 1.00197902092567
0.1325 1.00310056637495
0.1375 1.00264298610198
0.1425 1.00230583777794
0.1475 1.00166803751236
0.1525 1.00041083800487
0.1575 1.0013484954834
0.1625 1.00191593170166
0.1675 1.00190278811332
0.1725 1.00161350690402
0.1775 1.00206307875804
0.1825 1.0031028282948
0.1875 1.00196043650309
0.1925 1.00189838653956
0.1975 1.00154002507528
0.2025 0.999436256213066
0.2075 1.0013634730608
0.2125 1.00098970608834
0.2175 1.00079811536349
0.2225 1.00017345868624
0.2275 0.999446832216703
0.2325 0.999572582733937
0.2375 0.997920097448887
0.2425 0.996778928316556
0.2475 0.995197296142578
0.2525 0.991145830888014
0.2575 0.990359905438545
0.2625 0.987404432052221
0.2675 0.985120198665521
0.2725 0.981604686150184
0.2775 0.977559273059551
0.2825 0.97483420983339
0.2875 0.969669390947391
0.2925 0.965662185962384
0.2975 0.959623899215307
0.3025 0.950862566630046
0.3075 0.944992456680689
0.3125 0.940355887779823
0.3175 0.932353826669546
0.3225 0.923714454357441
0.3275 0.916518431443434
0.3325 0.908176471025516
0.3375 0.898656294896052
0.3425 0.893710270906106
0.3475 0.881794905051207
0.3525 0.871296845949613
0.3575 0.860777695973714
0.3625 0.855641059386424
0.3675 0.843016123160338
0.3725 0.830003237112974
0.3775 0.82032350393442
0.3825 0.80804372445131
0.3875 0.793726872175168
0.3925 0.788378043052478
0.3975 0.773017345330654
0.4025 0.759121271280142
0.4075 0.74831852546105
0.4125 0.737922925215501
0.4175 0.724308307354267
0.4225 0.708511059100811
0.4275 0.695907947344658
0.4325 0.683895563467955
0.4375 0.668823841290596
0.4425 0.660165028694348
0.4475 0.643280286055345
0.4525 0.62992621690799
0.4575 0.619602692432893
0.4625 0.609846298511212
0.4675 0.596762926150591
0.4725 0.584436991275885
0.4775 0.575734774271647
0.4825 0.571940617683606
0.4875 0.562261067903959
0.4925 0.558517773946126
0.4975 0.553749585763002
0.5025 0.550055992908967
0.5075 0.550579352256579
0.5125 0.552320113548866
0.5175 0.551389608627711
0.5225 0.547289970593575
0.5275 0.547026242965307
0.5325 0.545610464536227
0.5375 0.541077577150785
0.5425 0.538295171199701
0.5475 0.530666082333296
0.5525 0.515002654148982
0.5575 0.504624813030928
0.5625 0.487025059186495
0.5675 0.467985654488588
0.5725 0.440198397025084
0.5775 0.412148390060816
0.5825 0.379088016656729
0.5875 0.350631444882124
0.5925 0.32552040540255
0.5975 0.295807214883658
0.6025 0.270062104249612
0.6075 0.257281187253121
0.6125 0.244320967258551
0.6175 0.230089899821159
0.6225 0.221662123998006
0.6275 0.216149687767029
0.6325 0.206605058449965
0.6375 0.209021415465917
0.6425 0.206697491499094
0.6475 0.207482316555121
0.6525 0.200891158519647
0.6575 0.202509531607995
0.6625 0.204399564327338
0.6675 0.203319940811548
0.6725 0.203927388558021
0.6775 0.204356251618801
0.6825 0.204821198414534
0.6875 0.206904197350526
0.6925 0.203668887798603
0.6975 0.204570262860029
0.7025 0.198694345278618
0.7075 0.201351443926493
0.7125 0.201564446473733
0.7175 0.197433798741072
0.7225 0.192084633387052
0.7275 0.183948400693062
0.7325 0.17380511149382
0.7375 0.161192783942589
0.7425 0.148635521913186
0.7475 0.134891898204119
0.7525 0.128009418646495
0.7575 0.126178065935771
0.7625 0.12883050319476
0.7675 0.124788498267149
0.7725 0.126260175154759
0.7775 0.126406260025807
0.7825 0.124152524349017
0.7875 0.127139137341426
0.7925 0.126614364293905
0.7975 0.126168972406632
0.8025 0.125105999983274
0.8075 0.124403765568366
0.8125 0.128735242745815
0.8175 0.124683991456643
0.8225 0.126545711969718
0.8275 0.125889159165896
0.8325 0.12448312380375
0.8375 0.12731501689324
0.8425 0.125849651984679
0.8475 0.125490801456647
0.8525 0.124697043345525
0.8575 0.124707451233497
0.8625 0.129072467486064
0.8675 0.124883437768007
0.8725 0.126796662807465
0.8775 0.125949818354387
0.8825 0.12470733660918
0.8875 0.126544053737934
0.8925 0.125861175549336
0.8975 0.125687771882766
0.9025 0.123987350708399
0.9075 0.12474822692382
0.9125 0.128930776547163
0.9175 0.125443599162958
0.9225 0.126453729776236
0.9275 0.125518494691604
0.9325 0.124389139505533
0.9375 0.126425318228893
0.9425 0.126059972322904
0.9475 0.126345539704347
0.9525 0.12450581177687
0.9575 0.12480697570703
0.9625 0.128966294802152
0.9675 0.124523387505458
0.9725 0.126605813319866
0.9775 0.126972465943067
0.9825 0.12454058879461
0.9875 0.127211052637834
0.9925 0.126055853489118
0.9975 0.126497898346339
};
\addlegendentry{CNN}

\nextgroupplot[
legend cell align={left},
legend style={at={(0.0,1)},anchor=north west, opacity=0.1, draw opacity=1, text opacity=1,draw=none},
tick align=outside,
tick pos=left,
x grid style={white!69.0196078431373!black},
xmajorgrids,
xlabel={\(x\)},
xmin=-0.04725, xmax=1.04725,
xtick style={color=black},
y grid style={white!69.0196078431373!black},
ymajorgrids,
ylabel={\(\rho u\)},
ymin=-0.0232244711945436, ymax=0.473814120212489,
ytick style={color=black},
width=.37\textwidth,
height=.4\textwidth
]
\addplot [semithick, black, dashed, mark=x,mark size=2, mark repeat=25, mark options={solid}]
table {%
0.0025 4.4436745499885e-07
0.0075 5.97062470073428e-07
0.0125 7.99408270642964e-07
0.0175 1.04705922162596e-06
0.0225 1.35567708717111e-06
0.0275 1.75172552417413e-06
0.0325 2.30844744858165e-06
0.0375 2.9880815542495e-06
0.0425 3.90773662630503e-06
0.0475 5.09393142708624e-06
0.0525 6.65892163132399e-06
0.0575 8.65601756644673e-06
0.0625 1.12774571035243e-05
0.0675 1.46677716873861e-05
0.0725 1.90850619305997e-05
0.0775 2.48119359347914e-05
0.0825 3.21826385237286e-05
0.0875 4.1699756025494e-05
0.0925 5.3943741472899e-05
0.0975 6.96811966648594e-05
0.1025 8.97875285320438e-05
0.1075 0.000115478392117381
0.1125 0.000148168627473818
0.1175 0.00018966355103918
0.1225 0.000242071494594385
0.1275 0.000308100613199496
0.1325 0.000390947667786122
0.1375 0.000494485968892232
0.1425 0.000623332477072447
0.1475 0.000782953325326417
0.1525 0.000979808754148224
0.1575 0.00122144670973211
0.1625 0.00151658846851006
0.1675 0.00187525642143297
0.1725 0.00230887351456253
0.1775 0.00283029961996043
0.1825 0.00345390691125881
0.1875 0.00419554200717569
0.1925 0.00507257145137372
0.1975 0.006103693005683
0.2025 0.00730891394626128
0.2075 0.00870933779277978
0.2125 0.010326898468218
0.2175 0.0121841129234184
0.2225 0.0143037056518193
0.2275 0.0167083646052653
0.2325 0.0194201102405193
0.2375 0.0224601437913052
0.2425 0.0258481733743399
0.2475 0.0296022484508076
0.2525 0.0337381343206392
0.2575 0.0382691659979407
0.2625 0.0432057288000136
0.2675 0.0485552556828182
0.2725 0.0543217941540574
0.2775 0.0605060639270607
0.2825 0.0671053302971474
0.2875 0.0741133599483847
0.2925 0.0815206213645865
0.2975 0.0893142990379972
0.3025 0.0974785360270983
0.3075 0.105994578817519
0.3125 0.114841105037565
0.3175 0.12399437397005
0.3225 0.133428589794627
0.3275 0.143116059063904
0.3325 0.153027585489409
0.3375 0.163132590884043
0.3425 0.173399399495716
0.3475 0.183795462230394
0.3525 0.194287584801184
0.3575 0.204842058018844
0.3625 0.215424936160414
0.3675 0.226002099564956
0.3725 0.236539589269847
0.3775 0.247003563785125
0.3825 0.257360487140066
0.3875 0.26757719008974
0.3925 0.277620758515484
0.3975 0.287458476938765
0.4025 0.297057744161116
0.4075 0.306385832842246
0.4125 0.315409954800661
0.4175 0.324097309196729
0.4225 0.332415449030078
0.4275 0.340332824538059
0.4325 0.347819504512935
0.4375 0.354847972163575
0.4425 0.361393558522961
0.4475 0.367434776935423
0.4525 0.372952737717246
0.4575 0.377929946760492
0.4625 0.382348185014653
0.4675 0.386186386435815
0.4725 0.389419924337175
0.4775 0.392023888666256
0.4825 0.393981529642316
0.4875 0.395294662164388
0.4925 0.395987916377527
0.4975 0.39609954903512
0.5025 0.395658922689244
0.5075 0.394662327013748
0.5125 0.393063103862165
0.5175 0.390783103999483
0.5225 0.387738694978082
0.5275 0.383869346410937
0.5325 0.379155366346474
0.5375 0.373617303066399
0.5425 0.36730289652528
0.5475 0.360271903528538
0.5525 0.352586221791485
0.5575 0.344306220689253
0.5625 0.335491438546631
0.5675 0.326203071289521
0.5725 0.316507211331304
0.5775 0.30647779078902
0.5825 0.296198456339372
0.5875 0.285762286750092
0.5925 0.275268547265266
0.5975 0.264816739940731
0.6025 0.254498650937767
0.6075 0.244390586966174
0.6125 0.234547403068611
0.6175 0.225000009951617
0.6225 0.215756629720444
0.6275 0.206807463826392
0.6325 0.198131204950899
0.6375 0.189702048050083
0.6425 0.181495519782867
0.6475 0.173492243942148
0.6525 0.165679454926409
0.6575 0.15805024700771
0.6625 0.150601350812323
0.6675 0.143330324967283
0.6725 0.136232905729637
0.6775 0.129301264105613
0.6825 0.122523504835293
0.6875 0.115884391171121
0.6925 0.109367185701245
0.6975 0.102956018945085
0.7025 0.0966384288349653
0.7075 0.0904075365474779
0.7125 0.0842635834942107
0.7175 0.0782146371436074
0.7225 0.0722764392491349
0.7275 0.0664714987092887
0.7325 0.06082760069483
0.7375 0.0553758961220649
0.7425 0.0501487940419877
0.7475 0.0451777667426355
0.7525 0.0404913098241465
0.7575 0.0361131099955101
0.7625 0.032060594644974
0.7675 0.0283440095410646
0.7725 0.024966001985276
0.7775 0.0219218460910474
0.7825 0.0192001441371101
0.7875 0.016783957884182
0.7925 0.014652206355614
0.7975 0.0127811044817779
0.8025 0.0111455446111314
0.8075 0.009720295606457
0.8125 0.00848092617497202
0.8175 0.00740448493035834
0.8225 0.00646991604531763
0.8275 0.00565827669150929
0.8325 0.00495279857000134
0.8375 0.00433881857159492
0.8425 0.00380365125281355
0.8475 0.00333640757172043
0.8525 0.00292779445498711
0.8575 0.00256990433168742
0.8625 0.00225601870974582
0.8675 0.001980431041221
0.8725 0.00173826783925844
0.8775 0.00152536140111038
0.8825 0.00133812140898467
0.8875 0.00117343640379604
0.8925 0.00102860020684167
0.8975 0.000901238476548837
0.9025 0.000789272109464669
0.9075 0.000690863406680269
0.9125 0.000604396523084081
0.9175 0.000528444440882734
0.9225 0.000461749706104483
0.9275 0.000403204600710479
0.9325 0.000351834897180649
0.9375 0.000306778803075717
0.9425 0.000267281453650786
0.9475 0.000232677784990514
0.9525 0.000202381785840636
0.9575 0.000175874391806256
0.9625 0.000152694370315404
0.9675 0.000132436303187084
0.9725 0.000114735976118324
0.9775 9.92646520308334e-05
0.9825 8.57258983698519e-05
0.9875 7.38453284255894e-05
0.9925 6.33765864595079e-05
0.9975 5.41186895572472e-05
};
\addlegendentry{FOM}
\addplot [semithick, red, mark=o,mark size=2, mark repeat=25, mark options={solid}]
table {%
0.0025 0.0062600192592862
0.0075 0.00626007040551888
0.0125 0.00626014048425322
0.0175 0.00626023310166612
0.0225 0.0062603544523252
0.0275 0.00626051320389498
0.0325 0.00626072092607204
0.0375 0.00626099284656747
0.0425 0.00626134890672328
0.0475 0.00626181515624492
0.0525 0.00626242556201539
0.0575 0.00626322433412447
0.0625 0.00626426890058563
0.0675 0.00626563369374165
0.0725 0.00626741494718634
0.0775 0.00626973674293628
0.0825 0.00627275859452341
0.0875 0.00627668490238315
0.0925 0.00628177667226014
0.0975 0.00628836594373728
0.1025 0.00629687343171148
0.1075 0.0063078299349464
0.1125 0.00632190210782474
0.1175 0.00633992321766196
0.1225 0.00636292951292474
0.1275 0.00639220279841957
0.1325 0.00642931974238764
0.1375 0.00647620831748141
0.1425 0.00653521159337205
0.1475 0.0066091588458083
0.1525 0.0067014436207368
0.1575 0.00681610799309832
0.1625 0.00695793179465215
0.1675 0.00713252506825813
0.1725 0.00734642146033343
0.1775 0.00760716972062289
0.1825 0.00792341997850413
0.1875 0.00830500105299799
0.1925 0.00876298477672356
0.1975 0.00930973321688419
0.2025 0.00995892479567539
0.2075 0.0107255556715872
0.2125 0.0116259133468907
0.2175 0.0126775202989951
0.2225 0.0138990464555852
0.2275 0.0153101904866576
0.2325 0.0169315310958639
0.2375 0.0187843506753947
0.2425 0.0208904347590622
0.2475 0.0232718515916525
0.2525 0.025950716769693
0.2575 0.0289489482616618
0.2625 0.0322880171709425
0.2675 0.0359886993723744
0.2725 0.040070832662665
0.2775 0.0445530833594871
0.2825 0.0494527254137574
0.2875 0.0547854341150499
0.2925 0.0605650954185703
0.2975 0.066803630847288
0.3025 0.0735108368681565
0.3075 0.0806942366578606
0.3125 0.0883589413282355
0.3175 0.0965075170649424
0.3225 0.105139854359177
0.3275 0.11425303570711
0.3325 0.123841198924194
0.3375 0.133895394615914
0.3425 0.144403438280727
0.3475 0.155349759720937
0.3525 0.166715254391251
0.3575 0.178477142279737
0.3625 0.190608839017589
0.3675 0.203079840384468
0.3725 0.215855614950675
0.3775 0.228897491046852
0.3825 0.242162515915183
0.3875 0.255603260938862
0.3925 0.26916755281826
0.3975 0.282798131808214
0.4025 0.296432276902629
0.4075 0.310001489700602
0.4125 0.323431379819163
0.4175 0.336641922873029
0.4225 0.349548243042551
0.4275 0.362061991591608
0.4325 0.374093256647815
0.4375 0.385552776062503
0.4425 0.396354067658105
0.4475 0.40641495514114
0.4525 0.4156578592761
0.4575 0.424008231635126
0.4625 0.431390964931745
0.4675 0.437726087810794
0.4725 0.442927697436931
0.4775 0.44691221110926
0.4825 0.449619385635629
0.4875 0.451038801603048
0.4925 0.451221456966715
0.4975 0.45025830760472
0.5025 0.448225912227799
0.5075 0.445128950450256
0.5125 0.440879733560757
0.5175 0.435329845057778
0.5225 0.428333735334988
0.5275 0.419812428113606
0.5325 0.409786538264864
0.5375 0.398365263656221
0.5425 0.385708967235167
0.5475 0.37199218959285
0.5525 0.357382540369815
0.5575 0.342035653412166
0.5625 0.326099037378626
0.5675 0.309718342463012
0.5725 0.29304291179853
0.5775 0.276229592333591
0.5825 0.259444100961741
0.5875 0.242858989306278
0.5925 0.226647532951254
0.5975 0.210973932139066
0.6025 0.195981574076951
0.6075 0.181782059301807
0.6125 0.168447736664603
0.6175 0.156009575706663
0.6225 0.144460714223272
0.6275 0.133764540352922
0.6325 0.123865206389541
0.6375 0.114698247823318
0.6425 0.106199409401684
0.6475 0.0983105795742747
0.6525 0.0909825906472759
0.6575 0.0841753230563736
0.6625 0.0778559520770433
0.6675 0.071996293631247
0.6725 0.0665701045920575
0.6775 0.0615509567606551
0.6825 0.0569110147228712
0.6875 0.0526207735268128
0.6925 0.0486495997131755
0.6975 0.0449667932837786
0.7025 0.0415428512411749
0.7075 0.0383506496790804
0.7125 0.0353663441373671
0.7175 0.0325698873368719
0.7225 0.0299451543419271
0.7275 0.0274797316112294
0.7325 0.0251644623762688
0.7375 0.0229928488341954
0.7425 0.0209603995052474
0.7475 0.0190639874274071
0.7525 0.0173012604960941
0.7575 0.0156701255510268
0.7625 0.0141683158242827
0.7675 0.0127930468746204
0.7725 0.0115407663544456
0.7775 0.0104070038471787
0.7825 0.00938632497421577
0.7875 0.00847238737697508
0.7925 0.00765808607176346
0.7975 0.00693576518857187
0.8025 0.00629746587755054
0.8075 0.00573517859701834
0.8125 0.0052410722890083
0.8175 0.00480768133602611
0.8225 0.0044280409931152
0.8275 0.00409577077391929
0.8325 0.00380511158086698
0.8375 0.00355092583360814
0.8425 0.00332867087097416
0.8475 0.00313435523588559
0.8525 0.00296448586617855
0.8575 0.00281601230361408
0.8625 0.00268627218029253
0.8675 0.00257294064637399
0.8725 0.00247398514059931
0.8775 0.00238762597786761
0.8825 0.00231230260128805
0.8875 0.00224664496923474
0.8925 0.00218944936603677
0.8975 0.0021396578845305
0.9025 0.00209634088119668
0.9075 0.00205868180868918
0.9125 0.0020259639537252
0.9175 0.00199755872739317
0.9225 0.0019729152557654
0.9275 0.00195155109478082
0.9325 0.00193304394399322
0.9375 0.00191702426193665
0.9425 0.00190316869618747
0.9475 0.00189119423817449
0.9525 0.00188085299945855
0.9575 0.00187192748345165
0.9625 0.0018642261936353
0.9675 0.0018575793784486
0.9725 0.00185183468759296
0.9775 0.00184685261634736
0.9825 0.00184250226941407
0.9875 0.00183866060527194
0.9925 0.00183522710042123
0.9975 0.00183219110923913
};
\addlegendentry{POD}
\addplot [semithick, color0, mark=pentagon,mark size=2, mark repeat=25, mark options={solid}]
table {%
0.0025 -0.000631807948769392
0.0075 -0.000631490037736713
0.0125 -0.000631546125276371
0.0175 -0.000631447298541644
0.0225 -0.000631036194020578
0.0275 -0.000630776789149651
0.0325 -0.000630034731405866
0.0375 -0.000629418870626427
0.0425 -0.000628891213013844
0.0475 -0.000627896853188038
0.0525 -0.000626794206447513
0.0575 -0.000625252931879017
0.0625 -0.000623044178849921
0.0675 -0.000620400839234849
0.0725 -0.000616628156189952
0.0775 -0.000611449580181555
0.0825 -0.000605452866988469
0.0875 -0.00059756794472925
0.0925 -0.000587377157172608
0.0975 -0.000574367787054611
0.1025 -0.000557506347980095
0.1075 -0.000535701306623739
0.1125 -0.000507954629303415
0.1175 -0.000472606130968072
0.1225 -0.000428145489288693
0.1275 -0.000371216881449591
0.1325 -0.000300095401297734
0.1375 -0.000210248765380809
0.1425 -9.83303661392677e-05
0.1475 4.10902459005241e-05
0.1525 0.000213118537006446
0.1575 0.000426264015625842
0.1625 0.000687119416907268
0.1675 0.001005901735249
0.1725 0.00139287529656513
0.1775 0.0018594114515223
0.1825 0.00242111508849028
0.1875 0.00309184331993063
0.1925 0.00388878902798813
0.1975 0.0048310368246522
0.2025 0.00593784714655254
0.2075 0.0072308277489616
0.2125 0.00873312793725966
0.2175 0.0104677618801456
0.2225 0.0124594097176933
0.2275 0.014734013896389
0.2325 0.0173154684735179
0.2375 0.0202297125278649
0.2425 0.0235015607544912
0.2475 0.0271546055258476
0.2525 0.0312115789809308
0.2575 0.0356925009730848
0.2625 0.0406167005788213
0.2675 0.046001841331894
0.2725 0.0518597536925459
0.2775 0.0582043165714529
0.2825 0.06504124921381
0.2875 0.0723768230454162
0.2925 0.0802116460894907
0.2975 0.0885438763050951
0.3025 0.0973674643510447
0.3075 0.107021789041389
0.3125 0.116440720848842
0.3175 0.12490586230726
0.3225 0.133720003001093
0.3275 0.142862778316196
0.3325 0.152341120788812
0.3375 0.162497892875388
0.3425 0.172917859936176
0.3475 0.183567577877488
0.3525 0.194413266753941
0.3575 0.205418398391938
0.3625 0.216543749721818
0.3675 0.227748402924002
0.3725 0.238989836785558
0.3775 0.250076203785205
0.3825 0.259722955769845
0.3875 0.269352678236485
0.3925 0.278929552861797
0.3975 0.288415640588901
0.4025 0.297770398729563
0.4075 0.306953443918453
0.4125 0.315922608183921
0.4175 0.324631326162623
0.4225 0.33326601881453
0.4275 0.34142263404496
0.4325 0.349225434073584
0.4375 0.356678482797793
0.4425 0.363737897044887
0.4475 0.370362185506318
0.4525 0.376512436900364
0.4575 0.382150750684403
0.4625 0.38714794874348
0.4675 0.3913536713747
0.4725 0.394871373208366
0.4775 0.397388974356512
0.4825 0.398981831092228
0.4875 0.399835243726848
0.4925 0.399990595924931
0.4975 0.399508213203089
0.5025 0.398436639479984
0.5075 0.396693275689119
0.5125 0.394299353321648
0.5175 0.391241993459969
0.5225 0.387457434184736
0.5275 0.383064357158505
0.5325 0.37821331526778
0.5375 0.372720766784397
0.5425 0.36681159708666
0.5475 0.360588362026546
0.5525 0.355096663704128
0.5575 0.350327936584653
0.5625 0.343972632758207
0.5675 0.336688651870459
0.5725 0.328822274105152
0.5775 0.32041557638353
0.5825 0.311526879245742
0.5875 0.302190853608558
0.5925 0.292278158329648
0.5975 0.282171426890791
0.6025 0.271971512556488
0.6075 0.261768723888853
0.6125 0.251635805692066
0.6175 0.241624864362019
0.6225 0.231764247220923
0.6275 0.222064426718653
0.6325 0.212523519697695
0.6375 0.203134767425826
0.6425 0.193891589567163
0.6475 0.184792477054611
0.6525 0.175842542977689
0.6575 0.167050575659923
0.6625 0.158431217163791
0.6675 0.14999581762344
0.6725 0.141752426090658
0.6775 0.13370318635245
0.6825 0.125844669476259
0.6875 0.118165959822679
0.6925 0.110653909333516
0.6975 0.103294400083125
0.7025 0.0960750370007804
0.7075 0.0894654472710603
0.7125 0.0829750399811243
0.7175 0.0766050513223901
0.7225 0.0703740820599711
0.7275 0.0643055789897624
0.7325 0.058428695233126
0.7375 0.0527750463136163
0.7425 0.0473773895280973
0.7475 0.0422663940147378
0.7525 0.0374695763449913
0.7575 0.0330099585557321
0.7625 0.0289021234991204
0.7675 0.0251548345470746
0.7725 0.021769468875248
0.7775 0.0187375282416859
0.7825 0.0160458188486836
0.7875 0.0136745361921076
0.7925 0.0116006821457647
0.7975 0.00979790352603309
0.8025 0.00823835164484893
0.8075 0.0068948125034414
0.8125 0.00574062449795137
0.8175 0.00475215199626958
0.8225 0.00390651131282701
0.8275 0.00318400849594951
0.8325 0.00256670474819961
0.8375 0.00203905791230377
0.8425 0.00158857323356673
0.8475 0.00120324041647867
0.8525 0.000873574329433593
0.8575 0.000591753822194752
0.8625 0.000350643592267582
0.8675 0.000144431660200405
0.8725 -3.20908287142093e-05
0.8775 -0.000182570176864047
0.8825 -0.000311330115125025
0.8875 -0.000421055650765921
0.8925 -0.000514183530215367
0.8975 -0.000593810018340587
0.9025 -0.000616745505283568
0.9075 -0.000604407532406354
0.9125 -0.000591894867661592
0.9175 -0.000580566654191215
0.9225 -0.000569624930404658
0.9275 -0.000559549103934931
0.9325 -0.000550545278121026
0.9375 -0.000542154282155123
0.9425 -0.000534904477805547
0.9475 -0.000528278463641959
0.9525 -0.000522401303856251
0.9575 -0.000517344944801165
0.9625 -0.000513090221192944
0.9675 -0.000509410517328195
0.9725 -0.000506246714735798
0.9775 -0.00050340667073073
0.9825 -0.000501216899292553
0.9875 -0.000498651904663233
0.9925 -0.000496241671230978
0.9975 -0.000494182474769158
};
\addlegendentry{FCNN}
\addplot [semithick, green!50!black, mark=triangle,mark size=2, mark repeat=25, mark options={solid,rotate=180}, only marks]
table {%
0.0025 0.000990276688505998
0.0075 0.00187346175142161
0.0125 0.000956459126158295
0.0175 0.00111032418942201
0.0225 0.00118305443355243
0.0275 0.000827656629166061
0.0325 0.000691186990065389
0.0375 0.00166181710439562
0.0425 0.00175891415643347
0.0475 0.0025517450827575
0.0525 0.00173716383113934
0.0575 0.00150303271027979
0.0625 0.00176329713446111
0.0675 0.00073753527777329
0.0725 0.000993213476437076
0.0775 0.000639066816254604
0.0825 0.000734195993051566
0.0875 0.000873181985378367
0.0925 0.00145871509211356
0.0975 0.00220339796449235
0.1025 0.00121188850202064
0.1075 0.000607562745446116
0.1125 0.00126247139801679
0.1175 0.000905569549864286
0.1225 0.00139829101712817
0.1275 0.00129545967759409
0.1325 0.00117681401787554
0.1375 0.00116151532964278
0.1425 0.00202973556888819
0.1475 0.0013777459716733
0.1525 0.00113482140780733
0.1575 0.00134956276464138
0.1625 0.00129371936133599
0.1675 0.0013712722426807
0.1725 0.00103588676406005
0.1775 0.00101600286540407
0.1825 0.00100900922833961
0.1875 0.0013310273908215
0.1925 0.00167954033921098
0.1975 0.00210040316035736
0.2025 0.00142412265570725
0.2075 0.00218474513011934
0.2125 0.00185787549017274
0.2175 0.00242717031932594
0.2225 0.00272509950922413
0.2275 0.00554158091520655
0.2325 0.00501972896450751
0.2375 0.00741562516264531
0.2425 0.00903226747445517
0.2475 0.0102189644464867
0.2525 0.016028460442091
0.2575 0.0204425711567586
0.2625 0.0247487806306942
0.2675 0.0277115491715486
0.2725 0.0285183165264044
0.2775 0.0394620391275827
0.2825 0.0404231770026746
0.2875 0.0453527144118873
0.2925 0.0531651155353833
0.2975 0.057593675254728
0.3025 0.07810789210525
0.3075 0.0888079670094838
0.3125 0.100107916660267
0.3175 0.10611079366116
0.3225 0.111786485160396
0.3275 0.126498960023075
0.3325 0.135245227393099
0.3375 0.14244712586331
0.3425 0.152119756857271
0.3475 0.161680623422925
0.3525 0.17728119451156
0.3575 0.195771888660809
0.3625 0.213513431552896
0.3675 0.224904500811553
0.3725 0.22976089408802
0.3775 0.247539473043291
0.3825 0.259006029639144
0.3875 0.263381734217764
0.3925 0.278551351537491
0.3975 0.292413506808738
0.4025 0.28534933405143
0.4075 0.301925296693947
0.4125 0.325359533059993
0.4175 0.33721999117675
0.4225 0.340365561057586
0.4275 0.35717179868544
0.4325 0.366781022712937
0.4375 0.370393693424151
0.4425 0.386592727669981
0.4475 0.396976764144343
0.4525 0.376480683082908
0.4575 0.382533676415473
0.4625 0.396888720020967
0.4675 0.406607204118947
0.4725 0.407934181630439
0.4775 0.410997641751611
0.4825 0.41473210313585
0.4875 0.417731766439391
0.4925 0.423495037774944
0.4975 0.420958239502959
0.5025 0.416038392852477
0.5075 0.416642390696328
0.5125 0.416973869432982
0.5175 0.420014043483255
0.5225 0.41714056967714
0.5275 0.413960954320205
0.5325 0.411179128042918
0.5375 0.414924135386088
0.5425 0.410818195013714
0.5475 0.393717168755193
0.5525 0.40115810202693
0.5575 0.411407132842746
0.5625 0.365318496808962
0.5675 0.353533185299319
0.5725 0.338222763075476
0.5775 0.304273681451843
0.5825 0.272756410479436
0.5875 0.258566636841589
0.5925 0.226470683056211
0.5975 0.189270391042468
0.6025 0.18027032496892
0.6075 0.170689002187637
0.6125 0.159070458630774
0.6175 0.150926852958939
0.6225 0.139427758750391
0.6275 0.140715708293604
0.6325 0.122731400333482
0.6375 0.126532109934631
0.6425 0.121545605483923
0.6475 0.130906543066207
0.6525 0.115198879955057
0.6575 0.120824742002963
0.6625 0.119357763601003
0.6675 0.126076765620363
0.6725 0.124074290909135
0.6775 0.123734517871169
0.6825 0.12981831822082
0.6875 0.133076137436779
0.6925 0.12436016261217
0.6975 0.123142662336731
0.7025 0.105220869163334
0.7075 0.103484978271743
0.7125 0.112250033363591
0.7175 0.111705513781945
0.7225 0.0997635001904637
0.7275 0.0916693451345243
0.7325 0.0806143794716293
0.7375 0.053541960968245
0.7425 0.0449047103431749
0.7475 0.0167015871068529
0.7525 0.00579635965356813
0.7575 0.00438916708799925
0.7625 0.00245099068421994
0.7675 0.00142430349835415
0.7725 0.0016691646615766
0.7775 0.00221383891918055
0.7825 0.0022647340696426
0.7875 0.00112485616823951
0.7925 0.00121159903634666
0.7975 0.000618719086368582
0.8025 0.00195746239831248
0.8075 0.00211052270715228
0.8125 0.00226499978070228
0.8175 0.00204830467425802
0.8225 0.00213557941032228
0.8275 0.00215728376648924
0.8325 0.00215253516465513
0.8375 0.00149328793778869
0.8425 0.00210283045449903
0.8475 0.00190540611286073
0.8525 0.00128473917323556
0.8575 0.000803198318497643
0.8625 0.00101323873867382
0.8675 0.000237003719991305
0.8725 0.00221494467733775
0.8775 0.00183400172912735
0.8825 0.00137541812686063
0.8875 0.00195466124728841
0.8925 0.00177485548510151
0.8975 0.0018088387712864
0.9025 0.00146071593262078
0.9075 0.00274721246216447
0.9125 0.00167953047241935
0.9175 0.000550836205683706
0.9225 0.00199540733327809
0.9275 0.00287023866107769
0.9325 0.00141074914415169
0.9375 0.0013739437490657
0.9425 0.0017236385951483
0.9475 0.00093101545030355
0.9525 0.00180932305171964
0.9575 0.00075867152133363
0.9625 0.00141641223779588
0.9675 0.00112006187425264
0.9725 0.00176333489864244
0.9775 0.00194067231155268
0.9825 0.000818390347219144
0.9875 0.00369147800376881
0.9925 0.00273298329524285
0.9975 0.00306609981605222
};
\addlegendentry{CNN}

\nextgroupplot[
legend cell align={left},
legend style={at={(1,1)},anchor=north east,fill opacity=0.1, draw opacity=1, text opacity=1, draw=none},
tick align=outside,
tick pos=left,
x grid style={white!69.0196078431373!black},
xmajorgrids,
xlabel={\(x\)},
xmin=-0.04725, xmax=1.04725,
xtick style={color=black},
y grid style={white!69.0196078431373!black},
ymajorgrids,
ylabel={\(E\)},
ymin=-0.117868330334208, ymax=0.603154515104935,
ytick style={color=black},
width=.35\textwidth,
height=.4\textwidth,
clip=false,
%y label style={yshift=-.7em}
]
\addplot [semithick, black, dashed, mark=x,mark size=2, mark repeat=25, mark options={solid}]
table {%
0.0025 0.499999361504023
0.0075 0.499999183620267
0.0125 0.499998956656357
0.0175 0.499998657893175
0.0225 0.499998272625194
0.0275 0.499997777704083
0.0325 0.499997126285128
0.0375 0.499996272281854
0.0425 0.499995169976743
0.0475 0.499993730350822
0.0525 0.499991876410342
0.0575 0.499989476562542
0.0625 0.499986364006075
0.0675 0.499982355915585
0.0725 0.499977180296728
0.0775 0.499970525776456
0.0825 0.499961955650545
0.0875 0.499950973892023
0.0925 0.49993691017478
0.0975 0.499918981086569
0.1025 0.499896128536149
0.1075 0.499867124733552
0.1125 0.499830438144633
0.1175 0.499784161496544
0.1225 0.499725999880656
0.1275 0.499653169032844
0.1325 0.499562332427576
0.1375 0.499449480224796
0.1425 0.499309905902888
0.1475 0.499138042032686
0.1525 0.49892740118424
0.1575 0.498670486468602
0.1625 0.498358703749034
0.1675 0.497982270415046
0.1725 0.497530226757131
0.1775 0.496990335519934
0.1825 0.49634912991411
0.1875 0.495591999394454
0.1925 0.494703208363668
0.1975 0.493666077783127
0.2025 0.492463172397307
0.2075 0.491076549512914
0.2125 0.489488046580558
0.2175 0.487679600013574
0.2225 0.485633687799318
0.2275 0.483333679718255
0.2325 0.48076423294581
0.2375 0.477911799239919
0.2425 0.474764935511201
0.2475 0.471314719488362
0.2525 0.467555088721707
0.2575 0.463483090324178
0.2625 0.459099083657965
0.2675 0.454406855911048
0.2725 0.449413715086325
0.2775 0.444130393539245
0.2825 0.438571026778881
0.2875 0.432752851994598
0.2925 0.426696094320604
0.2975 0.420423598708122
0.3025 0.413960495202088
0.3075 0.407333857180233
0.3125 0.400572300712034
0.3175 0.393705637998278
0.3225 0.386764405139074
0.3275 0.379779558522867
0.3325 0.372782111535813
0.3375 0.365802766850548
0.3425 0.358871639307349
0.3475 0.352017958367442
0.3525 0.345269841759585
0.3575 0.338654021796531
0.3625 0.332195673534968
0.3675 0.325918213229631
0.3725 0.319843193919365
0.3775 0.313990117618382
0.3825 0.308376397774443
0.3875 0.303017309076339
0.3925 0.297925928448431
0.3975 0.293113196380939
0.4025 0.288587922654914
0.4075 0.284356778976805
0.4125 0.28042433520336
0.4175 0.276792931214978
0.4225 0.273462577948832
0.4275 0.270430830226565
0.4325 0.267692592262936
0.4375 0.265240081844257
0.4425 0.263062808250952
0.4475 0.26114780211783
0.4525 0.259479893722797
0.4575 0.258042107206346
0.4625 0.256815969293871
0.4675 0.255781716317408
0.4725 0.254918098730885
0.4775 0.254201785541831
0.4825 0.253606557360809
0.4875 0.253102949250893
0.4925 0.252659205893312
0.4975 0.252243789434365
0.5025 0.251827275987589
0.5075 0.251381014770881
0.5125 0.250873031148202
0.5175 0.250265601411364
0.5225 0.249517692496044
0.5275 0.248590686884939
0.5325 0.247453394320449
0.5375 0.246084020788545
0.5425 0.244469269449419
0.5475 0.242601861073284
0.5525 0.240478184700032
0.5575 0.238096661384523
0.5625 0.235457211563669
0.5675 0.23256160323498
0.5725 0.229414723541784
0.5775 0.226026208963633
0.5825 0.222411970099903
0.5875 0.21859481129097
0.5925 0.214603594912074
0.5975 0.21047092978495
0.6025 0.206229560100738
0.6075 0.201908555413813
0.6125 0.197530027012126
0.6175 0.19310734820258
0.6225 0.188645123357809
0.6275 0.184140898682197
0.6325 0.179587896810172
0.6375 0.174978196029754
0.6425 0.170305413552305
0.6475 0.165566403039851
0.6525 0.160761829825727
0.6575 0.155895508695644
0.6625 0.150972948924719
0.6675 0.14599961589413
0.6725 0.140979355280142
0.6775 0.135913453516059
0.6825 0.130800568473667
0.6875 0.125637492433925
0.6925 0.120420696567495
0.6975 0.115148229960878
0.7025 0.109821680117693
0.7075 0.104447879759985
0.7125 0.099040050456216
0.7175 0.0936183095804956
0.7225 0.0882094557276782
0.7275 0.0828461162686445
0.7325 0.0775653754621413
0.7375 0.0724069851934649
0.7425 0.0674113530543212
0.7475 0.0626174208730466
0.7525 0.0580606180437754
0.7575 0.0537710116311263
0.7625 0.0497718581564984
0.7675 0.0460786337784703
0.7725 0.0426986709975221
0.7775 0.0396313903583002
0.7825 0.0368690814934464
0.7875 0.0343980870855545
0.7925 0.0322002370953793
0.7975 0.0302543369596147
0.8025 0.0285375636073226
0.8075 0.0270266753831553
0.8125 0.0256989512538859
0.8175 0.0245328807748135
0.8225 0.0235086070033041
0.8275 0.0226081683524312
0.8325 0.0218155829214253
0.8375 0.0211168158856341
0.8425 0.0204996764296028
0.8475 0.0199536540088463
0.8525 0.0194697400800231
0.8575 0.0190402333117841
0.8625 0.0186585512793067
0.8675 0.0183190559120018
0.8725 0.0180168954550742
0.8775 0.017747867356557
0.8825 0.0175083087955527
0.8875 0.0172950032123299
0.8925 0.0171051056839067
0.8975 0.0169360900548237
0.9025 0.0167857058531322
0.9075 0.0166519413428778
0.9125 0.0165330009502372
0.9175 0.0164272791506258
0.9225 0.0163333443756653
0.9275 0.0162499199997569
0.9325 0.0161758679740341
0.9375 0.0161101745659652
0.9425 0.0160519363070998
0.9475 0.0160003481569746
0.9525 0.0159546901722751
0.9575 0.0159143188253597
0.9625 0.0158786592261423
0.9675 0.0158471966022788
0.9725 0.015819471627134
0.9775 0.0157950755882109
0.9825 0.0157736453028088
0.9875 0.0157548627175762
0.9925 0.0157384515158429
0.9975 0.0157241723861597
};
\addlegendentry{FOM}
\addplot [semithick, red, mark=o,mark size=2, mark repeat=25, mark options={solid}]
table {%
0.0025 0.496026834244761
0.0075 0.496026784168737
0.0125 0.49602671969299
0.0175 0.496026635691211
0.0225 0.496026525736975
0.0275 0.496026381621337
0.0325 0.496026192698993
0.0375 0.4960259450845
0.0425 0.496025620643744
0.0475 0.496025195709419
0.0525 0.496024639437358
0.0575 0.496023911704497
0.0625 0.496022960428407
0.0675 0.496021718162667
0.0725 0.49602009779221
0.0775 0.496017987118061
0.0825 0.496015242082117
0.0875 0.49601167834026
0.0925 0.496007060847421
0.0975 0.496001091072989
0.1025 0.495993391421833
0.1075 0.49598348639884
0.1125 0.495970780027982
0.1175 0.495954529026425
0.1225 0.495933811247129
0.1275 0.495907488947688
0.1325 0.495874166527358
0.1375 0.495832142506849
0.1425 0.495779355714125
0.1475 0.495713325889685
0.1525 0.495631089239132
0.1575 0.495529129836633
0.1625 0.495403308211861
0.1675 0.495248788919343
0.1725 0.4950599693693
0.1775 0.4948304126619
0.1825 0.494552787574625
0.1875 0.494218819163569
0.1925 0.493819253611612
0.1975 0.493343840951512
0.2025 0.492781339080651
0.2075 0.492119542051452
0.2125 0.491345334969818
0.2175 0.490444776986442
0.2225 0.489403212865514
0.2275 0.48820541252315
0.2325 0.486835736816765
0.2375 0.485278326815333
0.2425 0.483517312864128
0.2475 0.481537039040688
0.2525 0.479322298127445
0.2575 0.476858572025354
0.2625 0.474132272602276
0.2675 0.471130978289698
0.2725 0.467843662274004
0.2775 0.464260908826629
0.2825 0.460375115129565
0.2875 0.456180676829434
0.2925 0.451674156449146
0.2975 0.446854434658495
0.3025 0.441722845208988
0.3075 0.43628329501869
0.3125 0.430542371376156
0.3175 0.42450943842263
0.3225 0.41819672485441
0.3275 0.411619404049749
0.3325 0.404795666490254
0.3375 0.397746782426705
0.3425 0.390497150390604
0.3475 0.383074324728323
0.3525 0.375509013398015
0.3575 0.367835036545584
0.3625 0.360089237604699
0.3675 0.352311342335349
0.3725 0.344543767182771
0.3775 0.336831385399502
0.3825 0.329221264937432
0.3875 0.321762392261139
0.3925 0.314505386491973
0.3975 0.307502185474024
0.4025 0.300805650325001
0.4075 0.29446899563923
0.4125 0.288544924837368
0.4175 0.283084355835123
0.4225 0.278134680177799
0.4275 0.273737612933219
0.4325 0.269926838648716
0.4375 0.266725790115062
0.4425 0.264145944843835
0.4475 0.26218592775786
0.4525 0.260831433700222
0.4575 0.260055548831015
0.4625 0.259818585849822
0.4675 0.260066401147027
0.4725 0.260726916758803
0.4775 0.261706626314939
0.4825 0.26289137069062
0.4875 0.264155781204281
0.4925 0.265380662180395
0.4975 0.266469535232779
0.5025 0.267355228057877
0.5075 0.267993513015443
0.5125 0.268347498708088
0.5175 0.268368491860145
0.5225 0.2679821051476
0.5275 0.267089201906358
0.5325 0.265584916609639
0.5375 0.263384496046043
0.5425 0.260440887734434
0.5475 0.256746907668944
0.5525 0.252325328530938
0.5575 0.247215333450493
0.5625 0.24146244755571
0.5675 0.23511484113163
0.5725 0.228225222098359
0.5775 0.220855642520863
0.5825 0.213082150057144
0.5875 0.204996741427499
0.5925 0.196705161502968
0.5975 0.18832049028865
0.6025 0.17995383660725
0.6075 0.17170439747941
0.6125 0.163651330948107
0.6175 0.155849287584302
0.6225 0.148328318820406
0.6275 0.141097667981083
0.6325 0.134152057682715
0.6375 0.127478729864486
0.6425 0.121063659615039
0.6475 0.114895879850661
0.6525 0.108969498367532
0.6575 0.10328357404809
0.6625 0.097840430474844
0.6675 0.0926431812470018
0.6725 0.0876932349695655
0.6775 0.0829883867235971
0.6825 0.078521852113852
0.6875 0.0742823286931795
0.6925 0.0702549375263153
0.6975 0.0664227450843482
0.7025 0.0627685075233877
0.7075 0.0592763065346707
0.7125 0.05593283218737
0.7175 0.0527281805380399
0.7225 0.0496561421089895
0.7275 0.0467140407640087
0.7325 0.0439022317005707
0.7375 0.0412233832109181
0.7425 0.0386816575181199
0.7475 0.0362818824228978
0.7525 0.0340287783556193
0.7575 0.0319262824821773
0.7625 0.0299769963659542
0.7675 0.0281817757497395
0.7725 0.0265394766551976
0.7775 0.0250468665765341
0.7825 0.0236986997865967
0.7875 0.0224879414245178
0.7925 0.0214061090700217
0.7975 0.020443687675489
0.8025 0.0195905681785203
0.8075 0.0188364634737073
0.8125 0.0181712662333032
0.8175 0.0175853277785879
0.8225 0.017069651846122
0.8275 0.0166160087137515
0.8325 0.0162169824359945
0.8375 0.0158659670335385
0.8425 0.0155571273960618
0.8475 0.0152853386447708
0.8525 0.0150461148459901
0.8575 0.0148355350181559
0.8625 0.0146501717487152
0.8675 0.0144870256063356
0.8725 0.0143434669210281
0.8775 0.014217185360933
0.8825 0.0141061469833791
0.8875 0.0140085579990384
0.8925 0.0139228342861477
0.8975 0.0138475756591185
0.9025 0.013781543973461
0.9075 0.0137236442874623
0.9125 0.0136729084617042
0.9175 0.0136284807325308
0.9225 0.013589604927869
0.9275 0.0135556130956329
0.9325 0.0135259153859493
0.9375 0.0134999910732926
0.9425 0.0134773806308383
0.9475 0.0134576787855128
0.9525 0.0134405284968079
0.9575 0.0134256158233248
0.9625 0.0134126656754567
0.9675 0.0134014385076998
0.9725 0.0133917280890878
0.9775 0.0133833606280774
0.9825 0.0133761958024525
0.9875 0.0133701309602265
0.9925 0.0133651118920604
0.9975 0.0133611597771419
};
\addlegendentry{POD}
\addplot [semithick, color0, mark=pentagon,mark size=2, mark repeat=25, mark options={solid}]
table {%
0.0025 0.500140075687656
0.0075 0.500138569720573
0.0125 0.500139517339998
0.0175 0.500139966508182
0.0225 0.500139617444732
0.0275 0.500138978574465
0.0325 0.500137561965302
0.0375 0.500137891372051
0.0425 0.500137151996694
0.0475 0.500135950500453
0.0525 0.500134149494448
0.0575 0.50013480301304
0.0625 0.50013227644004
0.0675 0.500128712171409
0.0725 0.500126556371367
0.0775 0.50012183523443
0.0825 0.500116478500726
0.0875 0.500108110409841
0.0925 0.500100555359842
0.0975 0.50008712681842
0.1025 0.500072462342279
0.1075 0.50005143288653
0.1125 0.500025072417997
0.1175 0.499993383240689
0.1225 0.499952071574294
0.1275 0.499899056932826
0.1325 0.499833937676651
0.1375 0.499752629886652
0.1425 0.499648863699201
0.1475 0.499520221261541
0.1525 0.499364131411987
0.1575 0.49916976500376
0.1625 0.498929997942765
0.1675 0.498641273135382
0.1725 0.498286362495602
0.1775 0.497863030705589
0.1825 0.497352476264018
0.1875 0.49674421471504
0.1925 0.496023800388918
0.1975 0.495171691561157
0.2025 0.494173547647654
0.2075 0.493012143549371
0.2125 0.491663550133115
0.2175 0.490110957657565
0.2225 0.488335367971955
0.2275 0.486307187111791
0.2325 0.484015047494466
0.2375 0.481433638958292
0.2425 0.478543856838149
0.2475 0.475328681456947
0.2525 0.471765975824711
0.2575 0.467845790718573
0.2625 0.463552405181422
0.2675 0.458874772963551
0.2725 0.453802787195974
0.2775 0.448331265739512
0.2825 0.44245660179751
0.2875 0.43618238893917
0.2925 0.429509686820836
0.2975 0.422443585401646
0.3025 0.414998569884544
0.3075 0.407046027992034
0.3125 0.399572458871184
0.3175 0.393359932709707
0.3225 0.386936210855806
0.3275 0.38032201458913
0.3325 0.373537555422142
0.3375 0.366514761907843
0.3425 0.359369412256221
0.3475 0.352139880791348
0.3525 0.344852350357085
0.3575 0.337540151763602
0.3625 0.33024040984702
0.3675 0.322988159443128
0.3725 0.315820011220866
0.3775 0.309067613911723
0.3825 0.305252638261207
0.3875 0.301470778755394
0.3925 0.297735900818455
0.3975 0.294060506881717
0.4025 0.290467276731725
0.4075 0.286968448348876
0.4125 0.283576115118281
0.4175 0.280307067053909
0.4225 0.277093211811741
0.4275 0.274091854542008
0.4325 0.271228359319071
0.4375 0.268491574167801
0.4425 0.265886212245488
0.4475 0.263417712303893
0.4525 0.261083554784099
0.4575 0.258889325258665
0.4625 0.256871658033247
0.4675 0.255077939994129
0.4725 0.253439685098059
0.4775 0.252178161562171
0.4825 0.251203537504062
0.4875 0.250355023077618
0.4925 0.24961086561513
0.4975 0.248939584226531
0.5025 0.248307025224565
0.5075 0.247898040589204
0.5125 0.247589087674205
0.5175 0.247259400954689
0.5225 0.246882641620604
0.5275 0.24642810032311
0.5325 0.246407945928123
0.5375 0.246296296453594
0.5425 0.245794117179791
0.5475 0.244874819376307
0.5525 0.244878556155256
0.5575 0.246578866181811
0.5625 0.247233794988695
0.5675 0.248064861180355
0.5725 0.248387108505696
0.5775 0.248160905145571
0.5825 0.247361991845675
0.5875 0.245891200814941
0.5925 0.243324734959248
0.5975 0.240225771534656
0.6025 0.236624413495685
0.6075 0.232563868301012
0.6125 0.228080220049207
0.6175 0.223217221554655
0.6225 0.218011381509309
0.6275 0.212496395581232
0.6325 0.206702069633215
0.6375 0.200660723529083
0.6425 0.194402602655939
0.6475 0.187959083085118
0.6525 0.181371909801806
0.6575 0.174674502412807
0.6625 0.167904659596798
0.6675 0.161094638911197
0.6725 0.154270031999454
0.6775 0.147451874354061
0.6825 0.140652405505816
0.6875 0.133873729208463
0.6925 0.127121667213285
0.6975 0.120396667334319
0.7025 0.113704769881973
0.7075 0.107430020291282
0.7125 0.10119743837158
0.7175 0.0950233857125965
0.7225 0.0889374020070176
0.7275 0.0829754753150418
0.7325 0.0771778313762941
0.7375 0.0715822533419114
0.7425 0.0662310896868544
0.7475 0.0611603633611809
0.7525 0.0564013456174933
0.7575 0.0519827389056712
0.7625 0.0479214424584085
0.7675 0.0442280879134589
0.7725 0.0409012286467247
0.7775 0.0379353952752752
0.7825 0.0353156625537475
0.7875 0.0330195030899836
0.7925 0.0310227625308385
0.7975 0.0292979137542972
0.8025 0.0278166800774559
0.8075 0.0265473744857147
0.8125 0.025467407182384
0.8175 0.0245495167563215
0.8225 0.0237700109360146
0.8275 0.0231088655246475
0.8325 0.022548137113622
0.8375 0.0220720999297803
0.8425 0.0216695406139774
0.8475 0.0213286373442422
0.8525 0.0210385584028792
0.8575 0.0207929935020692
0.8625 0.0205818029161479
0.8675 0.0204047573195505
0.8725 0.0202528976133086
0.8775 0.0201238800497989
0.8825 0.0200144229372472
0.8875 0.0199204343023097
0.8925 0.0198409714397374
0.8975 0.0197748695722275
0.9025 0.0194386983628718
0.9075 0.0189553115588306
0.9125 0.0185324562397207
0.9175 0.0181694297491223
0.9225 0.0178544961414556
0.9275 0.0175845703861607
0.9325 0.0173515058952726
0.9375 0.017151094195737
0.9425 0.0169777655094346
0.9475 0.0168299780099334
0.9525 0.016702848274759
0.9575 0.0165933512992963
0.9625 0.0164995646126291
0.9675 0.0164194184912843
0.9725 0.0163503012120064
0.9775 0.0162894750767339
0.9825 0.0162413863425908
0.9875 0.0161963301884288
0.9925 0.0161580692750441
0.9975 0.0161276522725351
};
\addlegendentry{FCNN}
\addplot [semithick, green!50!black, mark=triangle, mark size=2, mark repeat=25, mark options={solid,rotate=180}, only marks]
table {%
0.0025 0.540835561662537
0.0075 0.562372304317069
0.0125 0.483907564200636
0.0175 0.457767292803346
0.0225 0.51470995927126
0.0275 0.473847052973453
0.0325 0.534276955996909
0.0375 0.491288420946959
0.0425 0.515868752391436
0.0475 0.56143257500371
0.0525 0.541937746317524
0.0575 0.527438171911746
0.0625 0.522615615964018
0.0675 0.478113660124478
0.0725 0.507392183159991
0.0775 0.460038010482089
0.0825 0.516525655689162
0.0875 0.492386635267604
0.0925 0.510942147458494
0.0975 0.515048093515374
0.1025 0.535655252686905
0.1075 0.536655297071117
0.1125 0.515821756565027
0.1175 0.433278222824374
0.1225 0.518693991180524
0.1275 0.475766934042409
0.1325 0.515051635122914
0.1375 0.513815867218078
0.1425 0.519967161841684
0.1475 0.52469247877082
0.1525 0.547397648421336
0.1575 0.529331671352966
0.1625 0.515243950294948
0.1675 0.462487580868375
0.1725 0.509411308624497
0.1775 0.477005882498008
0.1825 0.516264874780113
0.1875 0.492734387982292
0.1925 0.513785284293402
0.1975 0.526055971116259
0.2025 0.533226946816323
0.2075 0.570380749403156
0.2125 0.479142514787516
0.2175 0.448835549523125
0.2225 0.510543627106503
0.2275 0.481627288631868
0.2325 0.512759906795225
0.2375 0.487899997809227
0.2425 0.492551204481274
0.2475 0.509204558311209
0.2525 0.54509873391418
0.2575 0.542814448724928
0.2625 0.479764487968043
0.2675 0.420368944280517
0.2725 0.44830244630346
0.2775 0.419919389856223
0.2825 0.440432376199189
0.2875 0.392791632581989
0.2925 0.403022540951034
0.2975 0.382701462129915
0.3025 0.490342659851083
0.3075 0.486446848967694
0.3125 0.41353748909309
0.3175 0.32644804712946
0.3225 0.354234134479632
0.3275 0.306794339859329
0.3325 0.325794463568196
0.3375 0.294310611548813
0.3425 0.26230900490813
0.3475 0.249750247466072
0.3525 0.357291087341986
0.3575 0.380217514469521
0.3625 0.325692539883376
0.3675 0.258809749262796
0.3725 0.256510450260947
0.3775 0.24184143086567
0.3825 0.247076593402403
0.3875 0.213721906507143
0.3925 0.198700793306444
0.3975 0.226909198489349
0.4025 0.299187878335091
0.4075 0.312096233158667
0.4125 0.268826258298261
0.4175 0.256061881978995
0.4225 0.22992451728565
0.4275 0.221991503094745
0.4325 0.238327763547569
0.4375 0.203658882353152
0.4425 0.205188933648054
0.4475 0.191292131398734
0.4525 0.322939890147141
0.4575 0.273206309573891
0.4625 0.269655823144721
0.4675 0.23849276054424
0.4725 0.20566757207015
0.4775 0.209635324217316
0.4825 0.284326837035186
0.4875 0.201170555046328
0.4925 0.185655572179927
0.4975 0.186017410152496
0.5025 0.193435740089067
0.5075 0.189062989096581
0.5125 0.241557534829389
0.5175 0.214371684271138
0.5225 0.193369319165069
0.5275 0.194362872479937
0.5325 0.25682754289823
0.5375 0.21395566514719
0.5425 0.223067941297677
0.5475 0.305159778331963
0.5525 0.11052463829649
0.5575 0.037726287825739
0.5625 0.252894970040442
0.5675 0.25913883265
0.5725 0.145558679784774
0.5775 0.214837624835507
0.5825 0.21023289467547
0.5875 0.18875221391739
0.5925 0.260163439808786
0.5975 0.302784751177033
0.6025 0.219541584882384
0.6075 0.238422847456316
0.6125 0.231145352234404
0.6175 0.188995929904077
0.6225 0.2336674417095
0.6275 0.196290592292979
0.6325 0.269374006267635
0.6375 0.26475489656134
0.6425 0.248346528479347
0.6475 0.232284338599596
0.6525 0.252709700879148
0.6575 0.246958050388625
0.6625 0.300235943304961
0.6675 0.236172064325156
0.6725 0.268474100759644
0.6775 0.227413404635762
0.6825 0.222232455156302
0.6875 0.252738198692091
0.6925 0.242997613412264
0.6975 0.222262635987941
0.7025 0.274442195235224
0.7075 0.329702824116659
0.7125 0.287322740001272
0.7175 0.170264211683017
0.7225 0.166934719004126
0.7275 0.101441447449708
0.7325 0.029856793267574
0.7375 0.0160832618619325
0.7425 -0.0850945646324289
0.7475 -0.0180888654924372
0.7525 0.00794276096162504
0.7575 -5.22214564878203e-05
0.7625 0.0642889414200021
0.7675 0.0128788877571234
0.7725 0.0301322255798566
0.7775 0.0301881673002315
0.7825 -0.00317969349098216
0.7875 0.0346729031649259
0.7925 0.0210136826537895
0.7975 0.0159049082259858
0.8025 0.0153003539073819
0.8075 0.00686260852596309
0.8125 0.0852030032816548
0.8175 0.00635664329012142
0.8225 0.0504135567220288
0.8275 0.0239097275382885
0.8325 -0.00327520356726033
0.8375 0.0488823880799524
0.8425 0.0295404297362966
0.8475 0.0259471092714362
0.8525 0.00577727668384665
0.8575 0.00043477400994047
0.8625 0.0744034287237534
0.8675 0.00892546091535605
0.8725 0.0386614184715363
0.8775 0.0155194806224155
0.8825 -0.00939818035450396
0.8875 0.0464476761688731
0.8925 0.0335772337921785
0.8975 0.0292779511470415
0.9025 -0.0113625705386145
0.9075 0.01245426184423
0.9125 0.0806139203925452
0.9175 0.0240748950345743
0.9225 0.0340319409914282
0.9275 0.0215743667007377
0.9325 -0.00346177408640124
0.9375 0.0314157299573165
0.9425 0.0306417382387716
0.9475 0.0174471126382198
0.9525 0.0139400802928716
0.9575 0.00236124517937917
0.9625 0.0748931739596047
0.9675 -0.00165461394376987
0.9725 0.035573514366597
0.9775 0.0401613386810571
0.9825 -0.00486117144905422
0.9875 0.0545409534249203
0.9925 0.0270827195047629
0.9975 0.034283685895735
};
\addlegendentry{CNN}
\end{groupplot}

\end{tikzpicture}

	\caption{Matching of macroscopic quantities \(\rho\), \(\rho u\) and \(E\) reproduced by POD, the FCNN the and CNN with macroscopic quantities computed from the FOM. Top row shows results for \(\hy\), bottom row for \(\rare\) at time \(t_i=0.12s\). CNN is displayed with marks only because of trembles in the signal.}
	\label{Fig:ErrMacro}
\end{figure}
Loss of information described above can unfold in severe mistakes in \(\rho\), \(\rho u\) and \(E\), the macroscopic quantities, as displayed in \cref{Fig:ErrMacro}. Examining the macroscopic quantities enables a detailed look on the reconstruction errors. Features of the macroscopic quantities are expressed in terms of rarefaction wave, contact discontinuity and height as well as position of the shockfront. For a detailed elaboration of these terms see \cref{Ch:BGK}. Following the structure in the preceding figures, macroscopic quantities of \(\hy\) are displayed in the top row and for \(\rare\) in the bottom row of \cref{Fig:ErrMacro}. First the reproduction of the macroscopic quantities \(\rho\) and \(\rho u\)  obtained by the FCNN match the FOM exact for both levels of rarefaction \(\hy\) and \(\rare\). Interestingly, despite the overall impressive performance of the FCNN regarding the small number of parameters it uses, the total energy shows small deviations around the tail of the rarefaction wave for \(\hy\) and somewhat severe errors at the transition form rarefaction wave to shock front for \(\rare\). Second the CNN produces trembles in \(\rho u\) and especially in \(E\) which is why it's shown with marks only. The macroscopic quantities reproduced by the CNN show unmissable, it's inabillity to differentiate between \(\hy\) and \(\rare\). Specifically, seem the macroscopic quantities for \(\rare\) appear to be a copy of the the ones for \(\hy\). Additionally, considering \(\hy\), one can observe, that the momentum \(\rho u\) holds small errors for the tail of the rarefaction wave as well as the contact discontinuity. The value for the tip of the shockwave exceeds, comparable with POD, the exact solution. Third POD performs better on \(\rare\), which is not suprising considering the the difference in number of used parameters, holding only small deviations of the contact discontinuity and the shockwave for the momentum \(\rho u\) and the total energy \(E\). The density \(\rho\) matches the FOM solution exact. Pronounced deviations from the FOM solution occur using POD on \(\hy\). The density \(\rho\) undercuts the original shockwave. The momentum \(\rho u\) heavily exceeds the tail of the rarefaction wave and in the same extent undercuts the contact discontinuity. Hence in the total energy \(E\) the same is obersvable for the tail of the rarefaction wave and the contact discontinuity.
\begin{figure}[H]
	% This file was created by tikzplotlib v0.9.8.
\begin{tikzpicture}

\begin{groupplot}[
group style={group size=3 by 2,
	horizontal sep=.8cm,
	vertical sep=1.1cm
},
tick align=outside,
tick pos=left,
x grid style={white!69.0196078431373!black},
xlabel={\(t\)},
xmin=-0.006, xmax=0.126,
xtick style={color=black},
y grid style={white!69.0196078431373!black},
ymin=-0.031347233897592, ymax=0.112481710620415,
ytick style={color=black},
x tick label style={/pgf/number format/fixed},
y tick label style={/pgf/number format/fixed},
width=.55\textwidth,
height=.6\textwidth
]
\nextgroupplot[
legend cell align={left},
legend style={fill opacity=0,
	at={(1,1)},
	anchor=north east,
	draw opacity=1,
	text opacity=1,
	draw=none,
	nodes={
		scale=0.7,
		transform shape
	}
},
ylabel={\(\bar{\dot{\rho}}\)},
ymin=-0.0614423751831055, ymax=0.158164024353027,
y label style={yshift=-1em},
y label style={xshift=-.7em}
]
\addplot [semithick, black, mark=x, mark size=2.5, mark repeat=5, mark options={solid}]
table {%
0 -1.06764214677924e-05
0.005 -1.05043313283204e-05
0.01 -1.01943976815733e-05
0.015 -9.92951054712421e-06
0.02 -9.68121178601677e-06
0.025 -9.44233560318253e-06
0.03 -9.20958690642237e-06
0.035 -8.98114045355669e-06
0.04 -8.7558689685352e-06
0.045 -8.53302895365005e-06
0.05 -8.31210804363991e-06
0.055 -8.09274173718677e-06
0.06 -7.87466429130745e-06
0.065 -7.6576779619586e-06
0.07 -7.44163272514697e-06
0.075 -7.22641276951208e-06
0.08 -7.01192718111088e-06
0.085 -6.79810317905094e-06
0.09 -6.58488170302007e-06
0.095 -6.37221425847656e-06
0.1 -6.16006092712951e-06
0.105 -5.94838910217277e-06
0.11 -5.73717198193435e-06
0.115 -5.52638756801116e-06
0.12 -5.42110106493965e-06
};
\addlegendentry{FOM}
\addplot [semithick, red, mark=o, mark size=2.5, mark repeat=5, mark options={solid}]
table {%
0 0.00320090036619547
0.005 -0.00437489515648082
0.01 -0.0113141286949912
0.015 -0.00983423356615987
0.02 -0.00832201478583983
0.025 -0.00715067115115886
0.03 -0.00626926456104826
0.035 -0.00559929078165311
0.04 -0.00507985756838281
0.045 -0.00466825950748984
0.05 -0.00433514926056944
0.055 -0.00406028361365429
0.06 -0.00382951651894814
0.065 -0.00363279606843747
0.07 -0.00346284994708412
0.075 -0.00331432081637928
0.08 -0.0031831922008152
0.085 -0.00306640243334755
0.09 -0.00296158137682312
0.095 -0.00286686807133663
0.1 -0.00278078218987332
0.105 -0.00270213161850563
0.11 -0.00262994470215716
0.115 -0.0025634198784843
0.12 -0.00253148207930565
};
\addlegendentry{POD}
\addplot [semithick, color0, dashed, mark=pentagon, mark size=2.5, mark repeat=5, mark options={solid}]
table {%
0 -0.00746633686219411
0.005 -0.00191205168188446
0.01 0.00320702240208703
0.015 0.00214250930242343
0.02 0.00111506648273263
0.025 0.000475115820727012
0.03 0.00010403093992295
0.035 -7.12205054895776e-05
0.04 -0.000147954261507266
0.045 -0.00017581447766446
0.05 -0.000168761500205505
0.055 -0.000137564761239162
0.06 -0.000105405954101911
0.065 -7.944014357264e-05
0.07 -4.56097915417786e-05
0.075 -1.80734487642553e-05
0.08 1.01643112486727e-05
0.085 3.72892004847358e-05
0.09 5.28989428474347e-05
0.095 7.05332969062056e-05
0.1 9.31240780275289e-05
0.105 0.00010478244558243
0.11 0.000116572332629516
0.115 0.000132521768925642
0.12 0.000141214530785305
};
\addlegendentry{FCNN}
\addplot  [semithick, green!50!black, mark=triangle, mark size=2.5, mark repeat=5, mark options={solid,rotate=180}, only marks]
table {%
0 0.0842819213867188
0.005 0.0232086181640625
0.01 0.0284805297851562
0.015 0.148181915283203
0.02 0.046661376953125
0.025 -0.0418891906738281
0.03 0.0140495300292969
0.035 0.000995635986328125
0.04 0.027679443359375
0.045 0.0475044250488281
0.05 0.0190849304199219
0.055 -0.0116539001464844
0.06 -0.0280494689941406
0.065 -0.00231170654296875
0.07 0.0535316467285156
0.075 0.0304374694824219
0.08 -0.0279808044433594
0.085 -0.0463638305664062
0.09 -0.0435104370117188
0.095 0.0211029052734375
0.1 0.0492706298828125
0.105 -0.00998687744140625
0.11 -0.0488853454589844
0.115 -0.0514602661132812
0.12 -0.0494155883789062
};
\addlegendentry{CNN}

\nextgroupplot[
legend cell align={left},
legend style={fill opacity=0,
	draw opacity=1,
	text opacity=1, 
	at={(1,0)}, 
	anchor=south east, 
	draw=none,
	nodes={
		scale=0.7,
		transform shape
	}
},
ylabel={\(\bar{\dot{\rho u}}\)},
ymin=0.721508781709842, ymax=1.06553556938162,
width=.55\textwidth,
height=.6\textwidth,
y label style={yshift=-1.6em},
y label style={xshift=-1.4em}
]
\addplot  [semithick, black, mark=x, mark size=2.5, mark repeat=5, mark options={solid}]
table {%
0 0.968749498492742
0.005 0.968748999531741
0.01 0.968747998630038
0.015 0.968746993741499
0.02 0.968745987523224
0.025 0.968744980913048
0.03 0.968743974314931
0.035 0.968742967983464
0.04 0.968741962093837
0.045 0.968740956764444
0.05 0.968739952072257
0.055 0.968738948064989
0.06 0.968737944770258
0.065 0.968736942202105
0.07 0.968735940365403
0.075 0.968734939258829
0.08 0.968733938877292
0.085 0.968732939213847
0.09 0.968731940261284
0.095 0.968730942013192
0.1 0.968729944464235
0.105 0.968728947609632
0.11 0.968727951444045
0.115 0.968726955960737
0.12 0.968726458388709
};
\addlegendentry{FOM}
\addplot [semithick, red, mark=o, mark size=2.5, mark repeat=5, mark options={solid}]
table {%
0 0.776747752282815
0.005 0.799229546504149
0.01 0.83735394829241
0.015 0.861900922251396
0.02 0.876455586960931
0.025 0.885983570693113
0.03 0.892681143287458
0.035 0.897645432745145
0.04 0.901479309509238
0.045 0.904538271974328
0.05 0.90704386353466
0.055 0.909140538750041
0.06 0.910926147319329
0.065 0.912469178230276
0.07 0.91381896011011
0.075 0.91501192813239
0.08 0.916075602263358
0.085 0.917031185955863
0.09 0.917895307382834
0.095 0.918681213322456
0.1 0.919399605471469
0.105 0.920059238319471
0.11 0.920667354864786
0.115 0.921230009678775
0.12 0.921500662838827
};
\addlegendentry{POD}
\addplot  [semithick, color0, dashed, mark=pentagon, mark size=2.5, mark repeat=5, mark options={solid}]
table {%
0 0.981022530264175
0.005 0.969845267497955
0.01 0.959354321437879
0.015 0.961170023556659
0.02 0.963262696338745
0.025 0.964992540696261
0.03 0.966216851008718
0.035 0.967038764346892
0.04 0.967648145098694
0.045 0.96805996934129
0.05 0.96834140075498
0.055 0.968603158959088
0.06 0.968791679955521
0.065 0.968910933009226
0.07 0.969025615081555
0.075 0.969102313505783
0.08 0.969166812752425
0.085 0.969199221553223
0.09 0.969228125357358
0.095 0.969266452281406
0.1 0.969295363019938
0.105 0.96931896078058
0.11 0.96933353091193
0.115 0.969335049760666
0.12 0.969333696878397
};
\addlegendentry{FCNN}
\addplot [semithick, green!50!black, mark=triangle, mark size=2.5, mark repeat=5, mark options={solid,rotate=180}, only marks]
table {%
0 0.97954326764293
0.005 0.951945287849175
0.01 0.924409057007322
0.015 0.95474182480782
0.02 0.881638584425079
0.025 0.907850569358606
0.03 1.04989798812381
0.035 1.02313119627251
0.04 0.96617057868159
0.045 0.96673523863701
0.05 1.00865441136285
0.055 1.02940238470562
0.06 0.984113614670854
0.065 0.915812590632678
0.07 0.949019600881284
0.075 0.962200688930006
0.08 0.923812266239562
0.085 0.895938572593428
0.09 0.830964156112351
0.095 0.933549032946724
0.1 0.968627714754298
0.105 0.869239411782875
0.11 0.833597986548245
0.115 0.768703975596079
0.12 0.73714636296765
};
\addlegendentry{CNN}

\nextgroupplot[
legend cell align={left},
legend style={fill opacity=0,
	draw opacity=1, 
	text opacity=1, 
	at={(1,0)}, 
	anchor=south east, 
	draw=none,
	nodes={
		scale=0.7,
		transform shape
	}
},
ylabel={\(\bar{\dot{E}}\)},
ymin=-2.11101520609519, ymax=2.42930439772177,
width=.55\textwidth,
height=.6\textwidth,
y label style={yshift=-2em},
y label style={xshift=-3em}
]
\addplot [semithick, black, mark=x, mark size=2.5, mark repeat=5, mark options={solid}]
table {%
0 4.87748394917276e-05
0.005 4.79943104636504e-05
0.01 4.65877911999257e-05
0.015 4.53842673948657e-05
0.02 4.42552087491777e-05
0.025 4.31684471564608e-05
0.03 4.21090988496076e-05
0.035 4.1068895487939e-05
0.04 4.00427243398838e-05
0.045 3.90272160544214e-05
0.05 3.80200565288646e-05
0.055 3.70196084311658e-05
0.06 3.60246853468027e-05
0.065 3.50344095743083e-05
0.07 3.40481191756226e-05
0.075 3.30653053168817e-05
0.08 3.20855694617705e-05
0.085 3.11085940438716e-05
0.09 3.01341225821261e-05
0.095 2.91619465144777e-05
0.1 2.81918961952954e-05
0.105 2.72238339640296e-05
0.11 2.62576476686149e-05
0.115 2.52932443203235e-05
0.12 2.48114777576802e-05
};
\addlegendentry{FOM}
\addplot  [semithick, red, mark=o, mark size=2.5, mark repeat=5, mark options={solid}]
table {%
0 0.0143189698871922
0.005 -0.00810455140208077
0.01 -0.0298314956071479
0.015 -0.0275728368836106
0.02 -0.0246849481216564
0.025 -0.022325029104703
0.03 -0.0204887749707865
0.035 -0.0190569897727855
0.04 -0.0179250309049905
0.045 -0.0170152912259454
0.05 -0.0162722242749638
0.055 -0.015656108470349
0.06 -0.0151382259954005
0.065 -0.0146975051119611
0.07 -0.0143182441718253
0.075 -0.0139885690766448
0.08 -0.0136993767897806
0.085 -0.0134436012520851
0.09 -0.013215695339742
0.095 -0.0130112593925737
0.1 -0.0128267703619791
0.105 -0.0126593809300424
0.11 -0.0125067682475297
0.115 -0.0123670189805338
0.12 -0.0123001435222179
};
\addlegendentry{POD}
\addplot [semithick, color0, dashed, mark=pentagon, mark size=2.5, mark repeat=5, mark options={solid}]
table {%
0 -0.0144530996476604
0.005 -0.013301816060352
0.01 -0.002997673423355
0.015 0.00688496029257024
0.02 0.00809063479628591
0.025 0.00799179979590647
0.03 0.00721117010957428
0.035 0.0069053952409277
0.04 0.00646577613554911
0.045 0.00591729384263573
0.05 0.00574836602891438
0.055 0.00569547312151641
0.06 0.00551190072786412
0.065 0.00528559237558568
0.07 0.00511527266292688
0.075 0.00491798123422171
0.08 0.00473666708664311
0.085 0.00470337758398642
0.09 0.00442858501268262
0.095 0.0041957603795808
0.1 0.00419847093976955
0.105 0.00404308396729647
0.11 0.0038131760915121
0.115 0.00370871306318676
0.12 0.00370363698505116
};
\addlegendentry{FCNN}
\addplot [semithick, green!50!black, mark=triangle, mark size=2.5, mark repeat=5, mark options={solid,rotate=180}, only marks]
table {%
0 2.22292623391191
0.005 1.77012520595716
0.01 1.42936709093432
0.015 1.82681980799178
0.02 -1.3737616654558
0.025 -1.90463704228533
0.03 0.617198420515578
0.035 0.257909501820961
0.04 0.769456099482134
0.045 -0.134924309656782
0.05 -0.541436050787752
0.055 -0.213320312091664
0.06 -0.815704385410292
0.065 -0.232429535139634
0.07 1.1558903965101
0.075 0.825456903239349
0.08 -0.689314545307141
0.085 -1.07111149054275
0.09 -0.773864974842621
0.095 0.995079248783416
0.1 1.47938049120881
0.105 -0.469208151587615
0.11 -1.08921418156228
0.115 -0.206365288559116
0.12 0.327869451377538
};
\addlegendentry{CNN}

\nextgroupplot[
legend cell align={left},
legend style={fill opacity=0,
	text opacity=1, 
	at={(1,1)}, 
	anchor=north east, 
	draw=none,
	nodes={
		scale=0.7,
		transform shape
	}
},
ylabel={\(\bar{\dot{\rho}}\)},
ymin=-0.0869522094726562, ymax=0.153236389160156,
width=.55\textwidth,
height=.6\textwidth,
y label style={yshift=-1em},
y label style={xshift=-.7em}
]
\addplot [semithick, black, mark=x, mark size=2.5, mark repeat=5, mark options={solid}]
table {%
0 -1.0566054697847e-08
0.005 -1.02966097870194e-08
0.01 -9.79558478775289e-09
0.015 -9.34416988229714e-09
0.02 -8.91122198254379e-09
0.025 -8.48964276656261e-09
0.03 -8.07637690058982e-09
0.035 -7.66986119060675e-09
0.04 -7.26916482562956e-09
0.045 -6.87370516061492e-09
0.05 -6.48320508389588e-09
0.055 -6.1007909835098e-09
0.06 -5.7680509257807e-09
0.065 -5.8217608511768e-09
0.07 -8.19480305835896e-09
0.075 -2.12261852539086e-08
0.08 -7.32128810909671e-08
0.085 -2.42511106307575e-07
0.09 -7.11676761966373e-07
0.095 -1.84790528123813e-06
0.1 -4.3030998000404e-06
0.105 -9.11806750991673e-06
0.11 -1.78100265486592e-05
0.115 -3.24263887492293e-05
0.12 -4.15057958917942e-05
};
\addlegendentry{FOM}
\addplot [semithick, red, mark=o, mark size=2.5, mark repeat=5, mark options={solid}]
table {%
0 -0.00850277287727863
0.005 -0.00772076083175932
0.01 -0.00627892573997002
0.015 -0.00505141716242008
0.02 -0.00399637070279368
0.025 -0.00309606729447864
0.03 -0.00233937069619117
0.035 -0.00171506938696098
0.04 -0.00121061889818463
0.045 -0.000812734404206594
0.05 -0.000508152395383377
0.055 -0.000284182881259198
0.06 -0.000129037084739991
0.065 -3.19895342002496e-05
0.07 1.65679465879975e-05
0.075 2.51385344043342e-05
0.08 1.17137958000058e-06
0.085 -4.88665757245599e-05
0.09 -0.000119410074333359
0.095 -0.000205715696424136
0.1 -0.000303790813717342
0.105 -0.000410338307908376
0.11 -0.000522721868449594
0.115 -0.000638953255311492
0.12 -0.000697822820029614
};
\addlegendentry{POD}
\addplot[semithick, color0, dashed, mark=pentagon, mark size=2.5, mark repeat=5, mark options={solid}]
table {%
0 0.000326463574765512
0.005 -0.00131532022280823
0.01 -0.00240714307327039
0.015 -0.00113891668360822
0.02 -0.000588241155391245
0.025 -0.000590722846460778
0.03 -0.000656246120463777
0.035 -0.000607481205086913
0.04 -6.6451880215368e-05
0.045 4.51196531940923e-05
0.05 -8.27034739785404e-05
0.055 -0.000183065868625931
0.06 -0.000287167389060983
0.065 -0.000487041868908022
0.07 -0.000788421822448981
0.075 -0.000905088287311173
0.08 -0.00105589884350366
0.085 -0.00115068076524949
0.09 -0.0012397377247737
0.095 -0.00139431988364436
0.1 -0.00165710132255015
0.105 -0.0019471158124631
0.11 -0.00238798052932054
0.115 -0.00286884322223102
0.12 -0.0029792309312171
};
\addlegendentry{FCNN}
\addplot [semithick, green!50!black, mark=triangle, mark size=2.5, mark repeat=5, mark options={solid,rotate=180}, only marks]
table {%
0 0.0805206298828125
0.005 0.0176162719726562
0.01 0.0207061767578125
0.015 0.142318725585938
0.02 0.0116500854492188
0.025 -0.0760345458984375
0.03 0.00912857055664062
0.035 -0.0053253173828125
0.04 0.0281791687011719
0.045 0.009979248046875
0.05 -0.0237503051757812
0.055 -0.0180511474609375
0.06 -0.0328407287597656
0.065 0.0026397705078125
0.07 0.0280723571777344
0.075 -0.00083160400390625
0.08 -0.0287628173828125
0.085 -0.0436363220214844
0.09 -0.0279083251953125
0.095 0.0157318115234375
0.1 0.0327720642089844
0.105 -0.00761795043945312
0.11 -0.0422439575195312
0.115 -0.0342559814453125
0.12 -0.0254592895507812
};
\addlegendentry{CNN}

\nextgroupplot[
legend cell align={left},
legend style={fill opacity=0,
	text opacity=1, 
	at={(1,0)}, 
	anchor=south east, 
	draw=none,
	nodes={
		scale=0.7,
		transform shape
	}
},
ylabel={\(\bar{\dot{\rho u}}\)},
ytick={0.8,0.9,1},
ymin=0.701558316660603, ymax=1.04570506859874,
width=.55\textwidth,
height=.6\textwidth,
y label style={yshift=-1.6em},
y label style={xshift=-.9em}
]
\addplot [semithick, black, mark=x, mark size=2.5, mark repeat=5, mark options={solid}]
table {%
0 0.968749977995855
0.005 0.968749977493689
0.01 0.968749976488528
0.015 0.968749975482364
0.02 0.968749974475969
0.025 0.968749973469496
0.03 0.968749972463012
0.035 0.968749971456551
0.04 0.968749970450135
0.045 0.968749969443747
0.05 0.968749968436312
0.055 0.96874996740731
0.06 0.968749966119898
0.065 0.968749962757677
0.07 0.968749947403054
0.075 0.968749879330216
0.08 0.968749627063247
0.085 0.968748844536202
0.09 0.968746766180359
0.095 0.968741931728314
0.1 0.968731885809511
0.105 0.96871291830678
0.11 0.968679910009525
0.115 0.968626314303693
0.12 0.96859346363712
};
\addlegendentry{FOM}
\addplot  [semithick, red, mark=o, mark size=2.5, mark repeat=5, mark options={solid}]
table {%
0 0.874450523447081
0.005 0.882012675145708
0.01 0.896373906746789
0.015 0.908519731859784
0.02 0.918013001095483
0.025 0.92537536910019
0.03 0.931091478463895
0.035 0.935538828032017
0.04 0.939001516793107
0.045 0.941693218607178
0.05 0.943776372159542
0.055 0.945376223453617
0.06 0.946590699787999
0.065 0.947497302172493
0.07 0.948157958205335
0.075 0.948622499149348
0.08 0.948931205690461
0.085 0.949116712127438
0.09 0.949205454769583
0.095 0.949218783052661
0.1 0.949173810629732
0.105 0.949084060185012
0.11 0.948959943099316
0.115 0.948809107113092
0.12 0.948727801667509
};
\addlegendentry{POD}
\addplot [semithick, color0, dashed, mark=pentagon, mark size=2.5, mark repeat=5, mark options={solid}]
table {%
0 0.970399317976372
0.005 0.963503522566775
0.01 0.957508562394808
0.015 0.961566075034872
0.02 0.966404637762006
0.025 0.968473543644464
0.03 0.969027915221748
0.035 0.970405037818537
0.04 0.971588561357327
0.045 0.971385233690603
0.05 0.97084447858429
0.055 0.970142786797226
0.06 0.970058801045607
0.065 0.969591850053609
0.07 0.968916380492354
0.075 0.968742961697338
0.08 0.968382479535162
0.085 0.968240874778057
0.09 0.967734768725052
0.095 0.967435535078145
0.1 0.967173210304168
0.105 0.966540158655194
0.11 0.965694011500871
0.115 0.964573306518972
0.12 0.96413484084443
};
\addlegendentry{FCNN}
\addplot  [semithick, green!50!black, mark=triangle, mark size=2.5, mark repeat=5, mark options={solid,rotate=180}, only marks]
table {%
0 0.961906278822966
0.005 0.933425723879134
0.01 0.905878353170239
0.015 0.940320663551927
0.02 0.857029422900776
0.025 0.879651316447493
0.03 1.03006203441973
0.035 1.00402846909231
0.04 0.953349925570686
0.045 0.935897540926501
0.05 0.970320240998085
0.055 1.00498821285898
0.06 0.961259257132621
0.065 0.898568595199741
0.07 0.907571697214435
0.075 0.911803760208734
0.08 0.892125015958701
0.085 0.867297804450649
0.09 0.809100371138463
0.095 0.88698271808509
0.1 0.912923489640189
0.105 0.833686946039219
0.11 0.802155935069418
0.115 0.744677807304853
0.12 0.717201350839609
};
\addlegendentry{CNN}

\nextgroupplot[
legend cell align={left},
legend style={fill opacity=0, 
	text opacity=1, 
	at={(1,0)},
	 anchor=south east, 
	 draw=none,
	 nodes={
	 	scale=0.7,
	 	transform shape
	 }
},
ylabel={\(\bar{\dot{E}}\)},
ymin=-2.1075135100065, ymax=2.44711060492608,
width=.55\textwidth,
height=.6\textwidth,
y label style={yshift=-2em},
y label style={xshift=-3em}
]
\addplot [semithick, black, mark=x, mark size=2.5, mark repeat=5, mark options={solid}]
table {%
0 4.82813575786167e-08
0.005 4.7063597463648e-08
0.01 4.47984191964679e-08
0.015 4.27564792460089e-08
0.02 4.07976692429202e-08
0.025 3.88903416137509e-08
0.03 3.70209036759661e-08
0.035 3.51821611843661e-08
0.04 3.33698650933911e-08
0.045 3.15811483631023e-08
0.05 2.98109235075117e-08
0.055 2.79997891539097e-08
0.06 2.55037626573085e-08
0.065 1.76024457232415e-08
0.07 -2.03018117872489e-08
0.075 -1.84349591592081e-07
0.08 -7.69065788830403e-07
0.085 -2.50761453557402e-06
0.09 -6.92790824885492e-06
0.095 -1.676259263661e-05
0.1 -3.6295084708371e-05
0.105 -7.1516663975757e-05
0.11 -0.000130010286365945
0.115 -0.000220553638243359
0.12 -0.000274993156637038
};
\addlegendentry{FOM}
\addplot [semithick, red, mark=o, mark size=2.5, mark repeat=5, mark options={solid}]
table {%
0 -0.0369185874576559
0.005 -0.0334847636942506
0.01 -0.0271861464861836
0.015 -0.0218886278141817
0.02 -0.0173890616645664
0.025 -0.0135858142109271
0.03 -0.0104139688794262
0.035 -0.00781383933774293
0.04 -0.00572432983522475
0.045 -0.00408429540355471
0.05 -0.00283483794892092
0.055 -0.00192094953581545
0.06 -0.00129239866497599
0.065 -0.000904061088800034
0.07 -0.000715899060683256
0.075 -0.000692736728687748
0.08 -0.000803927287726935
0.085 -0.00102297077507174
0.09 -0.00132711814463704
0.095 -0.00169698339989566
0.1 -0.00211617764014349
0.105 -0.00257097444239918
0.11 -0.00305001321775222
0.115 -0.00354404418175491
0.12 -0.00379377610802578
};
\addlegendentry{POD}
\addplot  [semithick, color0, dashed, mark=pentagon, mark size=2.5, mark repeat=5, mark options={solid}]
table {%
0 -0.0220373594494561
0.005 -0.0187375871121702
0.01 -0.0137267475558431
0.015 -0.00708892810284567
0.02 -0.000805795336006554
0.025 0.00385607701109691
0.03 0.00688734023708193
0.035 0.00506515072139635
0.04 -0.0011157740152008
0.045 -0.00625783112414524
0.05 -0.00777156337428764
0.055 -0.0101094557751722
0.06 -0.0125639515699199
0.065 -0.0137910901982785
0.07 -0.0135298759123259
0.075 -0.0147397060042884
0.08 -0.0167050544650884
0.085 -0.0165941921786654
0.09 -0.0155688464916217
0.095 -0.0160095281063555
0.1 -0.0163660479894041
0.105 -0.0146851150221678
0.11 -0.0137112109201389
0.115 -0.0133200208008226
0.12 -0.0128320391498704
};
\addlegendentry{FCNN}
\addplot [semithick, green!50!black, mark=triangle, mark size=2.5, mark repeat=5, mark options={solid,rotate=180}, only marks]
table {%
0 2.2400822360655
0.005 1.78518401371828
0.01 1.44243569371781
0.015 1.84259146740263
0.02 -1.36916045778383
0.025 -1.90048514114593
0.03 0.63336327700209
0.035 0.272824821472579
0.04 0.786352591134989
0.045 -0.131427471443889
0.05 -0.540336303919108
0.055 -0.200548516735186
0.06 -0.803961333714966
0.065 -0.218569256769491
0.07 1.1578414467834
0.075 0.82358957236357
0.08 -0.681481865767363
0.085 -1.06250963505425
0.09 -0.761910306073361
0.095 0.997457794438301
0.1 1.47727023886346
0.105 -0.464125495002811
0.11 -1.08283991448095
0.115 -0.195796045692205
0.12 0.341143847513344
};
\addlegendentry{CNN}
\end{groupplot}

\end{tikzpicture}

	\caption{Comparison of the conservative properties of reconstructions obatined from POD, the FCNN and the CNN against the conservative properties of the FOM solution using the temporal mean.}
	\label{Fig:Conservation}
\end{figure}
The physical consistency of \(\tilde{f}\), in terms of conservation of mass momentum and energy is a critical criteria for it's validity. Hence conservation properties are analyzed in the following. In that respect, the temporal mean over the time derivative, which can be calculated exemplary for \(\rho\) with
\begin{equation}
	\frac{\mathrm{d}}{\mathrm{d}t}\int \rho(x,t)\, \mathrm{d}x\Delta t  =\overline{\dot{\rho}}\mathrm{,}
\end{equation}
of the macroscopic quantities is employed. \Cref{Fig:Conservation} shows the conservation of mass, momentum and total energy as a temporal mean for \(\hy\) in the top row and for \(\rare\) in the bottom row.\\
Conservation of mass is met using the FCNN, except for small deviations at the outset for both cases \(\hy\) and \(\rare\). Similarly, does POD meet conservation of mass for \(\rare\). The erroneous \(\hy\) case losses mass at the onset and gains mass towards the end with POD. Conservation of momentum meets the FOM solution, except for minor gains and losses, after \(t\approx 0.03s\) using the FCNN for both cases \(\hy\) and \(\rare\). POD gains momentum of 0.13 for \(\hy\) and 0.07 for \(\rare\). The conservation of total energy is met for \(\hy\) and \(\rare\) using POD and the FCNN. Finally the reconstructions of the CNN do not conserve mass, momentum nor total energy. All conservative properties behave comparable to a sawtooth wave. A gain and loss of either of the quantities can be observed.\\
In conclusion the error over time for the CNN is disordered, showing gain and subsequent loss of information from one timestep to another. The CNN performs slighly better with \(\hy\) than with \(\rare\). What is hidden when looking at reconstructions becomes visible when verifying over the macroscopic quantities. Reconstructions obtained from the CNN show oscillations in the momentum \(\rho u\) and the total energy \(E\). On top of that the CNN does not meet conservation in any of the conservative properties. All of this together makes the CNN with this setup, especially the access to only 40 samples, unsuited for building a ROM. Next POD shows a noticeable increase in loss of information over time for both cases \(\hy\) and \(\rare\). Reconstructions of the last timestep as well as the macroscopic quantities at that time reveal that the POD is unsuited for building a ROM with the \(\hy\)case. However, with \(\rare\) POD shows only slight deviations from the FOM solution. Taking conservative properties of the reconstructions obtained from POD into consideration only underlines aforementioned findings. Ultimately POD could be taken for building a ROM with \(\rare\). Finally the the FCNN is the best performing model out of the three for both cases \(\hy\) and \(\rare\), while the performance for \(\hy\) is slightly better than that for \(\rare\). The error over time reveals a constant low loss. Only at the first time steps a noticeable loss of information is observed. Reconstructions of the last time step and the macroscopic quantities at that time are close to exact to the FOM solution for both cases \(\hy\) and \(\rare\). The conservation of the macroscopic quantites only emphasize the proximity to the FOM solution. In total the FCNN is suited for building a ROM with both cases \(\hy\) and \(\rare\) and will be taken further into the online phase.\\
We now reached the online phase where we want to be independent from the FOM solution. A first ROM relying on pure interpolation in the intrinsic variables is performed for \(\hy\). The intrinsic variables of the FCNN for \(\hy\) are shown in ... . 

With POD one usually exploits the intrinsic variables within a Galerkin framework as in \cite{Bernard} to produce new states. The same can be done with the intrinsic variables obtained from autoencoders as in \cite{Carlberg}. Both won't be discussed in this contribution. Rather new states are obtained by interpolating \(\idhy\) and \(\idrare\) in time \(t\). This approach tests a different kind of generalization about the FOM solution. Therefore this kind of generalization ability of the proposed autoencoder architectures will be analyzed.

\begin{figure}[H]
	% This file was created by tikzplotlib v0.9.6.
\begin{tikzpicture}

\begin{groupplot}[group style={group size=1 by 3,vertical sep=1cm}]
\nextgroupplot[
colorbar,
colorbar style={ylabel={}},
colormap/blackwhite,
point meta max=0.532396674156189,
point meta min=-0.00969430990517139,
tick align=outside,
tick pos=left,
x grid style={white!69.0196078431373!black},
xmin=0.0025, xmax=0.9975,
xtick style={color=black},
xlabel={x},
y grid style={white!69.0196078431373!black},
ymin=0, ymax=0.12,
ytick style={color=black},
ylabel={t},
ytick={0,0.06,0.12},
width=.9\textwidth,
height=.25\textwidth,
y tick label style={/pgf/number format/fixed}
]
\addplot graphics [includegraphics cmd=\pgfimage,xmin=0.0025, xmax=0.9975, ymin=0, ymax=0.12] {Figures/Results/Code2D_hy_FCNN-000.png};
\node [draw,fill=white] at (0.95,0.1) {\(\alpha_1\)};

\nextgroupplot[
colorbar,
colorbar style={ylabel={}},
colormap/blackwhite,
point meta max=-0.197054535150528,
point meta min=-0.552837491035461,
tick align=outside,
tick pos=left,
x grid style={white!69.0196078431373!black},
xmin=0.0025, xmax=0.9975,
xtick style={color=black},
xlabel={x},
y grid style={white!69.0196078431373!black},
ymin=0, ymax=0.12,
ytick={0,0.06,0.12},
ytick style={color=black},
ylabel={t},
width=.9\textwidth,
height=.25\textwidth,
y tick label style={/pgf/number format/fixed}
]
\addplot graphics [includegraphics cmd=\pgfimage,xmin=0.0025, xmax=0.9975, ymin=0, ymax=0.12] {Figures/Results/Code2D_hy_FCNN-001.png};
\node [draw,fill=white] at (0.95,0.1) {\(\alpha_2\)};

\nextgroupplot[
colorbar,
colorbar style={ylabel={}},
colormap/blackwhite,
point meta max=-0.136639997363091,
point meta min=-0.320842951536179,
tick align=outside,
tick pos=left,
x grid style={white!69.0196078431373!black},
xmin=0.0025, xmax=0.9975,
xlabel={x},
xtick style={color=black},
y grid style={white!69.0196078431373!black},
ymin=0, ymax=0.12,
ytick={0,0.06,0.12},
ytick style={color=black},
ylabel={t},
width=.9\textwidth,
height=.25\textwidth,
y tick label style={/pgf/number format/fixed}
]
\addplot graphics [includegraphics cmd=\pgfimage,xmin=0.0025, xmax=0.9975, ymin=0, ymax=0.12] {Figures/Results/Code2D_hy_FCNN-002.png};
\node [draw,fill=white] at (0.95,0.1) {\(\alpha_3\)};
\end{groupplot}

\end{tikzpicture}

	\caption{\(\alpha_1\), \(\alpha_2\) and \(\alpha_3\), the reduced basis \(\idhy\) obtained from the FCNN.}
\end{figure}
\begin{figure}[H]
	% This file was created by tikzplotlib v0.9.8.
\begin{tikzpicture}

\begin{groupplot}[group style={group size=3 by 1},
legend cell align={left},
legend style={fill opacity=0.1, draw opacity=1, text opacity=1, at={(1,1)}, anchor=north east, draw=none},
tick align=outside,
tick pos=left,
x grid style={white!69.0196078431373!black},
xlabel={\(x\)},
xmin=-0.04725, xmax=1.04725,
xtick style={color=black},
y grid style={white!69.0196078431373!black},
ytick style={color=black},
width=.35\textwidth,
height=.4\textwidth,
y label style={yshift=-2em}
]
\nextgroupplot[
ylabel={\(\rho\)},
ytick={0.2,0.4,0.8,1},
ymin=0.0812408233619941, ymax=1.04383396967043,
]
\addplot [semithick, color0, mark=pentagon, mark size=2, mark options={solid},mark repeat=5]
table {%
0.0025 1.00007958960132
0.0075 1.00007958960132
0.0125 1.00007958960132
0.0175 1.00007958960132
0.0225 1.00007958960132
0.0275 1.00007958960132
0.0325 1.00007958960132
0.0375 1.00007958960132
0.0425 1.00007958960132
0.0475 1.00007958960132
0.0525 1.00007958960132
0.0575 1.00007958960132
0.0625 1.00007958960132
0.0675 1.00007958960132
0.0725 1.00007958960132
0.0775 1.00007958960132
0.0825 1.00007958960132
0.0875 1.00007958960132
0.0925 1.00007958960132
0.0975 1.00007960560096
0.1025 1.00007967294275
0.1075 1.00007964882388
0.1125 1.00007973574732
0.1175 1.00007942697807
0.1225 1.00007952906535
0.1275 1.00007935247227
0.1325 1.00007910507478
0.1375 1.0000786045486
0.1425 1.00007773758281
0.1475 1.00007644805925
0.1525 1.00007424166057
0.1575 1.00007047887858
0.1625 1.00006458671113
0.1675 1.00005566690952
0.1725 1.00004139821195
0.1775 1.00001937338839
0.1825 0.999986432031501
0.1875 0.999937488403667
0.1925 0.999865611912762
0.1975 0.999762540533591
0.2025 0.999616785088856
0.2075 0.999413966793886
0.2125 0.999135611671512
0.2175 0.998761416889719
0.2225 0.998265151151335
0.2275 0.997618268225603
0.2325 0.996788949596336
0.2375 0.99574209013388
0.2425 0.994440960625299
0.2475 0.992848293362174
0.2525 0.990926713533989
0.2575 0.988643694819403
0.2625 0.98597447156029
0.2675 0.982894724536353
0.2725 0.979384886899208
0.2775 0.97543120528311
0.2825 0.971025352350884
0.2875 0.966164945468956
0.2925 0.960852690965192
0.2975 0.955096454813896
0.3025 0.948908228748352
0.3075 0.942303409679839
0.3125 0.935299842942792
0.3175 0.927917993992475
0.3225 0.920179602285666
0.3275 0.912107081055203
0.3325 0.903720666985319
0.3375 0.895043348699038
0.3425 0.886099242316696
0.3475 0.876913020961165
0.3525 0.867506658352671
0.3575 0.8578984900834
0.3625 0.848106184178299
0.3675 0.838090457212586
0.3725 0.827832277131951
0.3775 0.817345481359971
0.3825 0.806643968094992
0.3875 0.795742298087786
0.3925 0.784654691798187
0.3975 0.773418127866418
0.4025 0.76206761590187
0.4075 0.750619441273673
0.4125 0.739091075712631
0.4175 0.727500537444857
0.4225 0.715867658268065
0.4275 0.704210523511352
0.4325 0.692526782779235
0.4375 0.680846175469963
0.4425 0.66921524980196
0.4475 0.657692996940784
0.4525 0.646356617796227
0.4575 0.635305164135372
0.4625 0.624665874322266
0.4675 0.61460028374918
0.4725 0.605298140195823
0.4775 0.596974775589291
0.4825 0.589842921078701
0.4875 0.584063832062555
0.4925 0.579672229440244
0.4975 0.576530995545067
0.5025 0.574369383563039
0.5075 0.572842073294033
0.5125 0.571619951938634
0.5175 0.570435925138936
0.5225 0.56906307830009
0.5275 0.567255114691168
0.5325 0.564675626033231
0.5375 0.560842670806613
0.5425 0.555107028704735
0.5475 0.546692667435545
0.5525 0.534817340636247
0.5575 0.5188380042211
0.5625 0.498502658421844
0.5675 0.473995488276239
0.5725 0.446001658485437
0.5775 0.415641278464789
0.5825 0.38431216452763
0.5875 0.353492774178785
0.5925 0.324502245123317
0.5975 0.298354483407366
0.6025 0.275683994082713
0.6075 0.256797161618465
0.6125 0.241638212449143
0.6175 0.229901758262591
0.6225 0.221134202819148
0.6275 0.214821467045285
0.6325 0.210452181777347
0.6375 0.207558450317622
0.6425 0.205741919376864
0.6475 0.204680016666904
0.6525 0.204124898744949
0.6575 0.203894491451661
0.6625 0.203860315972972
0.6675 0.203935182367526
0.6725 0.204061199146803
0.6775 0.204198292814041
0.6825 0.204315014755273
0.6875 0.204378813575093
0.6925 0.204344825913122
0.6975 0.204142378236774
0.7025 0.203654290704554
0.7075 0.202686372769895
0.7125 0.200921816195532
0.7175 0.197857485664529
0.7225 0.19278071511726
0.7275 0.184850776625979
0.7325 0.173437997270236
0.7375 0.159227317397964
0.7425 0.144888745564241
0.7475 0.134016824746112
0.7525 0.128206411531143
0.7575 0.125959403239509
0.7625 0.125261302826069
0.7675 0.125066164599088
0.7725 0.125013926119164
0.7775 0.125000055502295
0.7825 0.124996353614482
0.7875 0.124995377277988
0.7925 0.124995147432354
0.7975 0.124995057285105
0.8025 0.124995140387734
0.8075 0.124995080209968
0.8125 0.124995080209968
0.8175 0.124995080209968
0.8225 0.124995080209968
0.8275 0.124995080209968
0.8325 0.124995080209968
0.8375 0.124995080209968
0.8425 0.124995080209968
0.8475 0.124995080209968
0.8525 0.124995080209968
0.8575 0.124995080209968
0.8625 0.124995080209968
0.8675 0.124995080209968
0.8725 0.124995080209968
0.8775 0.124995080209968
0.8825 0.124995080209968
0.8875 0.124995080209968
0.8925 0.124995080209968
0.8975 0.124995080209968
0.9025 0.124995080209968
0.9075 0.124995080209968
0.9125 0.124995080209968
0.9175 0.124995080209968
0.9225 0.124995080209968
0.9275 0.124995080209968
0.9325 0.124995080209968
0.9375 0.124995080209968
0.9425 0.124995080209968
0.9475 0.124995080209968
0.9525 0.124995080209968
0.9575 0.124995080209968
0.9625 0.124995080209968
0.9675 0.124995080209968
0.9725 0.124995080209968
0.9775 0.124995080209968
0.9825 0.124995080209968
0.9875 0.124995080209968
0.9925 0.124995080209968
0.9975 0.124995080209968
};
\addlegendentry{prediction}
\addplot [semithick, black, mark=+, mark size=2, mark options={solid},
dashed,%only marks,
mark repeat=5]
table {%
0.0025 0.999999994499998
0.0075 0.999999994499989
0.0125 0.999999994499973
0.0175 0.999999994499938
0.0225 0.999999994499854
0.0275 0.999999994499657
0.0325 0.999999994499225
0.0375 0.999999994498307
0.0425 0.999999994496346
0.0475 0.999999994492153
0.0525 0.999999994483302
0.0575 0.999999994464812
0.0625 0.999999994426618
0.0675 0.999999994348593
0.0725 0.999999994191032
0.0775 0.999999993876523
0.0825 0.999999993255951
0.0875 0.999999992045765
0.0925 0.999999989713722
0.0975 0.999999985273699
0.1025 0.999999976922728
0.1075 0.999999961408873
0.1125 0.999999932947061
0.1175 0.999999881389466
0.1225 0.999999789188686
0.1275 0.999999626442624
0.1325 0.99999934295137
0.1375 0.999998855715861
0.1425 0.99999802963589
0.1475 0.99999664829786
0.1525 0.999994370682796
0.1575 0.999990668414943
0.1625 0.999984736919399
0.1675 0.999975372768066
0.1725 0.999960808885077
0.1775 0.999938499600449
0.1825 0.999904849332842
0.1875 0.999854882524753
0.1925 0.99978185882593
0.1975 0.999676846595887
0.2025 0.999528279209094
0.2075 0.999321531266986
0.2125 0.999038563648431
0.2175 0.998657694619451
0.2225 0.99815355591569
0.2275 0.997497285204109
0.2325 0.996656988391704
0.2375 0.995598477836268
0.2425 0.994286259099913
0.2475 0.992684705039226
0.2525 0.990759328214853
0.2575 0.988478046531036
0.2625 0.985812336037212
0.2675 0.982738179020057
0.2725 0.979236741747293
0.2775 0.975294749126552
0.2825 0.970904557028253
0.2875 0.96606395178504
0.2925 0.960775726924111
0.2975 0.955047098184107
0.3025 0.948889019907117
0.3075 0.942315460872332
0.3125 0.9353426880196
0.3175 0.927988594730014
0.3225 0.920272098375855
0.3275 0.912212621103049
0.3325 0.903829659028516
0.3375 0.895142438489912
0.3425 0.886169653587621
0.3475 0.876929276719067
0.3525 0.867438432750232
0.3575 0.857713327521522
0.3625 0.847769222220737
0.3675 0.837620446519981
0.3725 0.82728044509819
0.3775 0.816761854164244
0.3825 0.806076606850691
0.3875 0.795236068929107
0.3925 0.784251209358396
0.3975 0.773132813964529
0.4025 0.761891755435828
0.4075 0.750539339336509
0.4125 0.739087754739222
0.4175 0.727550670369378
0.4225 0.715944034148932
0.4275 0.704287157239119
0.4325 0.692604194420389
0.4375 0.680926170788684
0.4425 0.669293745622052
0.4475 0.657760931521731
0.4525 0.646399957544768
0.4575 0.635307281677215
0.4625 0.624610228710984
0.4675 0.614472544087769
0.4725 0.605094996557836
0.4775 0.596704251904308
0.4825 0.589521611837022
0.4875 0.583707948506557
0.4925 0.57929759656908
0.4975 0.576156313265513
0.5025 0.574000965899689
0.5075 0.57248017604117
0.5125 0.571264200514682
0.5175 0.570086379629996
0.5225 0.568721456981151
0.5275 0.566925162056573
0.5325 0.564365789892603
0.5375 0.560570505963129
0.5425 0.554906052745071
0.5475 0.546613496573865
0.5525 0.534908968811207
0.5575 0.519142277507728
0.5625 0.49897861904467
0.5675 0.474548678169257
0.5725 0.446512146926428
0.5775 0.41600335371414
0.5825 0.384466860249091
0.5875 0.353428287708994
0.5925 0.324264601694908
0.5975 0.298031573575595
0.6025 0.275379773169268
0.6075 0.256558296419699
0.6125 0.241481616612008
0.6175 0.229826468110973
0.6225 0.221130573161533
0.6275 0.214876144173828
0.6325 0.210551636151645
0.6375 0.207691966758486
0.6425 0.205900478048754
0.6475 0.20485677362768
0.6525 0.204314507250902
0.6575 0.204092812694663
0.6625 0.204064483070227
0.6675 0.204143251128216
0.6725 0.204271659470887
0.6775 0.204410148437548
0.6825 0.204527208694704
0.6875 0.204589750704457
0.6925 0.204552164518092
0.6975 0.204341776813738
0.7025 0.203837533803555
0.7075 0.202838125078778
0.7125 0.201017075159804
0.7175 0.197870633082619
0.7225 0.192691646081915
0.7275 0.18466735979763
0.7325 0.173289805104828
0.7375 0.159203830364094
0.7425 0.144954928763386
0.7475 0.134090935533034
0.7525 0.128237750573878
0.7575 0.125970038823052
0.7625 0.125265821320929
0.7675 0.125069121636156
0.7725 0.125016331920844
0.7775 0.125002354274959
0.7825 0.124998671121747
0.7875 0.124997702861611
0.7925 0.124997448751868
0.7975 0.124997382171878
0.8025 0.124997364756704
0.8075 0.124997360209753
0.8125 0.124997359024918
0.8175 0.124997358716831
0.8225 0.124997358636905
0.8275 0.124997358616222
0.8325 0.124997358610884
0.8375 0.12499735860951
0.8425 0.124997358609158
0.8475 0.124997358609067
0.8525 0.124997358609045
0.8575 0.124997358609039
0.8625 0.124997358609038
0.8675 0.124997358609038
0.8725 0.124997358609037
0.8775 0.124997358609037
0.8825 0.124997358609037
0.8875 0.124997358609037
0.8925 0.124997358609037
0.8975 0.124997358609037
0.9025 0.124997358609037
0.9075 0.124997358609037
0.9125 0.124997358609037
0.9175 0.124997358609037
0.9225 0.124997358609037
0.9275 0.124997358609037
0.9325 0.124997358609037
0.9375 0.124997358609037
0.9425 0.124997358609037
0.9475 0.124997358609037
0.9525 0.124997358609037
0.9575 0.124997358609037
0.9625 0.124997358609037
0.9675 0.124997358609037
0.9725 0.124997358609037
0.9775 0.124997358609037
0.9825 0.124997358609037
0.9875 0.124997358609037
0.9925 0.124997358609037
0.9975 0.124997358609037
};
\addlegendentry{truth}

\nextgroupplot[
ylabel={\(\rho u\)},
ytick = {0,0.1,0.3,0.4},
ymin=-0.0210742313599136, ymax=0.437621533883013,
]
\addplot [semithick, color0, mark=pentagon, mark size=2, mark options={solid},mark repeat=5]
table {%
0.0025 -0.000224018744984643
0.0075 -0.000224018744984643
0.0125 -0.000224018744984643
0.0175 -0.000224018744984643
0.0225 -0.000224018744984643
0.0275 -0.000224018744984643
0.0325 -0.000224018744984643
0.0375 -0.000224018744984643
0.0425 -0.000224018744984643
0.0475 -0.000224018744984643
0.0525 -0.000224018744984643
0.0575 -0.000224018744984643
0.0625 -0.000224018744984643
0.0675 -0.000224018744984643
0.0725 -0.000224018744984643
0.0775 -0.000224018744984643
0.0825 -0.000224018744984643
0.0875 -0.000224018744984643
0.0925 -0.000224018744984643
0.0975 -0.000224114204017176
0.1025 -0.000224420971017424
0.1075 -0.000224423848871476
0.1125 -0.000224399111572556
0.1175 -0.000224021377914952
0.1225 -0.000223996242614871
0.1275 -0.000223399761430633
0.1325 -0.000222882727388758
0.1375 -0.000222388961103418
0.1425 -0.000220583597507155
0.1475 -0.000218405245650912
0.1525 -0.000214517724002303
0.1575 -0.000207951256962543
0.1625 -0.000198811069898842
0.1675 -0.000182532549448376
0.1725 -0.000157972483380186
0.1775 -0.000120629754814146
0.1825 -6.45468077751703e-05
0.1875 1.94128097132156e-05
0.1925 0.000142050020835177
0.1975 0.000317399813851215
0.2025 0.000566233682376917
0.2075 0.000911136198722704
0.2125 0.00138317272628902
0.2175 0.00201805094390059
0.2225 0.00285827773621694
0.2275 0.00395250206330534
0.2325 0.00535272705479511
0.2375 0.00711846857027321
0.2425 0.00930975956459987
0.2475 0.0119877343535197
0.2525 0.0152145700462701
0.2575 0.0190384658616776
0.2625 0.023486006956628
0.2675 0.0285888621560336
0.2725 0.0343668356271487
0.2775 0.0408301942981457
0.2825 0.0479735901058079
0.2875 0.0557848471250038
0.2925 0.0642390732036719
0.2975 0.0733041575213734
0.3025 0.0829408320345379
0.3075 0.093101483810572
0.3125 0.103735478966086
0.3175 0.114789414264818
0.3225 0.126204462695237
0.3275 0.137922470816743
0.3325 0.149856288696278
0.3375 0.161922343022342
0.3425 0.17405417015264
0.3475 0.186189567965098
0.3525 0.198268231517027
0.3575 0.210235498357924
0.3625 0.222050954078593
0.3675 0.233822526473728
0.3725 0.245508581475077
0.3775 0.257068285670135
0.3825 0.268466084457423
0.3875 0.279669139654188
0.3925 0.290645325132257
0.3975 0.30128351815843
0.4025 0.311509890508708
0.4075 0.321307295897787
0.4125 0.330665317075539
0.4175 0.339573258160456
0.4225 0.348027166971018
0.4275 0.356030595024101
0.4325 0.363641410975208
0.4375 0.370859526743636
0.4425 0.3776673689889
0.4475 0.384045174629705
0.4525 0.389965828769702
0.4575 0.395394642704727
0.4625 0.400289485355001
0.4675 0.404598553234369
0.4725 0.408279565124794
0.4775 0.4112957652881
0.4825 0.413636596928851
0.4875 0.415293668469434
0.4925 0.416305254573296
0.4975 0.416742638645411
0.5025 0.416771726371971
0.5075 0.416527856893603
0.5125 0.416087128785646
0.5175 0.415465345538933
0.5225 0.414619796982088
0.5275 0.413435572513614
0.5325 0.411703304513104
0.5375 0.409091789631392
0.5425 0.405106248364163
0.5475 0.399101061056479
0.5525 0.39036861576975
0.5575 0.378290466986318
0.5625 0.363146681986418
0.5675 0.345323436214576
0.5725 0.32549162805932
0.5775 0.304513083434557
0.5825 0.283249913988369
0.5875 0.262475023127174
0.5925 0.242767803704885
0.5975 0.224524474631891
0.6025 0.208051999031094
0.6075 0.193581449267863
0.6125 0.181341328977864
0.6175 0.171418746395887
0.6225 0.163721182217289
0.6275 0.15801678880118
0.6325 0.153986948144191
0.6375 0.151285444173278
0.6425 0.149584317095146
0.6475 0.148598559150083
0.6525 0.148099745476024
0.6575 0.147915445005698
0.6625 0.14792264607018
0.6675 0.148034800207918
0.6725 0.148191477993727
0.6775 0.148346908903325
0.6825 0.148456643561773
0.6875 0.148466094832618
0.6925 0.148289537902271
0.6975 0.147783888314429
0.7025 0.146705594838524
0.7075 0.14462977813771
0.7125 0.140840847912772
0.7175 0.134184715481149
0.7225 0.123067795069719
0.7275 0.105610533503771
0.7325 0.0802160459178515
0.7375 0.049383167051561
0.7425 0.0224540554733486
0.7475 0.00768127561627308
0.7525 0.00231781407386183
0.7575 0.000701356647177434
0.7625 0.000246654138973484
0.7675 0.00012330573050479
0.7725 9.08364029228138e-05
0.7775 8.20810216900367e-05
0.7825 7.98726666741706e-05
0.7875 7.89232503624513e-05
0.7925 7.88321693419298e-05
0.7975 7.90195053972418e-05
0.8025 7.88600600342628e-05
0.8075 7.87138405539246e-05
0.8125 7.87138405539246e-05
0.8175 7.87138405539246e-05
0.8225 7.87138405539246e-05
0.8275 7.87138405539246e-05
0.8325 7.87138405539246e-05
0.8375 7.87138405539246e-05
0.8425 7.87138405539246e-05
0.8475 7.87138405539246e-05
0.8525 7.87138405539246e-05
0.8575 7.87138405539246e-05
0.8625 7.87138405539246e-05
0.8675 7.87138405539246e-05
0.8725 7.87138405539246e-05
0.8775 7.87138405539246e-05
0.8825 7.87138405539246e-05
0.8875 7.87138405539246e-05
0.8925 7.87138405539246e-05
0.8975 7.87138405539246e-05
0.9025 7.87138405539246e-05
0.9075 7.87138405539246e-05
0.9125 7.87138405539246e-05
0.9175 7.87138405539246e-05
0.9225 7.87138405539246e-05
0.9275 7.87138405539246e-05
0.9325 7.87138405539246e-05
0.9375 7.87138405539246e-05
0.9425 7.87138405539246e-05
0.9475 7.87138405539246e-05
0.9525 7.87138405539246e-05
0.9575 7.87138405539246e-05
0.9625 7.87138405539246e-05
0.9675 7.87138405539246e-05
0.9725 7.87138405539246e-05
0.9775 7.87138405539246e-05
0.9825 7.87138405539246e-05
0.9875 7.87138405539246e-05
0.9925 7.87138405539246e-05
0.9975 7.87138405539246e-05
};
\addlegendentry{pred.}
\addplot [semithick, black, mark=+, mark size=2, mark options={solid},dashed,
mark repeat=5]
table {%
0.0025 6.30909939231123e-15
0.0075 2.52743809394443e-14
0.0125 6.71665948905054e-14
0.0175 1.47797315006362e-13
0.0225 3.07924882400649e-13
0.0275 6.6172472758689e-13
0.0325 1.42564590009118e-12
0.0375 3.07415834705836e-12
0.0425 6.61513596957766e-12
0.0475 1.41586575371753e-11
0.0525 3.00685203765747e-11
0.0575 6.32804197847002e-11
0.0625 1.31840368981028e-10
0.0675 2.71816268178485e-10
0.0725 5.54376848420904e-10
0.0775 1.11825149293422e-09
0.0825 2.23051508115254e-09
0.0875 4.3988436607807e-09
0.0925 8.57587718345818e-09
0.0975 1.6525966137083e-08
0.1025 3.14735316685944e-08
0.1075 5.92319220162265e-08
0.1125 1.1013832102345e-07
0.1175 2.02317177002482e-07
0.1225 3.67093736874415e-07
0.1275 6.57821440259511e-07
0.1325 1.16402287545304e-06
0.1375 2.03362895936117e-06
0.1425 3.50728899710936e-06
0.1475 5.97025139178314e-06
0.1525 1.00291758876392e-05
0.1575 1.66233526126904e-05
0.1625 2.71819698904176e-05
0.1675 4.38409222097891e-05
0.1725 6.97336015392367e-05
0.1775 0.000109369362893115
0.1825 0.000169109904238073
0.1875 0.000257746595168556
0.1925 0.000387169949609523
0.1975 0.000573105626212421
0.2025 0.000835870200910254
0.2075 0.00120107654392011
0.2125 0.0017001966437239
0.2175 0.00237087431188884
0.2225 0.00325687722538598
0.2275 0.00440759236330102
0.2325 0.00587700372294412
0.2375 0.00772214475261246
0.2425 0.0100010837355784
0.2475 0.0127705675659017
0.2525 0.0160835049890066
0.2575 0.0199865026149812
0.2625 0.024517668517973
0.2675 0.0297048683221924
0.2725 0.0355645634604718
0.2775 0.0421012918433731
0.2825 0.0493077803908931
0.2875 0.0571656182273979
0.2925 0.0656463765219179
0.2975 0.0747130389685672
0.3025 0.0843216044817968
0.3075 0.0944227366293171
0.3125 0.104963357117122
0.3175 0.115888107826333
0.3225 0.127140633030916
0.3275 0.138664657456851
0.3325 0.150404855140034
0.3375 0.162307518113764
0.3425 0.174321043116104
0.3475 0.186396259514057
0.3525 0.198486623435526
0.3575 0.210548302610127
0.3625 0.222540174456852
0.3675 0.234423757164425
0.3725 0.2461630903626
0.3775 0.257724578799727
0.3825 0.269076809419216
0.3875 0.280190349466021
0.3925 0.291037530789031
0.3975 0.301592223328621
0.4025 0.311829598863364
0.4075 0.321725884407147
0.4125 0.33125810319655
0.4175 0.340403800052193
0.4225 0.34914074723873
0.4275 0.357446627257301
0.4325 0.36529869126666
0.4375 0.37267339798658
0.4425 0.379546051649685
0.4475 0.385890485406314
0.4525 0.391678889528999
0.4575 0.396881977899246
0.4625 0.401469838716726
0.4675 0.405414024841958
0.4725 0.408691630863368
0.4775 0.411292024184161
0.4825 0.413226029159821
0.4875 0.414535185870344
0.4925 0.415295776454039
0.4975 0.415611531428894
0.5025 0.41559377314098
0.5075 0.415336270392398
0.5125 0.414895446220246
0.5175 0.41427973531504
0.5225 0.413441913351203
0.5275 0.412265556853831
0.5325 0.410543069713507
0.5375 0.407952684100831
0.5425 0.404050220477402
0.5475 0.39829414322411
0.5525 0.390116259686097
0.5575 0.379034706827778
0.5625 0.364785475907473
0.5675 0.347433283903082
0.5725 0.327421672581876
0.5775 0.305538966019578
0.5825 0.282804899447141
0.5875 0.260309643125283
0.5925 0.239050561287535
0.5975 0.219807541815168
0.6025 0.20307934526991
0.6075 0.189081107060286
0.6125 0.177786616825983
0.6175 0.168992999905332
0.6225 0.162388541129918
0.6275 0.15761173233599
0.6325 0.154296586879928
0.6375 0.152103711892583
0.6425 0.150738704005009
0.6475 0.149960167756696
0.6525 0.149579858547135
0.6575 0.149457421963541
0.6625 0.149491943424773
0.6675 0.1496120131206
0.6725 0.149765299386181
0.6775 0.149907790936039
0.6825 0.149991956492704
0.6875 0.149952035270804
0.6925 0.149683390951819
0.6975 0.149011228649705
0.7025 0.147642219487674
0.7075 0.145092233973356
0.7125 0.140589880124687
0.7175 0.132983873988083
0.7225 0.120763965024808
0.7275 0.10247138733323
0.7325 0.0779153045242086
0.7375 0.0500927744393442
0.7425 0.0255869633865413
0.7475 0.0101029770077908
0.7525 0.00323067177688417
0.7575 0.000916685479303123
0.7625 0.000247535672051547
0.7675 6.57015394421608e-05
0.7725 1.733476779243e-05
0.7775 4.56166541654529e-06
0.7825 1.19832317168437e-06
0.7875 3.14296210643595e-07
0.7925 8.22998342934205e-08
0.7975 2.15132336229347e-08
0.8025 5.61307170845078e-09
0.8075 1.46156535792315e-09
0.8125 3.79741250295264e-10
0.8175 9.8431384383556e-11
0.8225 2.54493088990962e-11
0.8275 6.56196969727746e-12
0.8325 1.68704996773948e-12
0.8375 4.3243891523643e-13
0.8425 1.10497770422186e-13
0.8475 2.8177175461474e-14
0.8525 7.14463267733202e-15
0.8575 1.78835622885588e-15
0.8625 4.08881035414389e-16
0.8675 5.23823250656242e-17
0.8725 2.23070011976512e-18
0.8775 -1.70270269849063e-18
0.8825 1.91820139524602e-19
0.8875 -2.40410875285755e-18
0.8925 -2.60371400131214e-18
0.8975 1.34734437976866e-19
0.9025 1.3517563413719e-19
0.9075 1.3561001001166e-19
0.9125 1.3561001001166e-19
0.9175 1.3561001001166e-19
0.9225 1.3561001001166e-19
0.9275 1.3561001001166e-19
0.9325 1.3561001001166e-19
0.9375 1.3561001001166e-19
0.9425 1.3561001001166e-19
0.9475 1.3561001001166e-19
0.9525 1.3561001001166e-19
0.9575 1.3561001001166e-19
0.9625 1.3561001001166e-19
0.9675 1.3561001001166e-19
0.9725 1.3561001001166e-19
0.9775 1.3561001001166e-19
0.9825 1.3561001001166e-19
0.9875 1.3561001001166e-19
0.9925 1.3561001001166e-19
0.9975 1.3561001001166e-19
};
\addlegendentry{truth}

\nextgroupplot[
ytick={0,0.1,0.2,0.4,0.5},
ylabel={\(E\)},
ymin=-0.0100275328732878, ymax=0.52428702249394,
y label style={xshift=.8em}
]
\addplot [semithick, color0, mark=pentagon, mark size=2, mark options={solid},mark repeat=5]
table {%
0.0025 0.499566434696068
0.0075 0.499566434696068
0.0125 0.499566434696068
0.0175 0.499566434696068
0.0225 0.499566434696068
0.0275 0.499566434696068
0.0325 0.499566434696068
0.0375 0.499566434696068
0.0425 0.499566434696068
0.0475 0.499566434696068
0.0525 0.499566434696068
0.0575 0.499566434696068
0.0625 0.499566434696068
0.0675 0.499566434696068
0.0725 0.499566434696068
0.0775 0.499566434696068
0.0825 0.499566434696068
0.0875 0.499566434696068
0.0925 0.499566434696068
0.0975 0.499566370348632
0.1025 0.499567510892607
0.1075 0.499567527370009
0.1125 0.499568461188917
0.1175 0.499565495813832
0.1225 0.499567820889932
0.1275 0.499567439440812
0.1325 0.499567908092993
0.1375 0.499566768757936
0.1425 0.499564687891228
0.1475 0.499561228374623
0.1525 0.499559679675419
0.1575 0.499553906343711
0.1625 0.499542880865482
0.1675 0.499527347451757
0.1725 0.499503439223567
0.1775 0.499467262862731
0.1825 0.499412237656599
0.1875 0.499332065979628
0.1925 0.499212074995381
0.1975 0.499043806748767
0.2025 0.498802532356096
0.2075 0.498472301059849
0.2125 0.498020814721587
0.2175 0.497416882395611
0.2225 0.49662143605863
0.2275 0.495597837974689
0.2325 0.494300632658181
0.2375 0.492685228301581
0.2425 0.490716071969207
0.2475 0.488356221513139
0.2525 0.485586777843386
0.2575 0.482366202590156
0.2625 0.478636349469023
0.2675 0.474370902309825
0.2725 0.469563017550409
0.2775 0.464211038835019
0.2825 0.458326442635222
0.2875 0.451927760843783
0.2925 0.445051297040587
0.2975 0.437740810869383
0.3025 0.430049988082055
0.3075 0.422041759481815
0.3125 0.41377940822452
0.3175 0.405341573274708
0.3225 0.396805402694221
0.3275 0.388252174850362
0.3325 0.379800179743692
0.3375 0.371559981380593
0.3425 0.363620787652613
0.3475 0.356180819505493
0.3525 0.349377926001172
0.3575 0.3432583843994
0.3625 0.33782942289075
0.3675 0.332347496805064
0.3725 0.326394980320559
0.3775 0.320012823852642
0.3825 0.313236933914264
0.3875 0.306129853124111
0.3925 0.298745550819665
0.3975 0.291790480170699
0.4025 0.285852767651042
0.4075 0.281006252061158
0.4125 0.277299036681199
0.4175 0.274759348070896
0.4225 0.273400023151207
0.4275 0.272964188700684
0.4325 0.271815780753738
0.4375 0.269828438346021
0.4425 0.267249841352473
0.4475 0.264350149833369
0.4525 0.261423391158425
0.4575 0.258766695461318
0.4625 0.256654295660602
0.4675 0.25529988899042
0.4725 0.254513976549939
0.4775 0.254076380414675
0.4825 0.253857447711592
0.4875 0.253797470865467
0.4925 0.253808752077065
0.4975 0.253763714009828
0.5025 0.253699368179262
0.5075 0.25362855235414
0.5125 0.253532433385479
0.5175 0.253393969678747
0.5225 0.253173501656991
0.5275 0.252807950506907
0.5325 0.252185353063773
0.5375 0.251113538453846
0.5425 0.249354667481387
0.5475 0.246752242224898
0.5525 0.243418878017469
0.5575 0.239465106067682
0.5625 0.234279419366452
0.5675 0.227783804517479
0.5725 0.220192690876746
0.5775 0.211972309824358
0.5825 0.203765609306105
0.5875 0.196264027025048
0.5925 0.190010807978054
0.5975 0.184577986106031
0.6025 0.179033526307619
0.6075 0.173685600452327
0.6125 0.168782162839598
0.6175 0.164524354824771
0.6225 0.16104175884175
0.6275 0.158359583934681
0.6325 0.156417219529006
0.6375 0.15509608100915
0.6425 0.154262423100925
0.6475 0.153786435607038
0.6525 0.153558933343705
0.6575 0.153494577775065
0.6625 0.15352806838995
0.6675 0.153617849069447
0.6725 0.153725001225156
0.6775 0.15382214803754
0.6825 0.1538752986271
0.6875 0.153841953133464
0.6925 0.153640439458696
0.6975 0.153146652727694
0.7025 0.152149559024472
0.7075 0.150290444535956
0.7125 0.146908177847255
0.7175 0.140638036539332
0.7225 0.129557863401295
0.7275 0.11102893867151
0.7325 0.0820449516669314
0.7375 0.046488003712627
0.7425 0.0207998444859785
0.7475 0.0142594923706771
0.7525 0.014927622493576
0.7575 0.0156025802899868
0.7625 0.0158366322815483
0.7675 0.0158999766958207
0.7725 0.0159181611108628
0.7775 0.0159240230115524
0.7825 0.0159260359236937
0.7875 0.0159249571562401
0.7925 0.0159248001028088
0.7975 0.0159241176643776
0.8025 0.0159246474218944
0.8075 0.0159247248240804
0.8125 0.0159247248240804
0.8175 0.0159247248240804
0.8225 0.0159247248240804
0.8275 0.0159247248240804
0.8325 0.0159247248240804
0.8375 0.0159247248240804
0.8425 0.0159247248240804
0.8475 0.0159247248240804
0.8525 0.0159247248240804
0.8575 0.0159247248240804
0.8625 0.0159247248240804
0.8675 0.0159247248240804
0.8725 0.0159247248240804
0.8775 0.0159247248240804
0.8825 0.0159247248240804
0.8875 0.0159247248240804
0.8925 0.0159247248240804
0.8975 0.0159247248240804
0.9025 0.0159247248240804
0.9075 0.0159247248240804
0.9125 0.0159247248240804
0.9175 0.0159247248240804
0.9225 0.0159247248240804
0.9275 0.0159247248240804
0.9325 0.0159247248240804
0.9375 0.0159247248240804
0.9425 0.0159247248240804
0.9475 0.0159247248240804
0.9525 0.0159247248240804
0.9575 0.0159247248240804
0.9625 0.0159247248240804
0.9675 0.0159247248240804
0.9725 0.0159247248240804
0.9775 0.0159247248240804
0.9825 0.0159247248240804
0.9875 0.0159247248240804
0.9925 0.0159247248240804
0.9975 0.0159247248240804
};
\addlegendentry{prediction}
\addplot [semithick, black, mark=+, mark size=2, mark options={solid},dashed,mark repeat=5]
table {%
0.0025 0.499999997249975
0.0075 0.499999997249952
0.0125 0.49999999724991
0.0175 0.49999999724983
0.0225 0.499999997249673
0.0275 0.499999997249343
0.0325 0.499999997248633
0.0375 0.499999997247108
0.0425 0.499999997243849
0.0475 0.499999997236918
0.0525 0.49999999722231
0.0575 0.499999997191854
0.0625 0.499999997129034
0.0675 0.499999997000896
0.0725 0.499999996742452
0.0775 0.499999996227163
0.0825 0.499999995211631
0.0875 0.499999993233655
0.0925 0.499999989426775
0.0975 0.499999982187882
0.1025 0.499999968590229
0.1075 0.499999943362711
0.1125 0.49999989714242
0.1175 0.499999813531151
0.1225 0.499999664219035
0.1275 0.499999401043895
0.1325 0.499998943288962
0.1375 0.499998157735899
0.1425 0.499996827939107
0.1475 0.499994607841606
0.1525 0.499990953221392
0.1575 0.499985022618125
0.1625 0.499975537529369
0.1675 0.499960590121808
0.1725 0.499937386002681
0.1775 0.499901910468537
0.1825 0.499848509989294
0.1875 0.499769387444734
0.1925 0.499654020597409
0.1975 0.499488528760043
0.2025 0.499255032002629
0.2075 0.498931068658278
0.2125 0.498489156918177
0.2175 0.497896600156329
0.2225 0.497115637784343
0.2275 0.496104028964957
0.2325 0.494816122733982
0.2375 0.493204416224345
0.2425 0.491221538805241
0.2475 0.488822534483274
0.2525 0.48596726066466
0.2575 0.482622690300756
0.2625 0.478764904226945
0.2675 0.474380592224794
0.2725 0.469467939022164
0.2775 0.46403684380784
0.2825 0.458108495428486
0.2875 0.451714388097888
0.2925 0.444894905918178
0.2975 0.437697625542276
0.3025 0.430175486289908
0.3075 0.422384960603958
0.3125 0.414384331017942
0.3175 0.406232148833711
0.3225 0.397985919383409
0.3275 0.389701032374154
0.3325 0.381429935098179
0.3375 0.373221531647741
0.3425 0.365120782228682
0.3475 0.357168472254166
0.3525 0.349401119983176
0.3575 0.341850992947928
0.3625 0.334546206351425
0.3675 0.327510880277021
0.3725 0.320765336408751
0.3775 0.314326318665062
0.3825 0.308207225493483
0.3875 0.302418344457086
0.3925 0.296967082133645
0.3975 0.291858184256781
0.4025 0.287093942489105
0.4075 0.28267438527183
0.4125 0.278597450881622
0.4175 0.27485914117029
0.4225 0.271453654478241
0.4275 0.268373495894285
0.4325 0.265609562373013
0.4375 0.263151199234399
0.4425 0.260986223393296
0.4475 0.25910090775627
0.4525 0.257479921790285
0.4575 0.256106228028066
0.4625 0.254960948347607
0.4675 0.254023244450158
0.4725 0.253270307401023
0.4775 0.252677601550096
0.4825 0.252219485718782
0.4875 0.251870132202037
0.4925 0.251604330194165
0.4975 0.251397729084703
0.5025 0.25122669129834
0.5075 0.251068332781992
0.5125 0.250900333775037
0.5175 0.250698706378175
0.5225 0.250431897599141
0.5275 0.250051148444404
0.5325 0.249478662542547
0.5375 0.248596912673306
0.5425 0.247244381726547
0.5475 0.245223917214396
0.5525 0.242327893303772
0.5575 0.238378924611544
0.5625 0.233277606118821
0.5675 0.227043160220145
0.5725 0.219832439446087
0.5775 0.211928643264648
0.5825 0.203701164573315
0.5875 0.195547680233291
0.5925 0.187834659876278
0.5975 0.18085111996356
0.6025 0.174784098494021
0.6075 0.169716410533201
0.6125 0.165641200169253
0.6175 0.162485367499619
0.6225 0.160134752274621
0.6275 0.158456399070025
0.6325 0.157315740975606
0.6375 0.156588304507358
0.6425 0.156166477645325
0.6475 0.155962258709567
0.6525 0.155906982680582
0.6575 0.155948945374948
0.6625 0.156049656989133
0.6675 0.156179160619016
0.6725 0.156310449279297
0.6775 0.156412496184095
0.6825 0.156440733162282
0.6875 0.156322878793174
0.6925 0.155936716265043
0.6975 0.155074755703434
0.7025 0.15338930096658
0.7075 0.150312941539415
0.7125 0.144962247510296
0.7175 0.136077279136682
0.7225 0.122162286306937
0.7275 0.102173607088272
0.7325 0.0770700772857907
0.7375 0.0513960997098126
0.7425 0.0317715372289992
0.7475 0.0212761317050686
0.7525 0.0172929078157419
0.7575 0.0160879851700508
0.7625 0.0157570212643047
0.7675 0.0156687593965459
0.7725 0.0156454185249256
0.7775 0.0156392644262287
0.7825 0.015637644580833
0.7875 0.0156372188316351
0.7925 0.0156371070941755
0.7975 0.0156370778143173
0.8025 0.0156370701545981
0.8075 0.0156370681544042
0.8125 0.0156370676331101
0.8175 0.0156370674975363
0.8225 0.0156370674623578
0.8275 0.0156370674532522
0.8325 0.0156370674509016
0.8375 0.0156370674502965
0.8425 0.0156370674501412
0.8475 0.0156370674501014
0.8525 0.0156370674500913
0.8575 0.0156370674500887
0.8625 0.015637067450088
0.8675 0.0156370674500878
0.8725 0.0156370674500877
0.8775 0.0156370674500877
0.8825 0.0156370674500877
0.8875 0.0156370674500877
0.8925 0.0156370674500877
0.8975 0.0156370674500877
0.9025 0.0156370674500877
0.9075 0.0156370674500877
0.9125 0.0156370674500877
0.9175 0.0156370674500877
0.9225 0.0156370674500877
0.9275 0.0156370674500877
0.9325 0.0156370674500877
0.9375 0.0156370674500877
0.9425 0.0156370674500877
0.9475 0.0156370674500877
0.9525 0.0156370674500877
0.9575 0.0156370674500877
0.9625 0.0156370674500877
0.9675 0.0156370674500877
0.9725 0.0156370674500877
0.9775 0.0156370674500877
0.9825 0.0156370674500877
0.9875 0.0156370674500877
0.9925 0.0156370674500877
0.9975 0.0156370674500877
};
\addlegendentry{truth}
\end{groupplot}

\end{tikzpicture}

	\caption{Interpoaltionin in time for \(\hy\) from 25 snapshots to 241 snapshots with cubic splines using the FCNN.}
\end{figure}
\begin{figure}[H]
	\scalebox{1}{% This file was created by tikzplotlib v0.9.6.
\begin{tikzpicture}

\begin{groupplot}[group style={group size=3 by 1,horizontal sep=2cm, vertical sep=2cm}]
\nextgroupplot[
legend cell align={left},
legend style={draw=none},
tick align=outside,
tick pos=left,
x grid style={white!69.0196078431373!black},
xlabel={\(x\)},
xmin=-0.04725, xmax=1.04725,
xtick style={color=black},
y grid style={white!69.0196078431373!black},
ylabel={\(c_0\) and \(\rho\)},
ymin=-0.0600705937195856, ymax=1.05041666166284,
ytick style={color=black},
axis lines=left
]
\addplot [semithick, black, dashed]
table {%
0.0025 0.532396674156189
0.0075 0.532396674156189
0.0125 0.532396674156189
0.0175 0.532396674156189
0.0225 0.532396674156189
0.0275 0.532396674156189
0.0325 0.532396674156189
0.0375 0.532396674156189
0.0425 0.532396674156189
0.0475 0.532396674156189
0.0525 0.532396674156189
0.0575 0.532396674156189
0.0625 0.532396674156189
0.0675 0.532396674156189
0.0725 0.532396674156189
0.0775 0.532396674156189
0.0825 0.532396674156189
0.0875 0.532396674156189
0.0925 0.532396674156189
0.0975 0.532396674156189
0.1025 0.532396674156189
0.1075 0.532396674156189
0.1125 0.532396674156189
0.1175 0.532396674156189
0.1225 0.532396674156189
0.1275 0.532396674156189
0.1325 0.532396674156189
0.1375 0.532396674156189
0.1425 0.532396674156189
0.1475 0.532396674156189
0.1525 0.532396674156189
0.1575 0.532396674156189
0.1625 0.532396674156189
0.1675 0.532396674156189
0.1725 0.532396674156189
0.1775 0.532396674156189
0.1825 0.532396674156189
0.1875 0.532396674156189
0.1925 0.532396674156189
0.1975 0.532396674156189
0.2025 0.532396674156189
0.2075 0.532396674156189
0.2125 0.532396674156189
0.2175 0.532396674156189
0.2225 0.532396674156189
0.2275 0.532396674156189
0.2325 0.532396674156189
0.2375 0.532396674156189
0.2425 0.532396674156189
0.2475 0.532396674156189
0.2525 0.532396674156189
0.2575 0.532396674156189
0.2625 0.532396674156189
0.2675 0.532396674156189
0.2725 0.532396674156189
0.2775 0.532396674156189
0.2825 0.532396614551544
0.2875 0.532396554946899
0.2925 0.532396554946899
0.2975 0.532396376132965
0.3025 0.532396078109741
0.3075 0.532395303249359
0.3125 0.53239369392395
0.3175 0.532390356063843
0.3225 0.532383680343628
0.3275 0.532370090484619
0.3325 0.532343864440918
0.3375 0.532294631004333
0.3425 0.532204687595367
0.3475 0.532045602798462
0.3525 0.531773149967194
0.3575 0.531321406364441
0.3625 0.530597150325775
0.3675 0.529474556446075
0.3725 0.527792632579803
0.3775 0.525356471538544
0.3825 0.521943271160126
0.3875 0.517312824726105
0.3925 0.511222898960114
0.3975 0.503444910049438
0.4025 0.493780314922333
0.4075 0.482075095176697
0.4125 0.468230247497559
0.4175 0.452209562063217
0.4225 0.43382865190506
0.4275 0.413315415382385
0.4325 0.3905388712883
0.4375 0.365876168012619
0.4425 0.339789062738419
0.4475 0.312636196613312
0.4525 0.284825265407562
0.4575 0.256570726633072
0.4625 0.228063091635704
0.4675 0.200090900063515
0.4725 0.173177376389503
0.4775 0.147900074720383
0.4825 0.124915175139904
0.4875 0.106026835739613
0.4925 0.0917501226067543
0.4975 0.0818334966897964
0.5025 0.0766019970178604
0.5075 0.0748537853360176
0.5125 0.0742902159690857
0.5175 0.073918804526329
0.5225 0.0725244954228401
0.5275 0.0700906664133072
0.5325 0.066360130906105
0.5375 0.0614439621567726
0.5425 0.0554116740822792
0.5475 0.0496930554509163
0.5525 0.0449490025639534
0.5575 0.041465163230896
0.5625 0.0391549542546272
0.5675 0.0376920029520988
0.5725 0.0366387143731117
0.5775 0.0355018638074398
0.5825 0.0337106138467789
0.5875 0.0305485334247351
0.5925 0.0251204036176205
0.5975 0.016600364819169
0.6025 0.00535319838672876
0.6075 -0.00505704572424293
0.6125 -0.0092976912856102
0.6175 -0.0095939002931118
0.6225 -0.00928737316280603
0.6275 -0.0091561870649457
0.6325 -0.00911904219537973
0.6375 -0.00910959485918283
0.6425 -0.00910726375877857
0.6475 -0.0091067124158144
0.6525 -0.00910657085478306
0.6575 -0.00910652615129948
0.6625 -0.00910654105246067
0.6675 -0.00910652615129948
0.6725 -0.00910652615129948
0.6775 -0.00910652615129948
0.6825 -0.00910652615129948
0.6875 -0.00910652615129948
0.6925 -0.00910652615129948
0.6975 -0.00910652615129948
0.7025 -0.00910652615129948
0.7075 -0.00910652615129948
0.7125 -0.00910652615129948
0.7175 -0.00910652615129948
0.7225 -0.00910652615129948
0.7275 -0.00910652615129948
0.7325 -0.00910652615129948
0.7375 -0.00910652615129948
0.7425 -0.00910652615129948
0.7475 -0.00910652615129948
0.7525 -0.00910652615129948
0.7575 -0.00910652615129948
0.7625 -0.00910652615129948
0.7675 -0.00910652615129948
0.7725 -0.00910652615129948
0.7775 -0.00910652615129948
0.7825 -0.00910652615129948
0.7875 -0.00910652615129948
0.7925 -0.00910652615129948
0.7975 -0.00910652615129948
0.8025 -0.00910652615129948
0.8075 -0.00910652615129948
0.8125 -0.00910652615129948
0.8175 -0.00910652615129948
0.8225 -0.00910652615129948
0.8275 -0.00910652615129948
0.8325 -0.00910652615129948
0.8375 -0.00910652615129948
0.8425 -0.00910652615129948
0.8475 -0.00910652615129948
0.8525 -0.00910652615129948
0.8575 -0.00910652615129948
0.8625 -0.00910652615129948
0.8675 -0.00910652615129948
0.8725 -0.00910652615129948
0.8775 -0.00910652615129948
0.8825 -0.00910652615129948
0.8875 -0.00910652615129948
0.8925 -0.00910652615129948
0.8975 -0.00910652615129948
0.9025 -0.00910652615129948
0.9075 -0.00910652615129948
0.9125 -0.00910652615129948
0.9175 -0.00910652615129948
0.9225 -0.00910652615129948
0.9275 -0.00910652615129948
0.9325 -0.00910652615129948
0.9375 -0.00910652615129948
0.9425 -0.00910652615129948
0.9475 -0.00910652615129948
0.9525 -0.00910652615129948
0.9575 -0.00910652615129948
0.9625 -0.00910652615129948
0.9675 -0.00910652615129948
0.9725 -0.00910652615129948
0.9775 -0.00910652615129948
0.9825 -0.00910652615129948
0.9875 -0.00910652615129948
0.9925 -0.00910652615129948
0.9975 -0.00910652615129948
};
\addlegendentry{\(c_0\)}
\addplot [semithick, black]
table {%
0.0025 0.999939968236364
0.0075 0.999939968236364
0.0125 0.999939968236364
0.0175 0.999939968236364
0.0225 0.999939968236364
0.0275 0.999939968236364
0.0325 0.999939968236364
0.0375 0.999939968236364
0.0425 0.999939968236364
0.0475 0.999939968236364
0.0525 0.999939968236364
0.0575 0.999939968236364
0.0625 0.999939968236364
0.0675 0.999939968236364
0.0725 0.999939968236364
0.0775 0.999939968236364
0.0825 0.999939968236364
0.0875 0.999939968236364
0.0925 0.999939968236364
0.0975 0.999939968236364
0.1025 0.999939968236364
0.1075 0.999939968236364
0.1125 0.999939968236364
0.1175 0.999939968236364
0.1225 0.999939968236364
0.1275 0.999939968236364
0.1325 0.999939968236364
0.1375 0.999939968236364
0.1425 0.999939968236364
0.1475 0.999939968236364
0.1525 0.999939968236364
0.1575 0.999939968236364
0.1625 0.999939968236364
0.1675 0.999939968236364
0.1725 0.999939968236364
0.1775 0.999939968236364
0.1825 0.999939968236364
0.1875 0.999939968236364
0.1925 0.999939968236364
0.1975 0.999939968236364
0.2025 0.999939968236364
0.2075 0.999939968236364
0.2125 0.999939968236364
0.2175 0.999939968236364
0.2225 0.999939968236364
0.2275 0.999939968236364
0.2325 0.999939968236364
0.2375 0.999939968236364
0.2425 0.999939968236364
0.2475 0.999939968236364
0.2525 0.999939968236364
0.2575 0.999939968236364
0.2625 0.999939968236364
0.2675 0.999939968236364
0.2725 0.999939968236364
0.2775 0.999939968236364
0.2825 0.999939891820153
0.2875 0.999939795822287
0.2925 0.999939859343263
0.2975 0.999939768599012
0.3025 0.999939568006457
0.3075 0.999939004317499
0.3125 0.999937914670087
0.3175 0.999935610124316
0.3225 0.999930894050078
0.3275 0.999921475513241
0.3325 0.999903148159576
0.3375 0.999868639553778
0.3425 0.999805401198757
0.3475 0.999693513179246
0.3525 0.999501534403326
0.3575 0.999182681433665
0.3625 0.998670157785408
0.3675 0.997873304220728
0.3725 0.996675169787919
0.3775 0.994931922341959
0.3825 0.992476384585293
0.3875 0.989122819633056
0.3925 0.984677037057013
0.3975 0.978946031716007
0.4025 0.971748285974638
0.4075 0.962925586276329
0.4125 0.952351180573878
0.4175 0.939938538134671
0.4225 0.926414078388076
0.4275 0.911102869118062
0.4325 0.894411588565279
0.4375 0.87616816521264
0.4425 0.856281408252051
0.4475 0.834931061627009
0.4525 0.812339105476171
0.4575 0.788944653975658
0.4625 0.764412948240836
0.4675 0.73869173713506
0.4725 0.711952378758444
0.4775 0.684410636790861
0.4825 0.65641625963438
0.4875 0.62931059835813
0.4925 0.604090741477334
0.4975 0.582040463789151
0.5025 0.565329562180126
0.5075 0.552076098556893
0.5125 0.535170405697173
0.5175 0.509202869561238
0.5225 0.474026685771652
0.5275 0.429425106670421
0.5325 0.382033329671965
0.5375 0.335415347168843
0.5425 0.294759331318812
0.5475 0.262096491642296
0.5525 0.238131975086454
0.5575 0.221985514060809
0.5625 0.211974344073007
0.5675 0.206237247404762
0.5725 0.203080492524
0.5775 0.201084886868604
0.5825 0.199046708667316
0.5875 0.195821415609083
0.5925 0.190153305227749
0.5975 0.180685935088266
0.6025 0.166637054405724
0.6075 0.149738479596682
0.6125 0.136586853470176
0.6175 0.12911991096842
0.6225 0.126236281118905
0.6275 0.125364558771253
0.6325 0.125054053198069
0.6375 0.124968802198195
0.6425 0.124947830246618
0.6475 0.12494275226998
0.6525 0.124941546326646
0.6575 0.124941325913637
0.6625 0.124941193856872
0.6675 0.124941221796549
0.6725 0.124941221796549
0.6775 0.124941221796549
0.6825 0.124941221796549
0.6875 0.124941221796549
0.6925 0.124941221796549
0.6975 0.124941221796549
0.7025 0.124941221796549
0.7075 0.124941221796549
0.7125 0.124941221796549
0.7175 0.124941221796549
0.7225 0.124941221796549
0.7275 0.124941221796549
0.7325 0.124941221796549
0.7375 0.124941221796549
0.7425 0.124941221796549
0.7475 0.124941221796549
0.7525 0.124941221796549
0.7575 0.124941221796549
0.7625 0.124941221796549
0.7675 0.124941221796549
0.7725 0.124941221796549
0.7775 0.124941221796549
0.7825 0.124941221796549
0.7875 0.124941221796549
0.7925 0.124941221796549
0.7975 0.124941221796549
0.8025 0.124941221796549
0.8075 0.124941221796549
0.8125 0.124941221796549
0.8175 0.124941221796549
0.8225 0.124941221796549
0.8275 0.124941221796549
0.8325 0.124941221796549
0.8375 0.124941221796549
0.8425 0.124941221796549
0.8475 0.124941221796549
0.8525 0.124941221796549
0.8575 0.124941221796549
0.8625 0.124941221796549
0.8675 0.124941221796549
0.8725 0.124941221796549
0.8775 0.124941221796549
0.8825 0.124941221796549
0.8875 0.124941221796549
0.8925 0.124941221796549
0.8975 0.124941221796549
0.9025 0.124941221796549
0.9075 0.124941221796549
0.9125 0.124941221796549
0.9175 0.124941221796549
0.9225 0.124941221796549
0.9275 0.124941221796549
0.9325 0.124941221796549
0.9375 0.124941221796549
0.9425 0.124941221796549
0.9475 0.124941221796549
0.9525 0.124941221796549
0.9575 0.124941221796549
0.9625 0.124941221796549
0.9675 0.124941221796549
0.9725 0.124941221796549
0.9775 0.124941221796549
0.9825 0.124941221796549
0.9875 0.124941221796549
0.9925 0.124941221796549
0.9975 0.124941221796549
};
\addlegendentry{\(\rho\)}

\nextgroupplot[
legend cell align={left},
legend style={draw=none},
tick align=outside,
tick pos=left,
x grid style={white!69.0196078431373!black},
xlabel={\(x\)},
xmin=-0.04725, xmax=1.04725,
xtick style={color=black},
y grid style={white!69.0196078431373!black},
ylabel={\(c_1\) and \(\rho u\)},
ymin=-0.597515286427337, ymax=0.452524393651436,
ytick style={color=black},
axis lines=left
]
\addplot [semithick, black, dashed]
table {%
0.0025 -0.432732731103897
0.0075 -0.432732731103897
0.0125 -0.432732731103897
0.0175 -0.432732731103897
0.0225 -0.432732731103897
0.0275 -0.432732731103897
0.0325 -0.432732731103897
0.0375 -0.432732731103897
0.0425 -0.432732731103897
0.0475 -0.432732731103897
0.0525 -0.432732731103897
0.0575 -0.432732731103897
0.0625 -0.432732731103897
0.0675 -0.432732731103897
0.0725 -0.432732731103897
0.0775 -0.432732731103897
0.0825 -0.432732731103897
0.0875 -0.432732731103897
0.0925 -0.432732731103897
0.0975 -0.432732731103897
0.1025 -0.432732731103897
0.1075 -0.432732731103897
0.1125 -0.432732731103897
0.1175 -0.432732731103897
0.1225 -0.432732731103897
0.1275 -0.432732731103897
0.1325 -0.432732731103897
0.1375 -0.432732731103897
0.1425 -0.432732731103897
0.1475 -0.432732731103897
0.1525 -0.432732731103897
0.1575 -0.432732731103897
0.1625 -0.432732731103897
0.1675 -0.432732731103897
0.1725 -0.432732731103897
0.1775 -0.432732731103897
0.1825 -0.432732731103897
0.1875 -0.432732731103897
0.1925 -0.432732731103897
0.1975 -0.432732731103897
0.2025 -0.432732731103897
0.2075 -0.432732731103897
0.2125 -0.432732731103897
0.2175 -0.432732731103897
0.2225 -0.432732731103897
0.2275 -0.432732731103897
0.2325 -0.432732731103897
0.2375 -0.432732731103897
0.2425 -0.432732731103897
0.2475 -0.432732731103897
0.2525 -0.432732731103897
0.2575 -0.432732731103897
0.2625 -0.432732731103897
0.2675 -0.432732731103897
0.2725 -0.432732731103897
0.2775 -0.432732731103897
0.2825 -0.432732731103897
0.2875 -0.432732731103897
0.2925 -0.432732790708542
0.2975 -0.432732880115509
0.3025 -0.432733088731766
0.3075 -0.432733476161957
0.3125 -0.432734400033951
0.3175 -0.432736217975616
0.3225 -0.43274000287056
0.3275 -0.432747662067413
0.3325 -0.432762205600739
0.3375 -0.432789623737335
0.3425 -0.432839512825012
0.3475 -0.432927429676056
0.3525 -0.4330775141716
0.3575 -0.433325350284576
0.3625 -0.433720856904984
0.3675 -0.434330135583878
0.3725 -0.435236155986786
0.3775 -0.436536222696304
0.3825 -0.438336074352264
0.3875 -0.440741539001465
0.3925 -0.443847119808197
0.3975 -0.447724401950836
0.4025 -0.452411115169525
0.4075 -0.457903206348419
0.4125 -0.464150786399841
0.4175 -0.471056699752808
0.4225 -0.479671806097031
0.4275 -0.489005982875824
0.4325 -0.499341070652008
0.4375 -0.509970843791962
0.4425 -0.52016693353653
0.4475 -0.529586613178253
0.4525 -0.537876844406128
0.4575 -0.544366419315338
0.4625 -0.548331916332245
0.4675 -0.54978621006012
0.4725 -0.548344135284424
0.4775 -0.543678402900696
0.4825 -0.535618782043457
0.4875 -0.523471593856812
0.4925 -0.508366823196411
0.4975 -0.492929041385651
0.5025 -0.479555636644363
0.5075 -0.467656821012497
0.5125 -0.452263534069061
0.5175 -0.429323792457581
0.5225 -0.400096535682678
0.5275 -0.363904923200607
0.5325 -0.32466995716095
0.5375 -0.287306934595108
0.5425 -0.257477670907974
0.5475 -0.234640568494797
0.5525 -0.218610465526581
0.5575 -0.208207130432129
0.5625 -0.201991692185402
0.5675 -0.198689639568329
0.5725 -0.197360694408417
0.5775 -0.197444394230843
0.5825 -0.198761835694313
0.5875 -0.201499730348587
0.5925 -0.206139147281647
0.5975 -0.213125392794609
0.6025 -0.221724510192871
0.6075 -0.228136301040649
0.6125 -0.228484690189362
0.6175 -0.225195422768593
0.6225 -0.223013624548912
0.6275 -0.22223174571991
0.6325 -0.22201269865036
0.6375 -0.221956744790077
0.6425 -0.221942976117134
0.6475 -0.221939638257027
0.6525 -0.221938803792
0.6575 -0.221938639879227
0.6625 -0.221938580274582
0.6675 -0.221938580274582
0.6725 -0.221938580274582
0.6775 -0.221938580274582
0.6825 -0.221938580274582
0.6875 -0.221938580274582
0.6925 -0.221938580274582
0.6975 -0.221938580274582
0.7025 -0.221938580274582
0.7075 -0.221938580274582
0.7125 -0.221938580274582
0.7175 -0.221938580274582
0.7225 -0.221938580274582
0.7275 -0.221938580274582
0.7325 -0.221938580274582
0.7375 -0.221938580274582
0.7425 -0.221938580274582
0.7475 -0.221938580274582
0.7525 -0.221938580274582
0.7575 -0.221938580274582
0.7625 -0.221938580274582
0.7675 -0.221938580274582
0.7725 -0.221938580274582
0.7775 -0.221938580274582
0.7825 -0.221938580274582
0.7875 -0.221938580274582
0.7925 -0.221938580274582
0.7975 -0.221938580274582
0.8025 -0.221938580274582
0.8075 -0.221938580274582
0.8125 -0.221938580274582
0.8175 -0.221938580274582
0.8225 -0.221938580274582
0.8275 -0.221938580274582
0.8325 -0.221938580274582
0.8375 -0.221938580274582
0.8425 -0.221938580274582
0.8475 -0.221938580274582
0.8525 -0.221938580274582
0.8575 -0.221938580274582
0.8625 -0.221938580274582
0.8675 -0.221938580274582
0.8725 -0.221938580274582
0.8775 -0.221938580274582
0.8825 -0.221938580274582
0.8875 -0.221938580274582
0.8925 -0.221938580274582
0.8975 -0.221938580274582
0.9025 -0.221938580274582
0.9075 -0.221938580274582
0.9125 -0.221938580274582
0.9175 -0.221938580274582
0.9225 -0.221938580274582
0.9275 -0.221938580274582
0.9325 -0.221938580274582
0.9375 -0.221938580274582
0.9425 -0.221938580274582
0.9475 -0.221938580274582
0.9525 -0.221938580274582
0.9575 -0.221938580274582
0.9625 -0.221938580274582
0.9675 -0.221938580274582
0.9725 -0.221938580274582
0.9775 -0.221938580274582
0.9825 -0.221938580274582
0.9875 -0.221938580274582
0.9925 -0.221938580274582
0.9975 -0.221938580274582
};
\addlegendentry{\(c_1\)}
\addplot [semithick, black]
table {%
0.0025 -0.000632143126926788
0.0075 -0.000632143126926788
0.0125 -0.000632143126926788
0.0175 -0.000632143126926788
0.0225 -0.000632143126926788
0.0275 -0.000632143126926788
0.0325 -0.000632143126926788
0.0375 -0.000632143126926788
0.0425 -0.000632143126926788
0.0475 -0.000632143126926788
0.0525 -0.000632143126926788
0.0575 -0.000632143126926788
0.0625 -0.000632143126926788
0.0675 -0.000632143126926788
0.0725 -0.000632143126926788
0.0775 -0.000632143126926788
0.0825 -0.000632143126926788
0.0875 -0.000632143126926788
0.0925 -0.000632143126926788
0.0975 -0.000632143126926788
0.1025 -0.000632143126926788
0.1075 -0.000632143126926788
0.1125 -0.000632143126926788
0.1175 -0.000632143126926788
0.1225 -0.000632143126926788
0.1275 -0.000632143126926788
0.1325 -0.000632143126926788
0.1375 -0.000632143126926788
0.1425 -0.000632143126926788
0.1475 -0.000632143126926788
0.1525 -0.000632143126926788
0.1575 -0.000632143126926788
0.1625 -0.000632143126926788
0.1675 -0.000632143126926788
0.1725 -0.000632143126926788
0.1775 -0.000632143126926788
0.1825 -0.000632143126926788
0.1875 -0.000632143126926788
0.1925 -0.000632143126926788
0.1975 -0.000632143126926788
0.2025 -0.000632143126926788
0.2075 -0.000632143126926788
0.2125 -0.000632143126926788
0.2175 -0.000632143126926788
0.2225 -0.000632143126926788
0.2275 -0.000632143126926788
0.2325 -0.000632143126926788
0.2375 -0.000632143126926788
0.2425 -0.000632143126926788
0.2475 -0.000632143126926788
0.2525 -0.000632143126926788
0.2575 -0.000632143126926788
0.2625 -0.000632143126926788
0.2675 -0.000632143126926788
0.2725 -0.000632143126926788
0.2775 -0.000632143126926788
0.2825 -0.000632173375010415
0.2875 -0.000632007194243283
0.2925 -0.00063202397152044
0.2975 -0.000631875302801798
0.3025 -0.00063135079858255
0.3075 -0.000630262632958971
0.3125 -0.000628540512813578
0.3175 -0.000624436478545078
0.3225 -0.000615503068310432
0.3275 -0.000598422146936829
0.3325 -0.000565411660141781
0.3375 -0.000503633072125836
0.3425 -0.000390692391750638
0.3475 -0.000190644761955293
0.3525 0.000152100743401347
0.3575 0.000719298081402627
0.3625 0.0016283939171884
0.3675 0.0030357892123553
0.3725 0.0051411870195923
0.3775 0.00818680967574724
0.3825 0.0124447812790995
0.3875 0.0182065823746699
0.3925 0.0257615754068545
0.3975 0.0353748145966958
0.4025 0.0472672618276523
0.4075 0.0615958609348218
0.4125 0.0784439155078793
0.4175 0.0978072289313771
0.4225 0.119432971373759
0.4275 0.139301568471792
0.4325 0.161420454177454
0.4375 0.185197899467082
0.4425 0.209965857901879
0.4475 0.235314556396885
0.4525 0.26079012848863
0.4575 0.284682962747094
0.4625 0.306761641493759
0.4675 0.328102108626274
0.4725 0.348191657657738
0.4775 0.366437134843008
0.4825 0.382161402785851
0.4875 0.393995357938184
0.4925 0.401444049376963
0.4975 0.404795317284219
0.5025 0.404355673481913
0.5075 0.40062380593973
0.5125 0.392499756843175
0.5175 0.374914746655593
0.5225 0.349303765711053
0.5275 0.316143827329168
0.5325 0.284308368550769
0.5375 0.252408394559405
0.5425 0.221984836173089
0.5475 0.196588067901195
0.5525 0.17733245072819
0.5575 0.164008134440871
0.5625 0.155525898957658
0.5675 0.150401894747721
0.5725 0.147078615502454
0.5775 0.144024359437047
0.5825 0.139635926012114
0.5875 0.132013712461639
0.5925 0.118766937871451
0.5975 0.0973178915995406
0.6025 0.0669648065945081
0.6075 0.033727921251597
0.6125 0.0115989859263063
0.6175 0.00239796569126362
0.6225 -0.000184599186473184
0.6275 -0.000822004102253741
0.6325 -0.000605732814839292
0.6375 -0.000516643299476872
0.6425 -0.000494866148813
0.6475 -0.00048947271344146
0.6525 -0.000488321571797622
0.6575 -0.000488073157880118
0.6625 -0.000487950267386545
0.6675 -0.000487994537355084
0.6725 -0.000487994537355084
0.6775 -0.000487994537355084
0.6825 -0.000487994537355084
0.6875 -0.000487994537355084
0.6925 -0.000487994537355084
0.6975 -0.000487994537355084
0.7025 -0.000487994537355084
0.7075 -0.000487994537355084
0.7125 -0.000487994537355084
0.7175 -0.000487994537355084
0.7225 -0.000487994537355084
0.7275 -0.000487994537355084
0.7325 -0.000487994537355084
0.7375 -0.000487994537355084
0.7425 -0.000487994537355084
0.7475 -0.000487994537355084
0.7525 -0.000487994537355084
0.7575 -0.000487994537355084
0.7625 -0.000487994537355084
0.7675 -0.000487994537355084
0.7725 -0.000487994537355084
0.7775 -0.000487994537355084
0.7825 -0.000487994537355084
0.7875 -0.000487994537355084
0.7925 -0.000487994537355084
0.7975 -0.000487994537355084
0.8025 -0.000487994537355084
0.8075 -0.000487994537355084
0.8125 -0.000487994537355084
0.8175 -0.000487994537355084
0.8225 -0.000487994537355084
0.8275 -0.000487994537355084
0.8325 -0.000487994537355084
0.8375 -0.000487994537355084
0.8425 -0.000487994537355084
0.8475 -0.000487994537355084
0.8525 -0.000487994537355084
0.8575 -0.000487994537355084
0.8625 -0.000487994537355084
0.8675 -0.000487994537355084
0.8725 -0.000487994537355084
0.8775 -0.000487994537355084
0.8825 -0.000487994537355084
0.8875 -0.000487994537355084
0.8925 -0.000487994537355084
0.8975 -0.000487994537355084
0.9025 -0.000487994537355084
0.9075 -0.000487994537355084
0.9125 -0.000487994537355084
0.9175 -0.000487994537355084
0.9225 -0.000487994537355084
0.9275 -0.000487994537355084
0.9325 -0.000487994537355084
0.9375 -0.000487994537355084
0.9425 -0.000487994537355084
0.9475 -0.000487994537355084
0.9525 -0.000487994537355084
0.9575 -0.000487994537355084
0.9625 -0.000487994537355084
0.9675 -0.000487994537355084
0.9725 -0.000487994537355084
0.9775 -0.000487994537355084
0.9825 -0.000487994537355084
0.9875 -0.000487994537355084
0.9925 -0.000487994537355084
0.9975 -0.000487994537355084
};
\addlegendentry{\(rho u\)}

\nextgroupplot[
legend cell align={left},
legend style={draw=none},
tick align=outside,
tick pos=left,
x grid style={white!69.0196078431373!black},
xlabel={\(x\)},
xmin=-0.04725, xmax=1.04725,
xtick style={color=black},
y grid style={white!69.0196078431373!black},
ylabel={\(c_2\) and \(E\)},
ymin=-0.380284859894522, ymax=1.03982424471276,
ytick style={color=black},
axis lines=left
]
\addplot [semithick, black, dashed]
table {%
0.0025 -0.177741408348083
0.0075 -0.177741408348083
0.0125 -0.177741408348083
0.0175 -0.177741408348083
0.0225 -0.177741408348083
0.0275 -0.177741408348083
0.0325 -0.177741408348083
0.0375 -0.177741408348083
0.0425 -0.177741408348083
0.0475 -0.177741408348083
0.0525 -0.177741408348083
0.0575 -0.177741408348083
0.0625 -0.177741408348083
0.0675 -0.177741408348083
0.0725 -0.177741408348083
0.0775 -0.177741408348083
0.0825 -0.177741408348083
0.0875 -0.177741408348083
0.0925 -0.177741408348083
0.0975 -0.177741408348083
0.1025 -0.177741408348083
0.1075 -0.177741408348083
0.1125 -0.177741408348083
0.1175 -0.177741408348083
0.1225 -0.177741408348083
0.1275 -0.177741408348083
0.1325 -0.177741408348083
0.1375 -0.177741408348083
0.1425 -0.177741408348083
0.1475 -0.177741408348083
0.1525 -0.177741408348083
0.1575 -0.177741408348083
0.1625 -0.177741408348083
0.1675 -0.177741408348083
0.1725 -0.177741408348083
0.1775 -0.177741408348083
0.1825 -0.177741408348083
0.1875 -0.177741408348083
0.1925 -0.177741408348083
0.1975 -0.177741408348083
0.2025 -0.177741408348083
0.2075 -0.177741408348083
0.2125 -0.177741408348083
0.2175 -0.177741408348083
0.2225 -0.177741408348083
0.2275 -0.177741408348083
0.2325 -0.177741408348083
0.2375 -0.177741408348083
0.2425 -0.177741408348083
0.2475 -0.177741408348083
0.2525 -0.177741408348083
0.2575 -0.177741408348083
0.2625 -0.177741408348083
0.2675 -0.177741408348083
0.2725 -0.177741408348083
0.2775 -0.177741408348083
0.2825 -0.177741408348083
0.2875 -0.177741378545761
0.2925 -0.177741348743439
0.2975 -0.177741348743439
0.3025 -0.177741199731827
0.3075 -0.177740976214409
0.3125 -0.177740469574928
0.3175 -0.177739366889
0.3225 -0.177737087011337
0.3275 -0.17773263156414
0.3325 -0.177724033594131
0.3375 -0.177707955241203
0.3425 -0.177678778767586
0.3475 -0.177627399563789
0.3525 -0.17754003405571
0.3575 -0.177396222949028
0.3625 -0.177167773246765
0.3675 -0.176817715167999
0.3725 -0.176300853490829
0.3775 -0.17556619644165
0.3825 -0.174561873078346
0.3875 -0.173242062330246
0.3925 -0.171576172113419
0.3975 -0.169558435678482
0.4025 -0.167216286063194
0.4075 -0.164617672562599
0.4125 -0.161874487996101
0.4175 -0.159143015742302
0.4225 -0.156095817685127
0.4275 -0.153310760855675
0.4325 -0.151156857609749
0.4375 -0.149965643882751
0.4425 -0.150083541870117
0.4475 -0.151832059025764
0.4525 -0.155518546700478
0.4575 -0.161904186010361
0.4625 -0.171773359179497
0.4675 -0.18463471531868
0.4725 -0.200466051697731
0.4775 -0.218969643115997
0.4825 -0.239427492022514
0.4875 -0.262117713689804
0.4925 -0.284339308738708
0.4975 -0.301868826150894
0.5025 -0.312276989221573
0.5075 -0.315734446048737
0.5125 -0.314571976661682
0.5175 -0.310171991586685
0.5225 -0.30202442407608
0.5275 -0.291808098554611
0.5325 -0.280926644802094
0.5375 -0.270824879407883
0.5425 -0.261575758457184
0.5475 -0.254636913537979
0.5525 -0.249808222055435
0.5575 -0.246643751859665
0.5625 -0.244650661945343
0.5675 -0.243358820676804
0.5725 -0.242312699556351
0.5775 -0.241020604968071
0.5825 -0.238868311047554
0.5875 -0.23499296605587
0.5925 -0.228125914931297
0.5975 -0.216508731245995
0.6025 -0.19837874174118
0.6075 -0.174530372023582
0.6125 -0.1527980864048
0.6175 -0.141425430774689
0.6225 -0.137841701507568
0.6275 -0.136943116784096
0.6325 -0.136725559830666
0.6375 -0.136672586202621
0.6425 -0.136659666895866
0.6475 -0.136656552553177
0.6525 -0.136655792593956
0.6575 -0.136655628681183
0.6625 -0.136655569076538
0.6675 -0.136655569076538
0.6725 -0.136655569076538
0.6775 -0.136655569076538
0.6825 -0.136655569076538
0.6875 -0.136655569076538
0.6925 -0.136655569076538
0.6975 -0.136655569076538
0.7025 -0.136655569076538
0.7075 -0.136655569076538
0.7125 -0.136655569076538
0.7175 -0.136655569076538
0.7225 -0.136655569076538
0.7275 -0.136655569076538
0.7325 -0.136655569076538
0.7375 -0.136655569076538
0.7425 -0.136655569076538
0.7475 -0.136655569076538
0.7525 -0.136655569076538
0.7575 -0.136655569076538
0.7625 -0.136655569076538
0.7675 -0.136655569076538
0.7725 -0.136655569076538
0.7775 -0.136655569076538
0.7825 -0.136655569076538
0.7875 -0.136655569076538
0.7925 -0.136655569076538
0.7975 -0.136655569076538
0.8025 -0.136655569076538
0.8075 -0.136655569076538
0.8125 -0.136655569076538
0.8175 -0.136655569076538
0.8225 -0.136655569076538
0.8275 -0.136655569076538
0.8325 -0.136655569076538
0.8375 -0.136655569076538
0.8425 -0.136655569076538
0.8475 -0.136655569076538
0.8525 -0.136655569076538
0.8575 -0.136655569076538
0.8625 -0.136655569076538
0.8675 -0.136655569076538
0.8725 -0.136655569076538
0.8775 -0.136655569076538
0.8825 -0.136655569076538
0.8875 -0.136655569076538
0.8925 -0.136655569076538
0.8975 -0.136655569076538
0.9025 -0.136655569076538
0.9075 -0.136655569076538
0.9125 -0.136655569076538
0.9175 -0.136655569076538
0.9225 -0.136655569076538
0.9275 -0.136655569076538
0.9325 -0.136655569076538
0.9375 -0.136655569076538
0.9425 -0.136655569076538
0.9475 -0.136655569076538
0.9525 -0.136655569076538
0.9575 -0.136655569076538
0.9625 -0.136655569076538
0.9675 -0.136655569076538
0.9725 -0.136655569076538
0.9775 -0.136655569076538
0.9825 -0.136655569076538
0.9875 -0.136655569076538
0.9925 -0.136655569076538
0.9975 -0.136655569076538
};
\addlegendentry{\(c_2\)}
\addplot [semithick, black]
table {%
0.0025 0.975271997551442
0.0075 0.975271997551442
0.0125 0.975271997551442
0.0175 0.975271997551442
0.0225 0.975271997551442
0.0275 0.975271997551442
0.0325 0.975271997551442
0.0375 0.975271997551442
0.0425 0.975271997551442
0.0475 0.975271997551442
0.0525 0.975271997551442
0.0575 0.975271997551442
0.0625 0.975271997551442
0.0675 0.975271997551442
0.0725 0.975271997551442
0.0775 0.975271997551442
0.0825 0.975271997551442
0.0875 0.975271997551442
0.0925 0.975271997551442
0.0975 0.975271997551442
0.1025 0.975271997551442
0.1075 0.975271997551442
0.1125 0.975271997551442
0.1175 0.975271997551442
0.1225 0.975271997551442
0.1275 0.975271997551442
0.1325 0.975271997551442
0.1375 0.975271997551442
0.1425 0.975271997551442
0.1475 0.975271997551442
0.1525 0.975271997551442
0.1575 0.975271997551442
0.1625 0.975271997551442
0.1675 0.975271997551442
0.1725 0.975271997551442
0.1775 0.975271997551442
0.1825 0.975271997551442
0.1875 0.975271997551442
0.1925 0.975271997551442
0.1975 0.975271997551442
0.2025 0.975271997551442
0.2075 0.975271997551442
0.2125 0.975271997551442
0.2175 0.975271997551442
0.2225 0.975271997551442
0.2275 0.975271997551442
0.2325 0.975271997551442
0.2375 0.975271997551442
0.2425 0.975271997551442
0.2475 0.975271997551442
0.2525 0.975271997551442
0.2575 0.975271997551442
0.2625 0.975271997551442
0.2675 0.975271997551442
0.2725 0.975271997551442
0.2775 0.975271997551442
0.2825 0.975271708786336
0.2875 0.975273103351576
0.2925 0.975273260010932
0.2975 0.975273294116564
0.3025 0.975273830866972
0.3075 0.975269472624808
0.3125 0.975268942441419
0.3175 0.975262575181548
0.3225 0.975244515529545
0.3275 0.975219685511442
0.3325 0.975167416095829
0.3375 0.975063422755727
0.3425 0.974878929338046
0.3475 0.974552268136347
0.3525 0.973989893546062
0.3575 0.973066470642708
0.3625 0.971583337353127
0.3675 0.969285850920136
0.3725 0.965851531886018
0.3775 0.960886826437765
0.3825 0.953959686095954
0.3875 0.944591803509279
0.3925 0.932335515084378
0.3975 0.916774241351091
0.4025 0.897585975508623
0.4075 0.874560561862619
0.4125 0.847628385948788
0.4175 0.816861435642097
0.4225 0.783828633560638
0.4275 0.756724094544929
0.4325 0.72756249062352
0.4375 0.697009463673292
0.4425 0.665840944423745
0.4475 0.634763694939587
0.4525 0.604537001735654
0.4575 0.581086561681408
0.4625 0.564537958543868
0.4675 0.548732327810139
0.4725 0.533966945104955
0.4775 0.520505679868967
0.4825 0.508598048605189
0.4875 0.499451522060831
0.4925 0.492900883748927
0.4975 0.488098938523851
0.5025 0.485027230867931
0.5075 0.482804296370767
0.5125 0.478804101033184
0.5175 0.467466163240192
0.5225 0.448261610113506
0.5275 0.42189707957976
0.5325 0.406458525314381
0.5375 0.389607609105759
0.5425 0.367413519530812
0.5475 0.34647091804807
0.5525 0.329103946418212
0.5575 0.316313260132637
0.5625 0.307739278224139
0.5675 0.30212845296013
0.5725 0.29773094619696
0.5775 0.29247439288694
0.5825 0.283796024497446
0.5875 0.268315380084941
0.5925 0.241581481407818
0.5975 0.199022157756669
0.6025 0.140761272645911
0.6075 0.0815676370014747
0.6125 0.0484020940752493
0.6175 0.039030297553408
0.6225 0.037845432472689
0.6275 0.0378070142454444
0.6325 0.0332612882905262
0.6375 0.031708747956456
0.6425 0.0313277619849232
0.6475 0.0312349964318984
0.6525 0.0312143014462262
0.6575 0.0312102315022826
0.6625 0.0312069399566448
0.6675 0.031207593612221
0.6725 0.031207593612221
0.6775 0.031207593612221
0.6825 0.031207593612221
0.6875 0.031207593612221
0.6925 0.031207593612221
0.6975 0.031207593612221
0.7025 0.031207593612221
0.7075 0.031207593612221
0.7125 0.031207593612221
0.7175 0.031207593612221
0.7225 0.031207593612221
0.7275 0.031207593612221
0.7325 0.031207593612221
0.7375 0.031207593612221
0.7425 0.031207593612221
0.7475 0.031207593612221
0.7525 0.031207593612221
0.7575 0.031207593612221
0.7625 0.031207593612221
0.7675 0.031207593612221
0.7725 0.031207593612221
0.7775 0.031207593612221
0.7825 0.031207593612221
0.7875 0.031207593612221
0.7925 0.031207593612221
0.7975 0.031207593612221
0.8025 0.031207593612221
0.8075 0.031207593612221
0.8125 0.031207593612221
0.8175 0.031207593612221
0.8225 0.031207593612221
0.8275 0.031207593612221
0.8325 0.031207593612221
0.8375 0.031207593612221
0.8425 0.031207593612221
0.8475 0.031207593612221
0.8525 0.031207593612221
0.8575 0.031207593612221
0.8625 0.031207593612221
0.8675 0.031207593612221
0.8725 0.031207593612221
0.8775 0.031207593612221
0.8825 0.031207593612221
0.8875 0.031207593612221
0.8925 0.031207593612221
0.8975 0.031207593612221
0.9025 0.031207593612221
0.9075 0.031207593612221
0.9125 0.031207593612221
0.9175 0.031207593612221
0.9225 0.031207593612221
0.9275 0.031207593612221
0.9325 0.031207593612221
0.9375 0.031207593612221
0.9425 0.031207593612221
0.9475 0.031207593612221
0.9525 0.031207593612221
0.9575 0.031207593612221
0.9625 0.031207593612221
0.9675 0.031207593612221
0.9725 0.031207593612221
0.9775 0.031207593612221
0.9825 0.031207593612221
0.9875 0.031207593612221
0.9925 0.031207593612221
0.9975 0.031207593612221
};
\addlegendentry{\(E\)}
\end{groupplot}

\end{tikzpicture}
}
	\caption{Code variables \(c_1\), \(c_2\) and \(c_3\) (dashed lines - -) and macroscopic quantities \(\rho\), \(E\), \(\rho u\) (full lines --) for \(t=0.05s\).}
\end{figure}
\begin{figure}[H]
	\scalebox{1}{% This file was created by tikzplotlib v0.9.6.
\begin{tikzpicture}

\begin{groupplot}[group style={group size=3 by 1,horizontal sep=2cm, vertical sep=2cm}]
\nextgroupplot[
legend cell align={left},
legend style={draw=none},
tick align=outside,
tick pos=left,
x grid style={white!69.0196078431373!black},
xlabel={\(x\)},
xmin=-0.04725, xmax=1.04725,
xtick style={color=black},
y grid style={white!69.0196078431373!black},
ylabel={\(c_0\) and \(\rho\)},
ymin=-0.0601167461548287, ymax=1.05041885939785,
ytick style={color=black},
axis lines=left,
width=0.3\textwidth,
height =.4\textwidth
]
\addplot [semithick, black, dashed]
table {%
0.0025 0.532396674156189
0.0075 0.532396674156189
0.0125 0.532396674156189
0.0175 0.532396674156189
0.0225 0.532396674156189
0.0275 0.532396674156189
0.0325 0.532396674156189
0.0375 0.532396674156189
0.0425 0.532396674156189
0.0475 0.532396674156189
0.0525 0.532396674156189
0.0575 0.532396674156189
0.0625 0.532396674156189
0.0675 0.532396674156189
0.0725 0.532396674156189
0.0775 0.532396674156189
0.0825 0.532396674156189
0.0875 0.532396674156189
0.0925 0.532396674156189
0.0975 0.532396674156189
0.1025 0.532396674156189
0.1075 0.532396674156189
0.1125 0.532396674156189
0.1175 0.532396674156189
0.1225 0.532396674156189
0.1275 0.532396674156189
0.1325 0.532396674156189
0.1375 0.532396674156189
0.1425 0.532396674156189
0.1475 0.532396614551544
0.1525 0.532396614551544
0.1575 0.532396554946899
0.1625 0.532396554946899
0.1675 0.53239643573761
0.1725 0.532396256923676
0.1775 0.532395958900452
0.1825 0.532395362854004
0.1875 0.532394349575043
0.1925 0.532392561435699
0.1975 0.532389581203461
0.2025 0.53238433599472
0.2075 0.532375633716583
0.2125 0.532361447811127
0.2175 0.532338559627533
0.2225 0.53230232000351
0.2275 0.532245934009552
0.2325 0.532159745693207
0.2375 0.532030642032623
0.2425 0.531840622425079
0.2475 0.531566202640533
0.2525 0.531177699565887
0.2575 0.530638039112091
0.2625 0.529902994632721
0.2675 0.528921067714691
0.2725 0.527634024620056
0.2775 0.525978863239288
0.2825 0.523889541625977
0.2875 0.521298944950104
0.2925 0.518142342567444
0.2975 0.514359056949615
0.3025 0.509895980358124
0.3075 0.504708409309387
0.3125 0.498762756586075
0.3175 0.492036581039429
0.3225 0.484519153833389
0.3275 0.476211398839951
0.3325 0.46712526679039
0.3375 0.457282871007919
0.3425 0.446695625782013
0.3475 0.435274511575699
0.3525 0.423205614089966
0.3575 0.410542845726013
0.3625 0.397137641906738
0.3675 0.383145272731781
0.3725 0.368730843067169
0.3775 0.353964537382126
0.3825 0.338918328285217
0.3875 0.323665529489517
0.3925 0.308280169963837
0.3975 0.292836308479309
0.4025 0.277407735586166
0.4075 0.261997550725937
0.4125 0.246466621756554
0.4175 0.231102213263512
0.4225 0.21597096323967
0.4275 0.201139256358147
0.4325 0.186673760414124
0.4375 0.172642007470131
0.4425 0.159113585948944
0.4475 0.146161213517189
0.4525 0.133862689137459
0.4575 0.122302934527397
0.4625 0.11172292381525
0.4675 0.102868385612965
0.4725 0.0950575843453407
0.4775 0.0883814021945
0.4825 0.0829150080680847
0.4875 0.0786925405263901
0.4925 0.0756731256842613
0.4975 0.0737144201993942
0.5025 0.0727607756853104
0.5075 0.0723534971475601
0.5125 0.0722023397684097
0.5175 0.0721774846315384
0.5225 0.0722092762589455
0.5275 0.0722649097442627
0.5325 0.0723256096243858
0.5375 0.0723692774772644
0.5425 0.0723552107810974
0.5475 0.0722099766135216
0.5525 0.0714246407151222
0.5575 0.0701148957014084
0.5625 0.0682081282138824
0.5675 0.0656430199742317
0.5725 0.0624613836407661
0.5775 0.0586692057549953
0.5825 0.0544614866375923
0.5875 0.0504112876951694
0.5925 0.0467777512967587
0.5975 0.0437299720942974
0.6025 0.0413325913250446
0.6075 0.0395610518753529
0.6125 0.0383322313427925
0.6175 0.0375366881489754
0.6225 0.0370632223784924
0.6275 0.036813847720623
0.6325 0.036709651350975
0.6375 0.0366904102265835
0.6425 0.0367102921009064
0.6475 0.0367310643196106
0.6525 0.0367135368287563
0.6575 0.0366066433489323
0.6625 0.0363319627940655
0.6675 0.0357608906924725
0.6725 0.0346800498664379
0.6775 0.0327421240508556
0.6825 0.0294098276644945
0.6875 0.0239496249705553
0.6925 0.015686022117734
0.6975 0.00500450562685728
0.7025 -0.00498133525252342
0.7075 -0.00924072973430157
0.7125 -0.00963785499334335
0.7175 -0.00931992195546627
0.7225 -0.00916687026619911
0.7275 -0.00912009179592133
0.7325 -0.00910733081400394
0.7375 -0.00910396315157413
0.7425 -0.00910309888422489
0.7475 -0.00910286046564579
0.7525 -0.00910280831158161
0.7575 -0.00910280086100101
0.7625 -0.00910278595983982
0.7675 -0.00910278595983982
0.7725 -0.00910279341042042
0.7775 -0.00910279341042042
0.7825 -0.00910279341042042
0.7875 -0.00910279341042042
0.7925 -0.00910279341042042
0.7975 -0.00910279341042042
0.8025 -0.00910279341042042
0.8075 -0.00910279341042042
0.8125 -0.00910279341042042
0.8175 -0.00910279341042042
0.8225 -0.00910279341042042
0.8275 -0.00910279341042042
0.8325 -0.00910279341042042
0.8375 -0.00910279341042042
0.8425 -0.00910279341042042
0.8475 -0.00910279341042042
0.8525 -0.00910279341042042
0.8575 -0.00910279341042042
0.8625 -0.00910279341042042
0.8675 -0.00910279341042042
0.8725 -0.00910279341042042
0.8775 -0.00910279341042042
0.8825 -0.00910279341042042
0.8875 -0.00910279341042042
0.8925 -0.00910279341042042
0.8975 -0.00910279341042042
0.9025 -0.00910279341042042
0.9075 -0.00910279341042042
0.9125 -0.00910279341042042
0.9175 -0.00910279341042042
0.9225 -0.00910279341042042
0.9275 -0.00910279341042042
0.9325 -0.00910279341042042
0.9375 -0.00910279341042042
0.9425 -0.00910279341042042
0.9475 -0.00910279341042042
0.9525 -0.00910279341042042
0.9575 -0.00910279341042042
0.9625 -0.00910279341042042
0.9675 -0.00910279341042042
0.9725 -0.00910279341042042
0.9775 -0.00910279341042042
0.9825 -0.00910279341042042
0.9875 -0.00910279341042042
0.9925 -0.00910279341042042
0.9975 -0.00910279341042042
};
\addlegendentry{\(c_0\)}
\addplot [semithick, black]
table {%
0.0025 0.999939968236364
0.0075 0.999939968236364
0.0125 0.999939968236364
0.0175 0.999939968236364
0.0225 0.999939968236364
0.0275 0.999939968236364
0.0325 0.999939968236364
0.0375 0.999939968236364
0.0425 0.999939968236364
0.0475 0.999939968236364
0.0525 0.999939968236364
0.0575 0.999939968236364
0.0625 0.999939968236364
0.0675 0.999939968236364
0.0725 0.999939968236364
0.0775 0.999939968236364
0.0825 0.999939968236364
0.0875 0.999939968236364
0.0925 0.999939968236364
0.0975 0.999939968236364
0.1025 0.999939968236364
0.1075 0.999939968236364
0.1125 0.999939968236364
0.1175 0.999939968236364
0.1225 0.999939968236364
0.1275 0.999939968236364
0.1325 0.999939968236364
0.1375 0.999939968236364
0.1425 0.999939968236364
0.1475 0.999939891820153
0.1525 0.999939896596166
0.1575 0.999939795822287
0.1625 0.999939859343263
0.1675 0.999939805851915
0.1725 0.999939694093206
0.1775 0.999939501619874
0.1825 0.999939005750303
0.1875 0.999938344153074
0.1925 0.999937119822089
0.1975 0.999935072942231
0.2025 0.999931231236611
0.2075 0.999925225877609
0.2125 0.999915257502252
0.2175 0.999899133323477
0.2225 0.999873507624635
0.2275 0.999833609765539
0.2325 0.99977253900411
0.2375 0.9996809445035
0.2425 0.999546136993628
0.2475 0.999351227775407
0.2525 0.999074761325923
0.2575 0.998690082954291
0.2625 0.99816518680503
0.2675 0.997462545831998
0.2725 0.996539198005429
0.2775 0.995348231890836
0.2825 0.993840270795119
0.2875 0.991963298800282
0.2925 0.989666673092124
0.2975 0.986901124437841
0.3025 0.983621211221012
0.3075 0.979786665202715
0.3125 0.975364012858615
0.3175 0.97032644200879
0.3225 0.964655201189602
0.3275 0.958339407419165
0.3325 0.951376274013175
0.3375 0.943770287319636
0.3425 0.935610410017081
0.3475 0.927371961566118
0.3525 0.918455868840027
0.3575 0.908970900763495
0.3625 0.899218359890466
0.3675 0.889084333768831
0.3725 0.87847134958093
0.3775 0.867419432824812
0.3825 0.855971563320893
0.3875 0.844173005973108
0.3925 0.832070847256825
0.3975 0.819712711665302
0.4025 0.807146266556512
0.4075 0.794695000976133
0.4125 0.78190466747261
0.4175 0.768850565625307
0.4225 0.755560808600141
0.4275 0.742063220853034
0.4325 0.728385018614622
0.4375 0.714554377019596
0.4425 0.700602839963558
0.4475 0.686569614335894
0.4525 0.672506648235214
0.4575 0.658487629814026
0.4625 0.644717437979311
0.4675 0.631765780660013
0.4725 0.619325883423862
0.4775 0.607687344965644
0.4825 0.597212820743712
0.4875 0.588285788201178
0.4925 0.581195238285149
0.4975 0.575985645349973
0.5025 0.572734052936236
0.5075 0.570536559113325
0.5125 0.568678801497206
0.5175 0.566643481142819
0.5225 0.563857749773142
0.5275 0.559670780785382
0.5325 0.553210844190266
0.5375 0.543409478611862
0.5425 0.529145993387852
0.5475 0.50948945315889
0.5525 0.485234439181976
0.5575 0.455965806419651
0.5625 0.42263749986887
0.5675 0.388745872471004
0.5725 0.354970117004063
0.5775 0.32313509641263
0.5825 0.294892554147503
0.5875 0.270800437921515
0.5925 0.25120485562067
0.5975 0.235968408031532
0.6025 0.224628301743322
0.6075 0.216551893390715
0.6125 0.21106138980637
0.6175 0.207518658791788
0.6225 0.20537306029254
0.6275 0.204180987336888
0.6325 0.203605286466579
0.6375 0.203402295637016
0.6425 0.203401903048731
0.6475 0.203486189890939
0.6525 0.203567316445212
0.6575 0.203565239954071
0.6625 0.20338402965512
0.6675 0.202881565126471
0.6725 0.201828048492853
0.6775 0.199844798980615
0.6825 0.19632382867619
0.6875 0.190356519097128
0.6925 0.180826629631412
0.6975 0.16706815120788
0.7025 0.150496277671594
0.7075 0.137438638470112
0.7125 0.129664560111287
0.7175 0.126477104778855
0.7225 0.125452522307825
0.7275 0.125098642893136
0.7325 0.124985097477642
0.7375 0.124955146979445
0.7425 0.12494730631797
0.7475 0.124945280333169
0.7525 0.124944723211229
0.7575 0.124944570378806
0.7625 0.124944525484282
0.7675 0.124944538857119
0.7725 0.124944498738608
0.7775 0.124944498738608
0.7825 0.124944498738608
0.7875 0.124944498738608
0.7925 0.124944498738608
0.7975 0.124944498738608
0.8025 0.124944498738608
0.8075 0.124944498738608
0.8125 0.124944498738608
0.8175 0.124944498738608
0.8225 0.124944498738608
0.8275 0.124944498738608
0.8325 0.124944498738608
0.8375 0.124944498738608
0.8425 0.124944498738608
0.8475 0.124944498738608
0.8525 0.124944498738608
0.8575 0.124944498738608
0.8625 0.124944498738608
0.8675 0.124944498738608
0.8725 0.124944498738608
0.8775 0.124944498738608
0.8825 0.124944498738608
0.8875 0.124944498738608
0.8925 0.124944498738608
0.8975 0.124944498738608
0.9025 0.124944498738608
0.9075 0.124944498738608
0.9125 0.124944498738608
0.9175 0.124944498738608
0.9225 0.124944498738608
0.9275 0.124944498738608
0.9325 0.124944498738608
0.9375 0.124944498738608
0.9425 0.124944498738608
0.9475 0.124944498738608
0.9525 0.124944498738608
0.9575 0.124944498738608
0.9625 0.124944498738608
0.9675 0.124944498738608
0.9725 0.124944498738608
0.9775 0.124944498738608
0.9825 0.124944498738608
0.9875 0.124944498738608
0.9925 0.124944498738608
0.9975 0.124944498738608
};
\addlegendentry{\(\rho\)}

\nextgroupplot[
legend cell align={left},
legend style={draw=none},
tick align=outside,
tick pos=left,
x grid style={white!69.0196078431373!black},
xlabel={\(x\)},
xmin=-0.04725, xmax=1.04725,
xtick style={color=black},
y grid style={white!69.0196078431373!black},
ylabel={\(c_1\) and \(\rho u\)},
ymin=-0.600631769639265, ymax=0.461477020365247,
ytick style={color=black},
axis lines=left,
width=0.3\textwidth,
height =.4\textwidth
]
\addplot [semithick, black, dashed]
table {%
0.0025 -0.432732731103897
0.0075 -0.432732731103897
0.0125 -0.432732731103897
0.0175 -0.432732731103897
0.0225 -0.432732731103897
0.0275 -0.432732731103897
0.0325 -0.432732731103897
0.0375 -0.432732731103897
0.0425 -0.432732731103897
0.0475 -0.432732731103897
0.0525 -0.432732731103897
0.0575 -0.432732731103897
0.0625 -0.432732731103897
0.0675 -0.432732731103897
0.0725 -0.432732731103897
0.0775 -0.432732731103897
0.0825 -0.432732731103897
0.0875 -0.432732731103897
0.0925 -0.432732731103897
0.0975 -0.432732731103897
0.1025 -0.432732731103897
0.1075 -0.432732731103897
0.1125 -0.432732731103897
0.1175 -0.432732731103897
0.1225 -0.432732731103897
0.1275 -0.432732731103897
0.1325 -0.432732731103897
0.1375 -0.432732731103897
0.1425 -0.432732731103897
0.1475 -0.432732731103897
0.1525 -0.432732731103897
0.1575 -0.432732731103897
0.1625 -0.432732790708542
0.1675 -0.432732850313187
0.1725 -0.432732939720154
0.1775 -0.432733118534088
0.1825 -0.432733416557312
0.1875 -0.43273401260376
0.1925 -0.432734996080399
0.1975 -0.432736665010452
0.2025 -0.432739526033401
0.2075 -0.432744324207306
0.2125 -0.432752132415771
0.2175 -0.432764679193497
0.2225 -0.432784557342529
0.2275 -0.432815372943878
0.2325 -0.432862430810928
0.2375 -0.432932764291763
0.2425 -0.433035999536514
0.2475 -0.433184742927551
0.2525 -0.433394700288773
0.2575 -0.433685302734375
0.2625 -0.434079647064209
0.2675 -0.434604167938232
0.2725 -0.435287922620773
0.2775 -0.43616184592247
0.2825 -0.43725711107254
0.2875 -0.438603729009628
0.2925 -0.440228641033173
0.2975 -0.442154169082642
0.3025 -0.444396555423737
0.3075 -0.446964740753174
0.3125 -0.449859857559204
0.3175 -0.453074663877487
0.3225 -0.456594347953796
0.3275 -0.46039617061615
0.3325 -0.464451283216476
0.3375 -0.468724995851517
0.3425 -0.473291903734207
0.3475 -0.478819519281387
0.3525 -0.484451830387115
0.3575 -0.490132510662079
0.3625 -0.496311396360397
0.3675 -0.502679526805878
0.3725 -0.508933007717133
0.3775 -0.515013039112091
0.3825 -0.520862698554993
0.3875 -0.52642685174942
0.3925 -0.531652331352234
0.3975 -0.536488115787506
0.4025 -0.540884792804718
0.4075 -0.544702112674713
0.4125 -0.547621369361877
0.4175 -0.54989105463028
0.4225 -0.551464855670929
0.4275 -0.552299320697784
0.4325 -0.552354097366333
0.4375 -0.55159318447113
0.4425 -0.549987435340881
0.4475 -0.547518193721771
0.4525 -0.544182181358337
0.4575 -0.539999961853027
0.4625 -0.534905731678009
0.4675 -0.528467237949371
0.4725 -0.521527111530304
0.4775 -0.514382779598236
0.4825 -0.50743043422699
0.4875 -0.50112122297287
0.4925 -0.495852291584015
0.4975 -0.49181717634201
0.5025 -0.489068746566772
0.5075 -0.487084209918976
0.5125 -0.485352098941803
0.5175 -0.483432203531265
0.5225 -0.480851918458939
0.5275 -0.477020800113678
0.5325 -0.471169352531433
0.5375 -0.462372064590454
0.5425 -0.449680656194687
0.5475 -0.43235045671463
0.5525 -0.412041693925858
0.5575 -0.388003587722778
0.5625 -0.361093729734421
0.5675 -0.332924395799637
0.5725 -0.305303752422333
0.5775 -0.280411273241043
0.5825 -0.259629011154175
0.5875 -0.242484986782074
0.5925 -0.228977903723717
0.5975 -0.218766763806343
0.6025 -0.211337506771088
0.6075 -0.206131130456924
0.6125 -0.202622607350349
0.6175 -0.200358927249908
0.6225 -0.198971793055534
0.6275 -0.19817641377449
0.6325 -0.197763219475746
0.6375 -0.197586163878441
0.6425 -0.197550982236862
0.6475 -0.197604939341545
0.6525 -0.197729483246803
0.6575 -0.197938069701195
0.6625 -0.198280200362206
0.6675 -0.198854014277458
0.6725 -0.19982998073101
0.6775 -0.201485872268677
0.6825 -0.204242646694183
0.6875 -0.208646968007088
0.6925 -0.215106576681137
0.6975 -0.222911700606346
0.7025 -0.228644728660583
0.7075 -0.228919893503189
0.7125 -0.225593611598015
0.7175 -0.223220273852348
0.7225 -0.222305253148079
0.7275 -0.222030952572823
0.7325 -0.221956059336662
0.7375 -0.221936255693436
0.7425 -0.22193107008934
0.7475 -0.221929714083672
0.7525 -0.221929371356964
0.7575 -0.221929296851158
0.7625 -0.221929267048836
0.7675 -0.221929267048836
0.7725 -0.221929267048836
0.7775 -0.221929267048836
0.7825 -0.221929267048836
0.7875 -0.221929267048836
0.7925 -0.221929267048836
0.7975 -0.221929267048836
0.8025 -0.221929267048836
0.8075 -0.221929267048836
0.8125 -0.221929267048836
0.8175 -0.221929267048836
0.8225 -0.221929267048836
0.8275 -0.221929267048836
0.8325 -0.221929267048836
0.8375 -0.221929267048836
0.8425 -0.221929267048836
0.8475 -0.221929267048836
0.8525 -0.221929267048836
0.8575 -0.221929267048836
0.8625 -0.221929267048836
0.8675 -0.221929267048836
0.8725 -0.221929267048836
0.8775 -0.221929267048836
0.8825 -0.221929267048836
0.8875 -0.221929267048836
0.8925 -0.221929267048836
0.8975 -0.221929267048836
0.9025 -0.221929267048836
0.9075 -0.221929267048836
0.9125 -0.221929267048836
0.9175 -0.221929267048836
0.9225 -0.221929267048836
0.9275 -0.221929267048836
0.9325 -0.221929267048836
0.9375 -0.221929267048836
0.9425 -0.221929267048836
0.9475 -0.221929267048836
0.9525 -0.221929267048836
0.9575 -0.221929267048836
0.9625 -0.221929267048836
0.9675 -0.221929267048836
0.9725 -0.221929267048836
0.9775 -0.221929267048836
0.9825 -0.221929267048836
0.9875 -0.221929267048836
0.9925 -0.221929267048836
0.9975 -0.221929267048836
};
\addlegendentry{\(c_1\)}
\addplot [semithick, black]
table {%
0.0025 -0.000632143126926788
0.0075 -0.000632143126926788
0.0125 -0.000632143126926788
0.0175 -0.000632143126926788
0.0225 -0.000632143126926788
0.0275 -0.000632143126926788
0.0325 -0.000632143126926788
0.0375 -0.000632143126926788
0.0425 -0.000632143126926788
0.0475 -0.000632143126926788
0.0525 -0.000632143126926788
0.0575 -0.000632143126926788
0.0625 -0.000632143126926788
0.0675 -0.000632143126926788
0.0725 -0.000632143126926788
0.0775 -0.000632143126926788
0.0825 -0.000632143126926788
0.0875 -0.000632143126926788
0.0925 -0.000632143126926788
0.0975 -0.000632143126926788
0.1025 -0.000632143126926788
0.1075 -0.000632143126926788
0.1125 -0.000632143126926788
0.1175 -0.000632143126926788
0.1225 -0.000632143126926788
0.1275 -0.000632143126926788
0.1325 -0.000632143126926788
0.1375 -0.000632143126926788
0.1425 -0.000632143126926788
0.1475 -0.000632173375010415
0.1525 -0.000632104061587989
0.1575 -0.000632007194243283
0.1625 -0.00063202397152044
0.1675 -0.000631778068071429
0.1725 -0.000631768516045024
0.1775 -0.00063129115964851
0.1825 -0.000630613241312872
0.1875 -0.000629283825791557
0.1925 -0.000627327558537402
0.1975 -0.000623569540302226
0.2025 -0.000616716053202141
0.2075 -0.00060563643734209
0.2125 -0.000587832776585883
0.2175 -0.000559192954181093
0.2225 -0.000513500835866577
0.2275 -0.000442384529728912
0.2325 -0.000334737570141997
0.2375 -0.000173054613481597
0.2425 6.53383098567214e-05
0.2475 0.000409476114908478
0.2525 0.000895869769376475
0.2575 0.00157195114623076
0.2625 0.00249156578371079
0.2675 0.00371920516779354
0.2725 0.00532691352887426
0.2775 0.00739231173568737
0.2825 0.00999584380123348
0.2875 0.0132203304786222
0.2925 0.0171427676217835
0.2975 0.0218344117863193
0.3025 0.0273575593902926
0.3075 0.0337626894030973
0.3125 0.0410829401617279
0.3175 0.0493410337020757
0.3225 0.0585408233177164
0.3275 0.0686713618046942
0.3325 0.0797101026149877
0.3375 0.091618603067246
0.3425 0.10441242142248
0.3475 0.117959440694993
0.3525 0.129703089887525
0.3575 0.1419404308807
0.3625 0.154997461140499
0.3675 0.168616103466397
0.3725 0.182533790133918
0.3775 0.196672578352668
0.3825 0.210955368206738
0.3875 0.225302622115324
0.3925 0.239637028732087
0.3975 0.253881643912623
0.4025 0.267957338762031
0.4075 0.280961195646218
0.4125 0.293150727063378
0.4175 0.305144263114254
0.4225 0.316881644064995
0.4275 0.328302127814299
0.4325 0.33933957799887
0.4375 0.349926546168251
0.4425 0.359992724938231
0.4475 0.369462272596841
0.4525 0.378256193242776
0.4575 0.386292554491997
0.4625 0.393398801651021
0.4675 0.399102463756175
0.4725 0.40380815175692
0.4775 0.407486924329708
0.4825 0.41015787594471
0.4875 0.411898922695632
0.4925 0.412851790967964
0.4975 0.413199348092314
0.5025 0.413082789122706
0.5075 0.41266932194461
0.5125 0.412017518665487
0.5175 0.411084868389817
0.5225 0.409360494731603
0.5275 0.406620506213134
0.5325 0.402334269979953
0.5375 0.395758255469973
0.5425 0.386095798790533
0.5475 0.372655703278992
0.5525 0.355083416023198
0.5575 0.333510192122156
0.5625 0.308616539214803
0.5675 0.28590732584735
0.5725 0.263293272246809
0.5775 0.240949501315919
0.5825 0.219911543798145
0.5875 0.201477424806764
0.5925 0.186114280988547
0.5975 0.173920234187673
0.6025 0.164698581942979
0.6075 0.158058163201109
0.6125 0.153517864067732
0.6175 0.150590227613131
0.6225 0.148833462910912
0.6275 0.147882022023603
0.6325 0.147451653820631
0.6375 0.147331032974255
0.6425 0.147365106277065
0.6475 0.147433794470336
0.6525 0.147424972133024
0.6575 0.14720636093591
0.6625 0.146584147752742
0.6675 0.145247168924701
0.6725 0.142677445368771
0.6775 0.138024700209155
0.6825 0.129942269169191
0.6875 0.116484360353021
0.6925 0.0954700091989671
0.6975 0.0663669489355933
0.7025 0.0343868951022351
0.7075 0.0125485931000844
0.7125 0.00289424899953278
0.7175 1.92283516177756e-06
0.7225 -0.000752186045089806
0.7275 -0.000643932991058006
0.7325 -0.000525182076312849
0.7375 -0.000493739530160031
0.7425 -0.000485546555049242
0.7475 -0.000483613861706367
0.7525 -0.000483165345081839
0.7575 -0.000482871681500782
0.7625 -0.000482868742415733
0.7675 -0.000482851352829197
0.7725 -0.000482760731040217
0.7775 -0.000482760731040217
0.7825 -0.000482760731040217
0.7875 -0.000482760731040217
0.7925 -0.000482760731040217
0.7975 -0.000482760731040217
0.8025 -0.000482760731040217
0.8075 -0.000482760731040217
0.8125 -0.000482760731040217
0.8175 -0.000482760731040217
0.8225 -0.000482760731040217
0.8275 -0.000482760731040217
0.8325 -0.000482760731040217
0.8375 -0.000482760731040217
0.8425 -0.000482760731040217
0.8475 -0.000482760731040217
0.8525 -0.000482760731040217
0.8575 -0.000482760731040217
0.8625 -0.000482760731040217
0.8675 -0.000482760731040217
0.8725 -0.000482760731040217
0.8775 -0.000482760731040217
0.8825 -0.000482760731040217
0.8875 -0.000482760731040217
0.8925 -0.000482760731040217
0.8975 -0.000482760731040217
0.9025 -0.000482760731040217
0.9075 -0.000482760731040217
0.9125 -0.000482760731040217
0.9175 -0.000482760731040217
0.9225 -0.000482760731040217
0.9275 -0.000482760731040217
0.9325 -0.000482760731040217
0.9375 -0.000482760731040217
0.9425 -0.000482760731040217
0.9475 -0.000482760731040217
0.9525 -0.000482760731040217
0.9575 -0.000482760731040217
0.9625 -0.000482760731040217
0.9675 -0.000482760731040217
0.9725 -0.000482760731040217
0.9775 -0.000482760731040217
0.9825 -0.000482760731040217
0.9875 -0.000482760731040217
0.9925 -0.000482760731040217
0.9975 -0.000482760731040217
};
\addlegendentry{\(\rho u\)}

\nextgroupplot[
legend cell align={left},
legend style={draw=none},
tick align=outside,
tick pos=left,
x grid style={white!69.0196078431373!black},
xlabel={\(x\)},
xmin=-0.04725, xmax=1.04725,
xtick style={color=black},
y grid style={white!69.0196078431373!black},
ylabel={\(c_2\) amd \(E\)},
ymin=-0.385158878988706, ymax=1.04005574282044,
ytick style={color=black},
axis lines=left,
width=0.3\textwidth,
height =.4\textwidth
]
\addplot [semithick, black, dashed]
table {%
0.0025 -0.177741408348083
0.0075 -0.177741408348083
0.0125 -0.177741408348083
0.0175 -0.177741408348083
0.0225 -0.177741408348083
0.0275 -0.177741408348083
0.0325 -0.177741408348083
0.0375 -0.177741408348083
0.0425 -0.177741408348083
0.0475 -0.177741408348083
0.0525 -0.177741408348083
0.0575 -0.177741408348083
0.0625 -0.177741408348083
0.0675 -0.177741408348083
0.0725 -0.177741408348083
0.0775 -0.177741408348083
0.0825 -0.177741408348083
0.0875 -0.177741408348083
0.0925 -0.177741408348083
0.0975 -0.177741408348083
0.1025 -0.177741408348083
0.1075 -0.177741408348083
0.1125 -0.177741408348083
0.1175 -0.177741408348083
0.1225 -0.177741408348083
0.1275 -0.177741408348083
0.1325 -0.177741408348083
0.1375 -0.177741408348083
0.1425 -0.177741408348083
0.1475 -0.177741408348083
0.1525 -0.177741393446922
0.1575 -0.177741378545761
0.1625 -0.177741348743439
0.1675 -0.1777413636446
0.1725 -0.177741289138794
0.1775 -0.177741169929504
0.1825 -0.177741035819054
0.1875 -0.177740678191185
0.1925 -0.177740082144737
0.1975 -0.177739098668098
0.2025 -0.177737444639206
0.2075 -0.177734658122063
0.2125 -0.17773012816906
0.2175 -0.177722796797752
0.2225 -0.177711308002472
0.2275 -0.177693411707878
0.2325 -0.177666246891022
0.2375 -0.177625611424446
0.2425 -0.177566096186638
0.2475 -0.177480578422546
0.2525 -0.177360102534294
0.2575 -0.177193835377693
0.2625 -0.176969036459923
0.2675 -0.17667131125927
0.2725 -0.176285192370415
0.2775 -0.175794810056686
0.2825 -0.175185024738312
0.2875 -0.174442440271378
0.2925 -0.173556804656982
0.2975 -0.172522187232971
0.3025 -0.171337842941284
0.3075 -0.170009329915047
0.3125 -0.168548956513405
0.3175 -0.166975513100624
0.3225 -0.16531465947628
0.3275 -0.163597971200943
0.3325 -0.161862626671791
0.3375 -0.160150215029716
0.3425 -0.158456206321716
0.3475 -0.156514629721642
0.3525 -0.154727399349213
0.3575 -0.153149142861366
0.3625 -0.151801973581314
0.3675 -0.15075595676899
0.3725 -0.150080397725105
0.3775 -0.149827197194099
0.3825 -0.150045812129974
0.3875 -0.150782898068428
0.3925 -0.152081668376923
0.3975 -0.153981238603592
0.4025 -0.156516149640083
0.4075 -0.159861445426941
0.4125 -0.164472028613091
0.4175 -0.169904589653015
0.4225 -0.176169723272324
0.4275 -0.183266013860703
0.4325 -0.191177740693092
0.4375 -0.199871897697449
0.4425 -0.209294691681862
0.4475 -0.219366699457169
0.4525 -0.229977145791054
0.4575 -0.240976229310036
0.4625 -0.252393841743469
0.4675 -0.264954209327698
0.4725 -0.277084320783615
0.4775 -0.288353323936462
0.4825 -0.298288077116013
0.4875 -0.306449711322784
0.4925 -0.312558472156525
0.4975 -0.31661930680275
0.5025 -0.318940103054047
0.5075 -0.320032507181168
0.5125 -0.320376396179199
0.5175 -0.320279002189636
0.5225 -0.319854766130447
0.5275 -0.319063723087311
0.5325 -0.317756950855255
0.5375 -0.315721541643143
0.5425 -0.312739461660385
0.5475 -0.308664947748184
0.5525 -0.302765876054764
0.5575 -0.295719653367996
0.5625 -0.288034588098526
0.5675 -0.280246078968048
0.5725 -0.272880136966705
0.5775 -0.266055703163147
0.5825 -0.259876877069473
0.5875 -0.254898935556412
0.5925 -0.251044660806656
0.5975 -0.248163357377052
0.6025 -0.246080696582794
0.6075 -0.24462828040123
0.6125 -0.243656873703003
0.6175 -0.243040844798088
0.6225 -0.242678299546242
0.6275 -0.242489129304886
0.6325 -0.242411822080612
0.6375 -0.242399305105209
0.6425 -0.242413878440857
0.6475 -0.242420628666878
0.6525 -0.242379456758499
0.6575 -0.24223330616951
0.6625 -0.241891264915466
0.6675 -0.241201236844063
0.6725 -0.239906370639801
0.6775 -0.237575203180313
0.6825 -0.233497560024261
0.6875 -0.226554483175278
0.6925 -0.215164035558701
0.6975 -0.197748988866806
0.7025 -0.174994006752968
0.7075 -0.153826326131821
0.7125 -0.142082512378693
0.7175 -0.138100951910019
0.7225 -0.137038215994835
0.7275 -0.136766105890274
0.7325 -0.136695802211761
0.7375 -0.136677488684654
0.7425 -0.136672720313072
0.7475 -0.136671483516693
0.7525 -0.136671155691147
0.7575 -0.13667106628418
0.7625 -0.13667106628418
0.7675 -0.136671036481857
0.7725 -0.136671036481857
0.7775 -0.136671036481857
0.7825 -0.136671036481857
0.7875 -0.136671036481857
0.7925 -0.136671036481857
0.7975 -0.136671036481857
0.8025 -0.136671036481857
0.8075 -0.136671036481857
0.8125 -0.136671036481857
0.8175 -0.136671036481857
0.8225 -0.136671036481857
0.8275 -0.136671036481857
0.8325 -0.136671036481857
0.8375 -0.136671036481857
0.8425 -0.136671036481857
0.8475 -0.136671036481857
0.8525 -0.136671036481857
0.8575 -0.136671036481857
0.8625 -0.136671036481857
0.8675 -0.136671036481857
0.8725 -0.136671036481857
0.8775 -0.136671036481857
0.8825 -0.136671036481857
0.8875 -0.136671036481857
0.8925 -0.136671036481857
0.8975 -0.136671036481857
0.9025 -0.136671036481857
0.9075 -0.136671036481857
0.9125 -0.136671036481857
0.9175 -0.136671036481857
0.9225 -0.136671036481857
0.9275 -0.136671036481857
0.9325 -0.136671036481857
0.9375 -0.136671036481857
0.9425 -0.136671036481857
0.9475 -0.136671036481857
0.9525 -0.136671036481857
0.9575 -0.136671036481857
0.9625 -0.136671036481857
0.9675 -0.136671036481857
0.9725 -0.136671036481857
0.9775 -0.136671036481857
0.9825 -0.136671036481857
0.9875 -0.136671036481857
0.9925 -0.136671036481857
0.9975 -0.136671036481857
};
\addlegendentry{\(c_2\)}
\addplot [semithick, black]
table {%
0.0025 0.975271997551442
0.0075 0.975271997551442
0.0125 0.975271997551442
0.0175 0.975271997551442
0.0225 0.975271997551442
0.0275 0.975271997551442
0.0325 0.975271997551442
0.0375 0.975271997551442
0.0425 0.975271997551442
0.0475 0.975271997551442
0.0525 0.975271997551442
0.0575 0.975271997551442
0.0625 0.975271997551442
0.0675 0.975271997551442
0.0725 0.975271997551442
0.0775 0.975271997551442
0.0825 0.975271997551442
0.0875 0.975271997551442
0.0925 0.975271997551442
0.0975 0.975271997551442
0.1025 0.975271997551442
0.1075 0.975271997551442
0.1125 0.975271997551442
0.1175 0.975271997551442
0.1225 0.975271997551442
0.1275 0.975271997551442
0.1325 0.975271997551442
0.1375 0.975271997551442
0.1425 0.975271997551442
0.1475 0.975271708786336
0.1525 0.975272345894251
0.1575 0.975273103351576
0.1625 0.975273260010932
0.1675 0.975272412574743
0.1725 0.97527230670645
0.1775 0.975273202300772
0.1825 0.975268970378035
0.1875 0.975267240504522
0.1925 0.975263804729409
0.1975 0.975259757861875
0.2025 0.975245787479831
0.2075 0.975232480833507
0.2125 0.975201277983516
0.2175 0.975155565574799
0.2225 0.975082412268415
0.2275 0.97496418065326
0.2325 0.974788535550117
0.2375 0.974521618308885
0.2425 0.974134176621737
0.2475 0.973571806141579
0.2525 0.972779885457644
0.2575 0.97167609724229
0.2625 0.970175578606417
0.2675 0.968174981828938
0.2725 0.965556486713027
0.2775 0.962188162170118
0.2825 0.957952551255927
0.2875 0.952702259567103
0.2925 0.946329352458147
0.2975 0.938713770210657
0.3025 0.92976104825849
0.3075 0.919395214513698
0.3125 0.907574256447026
0.3175 0.894272008570637
0.3225 0.87948846452417
0.3275 0.863259905697576
0.3325 0.845644010019958
0.3375 0.826716533556512
0.3425 0.806526487257226
0.3475 0.78592180169722
0.3525 0.769855939811628
0.3575 0.753248773797193
0.3625 0.736026455965726
0.3675 0.718395550066959
0.3725 0.700563203113584
0.3775 0.682669806970439
0.3825 0.664834364425736
0.3875 0.647192430597676
0.3925 0.629871662976777
0.3975 0.612988455086655
0.4025 0.596668836356059
0.4075 0.584371643008347
0.4125 0.575277458384687
0.4175 0.566407026206238
0.4225 0.557804660198697
0.4275 0.549505581767932
0.4325 0.541548341644939
0.4375 0.533961068373786
0.4425 0.526777477745141
0.4475 0.520021691847957
0.4525 0.513720220578792
0.4575 0.507894285356309
0.4625 0.502712153720949
0.4675 0.498802386253967
0.4725 0.495425657542786
0.4775 0.492577143898264
0.4825 0.490251562093066
0.4875 0.488427958586226
0.4925 0.487078322863964
0.4975 0.486127402784057
0.5025 0.485625410333697
0.5075 0.485307492902361
0.5125 0.485025667879341
0.5175 0.484664115696879
0.5225 0.483770123030807
0.5275 0.482261706109022
0.5325 0.479816120351499
0.5375 0.47592702610259
0.5425 0.470015472339994
0.5475 0.461493012010696
0.5525 0.448663490545801
0.5575 0.432088647088566
0.5625 0.412086561692047
0.5675 0.401183319139373
0.5725 0.390479542315925
0.5775 0.377266867220081
0.5825 0.361998244598973
0.5875 0.347356304644505
0.5925 0.334243157982305
0.5975 0.323241188539588
0.6025 0.314581149233533
0.6075 0.308180152033538
0.6125 0.303743960941119
0.6175 0.300877062332272
0.6225 0.299185743285398
0.6275 0.298316712773795
0.6325 0.297972671667703
0.6375 0.297937113346249
0.6425 0.298031047055465
0.6475 0.298098991706142
0.6525 0.297969610809626
0.6575 0.297401101943191
0.6625 0.296029213290566
0.6675 0.293220715021089
0.6725 0.287944708909793
0.6775 0.278493335848964
0.6825 0.262217338283881
0.6875 0.23536113753485
0.6925 0.194066197961633
0.6975 0.13866791976028
0.7025 0.0820898085362342
0.7075 0.049327634382726
0.7125 0.0392961122167837
0.7175 0.0378800558053657
0.7225 0.0378259570083111
0.7275 0.0341393846395762
0.7325 0.0320726948273463
0.7375 0.031528125344399
0.7425 0.0313845556091583
0.7475 0.0313482451115482
0.7525 0.0313390442292665
0.7575 0.0313372785739238
0.7625 0.0313353665154072
0.7675 0.0313353054069306
0.7725 0.0313345729624442
0.7775 0.0313345729624442
0.7825 0.0313345729624442
0.7875 0.0313345729624442
0.7925 0.0313345729624442
0.7975 0.0313345729624442
0.8025 0.0313345729624442
0.8075 0.0313345729624442
0.8125 0.0313345729624442
0.8175 0.0313345729624442
0.8225 0.0313345729624442
0.8275 0.0313345729624442
0.8325 0.0313345729624442
0.8375 0.0313345729624442
0.8425 0.0313345729624442
0.8475 0.0313345729624442
0.8525 0.0313345729624442
0.8575 0.0313345729624442
0.8625 0.0313345729624442
0.8675 0.0313345729624442
0.8725 0.0313345729624442
0.8775 0.0313345729624442
0.8825 0.0313345729624442
0.8875 0.0313345729624442
0.8925 0.0313345729624442
0.8975 0.0313345729624442
0.9025 0.0313345729624442
0.9075 0.0313345729624442
0.9125 0.0313345729624442
0.9175 0.0313345729624442
0.9225 0.0313345729624442
0.9275 0.0313345729624442
0.9325 0.0313345729624442
0.9375 0.0313345729624442
0.9425 0.0313345729624442
0.9475 0.0313345729624442
0.9525 0.0313345729624442
0.9575 0.0313345729624442
0.9625 0.0313345729624442
0.9675 0.0313345729624442
0.9725 0.0313345729624442
0.9775 0.0313345729624442
0.9825 0.0313345729624442
0.9875 0.0313345729624442
0.9925 0.0313345729624442
0.9975 0.0313345729624442
};
\addlegendentry{\(E\)}
\end{groupplot}

\end{tikzpicture}
}
	\caption{Code variables \(c_1\), \(c_2\) and \(c_3\) (dashed lines - -) and macroscopic quantities \(\rho\), \(E\), \(\rho u\) (full lines --) for \(t=0.099s\).}
\end{figure}
\begin{figure}[H]
	\scalebox{.6}{% This file was created by tikzplotlib v0.9.6.
\begin{tikzpicture}

\begin{groupplot}[group style={group size=3 by 2,horizontal sep=2cm, vertical sep=2cm}]
\nextgroupplot[
tick align=outside,
tick pos=left,
x grid style={white!69.0196078431373!black},
xlabel={\(v\)},
xmin=-11, xmax=11,
xtick style={color=black},
y grid style={white!69.0196078431373!black},
ylabel={\(\beta_0\)},
ymin=-6.88802815675735, ymax=1.90020395517349,
ytick style={color=black},
axis lines=left
]
\addplot [semithick, black]
table {%
-10 1.50073885917664
-9.48717948717949 1.50073885917664
-8.97435897435897 1.50073885917664
-8.46153846153846 1.50073885917664
-7.94871794871795 1.50073885917664
-7.43589743589744 1.50073885917664
-6.92307692307692 1.50073885917664
-6.41025641025641 1.50073885917664
-5.8974358974359 1.50073885917664
-5.38461538461538 1.50073635578156
-4.87179487179487 1.500697016716
-4.35897435897436 1.50029397010803
-3.84615384615385 1.49704301357269
-3.33333333333333 1.47711193561554
-2.82051282051282 1.38468647003174
-2.30769230769231 1.06730175018311
-1.79487179487179 0.304895222187042
-1.28205128205128 -0.922769010066986
-0.769230769230769 -2.52028822898865
-0.256410256410256 -4.36609077453613
0.256410256410256 -6.18956422805786
0.769230769230769 -6.4885630607605
1.28205128205128 -4.33615875244141
1.79487179487179 -1.32077956199646
2.30769230769231 0.601730704307556
2.82051282051282 1.27828693389893
3.33333333333333 1.45472884178162
3.84615384615385 1.49290776252747
4.35897435897436 1.49966502189636
4.87179487179487 1.50062108039856
5.38461538461538 1.50072908401489
5.8974358974359 1.50073838233948
6.41025641025641 1.50073885917664
6.92307692307692 1.50073885917664
7.43589743589744 1.50073885917664
7.94871794871795 1.50073885917664
8.46153846153846 1.50073885917664
8.97435897435897 1.50073885917664
9.48717948717949 1.50073885917664
10 1.50073885917664
};

\nextgroupplot[
tick align=outside,
tick pos=left,
x grid style={white!69.0196078431373!black},
xlabel={\(v\)},
xmin=-11, xmax=11,
xtick style={color=black},
y grid style={white!69.0196078431373!black},
ylabel={\(\beta_1\)},
ymin=-3.78430556356907, ymax=0.961265376210213,
ytick style={color=black},
axis lines=left
]
\addplot [semithick, black]
table {%
-10 0.745557606220245
-9.48717948717949 0.745557606220245
-8.97435897435897 0.745557606220245
-8.46153846153846 0.745557606220245
-7.94871794871795 0.745557606220245
-7.43589743589744 0.745557606220245
-6.92307692307692 0.745557606220245
-6.41025641025641 0.745557606220245
-5.8974358974359 0.745557606220245
-5.38461538461538 0.745556771755219
-4.87179487179487 0.745545029640198
-4.35897435897436 0.745418906211853
-3.84615384615385 0.744408249855042
-3.33333333333333 0.738228678703308
-2.82051282051282 0.709602534770966
-2.30769230769231 0.610597670078278
-1.79487179487179 0.365436613559723
-1.28205128205128 -0.159861743450165
-0.769230769230769 -1.51125168800354
-0.256410256410256 -3.5189425945282
0.256410256410256 -3.5685977935791
0.769230769230769 -1.69052612781525
1.28205128205128 -0.361645400524139
1.79487179487179 0.245922803878784
2.30769230769231 0.561992526054382
2.82051282051282 0.694838762283325
3.33333333333333 0.734742224216461
3.84615384615385 0.743755459785461
4.35897435897436 0.745321333408356
4.87179487179487 0.745533525943756
5.38461538461538 0.745555460453033
5.8974358974359 0.745557606220245
6.41025641025641 0.745557546615601
6.92307692307692 0.745557606220245
7.43589743589744 0.745557606220245
7.94871794871795 0.745557606220245
8.46153846153846 0.745557606220245
8.97435897435897 0.745557606220245
9.48717948717949 0.745557606220245
10 0.745557606220245
};

\nextgroupplot[
tick align=outside,
tick pos=left,
x grid style={white!69.0196078431373!black},
xlabel={\(v\)},
xmin=-11, xmax=11,
xtick style={color=black},
y grid style={white!69.0196078431373!black},
ylabel={\(\beta_2\)},
ymin=-0.610432699322701, ymax=4.70868660509586,
ytick style={color=black},
axis lines=left
]
\addplot [semithick, black]
table {%
-10 -0.368653476238251
-9.48717948717949 -0.368653476238251
-8.97435897435897 -0.368653476238251
-8.46153846153846 -0.368653476238251
-7.94871794871795 -0.368653476238251
-7.43589743589744 -0.368653476238251
-6.92307692307692 -0.368653476238251
-6.41025641025641 -0.368653476238251
-5.8974358974359 -0.368653357028961
-5.38461538461538 -0.368648827075958
-4.87179487179487 -0.368583858013153
-4.35897435897436 -0.367914021015167
-3.84615384615385 -0.362561643123627
-3.33333333333333 -0.330129861831665
-2.82051282051282 -0.182376563549042
-2.30769230769231 0.309535503387451
-1.79487179487179 1.43555212020874
-1.28205128205128 3.11136317253113
-0.769230769230769 4.46307420730591
-0.256410256410256 4.46690845489502
0.256410256410256 3.28958964347839
0.769230769230769 1.9076247215271
1.28205128205128 0.956885397434235
1.79487179487179 0.434787750244141
2.30769230769231 0.0257612727582455
2.82051282051282 -0.250134825706482
3.33333333333333 -0.345528900623322
3.84615384615385 -0.365595459938049
4.35897435897436 -0.368394196033478
4.87179487179487 -0.368643701076508
5.38461538461538 -0.368654549121857
5.8974358974359 -0.368653655052185
6.41025641025641 -0.368653476238251
6.92307692307692 -0.368653476238251
7.43589743589744 -0.368653476238251
7.94871794871795 -0.368653476238251
8.46153846153846 -0.368653476238251
8.97435897435897 -0.368653476238251
9.48717948717949 -0.368653476238251
10 -0.368653476238251
};

\nextgroupplot[
tick align=outside,
tick pos=left,
x grid style={white!69.0196078431373!black},
xlabel={\(v\)},
xmin=-11, xmax=11,
xtick style={color=black},
y grid style={white!69.0196078431373!black},
ylabel={\(\gamma_0\)},
ymin=-0.568114146636858, ymax=0.0270530546017552,
ytick style={color=black},
axis lines=left
]
\addplot [semithick, black]
table {%
-10 1.11022302462516e-16
-9.48717948717949 -1.49756037906857e-20
-8.97435897435897 -1.27810118294275e-18
-8.46153846153846 -1.12744683074863e-16
-7.94871794871795 -7.61031278983007e-15
-7.43589743589744 -3.95050479625332e-13
-6.92307692307692 -1.57714583487996e-11
-6.41025641025641 -4.84275102585206e-10
-5.8974358974359 -1.14380972444433e-08
-5.38461538461538 -2.07830187863026e-07
-4.87179487179487 -2.90552413910108e-06
-4.35897435897436 -3.12601969992907e-05
-3.84615384615385 -0.000258900668414548
-3.33333333333333 -0.00165127621583845
-2.82051282051282 -0.00811512031118912
-2.30769230769231 -0.0307554404646581
-1.79487179487179 -0.0900252664219377
-1.28205128205128 -0.204408297707841
-0.769230769230769 -0.363493236763491
-0.256410256410256 -0.506703365379294
0.256410256410256 -0.541061092035103
0.769230769230769 -0.433340373945296
1.28205128205128 -0.256535943244692
1.79487179487179 -0.111602690436555
2.30769230769231 -0.0365475607273425
2.82051282051282 -0.00929664348004875
3.33333333333333 -0.00185629538558574
3.84615384615385 -0.000289820815853585
4.35897435897436 -3.5225241343174e-05
4.87179487179487 -3.32774351739608e-06
5.38461538461538 -2.4444710128489e-07
5.8974358974359 -1.39837998972549e-08
6.41025641025641 -6.24428204040681e-10
6.92307692307692 -2.18275076466274e-11
7.43589743589744 -5.9915325787844e-13
7.94871794871795 -1.2952475837196e-14
8.46153846153846 -2.21019702020631e-16
8.97435897435897 -2.9804272176168e-18
9.48717948717949 -3.1755125641467e-20
10 -2.66916408904971e-22
};

\nextgroupplot[
tick align=outside,
tick pos=left,
x grid style={white!69.0196078431373!black},
xlabel={\(v\)},
xmin=-11, xmax=11,
xtick style={color=black},
y grid style={white!69.0196078431373!black},
ylabel={\(\gamma_1\)},
ymin=-0.534209545581082, ymax=0.600709312993124,
ytick style={color=black},
axis lines=left
]
\addplot [semithick, black]
table {%
-10 -3.46944695195361e-16
-9.48717948717949 1.1102130508315e-16
-8.97435897435897 -6.00279985198224e-18
-8.46153846153846 -2.43089976887143e-16
-7.94871794871795 -1.63779640628311e-14
-7.43589743589744 -8.49092822076895e-13
-6.92307692307692 -3.38441280150103e-11
-6.41025641025641 -1.03712137046538e-09
-5.8974358974359 -2.44324195657029e-08
-5.38461538461538 -4.42430689096351e-07
-4.87179487179487 -6.15715736041038e-06
-4.35897435897436 -6.58295067224784e-05
-3.84615384615385 -0.000540376231108169
-3.33333333333333 -0.00340181835101554
-2.82051282051282 -0.0163877027864573
-2.30769230769231 -0.0601457264406097
-1.79487179487179 -0.166593776856988
-1.28205128205128 -0.340778444520313
-0.769230769230769 -0.4826223247368
-0.256410256410256 -0.350573153470886
0.256410256410256 0.16305672323786
0.769230769230769 0.549122092148842
1.28205128205128 0.395734276311465
1.79487179487179 0.107797605642259
2.30769230769231 0.00113955872203916
2.82051282051282 -0.00628828145589854
3.33333333333333 -0.00183037661651124
3.84615384615385 -0.000293980239536051
4.35897435897436 -3.06998183837196e-05
4.87179487179487 -1.99051118798304e-06
5.38461538461538 -4.83099134786566e-08
5.8974358974359 4.76124359665474e-09
6.41025641025641 6.42627485384193e-10
6.92307692307692 4.09436697221996e-11
7.43589743589744 1.72243753098337e-12
7.94871794871795 5.18312293205685e-14
8.46153846153846 1.15181996987467e-15
8.97435897435897 1.92044935175516e-17
9.48717948717949 2.42388524543346e-19
10 2.32839870106148e-21
};

\nextgroupplot[
tick align=outside,
tick pos=left,
x grid style={white!69.0196078431373!black},
xlabel={\(v\)},
xmin=-11, xmax=11,
xtick style={color=black},
y grid style={white!69.0196078431373!black},
ylabel={\(\gamma_2\)},
ymin=-0.502335108871341, ymax=0.505441492195811,
ytick style={color=black},
axis lines=left
]
\addplot [semithick, black]
table {%
-10 -1.27675647831893e-15
-9.48717948717949 -2.93023219381888e-22
-8.97435897435897 2.20235372015017e-16
-8.46153846153846 -4.11302163870099e-16
-7.94871794871795 -2.73855527116848e-14
-7.43589743589744 -1.41767600983015e-12
-6.92307692307692 -5.64079853813417e-11
-6.41025641025641 -1.724830854549e-09
-5.8974358974359 -4.05240621292349e-08
-5.38461538461538 -7.31333772813112e-07
-4.87179487179487 -1.01335627152193e-05
-4.35897435897436 -0.000107733076686539
-3.84615384615385 -0.000877781684076761
-3.33333333333333 -0.00547097471652983
-2.82051282051282 -0.0260021653540801
-2.30769230769231 -0.0936925974540094
-1.79487179487179 -0.251576815346058
-1.28205128205128 -0.456527081550107
-0.769230769230769 -0.317891765968506
-0.256410256410256 0.401768320918203
0.256410256410256 0.459633464874577
0.769230769230769 -0.213550725589106
1.28205128205128 -0.380666111280626
1.79487179487179 -0.215754658190286
2.30769230769231 -0.0852674972409338
2.82051282051282 -0.0264365709515214
3.33333333333333 -0.00630008262332984
3.84615384615385 -0.00113692600891485
4.35897435897436 -0.00015652926761505
4.87179487179487 -1.66465605259822e-05
5.38461538461538 -1.38045539282732e-06
5.8974358974359 -8.97854083246136e-08
6.41025641025641 -4.59329613409237e-09
6.92307692307692 -1.84947282959922e-10
7.43589743589744 -5.85480723152104e-12
7.94871794871795 -1.45400512715473e-13
8.46153846153846 -2.82487700068086e-15
8.97435897435897 -4.28096738270601e-17
9.48717948717949 -5.04652899526609e-19
10 -4.61634328769642e-21
};
\end{groupplot}

\end{tikzpicture}
}
	\caption{Comparison of the intrinsic variables generated by POD \(\gamma_1\), \(\gamma_2\) and \(\gamma_3\) with the intrinsic variables of the convolutional autoencoder \(\beta_1\), \(\beta_2\) and \(\beta_3\).}
\end{figure}
\begin{figure}[H]
	\scalebox{0.9}{% This file was created by tikzplotlib v0.9.6.
\begin{tikzpicture}

\begin{groupplot}[group style={group size=1 by 5,horizontal sep=2cm, vertical sep=2cm}]
\nextgroupplot[
colorbar,
colorbar style={ylabel={}},
colormap/blackwhite,
point meta max=-0.0167456492781639,
point meta min=-0.31139874458313,
tick align=outside,
tick pos=left,
x grid style={white!69.0196078431373!black},
xlabel={x},
xmin=0.0025, xmax=0.9975,
xtick style={color=black},
y grid style={white!69.0196078431373!black},
ylabel={t},
ymin=0, ymax=0.12,
ytick style={color=black},
ytick={0,0.06,0.12},
width=\textwidth,
height=.25\textwidth
]
\addplot graphics [includegraphics cmd=\pgfimage,xmin=0.0025, xmax=0.9975, ymin=0, ymax=0.12] {Figures/Results/Rare/Code-000.png};
\node [draw,fill=white] at (0.9,0.1) {\(c_1\)};

\nextgroupplot[
colorbar,
colorbar style={ylabel={}},
colormap/blackwhite,
point meta max=0.188924923539162,
point meta min=-0.245918944478035,
tick align=outside,
tick pos=left,
x grid style={white!69.0196078431373!black},
xlabel={x},
xmin=0.0025, xmax=0.9975,
xtick style={color=black},
y grid style={white!69.0196078431373!black},
ylabel={t},
ymin=0, ymax=0.12,
ytick style={color=black},
ytick={0,0.06,0.12},
width=\textwidth,
height=.25\textwidth
]
\addplot graphics [includegraphics cmd=\pgfimage,xmin=0.0025, xmax=0.9975, ymin=0, ymax=0.12] {Figures/Results/Rare/Code-001.png};
\node [draw,fill=white] at (0.9,0.1) {\(c_2\)};

\nextgroupplot[
colorbar,
colorbar style={ylabel={}},
colormap/blackwhite,
point meta max=0.397124499082565,
point meta min=0.0549342148005962,
tick align=outside,
tick pos=left,
x grid style={white!69.0196078431373!black},
xlabel={x},
xmin=0.0025, xmax=0.9975,
xtick style={color=black},
y grid style={white!69.0196078431373!black},
ylabel={t},
ymin=0, ymax=0.12,
ytick style={color=black},
ytick={0,0.06,0.12},
width=\textwidth,
height=.25\textwidth
]
\addplot graphics [includegraphics cmd=\pgfimage,xmin=0.0025, xmax=0.9975, ymin=0, ymax=0.12] {Figures/Results/Rare/Code-002.png};
\node [draw,fill=white] at (0.9,0.1) {\(c_3\)};

\nextgroupplot[
colorbar,
colorbar style={ylabel={}},
colormap/blackwhite,
point meta max=0.00300436303950846,
point meta min=-0.165321439504623,
tick align=outside,
tick pos=left,
x grid style={white!69.0196078431373!black},
xlabel={x},
xmin=0.0025, xmax=0.9975,
xtick style={color=black},
y grid style={white!69.0196078431373!black},
ylabel={t},
ymin=0, ymax=0.12,
ytick style={color=black},
ytick={0,0.06,0.12},
width=\textwidth,
height=.25\textwidth
]
\addplot graphics [includegraphics cmd=\pgfimage,xmin=0.0025, xmax=0.9975, ymin=0, ymax=0.12] {Figures/Results/Rare/Code-003.png};
\node [draw,fill=white] at (0.9,0.1) {\(c_4\)};

\nextgroupplot[
colorbar,
colorbar style={ylabel={}},
colormap/blackwhite,
point meta max=-0.157852962613106,
point meta min=-0.423109173774719,
tick align=outside,
tick pos=left,
x grid style={white!69.0196078431373!black},
xlabel={x},
xmin=0.0025, xmax=0.9975,
xtick style={color=black},
y grid style={white!69.0196078431373!black},
ylabel={t},
ymin=0, ymax=0.12,
ytick style={color=black},
ytick={0,0.06,0.12},
width=\textwidth,
height=.25\textwidth
]
\addplot graphics [includegraphics cmd=\pgfimage,xmin=0.0025, xmax=0.9975, ymin=0, ymax=0.12] {Figures/Results/Rare/Code-004.png};
\node [draw,fill=white] at (0.9,0.1) {\(c_5\)};
\end{groupplot}

\end{tikzpicture}
}
	\caption{Intrinsic Variables of $\rare$}
\end{figure}
\subsection{Discussion and Outlook}
 One reason of the lacking ability of the CNN is the small number of samples, as described in \cref{Ch:ApB} and \cref{Ch:ROM}. The resulting trembles of the signal when calculating the macroscopic quantities is due the kernel approach of the CNN. During reconstruction the resolution of the output image is bounded by the size of the kernel, which leads to pixelation.\\
  What we see here is the known drawback of POD. Sharp fronts and especially advection dominated problems lead to a fast decaying kogolomorov n-width. These problems need a nonlienar ansatz, as described in \cref{Ch:ROM}.