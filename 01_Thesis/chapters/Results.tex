% !TEX root = master.tex

\chapter{Results}
\label{Ch:Results}
%\pagenumbering{arabic}
This chapter covers the evaluation of reconstructions \(\tilde{f}\) obtained from the FCNN and the CNN through a comparison against reconstructions obtained from POD. Additionally, an analysis of the interpretability of the intrinsic variables \(\idhy\) and \(\idrare\) is provided. This chapter concludes with the attempt to create new states of the FOM with the FCNN and the CNN, which can be viewed as an online phase of MOR.\\ 
To begin with the benchmarking of POD and neural networks, the number of parameters to obtain \(\tilde{f}\) is contrasted. Beforehand solely the number of trainable parameters that compose both neural networks were called \(\frepar\). For this comparison, \(\frepar\) is extended to also include all elements of the left and right singular vectors as well as the singular values of POD. Additionally the amount of intrinsic variables used for reconstruction is set to \(p=3\) and \(p=5\) for \(\hy\) and \(\rare\) respectively. An exception is the CNN, that uses \(p=5\) independent from rarefaction level. A summary is provided in \cref{Tab: Parameters}.   
\begin{table}[htp]
	\centering
	\caption{Amount of parameters \(\frepar\) used to reconstruct \(f\), number  of intrinsic variables \(p\) and the corresponding $\L2$-Error for POD, FCNN and CNN.}
	\begin{tabular*}{16cm}{ @{\extracolsep{\fill}} c c c c c c c @{} }
		\toprule
		Algorithm & \multicolumn{2}{c}{Parameters \(\frepar\)} & \multicolumn{2}{c}{Int. variables \(p\)}& \multicolumn{2}{c}{$\L2$-Error} \\ [.5ex]
		 & \(\hy\)&\(\rare\)&\(\hy\)&\(\rare\)&\(\hy\)&\(\rare\)\\   
		\hline
		POD     & 15129 & 25225 & 3 & 5 & 0.0205 & 0.0087 \\
		FCNN 	& 2683 & 3725 & 3 & 5 & 0.0008 & 0.0009 \\
		CNN   	& 8246 & 8246 & 5 & 5 &	0.025 & 0.027\\
		\bottomrule
	\end{tabular*} \label{Tab: Parameters}
\end{table}
POD uses with 15129 and 25225 parameters to reconstruct \(\hy\) and \(\rare\) respectively the largest amount of parameters of all three algorithms. These yield \(\L2\)-errors of 0.0205 and 0.0087 respectively. Interestingly, the elevation of \(p\) amounts to an increase of paramters by approximately 1.7 which is comparable to the FCNN with an approximate increase of 1.4. The FCNN, which holds the best \(\L2\)-errors of 0.0008 and 0.0009 for \(\hy\) and \(\rare\) respectively, does so with the least amount of parameters. For reconstructing \(\hy\) solely 2683 and for the reconstruction of \(\rare\) solely 3725 parameters are used, which is a fraction of the need for POD. The second most populous alhorithm is the CNN which uses 8246 parameters for both rarefaction levels. The resulting \(\L2\)-errors with 0.025 for \(\hy\) and 0.027 for \(\rare\) are the largest of all three algorithms. \\

Next a qualitative analysis with actual reconstructions is presented. For this purpose the \(\L2\)-error over time \(t\), seen in \cref{Fig:ErrTime}, is used to localize the most challenging snapshot for each algorithm.\\ 
\begin{figure}[tp!]
	% This file was created by tikzplotlib v0.9.6.
\begin{tikzpicture}
\definecolor{color0}{rgb}{0.12156862745098,0.466666666666667,0.705882352941177}

\begin{groupplot}[group style={group size=2 by 1,horizontal sep=2cm}]
\nextgroupplot[
legend cell align={left},
legend style={draw=none, at={(0,1)},anchor= north west},
tick align=outside,
tick pos=left,
x grid style={white!69.0196078431373!black},
xmin=-0.006, xmax=0.126,
xtick style={color=black},
y grid style={white!69.0196078431373!black},
ymin=-0.000841513772702749, ymax=0.0304345512636816,
ytick style={color=black},
ylabel={Relative Error},
xlabel={\(t\)},
axis lines=left,
width=0.47\textwidth,
height =.45\textwidth,
x tick label style={/pgf/number format/fixed}
]
\addplot [semithick, red, mark=o, mark size=2, mark options={solid}]
table {%
0 0.0041745608742312
0.005 0.00772826090461208
0.01 0.00998129651030254
0.015 0.0114814636830063
0.02 0.0127128633875819
0.025 0.0138212914705674
0.03 0.0148590810264679
0.035 0.0158480900427887
0.04 0.0167989301972853
0.045 0.0177175305866807
0.05 0.0186076563966316
0.055 0.0194719780627452
0.06 0.0203125628459335
0.065 0.021131115773516
0.07 0.0219291044243208
0.075 0.0227078270777979
0.08 0.0234684520909674
0.085 0.0242120420959511
0.09 0.0249395698621951
0.095 0.0256519293912289
0.1 0.0263499442110898
0.105 0.0270343740437529
0.11 0.0277059205812353
0.115 0.0283652327959761
0.12 0.029012911943846
};
\addlegendentry{POD}
\addplot [semithick, color0, mark=pentagon, mark size=2, mark options={solid}]
table {%
0 0.000580125547132904
0.005 0.0016071267939448
0.01 0.00188504649223781
0.015 0.00133036247337402
0.02 0.00132628189845874
0.025 0.00132716953321487
0.03 0.00138970165580601
0.035 0.00142017716840674
0.04 0.00145638808517116
0.045 0.00148681063630736
0.05 0.00153865172400739
0.055 0.0015731085384213
0.06 0.00160068094345799
0.065 0.00164077357612924
0.07 0.00168465107889107
0.075 0.00170978225630885
0.08 0.00174559426233016
0.085 0.00178917826233095
0.09 0.00182042226136015
0.095 0.0018497979559777
0.1 0.00189400391034215
0.105 0.00192810267450896
0.11 0.00196043475749194
0.115 0.00199731114052969
0.12 0.00204309455217968
};
\addlegendentry{FCNN}
\addplot [semithick, green!50!black, mark=triangle, mark size=2, mark options={solid,rotate=180}]
table {%
0 0.00649221194908023
0.005 0.00637732213363051
0.01 0.00686583481729031
0.015 0.00666404515504837
0.02 0.00614608032628894
0.025 0.0067147696390748
0.03 0.00703709293156862
0.035 0.00886726658791304
0.04 0.00913859903812408
0.045 0.00808194372802973
0.05 0.00891359616070986
0.055 0.00899100676178932
0.06 0.00881004240363836
0.065 0.009485456161201
0.07 0.00957572367042303
0.075 0.00939757097512484
0.08 0.0101738022640347
0.085 0.0107985194772482
0.09 0.0102939195930958
0.095 0.0112069239839911
0.1 0.0109974481165409
0.105 0.0116838105022907
0.11 0.0127648552879691
0.115 0.0127696730196476
0.12 0.0131859742105007
};
\addlegendentry{CNN}

\nextgroupplot[
legend cell align={left},
legend style={draw=none,at={(0,1)},anchor= north west},
tick align=outside,
tick pos=left,
x grid style={white!69.0196078431373!black},
xmin=-0.006, xmax=0.126,
xtick style={color=black},
y grid style={white!69.0196078431373!black},
ymin=-0.00230221234608083, ymax=0.0611092213046212,
ytick style={color=black},
ylabel={Relative Error},
xlabel={\(t\)},
axis lines=left,
width=0.47\textwidth,
height =.45\textwidth,
x tick label style={/pgf/number format/fixed}
]
\addplot [semithick, red, mark=o, mark size=2, mark options={solid}]
table {%
0 0.00740144721634158
0.005 0.0106893808997395
0.01 0.0149150434602834
0.015 0.0187372384735488
0.02 0.0219134756297143
0.025 0.0245262733618508
0.03 0.0266961103723662
0.035 0.0285235128498648
0.04 0.0300844182519668
0.045 0.0314352845718027
0.05 0.0326184730907696
0.055 0.0336663066570384
0.06 0.0346038953801234
0.065 0.0354510718869172
0.07 0.0362237260413102
0.075 0.0369347412336944
0.08 0.0375946650603757
0.085 0.038212200327165
0.09 0.0387945721638091
0.095 0.0393478079345977
0.1 0.0398769544960087
0.105 0.0403862495460136
0.11 0.0408792586959919
0.115 0.0413589864772983
0.12 0.0418279671630029
};
\addlegendentry{POD}
\addplot [semithick, color0, mark=pentagon, mark size=2, mark options={solid}]
table {%
0 0.000580125547132904
0.005 0.00316589483113868
0.01 0.00644069989189273
0.015 0.0094435622115595
0.02 0.0117462910277023
0.025 0.0135083335876847
0.03 0.0147541401649302
0.035 0.0156776008463723
0.04 0.016359257000976
0.045 0.0168387643606902
0.05 0.0171780580682302
0.055 0.0174023878750501
0.06 0.0175410884132009
0.065 0.0176139846919089
0.07 0.0176235927054341
0.075 0.0175937947736521
0.08 0.0175227229486254
0.085 0.0174309194625182
0.09 0.0173139350118248
0.095 0.0171816191398723
0.1 0.0170365979594161
0.105 0.0168830016501064
0.11 0.0167221999919594
0.115 0.0165578630537945
0.12 0.0163915040253931
};
\addlegendentry{FCNN}
\addplot [semithick, green!50!black, mark=triangle, mark size=2, mark options={solid,rotate=180}]
table {%
0 0.00956460367888212
0.005 0.0124108670279384
0.01 0.0172932539135218
0.015 0.0221939664334059
0.02 0.0249469373375177
0.025 0.0283007752150297
0.03 0.0311273168772459
0.035 0.0343553461134434
0.04 0.036633376032114
0.045 0.0377688743174076
0.05 0.0399965308606625
0.055 0.041291106492281
0.06 0.0428216233849525
0.065 0.0447116009891033
0.07 0.0460423566401005
0.075 0.0471654385328293
0.08 0.048737995326519
0.085 0.049955528229475
0.09 0.0508080013096333
0.095 0.0526490993797779
0.1 0.053377740085125
0.105 0.0546315461397171
0.11 0.0561474151909351
0.115 0.0573081187903881
0.12 0.0582268834114075
};
\addlegendentry{CNN}
\end{groupplot}

\end{tikzpicture}

	\caption{\(\L2\)-error over time for POD, FCNN and CNN. Results for $\hy$ are displayed on the left, the results for $\rare$ are displayed on the right.}
	\label{Fig:ErrTime}
\end{figure}
With POD and the CNN the last timestep at $t=0.12s$ for both rarefaction levels is the most rich in the $L2$-Error. In contrast, the FCNN does not show a distinct time dependence of the \(\L2\)-error. Nonetheless, struggles at the onset at around $t=0.005s$ with $\hy$ and around \(t=0.005s\) and \(t=0.0115\) (in the beginning and at the end) with $\rare$ can be observed for the FCNN. Examples of reconstructions $\tilde{f}(x,v,t_i)$ with $t_i=0.12s$ and $x \in [0.375,0.75]$ are given in \cref{Fig: ErrWorst}.\\
\begin{figure}[tp!]
	% This file was created by tikzplotlib v0.9.6.
\begin{tikzpicture}

\begin{groupplot}[
group style={group size=4 by 2,
	horizontal sep= 1.1cm,
	vertical sep = 1.5cm},
tick align=outside,
tick pos=left,
x grid style={white!69.0196078431373!black},
xmin=0.375, xmax=0.75,
xtick style={color=black},
y grid style={white!69.0196078431373!black},
ymin=-10, ymax=10,
ytick style={color=black},
height=.26\textwidth,
width=.26\textwidth,
xlabel={\(x\)},
ylabel={\(v\)},
y label style={yshift=-1.4em}
]
\nextgroupplot[
]
\addplot graphics [
includegraphics cmd=\pgfimage,
xmin=0.375, xmax=0.75,
ymin=-10, ymax=10
] {Figures/Chapter_5/ErrWorst-000.png};
\node[fill=white] at (axis cs:0.65,15) {FOM};
\nextgroupplot[
]
\addplot graphics [
includegraphics cmd=\pgfimage,
xmin=0.375, xmax=0.75,
ymin=-10, ymax=10
] {Figures/Chapter_5/ErrWorst-001.png};
\node[fill=white] at (axis cs:0.65,15) {POD};
\nextgroupplot[
]
\addplot graphics [
includegraphics cmd=\pgfimage,
xmin=0.375, xmax=0.75,
ymin=-10, ymax=10
] {Figures/Chapter_5/ErrWorst-002.png};
\node[fill=white] at (axis cs:0.65,15) {FCNN};
\nextgroupplot[
colorbar,
colorbar style={
	ylabel={$\tilde{f}$},
	ytick={0,0.1,.3,.39},
	yticklabels={0,0.1,0.3,0.39},
	y label style={yshift=1.3cm},
	ticklabel style={font=\footnotesize},
	tick align=outside,
	tick pos=right,
	width=0.1*\pgfkeysvalueof{/pgfplots/parent axis width},
	xshift=-0.2cm
},
colormap/blackwhite,
point meta max=0.397007430832041,
point meta min=8.50982895819395e-73,
]
\addplot graphics [
includegraphics cmd=\pgfimage,
xmin=0.375, xmax=0.75,
ymin=-10, ymax=10
] {Figures/Chapter_5/ErrWorst-003.png};
\node[fill=white] at (axis cs:0.65,15) {CNN};
\nextgroupplot[
]
\addplot graphics [
includegraphics cmd=\pgfimage,
xmin=0.375, xmax=0.75,
ymin=-10, ymax=10
] {Figures/Chapter_5/ErrWorst-004.png};
\node[fill=white] at (axis cs:0.65,15) {FOM};
\nextgroupplot[
]
\addplot graphics [
includegraphics cmd=\pgfimage,
xmin=0.375, xmax=0.75,
ymin=-10, ymax=10
] {Figures/Chapter_5/ErrWorst-005.png};
\node[fill=white] at (axis cs:0.65,15) {POD};
\nextgroupplot[
]
\addplot graphics [
includegraphics cmd=\pgfimage,
xmin=0.375, xmax=0.75,
ymin=-10, ymax=10
] {Figures/Chapter_5/ErrWorst-006.png};
\node[fill=white] at (axis cs:0.65,15) {FCNN};
\nextgroupplot[
colorbar,
colorbar style={
	ylabel={$\tilde{f}$},
	ytick={0,0.1,.3,.4},
	yticklabels={0,0.1,0.3,0.4},
	y label style={yshift=1.3cm},
	ticklabel style={font=\footnotesize},
	tick align=outside,
	tick pos=right,
	width=0.1*\pgfkeysvalueof{/pgfplots/parent axis width},
	xshift=-0.2cm
},
colormap/blackwhite,
point meta max=0.406101604565777,
point meta min=1.26406789996295e-34
]
\addplot graphics [
includegraphics cmd=\pgfimage,
xmin=0.375, xmax=0.75,
ymin=-10, ymax=10
] {Figures/Chapter_5/ErrWorst-007.png};
\node[fill=white] at (axis cs:0.65,15) {CNN};
\end{groupplot}

\end{tikzpicture}

	\caption{Comparison of the FOM solution \(f\) with three reconstructions \(\tilde{f}\) obtained from POD, the FCNN and the CNN. Reconstrucions are shown at \(t=0.12s\) for \(x\in [0.375,0.75]\). Case $\hy$ is displayed in the top row, $\rare$ in the bottom row. The colobars reference \(f\) and \(\tilde{f}\).}
	\label{Fig: ErrWorst}
\end{figure}
The FOM solution viewed as $f(x,v,t_i)$ has been introduced in \cref{Ch:BGK}. There, $f(x_j,v,t_i)$ is the probability distribution of the microscopic velocities $v$ at point $x_j$ in space at one moment $t_i$ in time for a gas.
With this in mind, a qualitative comparison between the three algorithms can be made considering the rendition of the velocity probabilities. Starting with \(\hy\), seen in the top row of \cref{Fig: ErrWorst} one can observe that \(\tilde{f}(x,v,t_i)\) starting around \(x=0.6\) gets defective for POD and the CNN. Noteworthy, here the probability distribution is thinner as the original with POD. This in turn leads to errors in the temperature \(T\) once passing \(x\approx 0.6\). Prominent qualitative deviations using the CNN are especially blurriness/pixelation of \(\tilde{f}(x,v,t_i)\) after \(x\approx 0.6\). In contrast the FCNN seems to reproduce the FOM solution almost exactly.\\
Continuing with a row further down in \cref{Fig: ErrWorst} and therefore with \(\rare\). The FCNN seems to reproduce the FOM solution without any visible drawback. Also POD seems to reproduce all important structures, except after \(x\approx 0.7\) around the contact discontinuity, some values for velocities with \(v>0\) appear to be missing. Again the CNN struggles with blurriness making \(\tilde{f}\) for both rarefaction levels look largely similar.    
\begin{figure}[H]
	% This file was created by tikzplotlib v0.9.6.
\begin{tikzpicture}

\definecolor{color0}{rgb}{0.12156862745098,0.466666666666667,0.705882352941177}

\begin{groupplot}[group style={group size=3 by 2,horizontal sep=1.5cm,vertical sep=1.1cm}]
\nextgroupplot[
legend cell align={left},
legend style={at={(1,1)},anchor=north east,fill opacity=0.1, draw opacity=1, text opacity=1,draw=none},
tick align=outside,
tick pos=left,
x grid style={white!69.0196078431373!black},
xmajorgrids,
xlabel={\(x\)},
xmin=-0.04725, xmax=1.04725,
xtick style={color=black},
y grid style={white!69.0196078431373!black},
ymajorgrids,
ylabel={\(\rho\)},
ymin=0.0787652333577474, ymax=1.04575335673797,
ytick style={color=black},
width=.35\textwidth,
height=.4\textwidth,
x label style={xshift=-.10em}
]
\addplot [semithick, black, dashed, mark=x,mark size=2,mark repeat=25, mark options={solid}]
table {%
0.0025 0.999999963320219
0.0075 0.999999963320219
0.0125 0.999999963320219
0.0175 0.999999963320219
0.0225 0.999999963320219
0.0275 0.999999963320219
0.0325 0.999999963320219
0.0375 0.999999963320219
0.0425 0.999999963320219
0.0475 0.999999963320219
0.0525 0.999999963320219
0.0575 0.999999963320219
0.0625 0.999999963320219
0.0675 0.999999963320219
0.0725 0.999999963320219
0.0775 0.999999963320219
0.0825 0.999999963320219
0.0875 0.999999963320219
0.0925 0.999999963320219
0.0975 0.999999963320219
0.1025 0.999999963320219
0.1075 0.999999963320219
0.1125 0.999999963320219
0.1175 0.99999990218725
0.1225 0.999999779921312
0.1275 0.999999535389436
0.1325 0.999999413123497
0.1375 0.999998862926777
0.1425 0.99999800706521
0.1475 0.999996539873955
0.1525 0.999994400220039
0.1575 0.99999067110893
0.1625 0.999984741210937
0.1675 0.99997538786668
0.1725 0.999960654821151
0.1775 0.999938524686373
0.1825 0.999904779287485
0.1875 0.999854833651812
0.1925 0.999782024285732
0.1975 0.999676814446082
0.2025 0.999528261331411
0.2075 0.999321570763221
0.2125 0.99903846398378
0.2175 0.998657727852846
0.2225 0.998153625390468
0.2275 0.997497362968249
0.2325 0.99665702917637
0.2375 0.995598572951097
0.2425 0.994286231505565
0.2475 0.992684731116662
0.2525 0.990759225992056
0.2575 0.988477988120837
0.2625 0.985812468406482
0.2675 0.982738213661389
0.2725 0.979236639462984
0.2775 0.975294846754808
0.2825 0.970904521453075
0.2875 0.966063951834654
0.2925 0.960775827750181
0.2975 0.955047118358123
0.3025 0.948889010991806
0.3075 0.94231550510113
0.3125 0.935342617523976
0.3175 0.927988627018073
0.3225 0.920272118005997
0.3275 0.912212653037829
0.3325 0.903829550131773
0.3375 0.895142555236816
0.3425 0.886169641445845
0.3475 0.876929209782527
0.3525 0.867438438611153
0.3575 0.857713405902569
0.3625 0.84776915036715
0.3675 0.837620466183393
0.3725 0.827280374673697
0.3775 0.816761958293426
0.3825 0.806076526641846
0.3875 0.79523612291385
0.3925 0.784251078581199
0.3975 0.773132764376127
0.4025 0.761891756302271
0.4075 0.750539363958897
0.4125 0.739087813939804
0.4175 0.72755067776411
0.4225 0.715944155668601
0.4275 0.704287198873667
0.4325 0.692604199433938
0.4375 0.680926212897667
0.4425 0.669293831556271
0.4475 0.657760913555439
0.4525 0.646399901463435
0.4575 0.635307385371282
0.4625 0.624610216189653
0.4675 0.614472535940317
0.4725 0.605094983027532
0.4775 0.59670429963332
0.4825 0.589521603706555
0.4875 0.583707980620555
0.4925 0.579297664837959
0.4975 0.576156347225874
0.5025 0.574000982137827
0.5075 0.572480116135035
0.5125 0.571264242514586
0.5175 0.570086332467886
0.5225 0.568721416669014
0.5275 0.566925146640875
0.5325 0.564365814893674
0.5375 0.560570557912191
0.5425 0.554906038137583
0.5475 0.546613473158616
0.5525 0.534909016046769
0.5575 0.519142212011875
0.5625 0.498978602580535
0.5675 0.474548706641564
0.5725 0.446512118363992
0.5775 0.416003312820043
0.5825 0.384466831500714
0.5875 0.353428247647408
0.5925 0.324264611953344
0.5975 0.298031599093706
0.6025 0.275379786124596
0.6075 0.256558289894691
0.6125 0.241481600663601
0.6175 0.229826477857736
0.6225 0.221130557549305
0.6275 0.21487615047357
0.6325 0.210551619529724
0.6375 0.207691956789066
0.6425 0.205900455132509
0.6475 0.204856793085734
0.6525 0.204314497800974
0.6575 0.204092799088894
0.6625 0.204064494524247
0.6675 0.204143249071561
0.6725 0.204271674156189
0.6775 0.204410170897459
0.6825 0.204527209966611
0.6875 0.204589748993898
0.6925 0.204552182784447
0.6975 0.20434177838839
0.7025 0.203837538376833
0.7075 0.202838136599614
0.7125 0.201017092435788
0.7175 0.197870639654306
0.7225 0.192691668485984
0.7275 0.184667370258233
0.7325 0.173289806414873
0.7375 0.159203853362646
0.7425 0.144954935098306
0.7475 0.134090918761033
0.7525 0.12823774264409
0.7575 0.125970038083883
0.7625 0.125265816847483
0.7675 0.125069113878103
0.7725 0.125016333200993
0.7775 0.125002356675955
0.7825 0.124998665772952
0.7875 0.12499770292869
0.7925 0.124997443113572
0.7975 0.124997389622224
0.8025 0.124997374338981
0.8075 0.12499736669736
0.8125 0.12499736669736
0.8175 0.12499736669736
0.8225 0.12499736669736
0.8275 0.12499736669736
0.8325 0.12499736669736
0.8375 0.12499736669736
0.8425 0.12499736669736
0.8475 0.12499736669736
0.8525 0.12499736669736
0.8575 0.12499736669736
0.8625 0.12499736669736
0.8675 0.12499736669736
0.8725 0.12499736669736
0.8775 0.12499736669736
0.8825 0.12499736669736
0.8875 0.12499736669736
0.8925 0.12499736669736
0.8975 0.12499736669736
0.9025 0.12499736669736
0.9075 0.12499736669736
0.9125 0.12499736669736
0.9175 0.12499736669736
0.9225 0.12499736669736
0.9275 0.12499736669736
0.9325 0.12499736669736
0.9375 0.12499736669736
0.9425 0.12499736669736
0.9475 0.12499736669736
0.9525 0.12499736669736
0.9575 0.12499736669736
0.9625 0.12499736669736
0.9675 0.12499736669736
0.9725 0.12499736669736
0.9775 0.12499736669736
0.9825 0.12499736669736
0.9875 0.12499736669736
0.9925 0.12499736669736
0.9975 0.12499736669736
};
\addlegendentry{FOM}
\addplot [semithick, red, mark=o,mark size=2, mark repeat=25, mark options={solid}]
table {%
0.0025 0.999548771594793
0.0075 0.999548771594787
0.0125 0.999548771594777
0.0175 0.999548771594753
0.0225 0.999548771594688
0.0275 0.99954877159453
0.0325 0.999548771594179
0.0375 0.999548771593437
0.0425 0.999548771591852
0.0475 0.999548771588456
0.0525 0.999548771581279
0.0575 0.999548771566277
0.0625 0.999548771535268
0.0675 0.999548771471879
0.0725 0.999548771343803
0.0775 0.999548771088022
0.0825 0.999548770583062
0.0875 0.999548769597802
0.0925 0.999548767698169
0.0975 0.99954876407943
0.1025 0.999548757269338
0.1075 0.999548744610781
0.1125 0.999548721373772
0.1175 0.999548679255706
0.1225 0.999548603890035
0.1275 0.999548470777807
0.1325 0.999548238760097
0.1375 0.999547839734983
0.1425 0.999547162765054
0.1475 0.999546030000005
0.1525 0.999544160954158
0.1575 0.999541120663193
0.1625 0.999536246184591
0.1675 0.999528544966733
0.1725 0.999516558049316
0.1775 0.999498181231678
0.1825 0.999470438699194
0.1875 0.999429206591679
0.1925 0.9993688890128
0.1975 0.999282056142071
0.2025 0.99915906311425
0.2075 0.998987678260307
0.2125 0.998752758572666
0.2175 0.998436016733616
0.2225 0.998015925419097
0.2275 0.997467798986495
0.2325 0.996764079322538
0.2375 0.99587483249879
0.2425 0.994768438710398
0.2475 0.993412433898535
0.2525 0.991774442038987
0.2575 0.989823126091609
0.2625 0.98752908500575
0.2675 0.984865633633234
0.2725 0.981809419515683
0.2775 0.978340851622089
0.2825 0.9744443373183
0.2875 0.970108341936711
0.2925 0.965325298291781
0.2975 0.960091400703041
0.3025 0.954406319994776
0.3075 0.94827287369326
0.3125 0.941696680663758
0.3175 0.934685823088954
0.3225 0.927250532091985
0.3275 0.91940290722078
0.3325 0.911156674872146
0.3375 0.902526986722904
0.3425 0.893530256338715
0.3475 0.884184030229781
0.3525 0.874506888547891
0.3575 0.864518370193878
0.3625 0.854238917174887
0.3675 0.843689833494142
0.3725 0.832893254596429
0.3775 0.821872124393158
0.3825 0.810650178169559
0.3875 0.799251931294829
0.3925 0.78770267574189
0.3975 0.77602848917673
0.4025 0.764256265096905
0.4075 0.752413777619085
0.4125 0.740529801647178
0.4175 0.728634319137727
0.4225 0.716758856123944
0.4275 0.70493701438836
0.4325 0.693205287449573
0.4375 0.681604283152755
0.4425 0.670180511628393
0.4475 0.65898892591229
0.4525 0.648096390752349
0.4575 0.637586127827967
0.4625 0.627562790490795
0.4675 0.618156902786947
0.4725 0.609525670234001
0.4775 0.6018447263329
0.4825 0.595283734240775
0.4875 0.589962041804688
0.4925 0.585893694609532
0.4975 0.582950451517276
0.5025 0.580875840796191
0.5075 0.579352251627723
0.5125 0.578077723544553
0.5175 0.576800623627844
0.5225 0.575296690289324
0.5275 0.573308767286148
0.5325 0.570477432918538
0.5375 0.566285778369634
0.5425 0.560040704121852
0.5475 0.550913667192523
0.5525 0.538054899617929
0.5575 0.52077187876436
0.5625 0.49873191332678
0.5675 0.472125596353103
0.5725 0.441728225170466
0.5775 0.408825407012185
0.5825 0.3750163990988
0.5875 0.341952318250269
0.5925 0.311085403139529
0.5975 0.28349274631892
0.6025 0.259803155850325
0.6075 0.240218452263519
0.6125 0.224596872533648
0.6175 0.21256136281911
0.6225 0.203604094956206
0.6275 0.197171596305905
0.6325 0.192725777382
0.6375 0.189782476887888
0.6425 0.187931638026962
0.6475 0.18684363567553
0.6525 0.186265936683102
0.6575 0.186013763883529
0.6625 0.185957830374123
0.6675 0.186011486873066
0.6725 0.186118822997236
0.6775 0.186244486244602
0.6825 0.186365323201625
0.6875 0.186463437237978
0.6925 0.18651985312962
0.6975 0.186507573177517
0.7025 0.186382227399486
0.7075 0.186067492116135
0.7125 0.185430535378378
0.7175 0.18423927878628
0.7225 0.18208804768065
0.7275 0.178277990908776
0.7325 0.171696753937413
0.7375 0.161099168034382
0.7425 0.147018646050487
0.7475 0.134243307515287
0.7525 0.127109461113498
0.7575 0.124412936576949
0.7625 0.123592848531648
0.7675 0.123365803728222
0.7725 0.123305048460296
0.7775 0.123288975392124
0.7825 0.123284741006781
0.7875 0.123283627872577
0.7925 0.123283335737192
0.7975 0.123283259191473
0.8025 0.123283239168794
0.8075 0.123283233940817
0.8125 0.123283232578456
0.8175 0.123283232224191
0.8225 0.123283232132279
0.8275 0.123283232108493
0.8325 0.123283232102354
0.8375 0.123283232100774
0.8425 0.123283232100368
0.8475 0.123283232100264
0.8525 0.123283232100238
0.8575 0.123283232100232
0.8625 0.12328323210023
0.8675 0.12328323210023
0.8725 0.12328323210023
0.8775 0.12328323210023
0.8825 0.12328323210023
0.8875 0.123283232100229
0.8925 0.123283232100229
0.8975 0.123283232100229
0.9025 0.123283232100229
0.9075 0.123283232100229
0.9125 0.123283232100229
0.9175 0.123283232100229
0.9225 0.123283232100229
0.9275 0.123283232100229
0.9325 0.123283232100229
0.9375 0.123283232100229
0.9425 0.123283232100229
0.9475 0.123283232100229
0.9525 0.123283232100229
0.9575 0.123283232100229
0.9625 0.123283232100229
0.9675 0.123283232100229
0.9725 0.123283232100229
0.9775 0.123283232100229
0.9825 0.123283232100229
0.9875 0.123283232100229
0.9925 0.123283232100229
0.9975 0.123283232100229
};
\addlegendentry{POD}
\addplot [semithick, color0, mark=pentagon,mark size=2, mark repeat=25, mark options={solid}]
table {%
0.0025 0.999939968236364
0.0075 0.999939968236364
0.0125 0.999939968236364
0.0175 0.999939968236364
0.0225 0.999939968236364
0.0275 0.999939968236364
0.0325 0.999939968236364
0.0375 0.999939968236364
0.0425 0.999939968236364
0.0475 0.999939968236364
0.0525 0.999939968236364
0.0575 0.999939968236364
0.0625 0.999939968236364
0.0675 0.999939968236364
0.0725 0.999939968236364
0.0775 0.999939968236364
0.0825 0.999939968236364
0.0875 0.999939968236364
0.0925 0.999939968236364
0.0975 0.999939891820153
0.1025 0.999939896596166
0.1075 0.999939795822287
0.1125 0.999939862208871
0.1175 0.999939878924917
0.1225 0.999939654452296
0.1275 0.999939597617739
0.1325 0.999939325504387
0.1375 0.999938859007297
0.1425 0.999938065592104
0.1475 0.999936676130463
0.1525 0.999934615400166
0.1575 0.99993094897423
0.1625 0.999925315666657
0.1675 0.999916495683675
0.1725 0.999902459816673
0.1775 0.999881169782617
0.1825 0.999849121898222
0.1875 0.999801299439218
0.1925 0.999731557825819
0.1975 0.999631308472882
0.2025 0.999489218020477
0.2075 0.99929166957736
0.2125 0.999021155353731
0.2175 0.998656816828327
0.2225 0.998174267319532
0.2275 0.997545471391044
0.2325 0.996739041681091
0.2375 0.995721569260917
0.2425 0.994457296358469
0.2475 0.992910082046038
0.2525 0.991043596862791
0.2575 0.988823653509219
0.2625 0.986217566264363
0.2675 0.983196732779153
0.2725 0.979736232055494
0.2775 0.97581622369874
0.2825 0.971420934805885
0.2875 0.966540744456534
0.2925 0.961170660761686
0.2975 0.955310777331201
0.3025 0.948965688092777
0.3075 0.942144474396721
0.3125 0.934979659743989
0.3175 0.927785223421569
0.3225 0.920081615734559
0.3275 0.911964373543667
0.3325 0.903547288229068
0.3375 0.89501510720509
0.3425 0.886130499390837
0.3475 0.876919190662029
0.3525 0.867408619692119
0.3575 0.857626938571532
0.3625 0.847603971950519
0.3675 0.83736945851109
0.3725 0.826953578955279
0.3775 0.816385962594396
0.3825 0.805695764362239
0.3875 0.795205382224268
0.3925 0.784419378480659
0.3975 0.773457267727607
0.4025 0.762336483129706
0.4075 0.751073205819688
0.4125 0.739683521887622
0.4175 0.728183065732129
0.4225 0.716587809296564
0.4275 0.70491533439893
0.4325 0.693185986616673
0.4375 0.681425008612374
0.4425 0.669666346855079
0.4475 0.657956207720324
0.4525 0.646394694056839
0.4575 0.635514160952507
0.4625 0.624949089251459
0.4675 0.61485644382162
0.4725 0.605438914961922
0.4775 0.596935878722713
0.4825 0.589592156525797
0.4875 0.583598525979771
0.4925 0.579017400860977
0.4975 0.575732599036434
0.5025 0.573749198482778
0.5075 0.572390212462499
0.5125 0.571257876685988
0.5175 0.570111481043009
0.5225 0.568769076743569
0.5275 0.567027286339838
0.5325 0.564539783920806
0.5375 0.560837504334557
0.5425 0.555283192258615
0.5475 0.547092540117984
0.5525 0.535416085249147
0.5575 0.519480185392193
0.5625 0.499157757044603
0.5675 0.475012445105956
0.5725 0.446823057360374
0.5775 0.41563895650399
0.5825 0.384757045704203
0.5875 0.353900500788138
0.5925 0.324724445549341
0.5975 0.298561543727723
0.6025 0.275809307439396
0.6075 0.256807394086932
0.6125 0.241533189128416
0.6175 0.22969940975786
0.6225 0.220858842397156
0.6275 0.214495856362658
0.6325 0.210094252266945
0.6375 0.207182188661626
0.6425 0.205356042282895
0.6475 0.204289610354373
0.6525 0.203732133437044
0.6575 0.203499824692232
0.6625 0.203463445441463
0.6675 0.203535807772707
0.6725 0.203659435710273
0.6775 0.203795783603803
0.6825 0.20391557747737
0.6875 0.203990367575525
0.6925 0.203982203339155
0.6975 0.203831830444053
0.7025 0.203439521794327
0.7075 0.202635776681396
0.7125 0.201135637978904
0.7175 0.198469977133358
0.7225 0.193903454674933
0.7275 0.186410662837518
0.7325 0.174961739386886
0.7375 0.159579780645286
0.7425 0.144058139158938
0.7475 0.133231449872255
0.7525 0.127834568922527
0.7575 0.125873483574161
0.7625 0.125272463028056
0.7675 0.125032803998926
0.7725 0.124968871569786
0.7775 0.124951929856951
0.7825 0.124947448762564
0.7875 0.12494626478889
0.7925 0.124945930467966
0.7975 0.124945836858107
0.8025 0.124945856678562
0.8075 0.124945856678562
0.8125 0.124945856678562
0.8175 0.124945822052467
0.8225 0.124945822052467
0.8275 0.124945822052467
0.8325 0.124945822052467
0.8375 0.124945822052467
0.8425 0.124945822052467
0.8475 0.124945822052467
0.8525 0.124945822052467
0.8575 0.124945822052467
0.8625 0.124945822052467
0.8675 0.124945822052467
0.8725 0.124945822052467
0.8775 0.124945822052467
0.8825 0.124945822052467
0.8875 0.124945822052467
0.8925 0.124945822052467
0.8975 0.124945822052467
0.9025 0.124945822052467
0.9075 0.124945822052467
0.9125 0.124945822052467
0.9175 0.124945822052467
0.9225 0.124945822052467
0.9275 0.124945822052467
0.9325 0.124945822052467
0.9375 0.124945822052467
0.9425 0.124945822052467
0.9475 0.124945822052467
0.9525 0.124945822052467
0.9575 0.124945822052467
0.9625 0.124945822052467
0.9675 0.124945822052467
0.9725 0.124945822052467
0.9775 0.124945822052467
0.9825 0.124945822052467
0.9875 0.124945822052467
0.9925 0.124945857633765
0.9975 0.124945857633765
};
\addlegendentry{FCNN}
\addplot [semithick, green!50!black, mark=triangle,mark size=2, mark repeat=25, mark options={solid,rotate=180}, only marks]
table {%
0.0025 0.997826564006316
0.0075 0.999634082500751
0.0125 0.999670150952461
0.0175 1.00089586698092
0.0225 1.00048700968424
0.0275 0.999750051742945
0.0325 1.00179935112978
0.0375 0.998835991590451
0.0425 1.00054997664232
0.0475 1.00057424643101
0.0525 0.997986365587283
0.0575 0.998990902533898
0.0625 1.00076326957116
0.0675 1.00147754718096
0.0725 0.999688368577223
0.0775 0.999325605539175
0.0825 1.00170001005515
0.0875 1.00006812658065
0.0925 1.00029890353863
0.0975 0.999496227655655
0.1025 0.997886718847813
0.1075 0.999347613408015
0.1125 1.00085362409934
0.1175 1.00004006654788
0.1225 0.999976182595277
0.1275 1.00009961005969
0.1325 1.00124609776032
0.1375 1.00066771874061
0.1425 1.00034903257321
0.1475 0.999733912639129
0.1525 0.99842383311345
0.1575 0.999465783437093
0.1625 1.00011764428554
0.1675 1.00016074302869
0.1725 0.999795045608129
0.1775 1.00020836561154
0.1825 1.00125098839784
0.1875 1.00000894986666
0.1925 0.999992321699093
0.1975 0.999634877229348
0.2025 0.997431094829853
0.2075 0.999444631429819
0.2125 0.999194780985514
0.2175 0.999104304191394
0.2225 0.998374804472312
0.2275 0.99752358901195
0.2325 0.997723982884334
0.2375 0.995986400506435
0.2425 0.994922197782076
0.2475 0.993364713130853
0.2525 0.98899089373075
0.2575 0.988139922802265
0.2625 0.985307326683631
0.2675 0.983291161365998
0.2725 0.979989614242162
0.2775 0.975429767217391
0.2825 0.973076147910876
0.2875 0.968115024077587
0.2925 0.964070100050706
0.2975 0.958142158312675
0.3025 0.947710306216509
0.3075 0.941470892001421
0.3125 0.936301427009778
0.3175 0.929134442256047
0.3225 0.921500340486184
0.3275 0.913304365598238
0.3325 0.905694655883006
0.3375 0.898112517136794
0.3425 0.892012180426182
0.3475 0.880463001055595
0.3525 0.868978867164025
0.3575 0.857521142715063
0.3625 0.851284296084673
0.3675 0.839678996648544
0.3725 0.828442084483611
0.3775 0.818615632179456
0.3825 0.806652643741705
0.3875 0.796960194905599
0.3925 0.789229319645808
0.3975 0.774383300389999
0.4025 0.764610706231533
0.4075 0.752236109513503
0.4125 0.741357069749099
0.4175 0.728611640441112
0.4225 0.71491308701344
0.4275 0.703751185001471
0.4325 0.691442856421837
0.4375 0.681317219367394
0.4425 0.670965023529835
0.4475 0.654357885703062
0.4525 0.648130331283961
0.4575 0.637230200645251
0.4625 0.626791073725774
0.4675 0.613789008213923
0.4725 0.603192830697084
0.4775 0.596600556984926
0.4825 0.591698243067815
0.4875 0.585167469122471
0.4925 0.58006512813079
0.4975 0.574537179408929
0.5025 0.572046438852946
0.5075 0.571879484714606
0.5125 0.573569872440436
0.5175 0.571612089108198
0.5225 0.568160643944373
0.5275 0.568164800986265
0.5325 0.566617892338679
0.5375 0.561307026789739
0.5425 0.557518311035939
0.5475 0.549704967400967
0.5525 0.531966747381748
0.5575 0.516557510082538
0.5625 0.499046888106909
0.5675 0.47705916258005
0.5725 0.444945372067965
0.5775 0.41721414297055
0.5825 0.382595031689375
0.5875 0.351334412892659
0.5925 0.324990046329987
0.5975 0.29638042816749
0.6025 0.269737029686952
0.6075 0.255421980833396
0.6125 0.242452422777812
0.6175 0.226303415420728
0.6225 0.218887711182619
0.6275 0.211986211630014
0.6325 0.204669053737934
0.6375 0.205674156164512
0.6425 0.20386951091962
0.6475 0.203292201726865
0.6525 0.199740452644152
0.6575 0.199890916164105
0.6625 0.201896459628374
0.6675 0.199385866140708
0.6725 0.200673861381335
0.6775 0.201314122248919
0.6825 0.201220909754435
0.6875 0.202678839365641
0.6925 0.200926050161704
0.6975 0.201315543590448
0.7025 0.198644674741305
0.7075 0.20112170622899
0.7125 0.200702196512467
0.7175 0.195979078610738
0.7225 0.191836402966426
0.7275 0.183470860505715
0.7325 0.173079035221002
0.7375 0.160602942491189
0.7425 0.146515476397979
0.7475 0.133639100270394
0.7525 0.126855396307432
0.7575 0.12492203559631
0.7625 0.127516190210978
0.7675 0.123466971593025
0.7725 0.124990153007018
0.7775 0.125107138584822
0.7825 0.12285223374
0.7875 0.125900140175453
0.7925 0.125363919979487
0.7975 0.124922822683285
0.8025 0.123820465344649
0.8075 0.123094549545875
0.8125 0.127414900522966
0.8175 0.12337829822149
0.8225 0.125258610798762
0.8275 0.124610868784098
0.8325 0.12319470063234
0.8375 0.126051153892126
0.8425 0.124572874643864
0.8475 0.124192971449632
0.8525 0.123434219604883
0.8575 0.123454859623542
0.8625 0.127833386262258
0.8675 0.123612704949501
0.8725 0.125545607163356
0.8775 0.124685840728955
0.8825 0.123431055973738
0.8875 0.125218278322464
0.8925 0.124563834606073
0.8975 0.124385952949524
0.9025 0.122719238965939
0.9075 0.12347678343455
0.9125 0.12764199421956
0.9175 0.124149498267051
0.9225 0.125206250410814
0.9275 0.124200116365384
0.9325 0.123076607019473
0.9375 0.125165398304279
0.9425 0.124789606302212
0.9475 0.125109224747389
0.9525 0.123349282986079
0.9575 0.123584935298333
0.9625 0.127726541115687
0.9675 0.123196954910572
0.9725 0.12529276120357
0.9775 0.125617744066776
0.9825 0.123182252431527
0.9875 0.125945210456848
0.9925 0.124717201942053
0.9975 0.12518231685345
};
\addlegendentry{CNN}

\nextgroupplot[
legend cell align={left},
legend style={at={(0.0,1)},anchor=north west,fill opacity=0.1, draw opacity=1, text opacity=1, draw=none},
tick align=outside,
tick pos=left,
x grid style={white!69.0196078431373!black},
xmajorgrids,
xlabel={\(x\)},
xmin=-0.04725, xmax=1.04725,
xtick style={color=black},
y grid style={white!69.0196078431373!black},
ymajorgrids,
ylabel={\(\rho u\)},
ymin=-0.0237175295924032, ymax=0.479965021862334,
ytick style={color=black},
width=.37\textwidth,
height=.4\textwidth
]
\addplot [semithick, black, dashed, mark=x,mark size=2, mark repeat=25, mark options={solid}]
table {%
0.0025 -2.9701575942379e-17
0.0075 -2.9701575942379e-17
0.0125 -2.9701575942379e-17
0.0175 -2.9701575942379e-17
0.0225 -2.9701575942379e-17
0.0275 -2.9701575942379e-17
0.0325 -2.9701575942379e-17
0.0375 -2.9701575942379e-17
0.0425 -2.9701575942379e-17
0.0475 -2.9701575942379e-17
0.0525 -2.9701575942379e-17
0.0575 -2.9701575942379e-17
0.0625 -2.9701575942379e-17
0.0675 -3.58603575113451e-17
0.0725 -3.58638102426239e-17
0.0775 5.6805661435368e-13
0.0825 1.99042476361002e-10
0.0875 2.20837991649488e-09
0.0925 2.93997368028267e-09
0.0975 1.31357376888643e-08
0.1025 2.38145813217986e-08
0.1075 4.87195462126188e-08
0.1125 1.18482312201542e-07
0.1175 2.0095587206295e-07
0.1225 3.83396059768099e-07
0.1275 6.58121348779825e-07
0.1325 1.16128545818365e-06
0.1375 2.03225357455255e-06
0.1425 3.48646389712185e-06
0.1475 5.97480063111906e-06
0.1525 1.00168031352998e-05
0.1575 1.66154888208619e-05
0.1625 2.71792475439631e-05
0.1675 4.38360783100733e-05
0.1725 6.9738147530809e-05
0.1775 0.000109360034940665
0.1825 0.000169110916869244
0.1875 0.000257759063581096
0.1925 0.000387167364876908
0.1975 0.000573117748608961
0.2025 0.000835869786797033
0.2075 0.00120107973443098
0.2125 0.00170018895588471
0.2175 0.00237086844167696
0.2225 0.00325687357947661
0.2275 0.00440760492585291
0.2325 0.0058770045036118
0.2375 0.00772213506918395
0.2425 0.0100010852238978
0.2475 0.0127705763429908
0.2525 0.0160835079558624
0.2575 0.0199864997521351
0.2625 0.0245176736126449
0.2675 0.029704885273648
0.2725 0.0355645653482579
0.2775 0.0421012893858932
0.2825 0.0493077793042677
0.2875 0.0571655985840627
0.2925 0.0656463751672976
0.2975 0.07471303783688
0.3025 0.0843216061489801
0.3075 0.0944227378619742
0.3125 0.104963347960292
0.3175 0.115888113820293
0.3225 0.127140629164304
0.3275 0.138664664012591
0.3325 0.150404853532333
0.3375 0.162307518067251
0.3425 0.174321041457807
0.3475 0.186396266022466
0.3525 0.198486618269269
0.3575 0.210548312786717
0.3625 0.222540185550889
0.3675 0.234423749038626
0.3725 0.24616309786956
0.3775 0.257724580624995
0.3825 0.269076805027718
0.3875 0.280190349490242
0.3925 0.291037541508123
0.3975 0.301592220966071
0.4025 0.311829593524858
0.4075 0.321725891157009
0.4125 0.331258108006691
0.4175 0.34040380705866
0.4225 0.349140751931193
0.4275 0.357446640915221
0.4325 0.365298701484088
0.4375 0.372673393425291
0.4425 0.379546061984702
0.4475 0.3858904907899
0.4525 0.391678889356953
0.4575 0.396881980119472
0.4625 0.401469837523235
0.4675 0.405414023679009
0.4725 0.408691643888321
0.4775 0.411292021579195
0.4825 0.413226034663786
0.4875 0.41453519470013
0.4925 0.415295784669826
0.4975 0.415611525216486
0.5025 0.415593772491148
0.5075 0.415336270698027
0.5125 0.41489544485938
0.5175 0.414279735123743
0.5225 0.413441921116945
0.5275 0.412265562656191
0.5325 0.41054307470679
0.5375 0.407952698513731
0.5425 0.404050230313847
0.5475 0.398294142046233
0.5525 0.390116252725716
0.5575 0.379034700041386
0.5625 0.364785466199088
0.5675 0.347433290718267
0.5725 0.327421677785053
0.5775 0.305538970309093
0.5825 0.282804902037727
0.5875 0.260309643297353
0.5925 0.239050560230286
0.5975 0.219807538955069
0.6025 0.20307934623608
0.6075 0.189081108725716
0.6125 0.17778662249403
0.6175 0.168993003883264
0.6225 0.162388542113512
0.6275 0.157611733837737
0.6325 0.154296588877
0.6375 0.152103708794801
0.6425 0.150738703044433
0.6475 0.149960169583879
0.6525 0.149579864110399
0.6575 0.149457424423213
0.6625 0.149491946494713
0.6675 0.149612014228558
0.6725 0.149765301786734
0.6775 0.149907796060004
0.6825 0.149991959449217
0.6875 0.149952034921847
0.6925 0.149683391104505
0.6975 0.149011228633714
0.7025 0.147642222218352
0.7075 0.145092235684392
0.7125 0.140589882050874
0.7175 0.132983875095863
0.7225 0.120763964502009
0.7275 0.10247138701595
0.7325 0.0779153039253849
0.7375 0.0500927752607743
0.7425 0.0255869634889079
0.7475 0.010102976420237
0.7525 0.00323067095348623
0.7575 0.000916686735148521
0.7625 0.000247536233262668
0.7675 6.57017337512221e-05
0.7725 1.73342190031844e-05
0.7775 4.56262885444554e-06
0.7825 1.19902286403957e-06
0.7875 3.14993485349219e-07
0.7925 8.20702035475298e-08
0.7975 2.12618186030591e-08
0.8025 5.83523126442528e-09
0.8075 1.06852817499928e-09
0.8125 1.53347718367724e-10
0.8175 1.33942684000716e-11
0.8225 1.28520434948373e-15
0.8275 5.28135721761187e-19
0.8325 5.28135721758471e-19
0.8375 5.28135721758267e-19
0.8425 5.2813572175825e-19
0.8475 5.28135721758248e-19
0.8525 5.28135721758248e-19
0.8575 5.28135721758248e-19
0.8625 5.28135721758248e-19
0.8675 5.28135721758248e-19
0.8725 5.28135721758248e-19
0.8775 5.28135721758248e-19
0.8825 5.28135721758248e-19
0.8875 5.28135721758248e-19
0.8925 5.28135721758248e-19
0.8975 5.28135721758248e-19
0.9025 5.28135721758248e-19
0.9075 5.28135721758248e-19
0.9125 5.28135721758248e-19
0.9175 5.28135721758248e-19
0.9225 5.28135721758248e-19
0.9275 5.28135721758248e-19
0.9325 5.28135721758248e-19
0.9375 5.28135721758248e-19
0.9425 5.28135721758248e-19
0.9475 5.28135721758248e-19
0.9525 5.28135721758248e-19
0.9575 5.28135721758248e-19
0.9625 5.28135721758248e-19
0.9675 5.28135721758248e-19
0.9725 5.28135721758248e-19
0.9775 5.28135721758248e-19
0.9825 5.28135721758248e-19
0.9875 5.28135721758248e-19
0.9925 5.28135721758248e-19
0.9975 5.28135721758248e-19
};
\addlegendentry{FOM}
\addplot [semithick, red, mark=o,mark size=2, mark repeat=25, mark options={solid}]
table {%
0.0025 0.00393237574162307
0.0075 0.00393237574163614
0.0125 0.00393237574166477
0.0175 0.00393237574171964
0.0225 0.00393237574182809
0.0275 0.00393237574206728
0.0325 0.00393237574258346
0.0375 0.00393237574369718
0.0425 0.00393237574608892
0.0475 0.00393237575118406
0.0525 0.00393237576192968
0.0575 0.00393237578436057
0.0625 0.00393237583066469
0.0675 0.00393237592520066
0.0725 0.00393237611603382
0.0775 0.00393237649685655
0.0825 0.00393237724804053
0.0875 0.00393237871244595
0.0925 0.00393238153343673
0.0975 0.00393238690255335
0.1025 0.00393239699736137
0.1075 0.0039324157437772
0.1125 0.00393245012270637
0.1175 0.00393251237378752
0.1225 0.00393262365100663
0.1275 0.00393281998348956
0.1325 0.00393316182416166
0.1375 0.0039337490664989
0.1425 0.00393474421179744
0.1475 0.00393640740084916
0.1525 0.0039391482816405
0.1575 0.00394360111542185
0.1625 0.00395073099413663
0.1675 0.00396198030874983
0.1725 0.00397946528925886
0.1775 0.00400623200805109
0.1825 0.00404657906024217
0.1875 0.00410644954553326
0.1925 0.00419388745402029
0.1975 0.00431954295975657
0.2025 0.00449719796363842
0.2075 0.00474426891353296
0.2125 0.00508223086272865
0.2175 0.00553689809257864
0.2225 0.00613849581716569
0.2275 0.006921467195858
0.2325 0.00792398103448396
0.2375 0.00918713651402677
0.2425 0.0107538976560125
0.2475 0.012667825597595
0.2525 0.0149717041880112
0.2575 0.0177061683886042
0.2625 0.0209084428487252
0.2675 0.024611280815639
0.2725 0.0288421653284213
0.2775 0.0336228013599424
0.2825 0.0389688952699348
0.2875 0.0448901914153248
0.2925 0.0513907177843819
0.2975 0.0584691837024481
0.3025 0.0661194719396857
0.3075 0.0743311728872895
0.3125 0.0830901175039003
0.3175 0.0923788763200986
0.3225 0.102177202267955
0.3275 0.112462404393532
0.3325 0.12320964705733
0.3375 0.134392174881238
0.3425 0.145981467579303
0.3475 0.157947331179376
0.3525 0.170257933324466
0.3575 0.182879790651119
0.3625 0.195777715953219
0.3675 0.208914732174675
0.3725 0.222251959400929
0.3775 0.235748480055173
0.3825 0.249361186532075
0.3875 0.263044614573805
0.3925 0.276750764854394
0.3975 0.290428914537441
0.4025 0.304025420087843
0.4075 0.317483512493661
0.4125 0.330743086559333
0.4175 0.343740487563639
0.4225 0.356408302237943
0.4275 0.368675168322387
0.4325 0.380465630717854
0.4375 0.391700097277873
0.4425 0.402294991460353
0.4475 0.41216327443322
0.4525 0.421215631784808
0.4575 0.429362804071025
0.4625 0.436519778058944
0.4675 0.442612762846729
0.4725 0.447589790767143
0.4775 0.451434865822219
0.4825 0.454183160773785
0.4875 0.455930953641405
0.4925 0.45683136196469
0.4975 0.457070360432574
0.5025 0.456828526427732
0.5075 0.456244022962827
0.5125 0.455389804273891
0.5175 0.454263808217002
0.5225 0.45277891087297
0.5275 0.450740151095367
0.5325 0.44780822825044
0.5375 0.443463394119888
0.5425 0.436996432917189
0.5475 0.427557383232156
0.5525 0.414281774058628
0.5575 0.396486578936554
0.5625 0.373891731431095
0.5675 0.346795295447884
0.5725 0.316129707754751
0.5775 0.283360071350005
0.5825 0.250241893082791
0.5875 0.218509256206492
0.5925 0.189589086473467
0.5975 0.164421155244688
0.6025 0.143416227206516
0.6075 0.126531603798344
0.6125 0.113410083472511
0.6175 0.10352585862365
0.6225 0.0963008073731597
0.6275 0.0911796275383113
0.6325 0.0876693019577585
0.6375 0.0853544433776073
0.6425 0.0838988479913904
0.6475 0.0830399641168731
0.6525 0.0825799093956901
0.6575 0.0823749451883289
0.6625 0.0823245853726994
0.6675 0.0823612015926955
0.6725 0.08244072511115
0.6775 0.0825347283366435
0.6825 0.0826238225116309
0.6875 0.082691972635448
0.6925 0.0827209975330994
0.6975 0.0826841077247918
0.7025 0.0825366468183002
0.7075 0.0822008578347515
0.7125 0.0815386693538691
0.7175 0.080300412577288
0.7225 0.0780249726353877
0.7275 0.0738497032187504
0.7325 0.0662278274931491
0.7375 0.053036060513252
0.7425 0.0341561499016392
0.7475 0.0160919951404636
0.7525 0.00586903038490919
0.7575 0.0020760331315755
0.7625 0.000946018726904214
0.7675 0.00063641683158109
0.7725 0.000553883596154878
0.7775 0.000532074986526063
0.7825 0.000526331645229695
0.7875 0.000524822013836168
0.7925 0.000524425840860136
0.7975 0.00052432203885198
0.8025 0.000524294887453232
0.8075 0.000524287798399671
0.8125 0.000524285951129817
0.8175 0.00052428547078955
0.8225 0.000524285346174678
0.8275 0.000524285313925921
0.8325 0.00052428530560256
0.8375 0.000524285303460556
0.8425 0.000524285302910899
0.8475 0.000524285302770344
0.8525 0.000524285302734448
0.8575 0.000524285302725269
0.8625 0.000524285302722924
0.8675 0.000524285302722314
0.8725 0.000524285302722221
0.8775 0.000524285302722232
0.8825 0.000524285302722232
0.8875 0.000524285302722219
0.8925 0.000524285302722228
0.8975 0.000524285302722223
0.9025 0.000524285302722223
0.9075 0.000524285302722223
0.9125 0.000524285302722223
0.9175 0.000524285302722223
0.9225 0.000524285302722223
0.9275 0.000524285302722223
0.9325 0.000524285302722223
0.9375 0.000524285302722223
0.9425 0.000524285302722223
0.9475 0.000524285302722223
0.9525 0.000524285302722223
0.9575 0.000524285302722223
0.9625 0.000524285302722223
0.9675 0.000524285302722223
0.9725 0.000524285302722223
0.9775 0.000524285302722223
0.9825 0.000524285302722223
0.9875 0.000524285302722223
0.9925 0.000524285302722223
0.9975 0.000524285302722223
};
\addlegendentry{POD}
\addplot [semithick, color0, mark=pentagon,mark size=2, mark repeat=25, mark options={solid}]
table {%
0.0025 -0.000632143126926788
0.0075 -0.000632143126926788
0.0125 -0.000632143126926788
0.0175 -0.000632143126926788
0.0225 -0.000632143126926788
0.0275 -0.000632143126926788
0.0325 -0.000632143126926788
0.0375 -0.000632143126926788
0.0425 -0.000632143126926788
0.0475 -0.000632143126926788
0.0525 -0.000632143126926788
0.0575 -0.000632143126926788
0.0625 -0.000632143126926788
0.0675 -0.000632143126926788
0.0725 -0.000632143126926788
0.0775 -0.000632143126926788
0.0825 -0.000632143126926788
0.0875 -0.000632143126926788
0.0925 -0.000632143126926788
0.0975 -0.000632173375010415
0.1025 -0.000632104061587989
0.1075 -0.000632007194243283
0.1125 -0.000632120716403293
0.1175 -0.000631910449360469
0.1225 -0.000631615438698764
0.1275 -0.00063152885815176
0.1325 -0.000630726702241898
0.1375 -0.000630169347624127
0.1425 -0.000628958475199835
0.1475 -0.000626601114683022
0.1525 -0.000622838565358478
0.1575 -0.000616276415063646
0.1625 -0.000605655051547406
0.1675 -0.000589919465739111
0.1725 -0.00056527024728935
0.1775 -0.000527699678195745
0.1825 -0.00047049709252235
0.1875 -0.00038642168686745
0.1925 -0.000263097280248266
0.1975 -8.63857408109713e-05
0.2025 0.000164091843011961
0.2075 0.00051175814212581
0.2125 0.000986357952451564
0.2175 0.00162556869107044
0.2225 0.00246904134005801
0.2275 0.00356540961324542
0.2325 0.0049675836850553
0.2375 0.00673055279581158
0.2425 0.00891387756524476
0.2475 0.0115732176818399
0.2525 0.0147664071753239
0.2575 0.0185438069924565
0.2625 0.0229509243035026
0.2675 0.0280261224345165
0.2725 0.0337981312479983
0.2775 0.0402849387098288
0.2825 0.0474968248516576
0.2875 0.055432582168762
0.2925 0.0640818716947143
0.2975 0.0734254435246062
0.3025 0.0834375517326484
0.3075 0.0940848816061278
0.3125 0.105425631713164
0.3175 0.11735601301068
0.3225 0.127535347167322
0.3275 0.138073698338917
0.3325 0.149057239274458
0.3375 0.160685125254658
0.3425 0.1725432862216
0.3475 0.184582720861515
0.3525 0.196754146963701
0.3575 0.209007790479944
0.3625 0.221294751494907
0.3675 0.233566555893202
0.3725 0.245777175796661
0.3775 0.257878534356224
0.3825 0.269825895697143
0.3875 0.280762751134239
0.3925 0.291125823998974
0.3975 0.301344687318739
0.4025 0.311383799901434
0.4075 0.321208499280159
0.4125 0.330782891511378
0.4175 0.340069206688678
0.4225 0.349030239424282
0.4275 0.357625064030437
0.4325 0.365811650550028
0.4375 0.373547694037799
0.4425 0.380786886392192
0.4475 0.387482958791299
0.4525 0.393554584425063
0.4575 0.398544508187415
0.4625 0.40284485512099
0.4675 0.406430877456141
0.4725 0.409295340053331
0.4775 0.411457697175722
0.4825 0.412966894784319
0.4875 0.413909365308522
0.4925 0.414393553363354
0.4975 0.414537071733491
0.5025 0.414414817643677
0.5075 0.414082601808506
0.5125 0.413605133145019
0.5175 0.41292593676663
0.5225 0.412146663042196
0.5275 0.411037759265123
0.5325 0.409423356341843
0.5375 0.406986588819159
0.5425 0.403290774788338
0.5475 0.397791048844959
0.5525 0.389889244113869
0.5575 0.379022289640628
0.5625 0.364790908884524
0.5675 0.34711405654474
0.5725 0.326258519922801
0.5775 0.30293366629912
0.5825 0.282746635654171
0.5875 0.262080746005429
0.5925 0.241685011604755
0.5975 0.222285537913755
0.6025 0.20499204194821
0.6075 0.190213260490063
0.6125 0.178092157708225
0.6175 0.168543554417529
0.6225 0.161318627034879
0.6275 0.156073291148601
0.6325 0.152428916739367
0.6375 0.150018538710252
0.6425 0.148518155509561
0.6475 0.147659144652695
0.6525 0.147232993087688
0.6575 0.147083805344955
0.6625 0.147102255604422
0.6675 0.147211749258498
0.6725 0.147359006818357
0.6775 0.14750284267091
0.6825 0.14760310588347
0.6875 0.14760750756133
0.6925 0.147436108173194
0.6975 0.146956216540239
0.7025 0.145939461804833
0.7075 0.144000666584626
0.7125 0.140496822843116
0.7175 0.134379596080212
0.7225 0.124047144379205
0.7275 0.107373947854461
0.7325 0.0825959413308089
0.7375 0.0512412213766304
0.7425 0.0228344296569949
0.7475 0.0069091441862917
0.7525 0.00114053280204829
0.7575 -0.000450641031938506
0.7625 -0.00082286816264241
0.7675 -0.000571324762488414
0.7725 -0.000504635391973649
0.7775 -0.000486962000039126
0.7825 -0.000482494039687571
0.7875 -0.000480878645068014
0.7925 -0.000480640824102871
0.7975 -0.000480611188328638
0.8025 -0.000480510096049167
0.8075 -0.000480510096049167
0.8125 -0.000480510096049167
0.8175 -0.000480612045561773
0.8225 -0.000480612045561773
0.8275 -0.000480612045561773
0.8325 -0.000480612045561773
0.8375 -0.000480612045561773
0.8425 -0.000480612045561773
0.8475 -0.000480612045561773
0.8525 -0.000480612045561773
0.8575 -0.000480612045561773
0.8625 -0.000480612045561773
0.8675 -0.000480612045561773
0.8725 -0.000480612045561773
0.8775 -0.000480612045561773
0.8825 -0.000480612045561773
0.8875 -0.000480612045561773
0.8925 -0.000480612045561773
0.8975 -0.000480612045561773
0.9025 -0.000480612045561773
0.9075 -0.000480612045561773
0.9125 -0.000480612045561773
0.9175 -0.000480612045561773
0.9225 -0.000480612045561773
0.9275 -0.000480612045561773
0.9325 -0.000480612045561773
0.9375 -0.000480612045561773
0.9425 -0.000480612045561773
0.9475 -0.000480612045561773
0.9525 -0.000480612045561773
0.9575 -0.000480612045561773
0.9625 -0.000480612045561773
0.9675 -0.000480612045561773
0.9725 -0.000480612045561773
0.9775 -0.000480612045561773
0.9825 -0.000480612045561773
0.9875 -0.000480612045561773
0.9925 -0.000480903566059982
0.9975 -0.000480903566059982
};
\addlegendentry{FCNN}
\addplot [semithick, green!50!black, mark=triangle,mark size=2, mark repeat=25, mark options={solid,rotate=180}, only marks]
table {%
0.0025 0.000175403527955961
0.0075 0.000809469424986566
0.0125 -4.28012590885427e-05
0.0175 -2.13725949829784e-05
0.0225 0.000106659253842119
0.0275 -0.000207309840107127
0.0325 -0.000406330196533797
0.0375 0.000719278590769813
0.0425 0.000708382661566267
0.0475 0.00151094687892279
0.0525 0.000782135159085221
0.0575 0.000420549123567561
0.0625 0.000575461592621713
0.0675 -0.000334871587445141
0.0725 -4.22705439951491e-05
0.0775 -0.000391575842255389
0.0825 -0.000329514631361994
0.0875 8.62757567707992e-05
0.0925 0.000426264078291846
0.0975 0.00107902767427385
0.1025 0.00041452302526441
0.1075 -0.000274161092785773
0.1125 0.000149601210385744
0.1175 -0.000348618166103473
0.1225 0.000170061512221714
0.1275 0.000212061725450335
0.1325 -6.43727563173199e-05
0.1375 0.000390991689056366
0.1425 0.000945584923951971
0.1475 0.000425616888798382
0.1525 0.000315607140189136
0.1575 0.000298669304038044
0.1625 0.000260548123862224
0.1675 0.000238298448783892
0.1725 -4.78869264758811e-05
0.1775 -1.71785794584437e-05
0.1825 -0.000111327265237153
0.1875 0.000412756977481636
0.1925 0.000654780177145764
0.1975 0.00103517798150457
0.2025 0.000809445303823731
0.2075 0.00126831066111352
0.2125 0.00108979657995524
0.2175 0.00149758069507804
0.2225 0.00188770460444465
0.2275 0.00444920901053215
0.2325 0.00401811724628406
0.2375 0.0064961348410123
0.2425 0.00803347663949747
0.2475 0.00943346706116915
0.2525 0.0151807164319011
0.2575 0.0190670830415002
0.2625 0.0236958872570769
0.2675 0.0266541807716849
0.2725 0.0277527706143023
0.2775 0.038101595551501
0.2825 0.0394963452629499
0.2875 0.0450273758159197
0.2925 0.0525917736430445
0.2975 0.0573718901889903
0.3025 0.0772370961353912
0.3075 0.0874065494114768
0.3125 0.0990030909915725
0.3175 0.105361230914033
0.3225 0.111720061622446
0.3275 0.125981263173547
0.3325 0.135245487874055
0.3375 0.143755156614223
0.3425 0.153268859817011
0.3475 0.163206992293405
0.3525 0.179091370359411
0.3575 0.196997957939161
0.3625 0.214657940757866
0.3675 0.226622963454056
0.3725 0.232712489083021
0.3775 0.250367358968631
0.3825 0.262262702309606
0.3875 0.269055868021418
0.3925 0.283560391332872
0.3975 0.297825562408221
0.4025 0.294209449247869
0.4075 0.310386180107936
0.4125 0.333372264686348
0.4175 0.345845313835254
0.4225 0.350517336271742
0.4275 0.367641765355586
0.4325 0.377294950610159
0.4375 0.383325661719335
0.4425 0.398678684031367
0.4475 0.409361635625675
0.4525 0.393592872821909
0.4575 0.399633152323373
0.4625 0.413758055019732
0.4675 0.423186575706933
0.4725 0.425620536613638
0.4775 0.429338757215925
0.4825 0.432656046321236
0.4875 0.436466574663106
0.4925 0.441513350108103
0.4975 0.43907859374908
0.5025 0.434735403743101
0.5075 0.435016330394402
0.5125 0.435365325276201
0.5175 0.437391802621699
0.5225 0.435080408365847
0.5275 0.431875747774584
0.5325 0.428959858596801
0.5375 0.43189843558262
0.5425 0.427101737762629
0.5475 0.410456835376721
0.5525 0.415074614376898
0.5575 0.421190103317747
0.5625 0.376319102817588
0.5675 0.362212307262518
0.5725 0.343881160147155
0.5775 0.309516060181335
0.5825 0.27695907691785
0.5875 0.259431246163464
0.5925 0.226963440562187
0.5975 0.190669043669428
0.6025 0.179309015919869
0.6075 0.168748523285006
0.6125 0.157087029550389
0.6175 0.147946574849586
0.6225 0.136950639572348
0.6275 0.136942970613523
0.6325 0.120703155931207
0.6375 0.123694305986908
0.6425 0.119249708032055
0.6475 0.127434117045532
0.6525 0.113774696028229
0.6575 0.118041082733402
0.6625 0.117105936236927
0.6675 0.122547788002218
0.6725 0.121198179592782
0.6775 0.120762819193125
0.6825 0.126633200782824
0.6875 0.129049905226949
0.6925 0.121611179383643
0.6975 0.120297568589375
0.7025 0.1050542872494
0.7075 0.103155173282314
0.7125 0.11100755319202
0.7175 0.110079937771864
0.7225 0.0992148621297882
0.7275 0.0908871932615529
0.7325 0.0798180565013511
0.7375 0.0538235905708119
0.7425 0.0435336821651016
0.7475 0.0161004991377832
0.7525 0.00517258741972474
0.7575 0.00365303297016947
0.7625 0.00160971696834973
0.7675 0.000554130291658048
0.7725 0.000803267893108249
0.7775 0.00128989553153956
0.7825 0.00133569223327627
0.7875 0.000352735052620432
0.7925 0.000427837958454728
0.7975 -3.35467911604981e-05
0.8025 0.00117219282540369
0.8075 0.00136400393893383
0.8125 0.00135754588506148
0.8175 0.00118592402961761
0.8225 0.00127105991004131
0.8275 0.00130141694120046
0.8325 0.00127627738790659
0.8375 0.00078850483399089
0.8425 0.00131854202754108
0.8475 0.001184829884171
0.8525 0.000649235891942487
0.8575 0.000342584167095695
0.8625 0.000277490639874809
0.8675 -0.00046968056025785
0.8725 0.00147115502694091
0.8775 0.00119081598171926
0.8825 0.000625006123360205
0.8875 0.00139212050039445
0.8925 0.00112974630514878
0.8975 0.00114711828360282
0.9025 0.000652579252109391
0.9075 0.00199950188595975
0.9125 0.000841037697121741
0.9175 -0.000248485644763418
0.9225 0.00121683763230818
0.9275 0.0019918280838506
0.9325 0.000553619813658961
0.9375 0.000615889055948389
0.9425 0.000935519871257249
0.9475 0.000247785696848638
0.9525 0.000895704595459115
0.9575 -2.83251234849078e-05
0.9625 0.000475915676648202
0.9675 0.000102767374729777
0.9725 0.000779575344928168
0.9775 0.00105107507539391
0.9825 -0.000154752364770014
0.9875 0.00275949208665127
0.9925 0.00177268677665771
0.9975 0.00200581502443838
};
\addlegendentry{CNN}

\nextgroupplot[
legend cell align={left},
legend style={at={(1,1)},anchor=north east,fill opacity=0.1, draw opacity=1, text opacity=1, draw=none},
tick align=outside,
tick pos=left,
x grid style={white!69.0196078431373!black},
xmajorgrids,
xlabel={\(x\)},
xmin=-0.04725, xmax=1.04725,
xtick style={color=black},
y grid style={white!69.0196078431373!black},
ymajorgrids,
ylabel={\(E\)},
ymin=-0.120552432712838, ymax=0.605980670973315,
ytick style={color=black},
width=.35\textwidth,
height=.4\textwidth
]
\addplot [semithick, black, dashed, mark=x,mark size=2, mark repeat=25, mark options={solid}]
table {%
0.0025 0.499999997457854
0.0075 0.499999997457854
0.0125 0.499999997457854
0.0175 0.499999997457854
0.0225 0.499999997457854
0.0275 0.499999997457854
0.0325 0.499999997457854
0.0375 0.499999997457854
0.0425 0.499999997457854
0.0475 0.499999997457854
0.0525 0.499999997457854
0.0575 0.499999997457854
0.0625 0.499999997457854
0.0675 0.499999997457854
0.0725 0.499999997457854
0.0775 0.49999999745647
0.0825 0.499999997123304
0.0875 0.499999994239222
0.0925 0.49999999315962
0.0975 0.499999982706901
0.1025 0.499999967212159
0.1075 0.499999945766997
0.1125 0.499999908329267
0.1175 0.4999998219352
0.1225 0.499999664287051
0.1275 0.499999408709247
0.1325 0.499998942510754
0.1375 0.4999981652197
0.1425 0.499996830561658
0.1475 0.499994612526022
0.1525 0.499990956182364
0.1575 0.499985022918745
0.1625 0.499975537745301
0.1675 0.499960592267752
0.1725 0.499937393293539
0.1775 0.499901909902479
0.1825 0.499848511035655
0.1875 0.499769387782158
0.1925 0.499654031258562
0.1975 0.499488526210481
0.2025 0.499255030732484
0.2075 0.498931073732302
0.2125 0.498489154452836
0.2175 0.497896602633831
0.2225 0.49711563857505
0.2275 0.496104031920521
0.2325 0.494816125131012
0.2375 0.493204409942022
0.2425 0.491221536834207
0.2475 0.488822538745084
0.2525 0.485967268521428
0.2575 0.482622689956056
0.2625 0.478764910116803
0.2675 0.474380595345601
0.2725 0.469467933202883
0.2775 0.464036847019814
0.2825 0.458108501333654
0.2875 0.451714388147804
0.2925 0.444894901525337
0.2975 0.437697625718639
0.3025 0.43017548415983
0.3075 0.422384967214943
0.3125 0.41438433189132
0.3175 0.406232146680525
0.3225 0.397985925585357
0.3275 0.389701040596312
0.3325 0.381429940465515
0.3375 0.373221536863317
0.3425 0.365120781658711
0.3475 0.35716847155665
0.3525 0.349401123148409
0.3575 0.341850998944951
0.3625 0.334546205617342
0.3675 0.327510880106934
0.3725 0.320765342451106
0.3775 0.314326317516064
0.3825 0.308207224996573
0.3875 0.302418348265026
0.3925 0.296967089557285
0.3975 0.291858183808528
0.4025 0.287093943575464
0.4075 0.282674389872564
0.4125 0.27859745148855
0.4175 0.274859147252174
0.4225 0.271453659257701
0.4275 0.268373503265531
0.4325 0.265609568116938
0.4375 0.263151195986163
0.4425 0.260986230966502
0.4475 0.259100912260108
0.4525 0.257479921599843
0.4575 0.2561062289821
0.4625 0.254960947439195
0.4675 0.254023244845512
0.4725 0.253270314529259
0.4775 0.252677600104063
0.4825 0.252219487436491
0.4875 0.251870137005716
0.4925 0.251604335682046
0.4975 0.251397725545779
0.5025 0.251226691525528
0.5075 0.251068334142746
0.5125 0.250900335380158
0.5175 0.250698705687687
0.5225 0.250431901804233
0.5275 0.250051150241851
0.5325 0.249478665887373
0.5375 0.248596919839037
0.5425 0.247244384869381
0.5475 0.245223917976736
0.5525 0.242327889855065
0.5575 0.238378920422901
0.5625 0.233277600007316
0.5675 0.227043163796654
0.5725 0.219832444821558
0.5775 0.211928646571401
0.5825 0.203701167142895
0.5875 0.195547681866796
0.5925 0.187834659992751
0.5975 0.18085111835914
0.6025 0.174784099823094
0.6075 0.169716411704117
0.6125 0.165641205077031
0.6175 0.162485369842056
0.6225 0.160134753015422
0.6275 0.158456400881594
0.6325 0.157315743947614
0.6375 0.156588301770535
0.6425 0.156166478079637
0.6475 0.15596226117267
0.6525 0.155906986609067
0.6575 0.155948948709906
0.6625 0.156049660237618
0.6675 0.156179161378539
0.6725 0.156310450227518
0.6775 0.156412499238597
0.6825 0.156440735266932
0.6875 0.156322878209411
0.6925 0.155936716342982
0.6975 0.155074757600502
0.7025 0.153389302709368
0.7075 0.150312942138363
0.7125 0.144962248699927
0.7175 0.136077279393745
0.7225 0.122162284914289
0.7275 0.102173606704146
0.7325 0.0770700776468148
0.7375 0.051396100443933
0.7425 0.031771537528569
0.7475 0.0212761318479062
0.7525 0.017292907803068
0.7575 0.0160879855050049
0.7625 0.0157570211704135
0.7675 0.0156687592971979
0.7725 0.0156454184213528
0.7775 0.0156392644073019
0.7825 0.0156376445647867
0.7875 0.0156372187972693
0.7925 0.0156371070022085
0.7975 0.0156370779062883
0.8025 0.0156370704822387
0.8075 0.015637068192549
0.8125 0.0156370675656021
0.8175 0.0156370674551392
0.8225 0.0156370674431169
0.8275 0.0156370674431151
0.8325 0.0156370674431151
0.8375 0.0156370674431151
0.8425 0.0156370674431151
0.8475 0.0156370674431151
0.8525 0.0156370674431151
0.8575 0.0156370674431151
0.8625 0.0156370674431151
0.8675 0.0156370674431151
0.8725 0.0156370674431151
0.8775 0.0156370674431151
0.8825 0.0156370674431151
0.8875 0.0156370674431151
0.8925 0.0156370674431151
0.8975 0.0156370674431151
0.9025 0.0156370674431151
0.9075 0.0156370674431151
0.9125 0.0156370674431151
0.9175 0.0156370674431151
0.9225 0.0156370674431151
0.9275 0.0156370674431151
0.9325 0.0156370674431151
0.9375 0.0156370674431151
0.9425 0.0156370674431151
0.9475 0.0156370674431151
0.9525 0.0156370674431151
0.9575 0.0156370674431151
0.9625 0.0156370674431151
0.9675 0.0156370674431151
0.9725 0.0156370674431151
0.9775 0.0156370674431151
0.9825 0.0156370674431151
0.9875 0.0156370674431151
0.9925 0.0156370674431151
0.9975 0.0156370674431151
};
\addlegendentry{FOM}
\addplot [semithick, red, mark=o,mark size=2, mark repeat=25, mark options={solid}]
table {%
0.0025 0.49862507099537
0.0075 0.498625070995358
0.0125 0.498625070995335
0.0175 0.498625070995289
0.0225 0.498625070995195
0.0275 0.498625070994987
0.0325 0.498625070994536
0.0375 0.498625070993572
0.0425 0.498625070991509
0.0475 0.498625070987111
0.0525 0.498625070977834
0.0575 0.498625070958476
0.0625 0.49862507091852
0.0675 0.498625070836961
0.0725 0.498625070672361
0.0775 0.498625070343984
0.0825 0.498625069696427
0.0875 0.498625068434377
0.0925 0.498625066003869
0.0975 0.498625061379234
0.1025 0.49862505268657
0.1075 0.498625036548459
0.1125 0.498625006961122
0.1175 0.498624953401292
0.1225 0.498624857686967
0.1275 0.498624688860349
0.1325 0.498624394993753
0.1375 0.498623890307746
0.1425 0.498623035305027
0.1475 0.49862160674214
0.1525 0.498619253187858
0.1575 0.498615430698208
0.1625 0.498609311888697
0.1675 0.498599660615519
0.1725 0.498584663918093
0.1775 0.498561713278738
0.1825 0.498527129175792
0.1875 0.498475826919541
0.1925 0.498400928328475
0.1975 0.498293333066813
0.2025 0.498141275003092
0.2075 0.497929901566226
0.2125 0.497640925705907
0.2175 0.497252407933146
0.2225 0.496738727011619
0.2275 0.496070789730308
0.2325 0.495216511824596
0.2375 0.494141574769251
0.2425 0.492810430540245
0.2475 0.49118749413781
0.2525 0.489238437906705
0.2575 0.486931487672566
0.2625 0.484238621141534
0.2675 0.481136583397302
0.2725 0.477607659268651
0.2775 0.473640172608262
0.2825 0.469228712554903
0.2875 0.464374112063574
0.2925 0.45908322158397
0.2975 0.453368529902798
0.3025 0.447247685609862
0.3075 0.440742968179374
0.3125 0.433880749428725
0.3175 0.426690976158106
0.3225 0.419206694698188
0.3275 0.41146362898685
0.3325 0.403499816247336
0.3375 0.395355298541466
0.3425 0.387071864341846
0.3475 0.378692831569695
0.3525 0.370262861977174
0.3575 0.361827796005775
0.3625 0.353434497046703
0.3675 0.345130694129389
0.3725 0.336964812281608
0.3775 0.328985779986757
0.3825 0.321242803186622
0.3875 0.313785095026547
0.3925 0.306661549894885
0.3975 0.299920349123918
0.4025 0.293608483800653
0.4075 0.287771177213868
0.4125 0.282451185164978
0.4175 0.277687946194899
0.4225 0.273516545106461
0.4275 0.269966441344469
0.4325 0.267059898486333
0.4375 0.264810033124293
0.4425 0.263218384890518
0.4475 0.262271906267568
0.4525 0.261939310151276
0.4575 0.262166856616607
0.4625 0.262874021462233
0.4675 0.263950230587403
0.4725 0.265255071304541
0.4775 0.266625778589816
0.4825 0.267895891198504
0.4875 0.268925021887292
0.4925 0.269630339497512
0.4975 0.270001703213151
0.5025 0.270087324494209
0.5075 0.269957249654707
0.5125 0.269667082626097
0.5175 0.269236878201456
0.5225 0.268641994106717
0.5275 0.267804764421337
0.5325 0.266581191454045
0.5375 0.264746631547403
0.5425 0.261992020488676
0.5475 0.25794461121141
0.5525 0.252222458497304
0.5575 0.244519462099447
0.5625 0.234701241014227
0.5675 0.222879634844955
0.5725 0.209434201254302
0.5775 0.194965968684945
0.5825 0.180195231913223
0.5875 0.165837357957321
0.5925 0.152496015577265
0.5975 0.140600034540706
0.6025 0.130388159134685
0.6075 0.121928998813938
0.6125 0.115158489735012
0.6175 0.10992129549136
0.6225 0.106008879051785
0.6275 0.1031912453944
0.6325 0.101241319233524
0.6375 0.0999518175373785
0.6425 0.099145116658001
0.6475 0.0986771620948313
0.6525 0.098436837862418
0.6575 0.0983423287272417
0.6625 0.0983358767613919
0.6675 0.098378019269526
0.6725 0.0984419881410418
0.6775 0.0985085288225847
0.6825 0.098560999625045
0.6875 0.0985802271650378
0.6925 0.0985381545880715
0.6975 0.0983886978519858
0.7025 0.098053213392447
0.7075 0.0973962545569519
0.7125 0.0961844117902861
0.7175 0.0940167796372562
0.7225 0.0902125241779154
0.7275 0.083658631856229
0.7325 0.0727603303030725
0.7375 0.0562332992635219
0.7425 0.0362830771816603
0.7475 0.0206913795284433
0.7525 0.0135920223442743
0.7575 0.0113884739044625
0.7625 0.0107936392393487
0.7675 0.0106369031078263
0.7725 0.0105956420995906
0.7775 0.0105847780160647
0.7825 0.0105819194090026
0.7875 0.0105811681010425
0.7925 0.0105809709113385
0.7975 0.0105809192354426
0.8025 0.0105809057156192
0.8075 0.0105809021848178
0.8125 0.010580901264515
0.8175 0.0105809010251426
0.8225 0.0105809009630227
0.8275 0.0105809009469414
0.8325 0.0105809009427893
0.8375 0.0105809009417203
0.8425 0.0105809009414458
0.8475 0.0105809009413756
0.8525 0.0105809009413575
0.8575 0.0105809009413528
0.8625 0.0105809009413515
0.8675 0.0105809009413511
0.8725 0.010580900941351
0.8775 0.010580900941351
0.8825 0.0105809009413509
0.8875 0.0105809009413509
0.8925 0.010580900941351
0.8975 0.010580900941351
0.9025 0.010580900941351
0.9075 0.010580900941351
0.9125 0.010580900941351
0.9175 0.010580900941351
0.9225 0.010580900941351
0.9275 0.010580900941351
0.9325 0.010580900941351
0.9375 0.010580900941351
0.9425 0.010580900941351
0.9475 0.010580900941351
0.9525 0.010580900941351
0.9575 0.010580900941351
0.9625 0.010580900941351
0.9675 0.010580900941351
0.9725 0.010580900941351
0.9775 0.010580900941351
0.9825 0.010580900941351
0.9875 0.010580900941351
0.9925 0.010580900941351
0.9975 0.010580900941351
};
\addlegendentry{POD}
\addplot [semithick, color0, mark=pentagon,mark size=2, mark repeat=25, mark options={solid}]
table {%
0.0025 0.500139485923816
0.0075 0.500139485923816
0.0125 0.500139485923816
0.0175 0.500139485923816
0.0225 0.500139485923816
0.0275 0.500139485923816
0.0325 0.500139485923816
0.0375 0.500139485923816
0.0425 0.500139485923816
0.0475 0.500139485923816
0.0525 0.500139485923816
0.0575 0.500139485923816
0.0625 0.500139485923816
0.0675 0.500139485923816
0.0725 0.500139485923816
0.0775 0.500139485923816
0.0825 0.500139485923816
0.0875 0.500139485923816
0.0925 0.500139485923816
0.0975 0.500139337839146
0.1025 0.500139664561154
0.1075 0.500140053000808
0.1125 0.500140032449193
0.1175 0.50014009944212
0.1225 0.500139170521682
0.1275 0.500139534782965
0.1325 0.500139105542311
0.1375 0.500138185339432
0.1425 0.500137333958558
0.1475 0.500134796623042
0.1525 0.500132729140778
0.1575 0.500127458652421
0.1625 0.500117775701708
0.1675 0.5001058112195
0.1725 0.500085240854488
0.1775 0.500052333134403
0.1825 0.500006052722577
0.1875 0.499933946564938
0.1925 0.49983115302702
0.1975 0.49968495226449
0.2025 0.499474429488304
0.2075 0.499182689120001
0.2125 0.498787126899044
0.2175 0.49825185877148
0.2225 0.497545480005185
0.2275 0.496627887781643
0.2325 0.495456468370328
0.2375 0.493983405188992
0.2425 0.492159893074038
0.2475 0.489941042749161
0.2525 0.487277898752358
0.2575 0.484130465234777
0.2625 0.480462314949465
0.2675 0.476245331724927
0.2725 0.471456610676344
0.2775 0.466082120062974
0.2825 0.460123431679082
0.2875 0.453579679275667
0.2925 0.446465684068298
0.2975 0.438804387185797
0.3025 0.430622934544413
0.3075 0.42195525025856
0.3125 0.412802485018499
0.3175 0.403466877668696
0.3225 0.396321453837362
0.3275 0.388971444589845
0.3325 0.381429485228611
0.3375 0.373672873009438
0.3425 0.365829630159456
0.3475 0.357941948292521
0.3525 0.350050858715831
0.3575 0.342199158695723
0.3625 0.334427537992437
0.3675 0.326778298152913
0.3725 0.319286767579964
0.3775 0.311991369028979
0.3825 0.304923709227454
0.3875 0.299794630676976
0.3925 0.295833990796102
0.3975 0.291956744780501
0.4025 0.288179524986253
0.4075 0.28451025739724
0.4125 0.280960201739344
0.4175 0.277541744580041
0.4225 0.274261937117206
0.4275 0.271129019445513
0.4325 0.268151836758685
0.4375 0.265335838249142
0.4425 0.262687329086426
0.4475 0.26021625488235
0.4525 0.257950575928568
0.4575 0.256232802037641
0.4625 0.254710041212527
0.4675 0.253377003987205
0.4725 0.252237557617474
0.4775 0.251288316860622
0.4825 0.250525685925861
0.4875 0.249938056392815
0.4925 0.249507245306465
0.4975 0.24920629573436
0.5025 0.249061294540865
0.5075 0.249070943810637
0.5125 0.248972070308803
0.5175 0.248725492797423
0.5225 0.248417009484471
0.5275 0.248128541729278
0.5325 0.247693876861861
0.5375 0.247010142148361
0.5425 0.245931538450181
0.5475 0.244273021341647
0.5525 0.241818063622992
0.5575 0.238336672633911
0.5625 0.233419846570831
0.5675 0.226627119072287
0.5725 0.218305180222336
0.5775 0.20864871476336
0.5825 0.20447698054689
0.5875 0.199437463367592
0.5925 0.193334002919517
0.5975 0.18621671764822
0.6025 0.179304742818442
0.6075 0.172967307544202
0.6125 0.167468696923938
0.6175 0.162950056093543
0.6225 0.159424681331687
0.6275 0.15681329327236
0.6325 0.154977903434728
0.6375 0.153765165370411
0.6425 0.153018411512027
0.6475 0.152607068606321
0.6525 0.152421798085408
0.6575 0.152384467101201
0.6625 0.152435486528614
0.6675 0.152533989084787
0.6725 0.152645844095092
0.6775 0.152747666541726
0.6825 0.152797543317399
0.6875 0.152754939289622
0.6925 0.152531504325453
0.6975 0.151994031577948
0.7025 0.150906772666807
0.7075 0.148869610413333
0.7125 0.145221857209477
0.7175 0.138899891244964
0.7225 0.128290490040301
0.7275 0.111354335000316
0.7325 0.086708262148705
0.7375 0.0569398717137683
0.7425 0.03263693492199
0.7475 0.0219526916475295
0.7525 0.0196229218111859
0.7575 0.0194009471028487
0.7625 0.0191529856545062
0.7675 0.0169050878237854
0.7725 0.0163095745326533
0.7775 0.0161520862273404
0.7825 0.0161104863280783
0.7875 0.0160987477134576
0.7925 0.0160964296671283
0.7975 0.0160953086071456
0.8025 0.0160959472536843
0.8075 0.0160959472536843
0.8125 0.0160959472536843
0.8175 0.0160953520811119
0.8225 0.0160953520811119
0.8275 0.0160953520811119
0.8325 0.0160953520811119
0.8375 0.0160953520811119
0.8425 0.0160953520811119
0.8475 0.0160953520811119
0.8525 0.0160953520811119
0.8575 0.0160953520811119
0.8625 0.0160953520811119
0.8675 0.0160953520811119
0.8725 0.0160953520811119
0.8775 0.0160953520811119
0.8825 0.0160953520811119
0.8875 0.0160953520811119
0.8925 0.0160953520811119
0.8975 0.0160953520811119
0.9025 0.0160953520811119
0.9075 0.0160953520811119
0.9125 0.0160953520811119
0.9175 0.0160953520811119
0.9225 0.0160953520811119
0.9275 0.0160953520811119
0.9325 0.0160953520811119
0.9375 0.0160953520811119
0.9425 0.0160953520811119
0.9475 0.0160953520811119
0.9525 0.0160953520811119
0.9575 0.0160953520811119
0.9625 0.0160953520811119
0.9675 0.0160953520811119
0.9725 0.0160953520811119
0.9775 0.0160953520811119
0.9825 0.0160953520811119
0.9875 0.0160953520811119
0.9925 0.0160963666699128
0.9975 0.0160963666699128
};
\addlegendentry{FCNN}
\addplot [semithick, green!50!black, mark=triangle,mark size=2, mark repeat=25, mark options={solid,rotate=180}, only marks]
table {%
0.0025 0.543558293894175
0.0075 0.564719463902347
0.0125 0.486404641787632
0.0175 0.4601241306863
0.0225 0.517028663357154
0.0275 0.476395964376408
0.0325 0.536590539303784
0.0375 0.494025085020027
0.0425 0.518386156252738
0.0475 0.563932195575686
0.0525 0.544471356407682
0.0575 0.529741751859604
0.0625 0.524852769033439
0.0675 0.48054148857693
0.0725 0.509762738907053
0.0775 0.46260325847772
0.0825 0.518876710893644
0.0875 0.49531200811389
0.0925 0.513481317519518
0.0975 0.517420025262313
0.1025 0.538449224338755
0.1075 0.539262363166237
0.1125 0.518216694746861
0.1175 0.435467282418118
0.1225 0.52076085186127
0.1275 0.47820874969494
0.1325 0.517141735350883
0.1375 0.516765930194936
0.1425 0.52243399627489
0.1475 0.527320419769502
0.1525 0.550114241409219
0.1575 0.531695089030882
0.1625 0.51769304017175
0.1675 0.464866874134334
0.1725 0.511729709721504
0.1775 0.479527185106051
0.1825 0.518568462647271
0.1875 0.495456249252165
0.1925 0.516289168802967
0.1975 0.528494285579435
0.2025 0.536252548514671
0.2075 0.572956438987581
0.2125 0.481976767684064
0.2175 0.451413984658676
0.2225 0.513133040687513
0.2275 0.483977520622519
0.2325 0.515088378362418
0.2375 0.490332705301279
0.2425 0.494783952540496
0.2475 0.511628825333278
0.2525 0.547262596371474
0.2575 0.544387907382801
0.2625 0.48148458858107
0.2675 0.421920586146238
0.2725 0.449918034460684
0.2775 0.420957393266648
0.2825 0.441483571354365
0.2875 0.394088438775616
0.2925 0.403846208936138
0.2975 0.383613786105359
0.3025 0.490328932948911
0.3075 0.485678217644804
0.3125 0.412746727264713
0.3175 0.325566006093673
0.3225 0.353423677089089
0.3275 0.305282501316173
0.3325 0.324139350748391
0.3375 0.292916635336728
0.3425 0.26059866911807
0.3475 0.247887060706375
0.3525 0.354876640329637
0.3575 0.376811035625196
0.3625 0.321815097377393
0.3675 0.254753381668985
0.3725 0.252858442030585
0.3775 0.237525447295275
0.3825 0.24259423141391
0.3875 0.210036117012141
0.3925 0.194386506348762
0.3975 0.222260084215196
0.4025 0.296486751089792
0.4075 0.308673791705261
0.4125 0.26453340961612
0.4175 0.251622232447928
0.4225 0.226146376265577
0.4275 0.21788043668769
0.4325 0.234018365552338
0.4375 0.200193079577118
0.4425 0.201033990521449
0.4475 0.186992294735369
0.4525 0.32156011729418
0.4575 0.271817326626354
0.4625 0.267915981140165
0.4675 0.236352221863691
0.4725 0.204120746454436
0.4775 0.208420357613006
0.4825 0.282898340619552
0.4875 0.199971060732828
0.4925 0.18401794121326
0.4975 0.184593350978308
0.5025 0.192440951769475
0.5075 0.187925647335276
0.5125 0.240410123786211
0.5175 0.212681354709498
0.5225 0.191993619425629
0.5275 0.193111172736345
0.5325 0.255530934199263
0.5375 0.212191427819647
0.5425 0.221085225203747
0.5475 0.303845017639265
0.5525 0.107611224802143
0.5575 0.0325796382493863
0.5625 0.249607360868059
0.5675 0.255060824832714
0.5725 0.14047186234441
0.5775 0.210359369342283
0.5825 0.206080529708012
0.5875 0.183253303998363
0.5925 0.255374057500228
0.5975 0.299364379671689
0.6025 0.215090411894404
0.6075 0.233807654925717
0.6125 0.226816462507188
0.6175 0.184526397879567
0.6225 0.229623984801968
0.6275 0.191642649041512
0.6325 0.265899799760188
0.6375 0.260917350698222
0.6425 0.244879923956296
0.6475 0.228028651938125
0.6525 0.249847368340966
0.6575 0.243229273804138
0.6625 0.296909367638542
0.6675 0.232077167832147
0.6725 0.264724841277657
0.6775 0.223377018923797
0.6825 0.21815616185635
0.6875 0.248097024084459
0.6925 0.239170664698377
0.6975 0.218609684796004
0.7025 0.272515116035421
0.7075 0.327595080283301
0.7125 0.284486908898688
0.7175 0.167183262653453
0.7225 0.164428335091317
0.7275 0.0991260432094094
0.7325 0.0273982976520321
0.7375 0.014485525287322
0.7425 -0.0875282007271037
0.7475 -0.0199902619341918
0.7525 0.00654694414887855
0.7575 -0.00144983336186735
0.7625 0.0627840533342506
0.7675 0.0114020638858208
0.7725 0.0286057376945837
0.7775 0.0286284919462966
0.7825 -0.00474399228678573
0.7875 0.0332732077510281
0.7925 0.0195773043842302
0.7975 0.014646074613457
0.8025 0.0139425280513923
0.8075 0.00552939850931217
0.8125 0.083673117384618
0.8175 0.00489534613105954
0.8225 0.048903442754067
0.8275 0.0224436954248665
0.8325 -0.00475785971724294
0.8375 0.0476093448454647
0.8425 0.0281479499324392
0.8475 0.0246747690691391
0.8525 0.00463177649102247
0.8575 -0.000518744724103201
0.8625 0.0730632628028802
0.8675 0.00764964093203847
0.8725 0.0373482973088953
0.8775 0.014372019074874
0.8825 -0.0106741840211157
0.8875 0.0454520033668718
0.8925 0.0324108889687051
0.8975 0.0281011226368596
0.9025 -0.0127392373305129
0.9075 0.0110874287660044
0.9125 0.0791579655843602
0.9175 0.0226732046121916
0.9225 0.0326002872797183
0.9275 0.0200573920925657
0.9325 -0.00491472365550866
0.9375 0.030061259064328
0.9425 0.0292240598674576
0.9475 0.0162076003648492
0.9525 0.0122862958556062
0.9575 0.000873731331321777
0.9625 0.073119140238547
0.9675 -0.00348471523310695
0.9725 0.0338053387048922
0.9775 0.0386164095534954
0.9825 -0.00653518769833601
0.9875 0.0529919584102714
0.9925 0.0254556437456522
0.9975 0.0325447631810872
};
\addlegendentry{CNN}

\nextgroupplot[
legend cell align={left},
legend style={at={(1,1)},anchor=north east,fill opacity=0.1, draw opacity=1, text opacity=1,draw=none},
tick align=outside,
tick pos=left,
x grid style={white!69.0196078431373!black},
xmajorgrids,
xlabel={\(x\)},
xmin=-0.04725, xmax=1.04725,
xtick style={color=black},
y grid style={white!69.0196078431373!black},
ymajorgrids,
ylabel={\(\rho\)},
ymin=0.0798726185710705, ymax=1.04764448950172,
ytick style={color=black},
width=.35\textwidth,
height=.4\textwidth
]
\addplot [semithick, black, dashed, mark=x,mark size=2, mark repeat=25, mark options={solid}]
table {%
0.0025 0.999999779921312
0.0075 0.999999779921312
0.0125 0.999999657655373
0.0175 0.999999535389436
0.0225 0.999999413123497
0.0275 0.999999290857559
0.0325 0.999999046325683
0.0375 0.999998740660838
0.0425 0.999998373863024
0.0475 0.999997823666304
0.0525 0.999997273469582
0.0575 0.999996295342078
0.0625 0.999995194948637
0.0675 0.99999378889035
0.0725 0.999991954901279
0.0775 0.999989570715488
0.0825 0.9999863918011
0.0875 0.999982295892177
0.0925 0.999976855057936
0.0975 0.999969947032439
0.1025 0.999961082751934
0.1075 0.999949650886731
0.1125 0.999934978974171
0.1175 0.999916394551595
0.1225 0.999892797225561
0.1275 0.999862780937782
0.1325 0.999825061895908
0.1375 0.999777378180088
0.1425 0.999717895801251
0.1475 0.999643741509853
0.1525 0.999551736391508
0.1575 0.999437967936198
0.1625 0.999298279102032
0.1675 0.999127473586645
0.1725 0.998919682625012
0.1775 0.998667937058669
0.1825 0.998364900931334
0.1875 0.998001954494378
0.1925 0.997569927802453
0.1975 0.99705818371895
0.2025 0.996455473777575
0.2075 0.995749815916404
0.2125 0.9949280665471
0.2175 0.993976654150547
0.2225 0.992881823808719
0.2275 0.991628597944211
0.2325 0.990202182378524
0.2375 0.988587660667224
0.2425 0.986770177498842
0.2475 0.984735733423477
0.2525 0.982469656528571
0.2575 0.9799591700236
0.2625 0.977191680516952
0.2675 0.974155389345609
0.2725 0.970840392968593
0.2775 0.967237154642741
0.2825 0.963338277278802
0.2875 0.959137158516126
0.2925 0.954628418653439
0.2975 0.94980857311151
0.3025 0.944675054305639
0.3075 0.939226700709416
0.3125 0.933463695721748
0.3175 0.927387078603109
0.3225 0.920999539204133
0.3275 0.914303950774364
0.3325 0.907304959419446
0.3375 0.900007516909868
0.3425 0.892417308611748
0.3475 0.884541609348395
0.3525 0.8763871437464
0.3575 0.867962531554393
0.3625 0.859276147989126
0.3675 0.85033716299595
0.3725 0.841155174450997
0.3775 0.831740636091966
0.3825 0.822103879390619
0.3875 0.812256152813251
0.3925 0.802208949358035
0.3975 0.791975351480337
0.4025 0.78156923636412
0.4075 0.771006681980231
0.4125 0.760306517283122
0.4175 0.749489466349284
0.4225 0.738579676701472
0.4275 0.727603680048233
0.4325 0.716589719821245
0.4375 0.705566895313752
0.4425 0.694562043899145
0.4475 0.683599679897993
0.4525 0.672698754530687
0.4575 0.661872717050406
0.4625 0.651134038582826
0.4675 0.640501914880215
0.4725 0.630014064984444
0.4775 0.619732783390925
0.4825 0.609731490795429
0.4875 0.600058298844558
0.4925 0.59069615143996
0.4975 0.581551392873128
0.5025 0.572476509289864
0.5075 0.56329397054819
0.5125 0.553800264994303
0.5175 0.543781549502642
0.5225 0.533063533978584
0.5275 0.521566317631648
0.5325 0.509313956285134
0.5375 0.496400778110211
0.5425 0.482943394245245
0.5475 0.469048328888722
0.5525 0.454804377678113
0.5575 0.440289851946708
0.5625 0.425583521525065
0.5675 0.410771675598927
0.5725 0.395950781993377
0.5775 0.381227089808537
0.5825 0.366713603337606
0.5875 0.352526994851919
0.5925 0.338781399604602
0.5975 0.325581202140221
0.6025 0.31301281391046
0.6075 0.301137948647524
0.6125 0.289988915125529
0.6175 0.279568556027535
0.6225 0.269853487992898
0.6275 0.260800795677381
0.6325 0.25235619300451
0.6375 0.244461878752097
0.6425 0.237063108346401
0.6475 0.230112029955937
0.6525 0.223569457347576
0.6575 0.217404182140644
0.6625 0.211590895285973
0.6675 0.206107787596874
0.6725 0.200933844615252
0.6775 0.196047272437658
0.6825 0.191424565437512
0.6875 0.187040689664009
0.6925 0.182870030403137
0.6975 0.178888042767843
0.7025 0.175072520207136
0.7075 0.171405245096256
0.7125 0.167872875164717
0.7175 0.164467356143854
0.7225 0.161185830067366
0.7275 0.158030222623776
0.7325 0.155006219179202
0.7375 0.15212214910067
0.7425 0.149387808946463
0.7475 0.146813270373222
0.7525 0.144407795025752
0.7575 0.142178871692755
0.7625 0.140131299312298
0.7675 0.138266850740482
0.7725 0.136584028219565
0.7775 0.135078032811483
0.7825 0.133741039496202
0.7875 0.132562930767353
0.7925 0.131531541164105
0.7975 0.130633803514334
0.8025 0.129855947616773
0.8075 0.12918458535121
0.8125 0.128606618979038
0.8175 0.128109967097258
0.8225 0.127683602846586
0.8275 0.127317637969286
0.8325 0.127003406867003
0.8375 0.12673332905158
0.8425 0.126500924428304
0.8475 0.126300606972132
0.8525 0.126127707652557
0.8575 0.125978222260108
0.8625 0.125848811406356
0.8675 0.125736601841755
0.8725 0.125639255230243
0.8775 0.125554723617358
0.8825 0.125481257071862
0.8875 0.125417434252225
0.8925 0.125361956082858
0.8975 0.125313752736801
0.9025 0.125271861369793
0.9075 0.125235441403511
0.9125 0.125203828016917
0.9175 0.125176386955457
0.9225 0.125152552739168
0.9275 0.12513185158754
0.9325 0.125113924344381
0.9375 0.125098358362149
0.9425 0.125084847976
0.9475 0.1250731639373
0.9525 0.12506304643093
0.9575 0.125054266208257
0.9625 0.125046716286586
0.9675 0.125040175058903
0.9725 0.125034558467376
0.9775 0.125029751887688
0.9825 0.125025663620386
0.9875 0.125022209607638
0.9925 0.125019313433231
0.9975 0.12501672292367
};
\addlegendentry{FOM}
\addplot [semithick, red, mark=o,mark size=2, mark repeat=25, mark options={solid}]
table {%
0.0025 0.998698302494394
0.0075 0.998698273326595
0.0125 0.998698239455942
0.0175 0.998698196204499
0.0225 0.998698139392356
0.0275 0.998698064184519
0.0325 0.998697964451367
0.0375 0.998697832197717
0.0425 0.998697656911356
0.0475 0.998697424751913
0.0525 0.998697117513723
0.0575 0.998696711291434
0.0625 0.998696174764184
0.0675 0.998695466996326
0.0725 0.998694534630346
0.0775 0.998693308321378
0.0825 0.998691698232338
0.0875 0.998689588374572
0.0925 0.99868682954154
0.0975 0.998683230543227
0.1025 0.998678547408363
0.1075 0.998672470182162
0.1125 0.998664606912185
0.1175 0.998654464388049
0.1225 0.998641425186778
0.1275 0.9986247205804
0.1325 0.998603398892314
0.1375 0.99857628895073
0.1425 0.998541958387887
0.1475 0.99849866667837
0.1525 0.998444313002865
0.1575 0.998376379266358
0.1625 0.998291868889386
0.1675 0.998187242320423
0.1725 0.998058350573667
0.1775 0.997900368461072
0.1825 0.997707729536365
0.1875 0.9974740650741
0.1925 0.997192149638342
0.1975 0.996853855924052
0.2025 0.996450121554054
0.2075 0.99597093036793
0.2125 0.995405310438635
0.2175 0.99474135060363
0.2225 0.993966236718855
0.2275 0.993066308167807
0.2325 0.992027134426715
0.2375 0.990833610749002
0.2425 0.989470071338078
0.2475 0.987920417773415
0.2525 0.986168259978534
0.2575 0.9841970666972
0.2625 0.981990322287675
0.2675 0.979531686654096
0.2725 0.976805155297111
0.2775 0.973795216764851
0.2825 0.970487005198974
0.2875 0.966866446178896
0.2925 0.962920394652705
0.2975 0.958636764388681
0.3025 0.954004649065567
0.3075 0.949014435808484
0.3125 0.943657912613614
0.3175 0.937928371599843
0.3225 0.93182071025845
0.3275 0.925331532698594
0.3325 0.918459252169867
0.3375 0.911204194800864
0.3425 0.903568702564973
0.3475 0.895557231209375
0.3525 0.887176436759925
0.3575 0.878435243021843
0.3625 0.869344883220239
0.3675 0.859918912545435
0.3725 0.850173195462653
0.3775 0.840125881810801
0.3825 0.829797396918779
0.3875 0.819210479104472
0.3925 0.808390297036259
0.3975 0.797364663162253
0.4025 0.786164323909006
0.4075 0.774823255248402
0.4125 0.763378835781004
0.4175 0.751871729969605
0.4225 0.740345314837129
0.4275 0.72884453647866
0.4325 0.717414177076701
0.4375 0.706096617458577
0.4425 0.694929275129902
0.4475 0.683942030409673
0.4525 0.673155258667752
0.4575 0.662579656940736
0.4625 0.652219569245489
0.4675 0.642080857227587
0.4725 0.632181024694021
0.4775 0.622553591742255
0.4825 0.613235751217255
0.4875 0.604236632686526
0.4925 0.595503679651282
0.4975 0.58691746306255
0.5025 0.578316110388185
0.5075 0.569516845631515
0.5125 0.560315441909217
0.5175 0.550494569498016
0.5225 0.53986720883965
0.5275 0.528329501861953
0.5325 0.515876607865928
0.5375 0.502578645462612
0.5425 0.488540690791141
0.5475 0.47387225333178
0.5525 0.45867571445608
0.5575 0.443048559957467
0.5625 0.427090122284292
0.5675 0.410907036722874
0.5725 0.394616081166999
0.5775 0.378344992711069
0.5825 0.362231527641442
0.5875 0.346420198056971
0.5925 0.331056022544265
0.5975 0.316275462181985
0.6025 0.302195939753015
0.6075 0.28890623510095
0.6125 0.276460125959793
0.6175 0.26487484970709
0.6225 0.254134642782134
0.6275 0.244198300488583
0.6325 0.235008843217355
0.6375 0.226503176654411
0.6425 0.218620019900139
0.6475 0.211305100244882
0.6525 0.204513397474426
0.6575 0.198208855190955
0.6625 0.192362360243628
0.6675 0.186948915262344
0.6725 0.181944842188357
0.6775 0.177325630078609
0.6825 0.173064755079178
0.6875 0.169133521968747
0.6925 0.165501755966607
0.6975 0.162139039797925
0.7025 0.159016150284471
0.7075 0.156106386657481
0.7125 0.153386571843381
0.7175 0.150837616519517
0.7225 0.14844463628612
0.7275 0.146196686747079
0.7325 0.14408622243703
0.7375 0.142108395532208
0.7425 0.140260297366477
0.7475 0.138540220581907
0.7525 0.136946992218088
0.7575 0.135479405324883
0.7625 0.134135762100721
0.7675 0.132913534690799
0.7725 0.131809147455078
0.7775 0.130817882532615
0.7825 0.129933905610679
0.7875 0.129150400077716
0.7925 0.128459787035304
0.7975 0.127853999478442
0.8025 0.12732477467171
0.8075 0.126863930901653
0.8125 0.126463602559474
0.8175 0.126116418355617
0.8225 0.125815618386506
0.8275 0.125555114458726
0.8325 0.125329503464708
0.8375 0.125134045782418
0.8425 0.124964620423029
0.8475 0.124817666963882
0.8525 0.124690122035919
0.8575 0.12457935586753
0.8625 0.124483112430963
0.8675 0.124399455202832
0.8725 0.124326719429532
0.8775 0.124263471020959
0.8825 0.124208471708138
0.8875 0.124160649823498
0.8925 0.124119075938947
0.8975 0.124082942580037
0.9025 0.124051547286842
0.9075 0.124024278383951
0.9125 0.124000602930333
0.9175 0.12398005642796
0.9225 0.123962233965332
0.9275 0.123946782552754
0.9325 0.123933394468968
0.9375 0.123921801485361
0.9425 0.123911769868239
0.9475 0.123903096086674
0.9525 0.123895603178997
0.9575 0.123889137761161
0.9625 0.123883567699913
0.9675 0.123878780521018
0.9725 0.123874682643982
0.9775 0.12387119937193
0.9825 0.123868274619189
0.9875 0.123865865617798
0.9925 0.123863915343346
0.9975 0.123862249067918
};
\addlegendentry{POD}
\addplot [semithick, color0, mark=pentagon,mark size=2, mark repeat=25, mark options={solid}]
table {%
0.0025 0.999939748539756
0.0075 0.999939691227598
0.0125 0.999939575170477
0.0175 0.999939472486193
0.0225 0.999939326817791
0.0275 0.999939119658218
0.0325 0.999938851962678
0.0375 0.999938460807197
0.0425 0.999938105830015
0.0475 0.999937524469808
0.0525 0.999936729741211
0.0575 0.999936003190202
0.0625 0.999934709965227
0.0675 0.999932965765206
0.0725 0.999930800320819
0.0775 0.999928111067185
0.0825 0.999924475565935
0.0875 0.999919676030867
0.0925 0.999913643926191
0.0975 0.999905828458185
0.1025 0.999895610297337
0.1075 0.999882567363481
0.1125 0.999865992209659
0.1175 0.99984476951739
0.1225 0.99981783853414
0.1275 0.999783773619968
0.1325 0.999740784844527
0.1375 0.999687083949072
0.1425 0.999619722103652
0.1475 0.999535844685175
0.1525 0.999432137259879
0.1575 0.999304587976673
0.1625 0.999147707763582
0.1675 0.998956547238124
0.1725 0.998724333297175
0.1775 0.998444300837433
0.1825 0.998107602055638
0.1875 0.99770568526135
0.1925 0.997228300055632
0.1975 0.9966642386877
0.2025 0.996001928399962
0.2075 0.995228307154507
0.2125 0.994329543975301
0.2175 0.993291718813662
0.2225 0.992099572904408
0.2275 0.990737577995811
0.2325 0.989190108524874
0.2375 0.987441065745094
0.2425 0.985474502070783
0.2475 0.98327452615381
0.2525 0.980825564489724
0.2575 0.978112902778845
0.2625 0.975121886541064
0.2675 0.971839267115753
0.2725 0.968252209564432
0.2775 0.964349209665297
0.2825 0.960120758017859
0.2875 0.95555768861698
0.2925 0.950652893322209
0.2975 0.945400731781354
0.3025 0.939797690878503
0.3075 0.934260087445951
0.3125 0.928395335825208
0.3175 0.922133065401935
0.3225 0.915521407284989
0.3275 0.908563889276523
0.3325 0.901286943624608
0.3375 0.893947889264195
0.3425 0.886281654238701
0.3475 0.878299870528281
0.3525 0.870015412115325
0.3575 0.861444172138969
0.3625 0.852603819460059
0.3675 0.843514560435254
0.3725 0.834197588145542
0.3775 0.824746730832908
0.3825 0.815773046551607
0.3875 0.806606280593536
0.3925 0.797267998807514
0.3975 0.787780041424319
0.4025 0.778165214981597
0.4075 0.768446971256381
0.4125 0.758649606544238
0.4175 0.748798172586621
0.4225 0.738154956115744
0.4275 0.727441433506707
0.4325 0.716651244184528
0.4375 0.705778407625472
0.4425 0.694845443925796
0.4475 0.683870809510923
0.4525 0.672867709579758
0.4575 0.661843689994361
0.4625 0.650858930192697
0.4675 0.639988203795674
0.4725 0.629177449438243
0.4775 0.61877694995835
0.4825 0.608734680124773
0.4875 0.598905063950672
0.4925 0.589268625331804
0.4975 0.579727338746381
0.5025 0.570124700569954
0.5075 0.561270500557163
0.5125 0.552479314068571
0.5175 0.543034219612869
0.5225 0.53274636144917
0.5275 0.521713407375874
0.5325 0.510640766304464
0.5375 0.498740428413909
0.5425 0.48603227457557
0.5475 0.47260600894403
0.5525 0.458931265733181
0.5575 0.445353894924315
0.5625 0.431195471244745
0.5675 0.416837310752807
0.5725 0.402233228445626
0.5775 0.387502070158147
0.5825 0.372775582333979
0.5875 0.358234208960755
0.5925 0.344228067268164
0.5975 0.330641503708485
0.6025 0.317592416197444
0.6075 0.3051739003366
0.6125 0.293445707394336
0.6175 0.282433894701684
0.6225 0.272132290097383
0.6275 0.262509962209524
0.6325 0.253519554717992
0.6375 0.245107069301109
0.6425 0.237219013775197
0.6475 0.229807604725162
0.6525 0.222833372939091
0.6575 0.216264377754086
0.6625 0.210074918607298
0.6675 0.204242818559018
0.6725 0.198746776590363
0.6775 0.193564373020751
0.6825 0.188671609339042
0.6875 0.184042406125137
0.6925 0.179650091494505
0.6975 0.175468535281909
0.7025 0.171473949717788
0.7075 0.168071820114094
0.7125 0.164805877452286
0.7175 0.161660612823489
0.7225 0.158631950497436
0.7275 0.155719912443788
0.7325 0.152928386217891
0.7375 0.150263438908718
0.7425 0.147733457076053
0.7475 0.145346669910046
0.7525 0.143111342301544
0.7575 0.141034556839329
0.7625 0.139121096103619
0.7675 0.137373151329274
0.7725 0.135790394762388
0.7775 0.134369332749301
0.7825 0.133103761248864
0.7875 0.131985244985956
0.7925 0.131003591829003
0.7975 0.130147187946699
0.8025 0.129404256168084
0.8075 0.128762269846331
0.8125 0.128209807384664
0.8175 0.127735711538639
0.8225 0.127329558778841
0.8275 0.126982382140481
0.8325 0.126685762109283
0.8375 0.126432687378465
0.8425 0.126216820297906
0.8475 0.126032876328398
0.8525 0.125876132112283
0.8575 0.125742860926458
0.8625 0.125629545476001
0.8675 0.125533380091954
0.8725 0.125452004229793
0.8775 0.125383300682864
0.8825 0.125325494804061
0.8875 0.125277080597022
0.8925 0.125236646272242
0.8975 0.125203171434502
0.9025 0.125166063364118
0.9075 0.125128739537337
0.9125 0.125097310027251
0.9175 0.125071190966245
0.9225 0.125049355989083
0.9275 0.125031324748236
0.9325 0.125016331648788
0.9375 0.125003891686598
0.9425 0.124993549468808
0.9475 0.124984951570439
0.9525 0.124977916861192
0.9575 0.124972011440266
0.9625 0.124967053580361
0.9675 0.124962959462442
0.9725 0.124959461987974
0.9775 0.124956497635979
0.9825 0.124954251118768
0.9875 0.124952227641375
0.9925 0.124950619318928
0.9975 0.124949141142842
};
\addlegendentry{FCNN}
\addplot [semithick, green!50!black, mark=triangle,mark size=2, mark repeat=25, mark options={solid,rotate=180}, only marks]
table {%
0.0025 0.999841690063476
0.0075 1.00157768298418
0.0125 1.00145187133398
0.0175 1.0025885166266
0.0225 1.00229238852476
0.0275 1.00162512216813
0.0325 1.00365485900488
0.0375 1.00085276823777
0.0425 1.00248098373413
0.0475 1.00255330403646
0.0525 1.00001897567358
0.0575 1.0008953167842
0.0625 1.00257946894719
0.0675 1.00317575992682
0.0725 1.00149668180026
0.0775 1.00117781223395
0.0825 1.00353552744939
0.0875 1.00201038213877
0.0925 1.00221505531898
0.0975 1.00140865032489
0.1025 0.999895792741042
0.1075 1.00124542529766
0.1125 1.00264225250635
0.1175 1.0017083852719
0.1225 1.00179916773087
0.1275 1.00197902092567
0.1325 1.00310056637495
0.1375 1.00264298610198
0.1425 1.00230583777794
0.1475 1.00166803751236
0.1525 1.00041083800487
0.1575 1.0013484954834
0.1625 1.00191593170166
0.1675 1.00190278811332
0.1725 1.00161350690402
0.1775 1.00206307875804
0.1825 1.0031028282948
0.1875 1.00196043650309
0.1925 1.00189838653956
0.1975 1.00154002507528
0.2025 0.999436256213066
0.2075 1.0013634730608
0.2125 1.00098970608834
0.2175 1.00079811536349
0.2225 1.00017345868624
0.2275 0.999446832216703
0.2325 0.999572582733937
0.2375 0.997920097448887
0.2425 0.996778928316556
0.2475 0.995197296142578
0.2525 0.991145830888014
0.2575 0.990359905438545
0.2625 0.987404432052221
0.2675 0.985120198665521
0.2725 0.981604686150184
0.2775 0.977559273059551
0.2825 0.97483420983339
0.2875 0.969669390947391
0.2925 0.965662185962384
0.2975 0.959623899215307
0.3025 0.950862566630046
0.3075 0.944992456680689
0.3125 0.940355887779823
0.3175 0.932353826669546
0.3225 0.923714454357441
0.3275 0.916518431443434
0.3325 0.908176471025516
0.3375 0.898656294896052
0.3425 0.893710270906106
0.3475 0.881794905051207
0.3525 0.871296845949613
0.3575 0.860777695973714
0.3625 0.855641059386424
0.3675 0.843016123160338
0.3725 0.830003237112974
0.3775 0.82032350393442
0.3825 0.80804372445131
0.3875 0.793726872175168
0.3925 0.788378043052478
0.3975 0.773017345330654
0.4025 0.759121271280142
0.4075 0.74831852546105
0.4125 0.737922925215501
0.4175 0.724308307354267
0.4225 0.708511059100811
0.4275 0.695907947344658
0.4325 0.683895563467955
0.4375 0.668823841290596
0.4425 0.660165028694348
0.4475 0.643280286055345
0.4525 0.62992621690799
0.4575 0.619602692432893
0.4625 0.609846298511212
0.4675 0.596762926150591
0.4725 0.584436991275885
0.4775 0.575734774271647
0.4825 0.571940617683606
0.4875 0.562261067903959
0.4925 0.558517773946126
0.4975 0.553749585763002
0.5025 0.550055992908967
0.5075 0.550579352256579
0.5125 0.552320113548866
0.5175 0.551389608627711
0.5225 0.547289970593575
0.5275 0.547026242965307
0.5325 0.545610464536227
0.5375 0.541077577150785
0.5425 0.538295171199701
0.5475 0.530666082333296
0.5525 0.515002654148982
0.5575 0.504624813030928
0.5625 0.487025059186495
0.5675 0.467985654488588
0.5725 0.440198397025084
0.5775 0.412148390060816
0.5825 0.379088016656729
0.5875 0.350631444882124
0.5925 0.32552040540255
0.5975 0.295807214883658
0.6025 0.270062104249612
0.6075 0.257281187253121
0.6125 0.244320967258551
0.6175 0.230089899821159
0.6225 0.221662123998006
0.6275 0.216149687767029
0.6325 0.206605058449965
0.6375 0.209021415465917
0.6425 0.206697491499094
0.6475 0.207482316555121
0.6525 0.200891158519647
0.6575 0.202509531607995
0.6625 0.204399564327338
0.6675 0.203319940811548
0.6725 0.203927388558021
0.6775 0.204356251618801
0.6825 0.204821198414534
0.6875 0.206904197350526
0.6925 0.203668887798603
0.6975 0.204570262860029
0.7025 0.198694345278618
0.7075 0.201351443926493
0.7125 0.201564446473733
0.7175 0.197433798741072
0.7225 0.192084633387052
0.7275 0.183948400693062
0.7325 0.17380511149382
0.7375 0.161192783942589
0.7425 0.148635521913186
0.7475 0.134891898204119
0.7525 0.128009418646495
0.7575 0.126178065935771
0.7625 0.12883050319476
0.7675 0.124788498267149
0.7725 0.126260175154759
0.7775 0.126406260025807
0.7825 0.124152524349017
0.7875 0.127139137341426
0.7925 0.126614364293905
0.7975 0.126168972406632
0.8025 0.125105999983274
0.8075 0.124403765568366
0.8125 0.128735242745815
0.8175 0.124683991456643
0.8225 0.126545711969718
0.8275 0.125889159165896
0.8325 0.12448312380375
0.8375 0.12731501689324
0.8425 0.125849651984679
0.8475 0.125490801456647
0.8525 0.124697043345525
0.8575 0.124707451233497
0.8625 0.129072467486064
0.8675 0.124883437768007
0.8725 0.126796662807465
0.8775 0.125949818354387
0.8825 0.12470733660918
0.8875 0.126544053737934
0.8925 0.125861175549336
0.8975 0.125687771882766
0.9025 0.123987350708399
0.9075 0.12474822692382
0.9125 0.128930776547163
0.9175 0.125443599162958
0.9225 0.126453729776236
0.9275 0.125518494691604
0.9325 0.124389139505533
0.9375 0.126425318228893
0.9425 0.126059972322904
0.9475 0.126345539704347
0.9525 0.12450581177687
0.9575 0.12480697570703
0.9625 0.128966294802152
0.9675 0.124523387505458
0.9725 0.126605813319866
0.9775 0.126972465943067
0.9825 0.12454058879461
0.9875 0.127211052637834
0.9925 0.126055853489118
0.9975 0.126497898346339
};
\addlegendentry{CNN}

\nextgroupplot[
legend cell align={left},
legend style={at={(0.0,1)},anchor=north west, opacity=0.1, draw opacity=1, text opacity=1,draw=none},
tick align=outside,
tick pos=left,
x grid style={white!69.0196078431373!black},
xmajorgrids,
xlabel={\(x\)},
xmin=-0.04725, xmax=1.04725,
xtick style={color=black},
y grid style={white!69.0196078431373!black},
ymajorgrids,
ylabel={\(\rho u\)},
ymin=-0.0232244711945436, ymax=0.473814120212489,
ytick style={color=black},
width=.37\textwidth,
height=.4\textwidth
]
\addplot [semithick, black, dashed, mark=x,mark size=2, mark repeat=25, mark options={solid}]
table {%
0.0025 4.4436745499885e-07
0.0075 5.97062470073428e-07
0.0125 7.99408270642964e-07
0.0175 1.04705922162596e-06
0.0225 1.35567708717111e-06
0.0275 1.75172552417413e-06
0.0325 2.30844744858165e-06
0.0375 2.9880815542495e-06
0.0425 3.90773662630503e-06
0.0475 5.09393142708624e-06
0.0525 6.65892163132399e-06
0.0575 8.65601756644673e-06
0.0625 1.12774571035243e-05
0.0675 1.46677716873861e-05
0.0725 1.90850619305997e-05
0.0775 2.48119359347914e-05
0.0825 3.21826385237286e-05
0.0875 4.1699756025494e-05
0.0925 5.3943741472899e-05
0.0975 6.96811966648594e-05
0.1025 8.97875285320438e-05
0.1075 0.000115478392117381
0.1125 0.000148168627473818
0.1175 0.00018966355103918
0.1225 0.000242071494594385
0.1275 0.000308100613199496
0.1325 0.000390947667786122
0.1375 0.000494485968892232
0.1425 0.000623332477072447
0.1475 0.000782953325326417
0.1525 0.000979808754148224
0.1575 0.00122144670973211
0.1625 0.00151658846851006
0.1675 0.00187525642143297
0.1725 0.00230887351456253
0.1775 0.00283029961996043
0.1825 0.00345390691125881
0.1875 0.00419554200717569
0.1925 0.00507257145137372
0.1975 0.006103693005683
0.2025 0.00730891394626128
0.2075 0.00870933779277978
0.2125 0.010326898468218
0.2175 0.0121841129234184
0.2225 0.0143037056518193
0.2275 0.0167083646052653
0.2325 0.0194201102405193
0.2375 0.0224601437913052
0.2425 0.0258481733743399
0.2475 0.0296022484508076
0.2525 0.0337381343206392
0.2575 0.0382691659979407
0.2625 0.0432057288000136
0.2675 0.0485552556828182
0.2725 0.0543217941540574
0.2775 0.0605060639270607
0.2825 0.0671053302971474
0.2875 0.0741133599483847
0.2925 0.0815206213645865
0.2975 0.0893142990379972
0.3025 0.0974785360270983
0.3075 0.105994578817519
0.3125 0.114841105037565
0.3175 0.12399437397005
0.3225 0.133428589794627
0.3275 0.143116059063904
0.3325 0.153027585489409
0.3375 0.163132590884043
0.3425 0.173399399495716
0.3475 0.183795462230394
0.3525 0.194287584801184
0.3575 0.204842058018844
0.3625 0.215424936160414
0.3675 0.226002099564956
0.3725 0.236539589269847
0.3775 0.247003563785125
0.3825 0.257360487140066
0.3875 0.26757719008974
0.3925 0.277620758515484
0.3975 0.287458476938765
0.4025 0.297057744161116
0.4075 0.306385832842246
0.4125 0.315409954800661
0.4175 0.324097309196729
0.4225 0.332415449030078
0.4275 0.340332824538059
0.4325 0.347819504512935
0.4375 0.354847972163575
0.4425 0.361393558522961
0.4475 0.367434776935423
0.4525 0.372952737717246
0.4575 0.377929946760492
0.4625 0.382348185014653
0.4675 0.386186386435815
0.4725 0.389419924337175
0.4775 0.392023888666256
0.4825 0.393981529642316
0.4875 0.395294662164388
0.4925 0.395987916377527
0.4975 0.39609954903512
0.5025 0.395658922689244
0.5075 0.394662327013748
0.5125 0.393063103862165
0.5175 0.390783103999483
0.5225 0.387738694978082
0.5275 0.383869346410937
0.5325 0.379155366346474
0.5375 0.373617303066399
0.5425 0.36730289652528
0.5475 0.360271903528538
0.5525 0.352586221791485
0.5575 0.344306220689253
0.5625 0.335491438546631
0.5675 0.326203071289521
0.5725 0.316507211331304
0.5775 0.30647779078902
0.5825 0.296198456339372
0.5875 0.285762286750092
0.5925 0.275268547265266
0.5975 0.264816739940731
0.6025 0.254498650937767
0.6075 0.244390586966174
0.6125 0.234547403068611
0.6175 0.225000009951617
0.6225 0.215756629720444
0.6275 0.206807463826392
0.6325 0.198131204950899
0.6375 0.189702048050083
0.6425 0.181495519782867
0.6475 0.173492243942148
0.6525 0.165679454926409
0.6575 0.15805024700771
0.6625 0.150601350812323
0.6675 0.143330324967283
0.6725 0.136232905729637
0.6775 0.129301264105613
0.6825 0.122523504835293
0.6875 0.115884391171121
0.6925 0.109367185701245
0.6975 0.102956018945085
0.7025 0.0966384288349653
0.7075 0.0904075365474779
0.7125 0.0842635834942107
0.7175 0.0782146371436074
0.7225 0.0722764392491349
0.7275 0.0664714987092887
0.7325 0.06082760069483
0.7375 0.0553758961220649
0.7425 0.0501487940419877
0.7475 0.0451777667426355
0.7525 0.0404913098241465
0.7575 0.0361131099955101
0.7625 0.032060594644974
0.7675 0.0283440095410646
0.7725 0.024966001985276
0.7775 0.0219218460910474
0.7825 0.0192001441371101
0.7875 0.016783957884182
0.7925 0.014652206355614
0.7975 0.0127811044817779
0.8025 0.0111455446111314
0.8075 0.009720295606457
0.8125 0.00848092617497202
0.8175 0.00740448493035834
0.8225 0.00646991604531763
0.8275 0.00565827669150929
0.8325 0.00495279857000134
0.8375 0.00433881857159492
0.8425 0.00380365125281355
0.8475 0.00333640757172043
0.8525 0.00292779445498711
0.8575 0.00256990433168742
0.8625 0.00225601870974582
0.8675 0.001980431041221
0.8725 0.00173826783925844
0.8775 0.00152536140111038
0.8825 0.00133812140898467
0.8875 0.00117343640379604
0.8925 0.00102860020684167
0.8975 0.000901238476548837
0.9025 0.000789272109464669
0.9075 0.000690863406680269
0.9125 0.000604396523084081
0.9175 0.000528444440882734
0.9225 0.000461749706104483
0.9275 0.000403204600710479
0.9325 0.000351834897180649
0.9375 0.000306778803075717
0.9425 0.000267281453650786
0.9475 0.000232677784990514
0.9525 0.000202381785840636
0.9575 0.000175874391806256
0.9625 0.000152694370315404
0.9675 0.000132436303187084
0.9725 0.000114735976118324
0.9775 9.92646520308334e-05
0.9825 8.57258983698519e-05
0.9875 7.38453284255894e-05
0.9925 6.33765864595079e-05
0.9975 5.41186895572472e-05
};
\addlegendentry{FOM}
\addplot [semithick, red, mark=o,mark size=2, mark repeat=25, mark options={solid}]
table {%
0.0025 0.0062600192592862
0.0075 0.00626007040551888
0.0125 0.00626014048425322
0.0175 0.00626023310166612
0.0225 0.0062603544523252
0.0275 0.00626051320389498
0.0325 0.00626072092607204
0.0375 0.00626099284656747
0.0425 0.00626134890672328
0.0475 0.00626181515624492
0.0525 0.00626242556201539
0.0575 0.00626322433412447
0.0625 0.00626426890058563
0.0675 0.00626563369374165
0.0725 0.00626741494718634
0.0775 0.00626973674293628
0.0825 0.00627275859452341
0.0875 0.00627668490238315
0.0925 0.00628177667226014
0.0975 0.00628836594373728
0.1025 0.00629687343171148
0.1075 0.0063078299349464
0.1125 0.00632190210782474
0.1175 0.00633992321766196
0.1225 0.00636292951292474
0.1275 0.00639220279841957
0.1325 0.00642931974238764
0.1375 0.00647620831748141
0.1425 0.00653521159337205
0.1475 0.0066091588458083
0.1525 0.0067014436207368
0.1575 0.00681610799309832
0.1625 0.00695793179465215
0.1675 0.00713252506825813
0.1725 0.00734642146033343
0.1775 0.00760716972062289
0.1825 0.00792341997850413
0.1875 0.00830500105299799
0.1925 0.00876298477672356
0.1975 0.00930973321688419
0.2025 0.00995892479567539
0.2075 0.0107255556715872
0.2125 0.0116259133468907
0.2175 0.0126775202989951
0.2225 0.0138990464555852
0.2275 0.0153101904866576
0.2325 0.0169315310958639
0.2375 0.0187843506753947
0.2425 0.0208904347590622
0.2475 0.0232718515916525
0.2525 0.025950716769693
0.2575 0.0289489482616618
0.2625 0.0322880171709425
0.2675 0.0359886993723744
0.2725 0.040070832662665
0.2775 0.0445530833594871
0.2825 0.0494527254137574
0.2875 0.0547854341150499
0.2925 0.0605650954185703
0.2975 0.066803630847288
0.3025 0.0735108368681565
0.3075 0.0806942366578606
0.3125 0.0883589413282355
0.3175 0.0965075170649424
0.3225 0.105139854359177
0.3275 0.11425303570711
0.3325 0.123841198924194
0.3375 0.133895394615914
0.3425 0.144403438280727
0.3475 0.155349759720937
0.3525 0.166715254391251
0.3575 0.178477142279737
0.3625 0.190608839017589
0.3675 0.203079840384468
0.3725 0.215855614950675
0.3775 0.228897491046852
0.3825 0.242162515915183
0.3875 0.255603260938862
0.3925 0.26916755281826
0.3975 0.282798131808214
0.4025 0.296432276902629
0.4075 0.310001489700602
0.4125 0.323431379819163
0.4175 0.336641922873029
0.4225 0.349548243042551
0.4275 0.362061991591608
0.4325 0.374093256647815
0.4375 0.385552776062503
0.4425 0.396354067658105
0.4475 0.40641495514114
0.4525 0.4156578592761
0.4575 0.424008231635126
0.4625 0.431390964931745
0.4675 0.437726087810794
0.4725 0.442927697436931
0.4775 0.44691221110926
0.4825 0.449619385635629
0.4875 0.451038801603048
0.4925 0.451221456966715
0.4975 0.45025830760472
0.5025 0.448225912227799
0.5075 0.445128950450256
0.5125 0.440879733560757
0.5175 0.435329845057778
0.5225 0.428333735334988
0.5275 0.419812428113606
0.5325 0.409786538264864
0.5375 0.398365263656221
0.5425 0.385708967235167
0.5475 0.37199218959285
0.5525 0.357382540369815
0.5575 0.342035653412166
0.5625 0.326099037378626
0.5675 0.309718342463012
0.5725 0.29304291179853
0.5775 0.276229592333591
0.5825 0.259444100961741
0.5875 0.242858989306278
0.5925 0.226647532951254
0.5975 0.210973932139066
0.6025 0.195981574076951
0.6075 0.181782059301807
0.6125 0.168447736664603
0.6175 0.156009575706663
0.6225 0.144460714223272
0.6275 0.133764540352922
0.6325 0.123865206389541
0.6375 0.114698247823318
0.6425 0.106199409401684
0.6475 0.0983105795742747
0.6525 0.0909825906472759
0.6575 0.0841753230563736
0.6625 0.0778559520770433
0.6675 0.071996293631247
0.6725 0.0665701045920575
0.6775 0.0615509567606551
0.6825 0.0569110147228712
0.6875 0.0526207735268128
0.6925 0.0486495997131755
0.6975 0.0449667932837786
0.7025 0.0415428512411749
0.7075 0.0383506496790804
0.7125 0.0353663441373671
0.7175 0.0325698873368719
0.7225 0.0299451543419271
0.7275 0.0274797316112294
0.7325 0.0251644623762688
0.7375 0.0229928488341954
0.7425 0.0209603995052474
0.7475 0.0190639874274071
0.7525 0.0173012604960941
0.7575 0.0156701255510268
0.7625 0.0141683158242827
0.7675 0.0127930468746204
0.7725 0.0115407663544456
0.7775 0.0104070038471787
0.7825 0.00938632497421577
0.7875 0.00847238737697508
0.7925 0.00765808607176346
0.7975 0.00693576518857187
0.8025 0.00629746587755054
0.8075 0.00573517859701834
0.8125 0.0052410722890083
0.8175 0.00480768133602611
0.8225 0.0044280409931152
0.8275 0.00409577077391929
0.8325 0.00380511158086698
0.8375 0.00355092583360814
0.8425 0.00332867087097416
0.8475 0.00313435523588559
0.8525 0.00296448586617855
0.8575 0.00281601230361408
0.8625 0.00268627218029253
0.8675 0.00257294064637399
0.8725 0.00247398514059931
0.8775 0.00238762597786761
0.8825 0.00231230260128805
0.8875 0.00224664496923474
0.8925 0.00218944936603677
0.8975 0.0021396578845305
0.9025 0.00209634088119668
0.9075 0.00205868180868918
0.9125 0.0020259639537252
0.9175 0.00199755872739317
0.9225 0.0019729152557654
0.9275 0.00195155109478082
0.9325 0.00193304394399322
0.9375 0.00191702426193665
0.9425 0.00190316869618747
0.9475 0.00189119423817449
0.9525 0.00188085299945855
0.9575 0.00187192748345165
0.9625 0.0018642261936353
0.9675 0.0018575793784486
0.9725 0.00185183468759296
0.9775 0.00184685261634736
0.9825 0.00184250226941407
0.9875 0.00183866060527194
0.9925 0.00183522710042123
0.9975 0.00183219110923913
};
\addlegendentry{POD}
\addplot [semithick, color0, mark=pentagon,mark size=2, mark repeat=25, mark options={solid}]
table {%
0.0025 -0.000631807948769392
0.0075 -0.000631490037736713
0.0125 -0.000631546125276371
0.0175 -0.000631447298541644
0.0225 -0.000631036194020578
0.0275 -0.000630776789149651
0.0325 -0.000630034731405866
0.0375 -0.000629418870626427
0.0425 -0.000628891213013844
0.0475 -0.000627896853188038
0.0525 -0.000626794206447513
0.0575 -0.000625252931879017
0.0625 -0.000623044178849921
0.0675 -0.000620400839234849
0.0725 -0.000616628156189952
0.0775 -0.000611449580181555
0.0825 -0.000605452866988469
0.0875 -0.00059756794472925
0.0925 -0.000587377157172608
0.0975 -0.000574367787054611
0.1025 -0.000557506347980095
0.1075 -0.000535701306623739
0.1125 -0.000507954629303415
0.1175 -0.000472606130968072
0.1225 -0.000428145489288693
0.1275 -0.000371216881449591
0.1325 -0.000300095401297734
0.1375 -0.000210248765380809
0.1425 -9.83303661392677e-05
0.1475 4.10902459005241e-05
0.1525 0.000213118537006446
0.1575 0.000426264015625842
0.1625 0.000687119416907268
0.1675 0.001005901735249
0.1725 0.00139287529656513
0.1775 0.0018594114515223
0.1825 0.00242111508849028
0.1875 0.00309184331993063
0.1925 0.00388878902798813
0.1975 0.0048310368246522
0.2025 0.00593784714655254
0.2075 0.0072308277489616
0.2125 0.00873312793725966
0.2175 0.0104677618801456
0.2225 0.0124594097176933
0.2275 0.014734013896389
0.2325 0.0173154684735179
0.2375 0.0202297125278649
0.2425 0.0235015607544912
0.2475 0.0271546055258476
0.2525 0.0312115789809308
0.2575 0.0356925009730848
0.2625 0.0406167005788213
0.2675 0.046001841331894
0.2725 0.0518597536925459
0.2775 0.0582043165714529
0.2825 0.06504124921381
0.2875 0.0723768230454162
0.2925 0.0802116460894907
0.2975 0.0885438763050951
0.3025 0.0973674643510447
0.3075 0.107021789041389
0.3125 0.116440720848842
0.3175 0.12490586230726
0.3225 0.133720003001093
0.3275 0.142862778316196
0.3325 0.152341120788812
0.3375 0.162497892875388
0.3425 0.172917859936176
0.3475 0.183567577877488
0.3525 0.194413266753941
0.3575 0.205418398391938
0.3625 0.216543749721818
0.3675 0.227748402924002
0.3725 0.238989836785558
0.3775 0.250076203785205
0.3825 0.259722955769845
0.3875 0.269352678236485
0.3925 0.278929552861797
0.3975 0.288415640588901
0.4025 0.297770398729563
0.4075 0.306953443918453
0.4125 0.315922608183921
0.4175 0.324631326162623
0.4225 0.33326601881453
0.4275 0.34142263404496
0.4325 0.349225434073584
0.4375 0.356678482797793
0.4425 0.363737897044887
0.4475 0.370362185506318
0.4525 0.376512436900364
0.4575 0.382150750684403
0.4625 0.38714794874348
0.4675 0.3913536713747
0.4725 0.394871373208366
0.4775 0.397388974356512
0.4825 0.398981831092228
0.4875 0.399835243726848
0.4925 0.399990595924931
0.4975 0.399508213203089
0.5025 0.398436639479984
0.5075 0.396693275689119
0.5125 0.394299353321648
0.5175 0.391241993459969
0.5225 0.387457434184736
0.5275 0.383064357158505
0.5325 0.37821331526778
0.5375 0.372720766784397
0.5425 0.36681159708666
0.5475 0.360588362026546
0.5525 0.355096663704128
0.5575 0.350327936584653
0.5625 0.343972632758207
0.5675 0.336688651870459
0.5725 0.328822274105152
0.5775 0.32041557638353
0.5825 0.311526879245742
0.5875 0.302190853608558
0.5925 0.292278158329648
0.5975 0.282171426890791
0.6025 0.271971512556488
0.6075 0.261768723888853
0.6125 0.251635805692066
0.6175 0.241624864362019
0.6225 0.231764247220923
0.6275 0.222064426718653
0.6325 0.212523519697695
0.6375 0.203134767425826
0.6425 0.193891589567163
0.6475 0.184792477054611
0.6525 0.175842542977689
0.6575 0.167050575659923
0.6625 0.158431217163791
0.6675 0.14999581762344
0.6725 0.141752426090658
0.6775 0.13370318635245
0.6825 0.125844669476259
0.6875 0.118165959822679
0.6925 0.110653909333516
0.6975 0.103294400083125
0.7025 0.0960750370007804
0.7075 0.0894654472710603
0.7125 0.0829750399811243
0.7175 0.0766050513223901
0.7225 0.0703740820599711
0.7275 0.0643055789897624
0.7325 0.058428695233126
0.7375 0.0527750463136163
0.7425 0.0473773895280973
0.7475 0.0422663940147378
0.7525 0.0374695763449913
0.7575 0.0330099585557321
0.7625 0.0289021234991204
0.7675 0.0251548345470746
0.7725 0.021769468875248
0.7775 0.0187375282416859
0.7825 0.0160458188486836
0.7875 0.0136745361921076
0.7925 0.0116006821457647
0.7975 0.00979790352603309
0.8025 0.00823835164484893
0.8075 0.0068948125034414
0.8125 0.00574062449795137
0.8175 0.00475215199626958
0.8225 0.00390651131282701
0.8275 0.00318400849594951
0.8325 0.00256670474819961
0.8375 0.00203905791230377
0.8425 0.00158857323356673
0.8475 0.00120324041647867
0.8525 0.000873574329433593
0.8575 0.000591753822194752
0.8625 0.000350643592267582
0.8675 0.000144431660200405
0.8725 -3.20908287142093e-05
0.8775 -0.000182570176864047
0.8825 -0.000311330115125025
0.8875 -0.000421055650765921
0.8925 -0.000514183530215367
0.8975 -0.000593810018340587
0.9025 -0.000616745505283568
0.9075 -0.000604407532406354
0.9125 -0.000591894867661592
0.9175 -0.000580566654191215
0.9225 -0.000569624930404658
0.9275 -0.000559549103934931
0.9325 -0.000550545278121026
0.9375 -0.000542154282155123
0.9425 -0.000534904477805547
0.9475 -0.000528278463641959
0.9525 -0.000522401303856251
0.9575 -0.000517344944801165
0.9625 -0.000513090221192944
0.9675 -0.000509410517328195
0.9725 -0.000506246714735798
0.9775 -0.00050340667073073
0.9825 -0.000501216899292553
0.9875 -0.000498651904663233
0.9925 -0.000496241671230978
0.9975 -0.000494182474769158
};
\addlegendentry{FCNN}
\addplot [semithick, green!50!black, mark=triangle,mark size=2, mark repeat=25, mark options={solid,rotate=180}, only marks]
table {%
0.0025 0.000990276688505998
0.0075 0.00187346175142161
0.0125 0.000956459126158295
0.0175 0.00111032418942201
0.0225 0.00118305443355243
0.0275 0.000827656629166061
0.0325 0.000691186990065389
0.0375 0.00166181710439562
0.0425 0.00175891415643347
0.0475 0.0025517450827575
0.0525 0.00173716383113934
0.0575 0.00150303271027979
0.0625 0.00176329713446111
0.0675 0.00073753527777329
0.0725 0.000993213476437076
0.0775 0.000639066816254604
0.0825 0.000734195993051566
0.0875 0.000873181985378367
0.0925 0.00145871509211356
0.0975 0.00220339796449235
0.1025 0.00121188850202064
0.1075 0.000607562745446116
0.1125 0.00126247139801679
0.1175 0.000905569549864286
0.1225 0.00139829101712817
0.1275 0.00129545967759409
0.1325 0.00117681401787554
0.1375 0.00116151532964278
0.1425 0.00202973556888819
0.1475 0.0013777459716733
0.1525 0.00113482140780733
0.1575 0.00134956276464138
0.1625 0.00129371936133599
0.1675 0.0013712722426807
0.1725 0.00103588676406005
0.1775 0.00101600286540407
0.1825 0.00100900922833961
0.1875 0.0013310273908215
0.1925 0.00167954033921098
0.1975 0.00210040316035736
0.2025 0.00142412265570725
0.2075 0.00218474513011934
0.2125 0.00185787549017274
0.2175 0.00242717031932594
0.2225 0.00272509950922413
0.2275 0.00554158091520655
0.2325 0.00501972896450751
0.2375 0.00741562516264531
0.2425 0.00903226747445517
0.2475 0.0102189644464867
0.2525 0.016028460442091
0.2575 0.0204425711567586
0.2625 0.0247487806306942
0.2675 0.0277115491715486
0.2725 0.0285183165264044
0.2775 0.0394620391275827
0.2825 0.0404231770026746
0.2875 0.0453527144118873
0.2925 0.0531651155353833
0.2975 0.057593675254728
0.3025 0.07810789210525
0.3075 0.0888079670094838
0.3125 0.100107916660267
0.3175 0.10611079366116
0.3225 0.111786485160396
0.3275 0.126498960023075
0.3325 0.135245227393099
0.3375 0.14244712586331
0.3425 0.152119756857271
0.3475 0.161680623422925
0.3525 0.17728119451156
0.3575 0.195771888660809
0.3625 0.213513431552896
0.3675 0.224904500811553
0.3725 0.22976089408802
0.3775 0.247539473043291
0.3825 0.259006029639144
0.3875 0.263381734217764
0.3925 0.278551351537491
0.3975 0.292413506808738
0.4025 0.28534933405143
0.4075 0.301925296693947
0.4125 0.325359533059993
0.4175 0.33721999117675
0.4225 0.340365561057586
0.4275 0.35717179868544
0.4325 0.366781022712937
0.4375 0.370393693424151
0.4425 0.386592727669981
0.4475 0.396976764144343
0.4525 0.376480683082908
0.4575 0.382533676415473
0.4625 0.396888720020967
0.4675 0.406607204118947
0.4725 0.407934181630439
0.4775 0.410997641751611
0.4825 0.41473210313585
0.4875 0.417731766439391
0.4925 0.423495037774944
0.4975 0.420958239502959
0.5025 0.416038392852477
0.5075 0.416642390696328
0.5125 0.416973869432982
0.5175 0.420014043483255
0.5225 0.41714056967714
0.5275 0.413960954320205
0.5325 0.411179128042918
0.5375 0.414924135386088
0.5425 0.410818195013714
0.5475 0.393717168755193
0.5525 0.40115810202693
0.5575 0.411407132842746
0.5625 0.365318496808962
0.5675 0.353533185299319
0.5725 0.338222763075476
0.5775 0.304273681451843
0.5825 0.272756410479436
0.5875 0.258566636841589
0.5925 0.226470683056211
0.5975 0.189270391042468
0.6025 0.18027032496892
0.6075 0.170689002187637
0.6125 0.159070458630774
0.6175 0.150926852958939
0.6225 0.139427758750391
0.6275 0.140715708293604
0.6325 0.122731400333482
0.6375 0.126532109934631
0.6425 0.121545605483923
0.6475 0.130906543066207
0.6525 0.115198879955057
0.6575 0.120824742002963
0.6625 0.119357763601003
0.6675 0.126076765620363
0.6725 0.124074290909135
0.6775 0.123734517871169
0.6825 0.12981831822082
0.6875 0.133076137436779
0.6925 0.12436016261217
0.6975 0.123142662336731
0.7025 0.105220869163334
0.7075 0.103484978271743
0.7125 0.112250033363591
0.7175 0.111705513781945
0.7225 0.0997635001904637
0.7275 0.0916693451345243
0.7325 0.0806143794716293
0.7375 0.053541960968245
0.7425 0.0449047103431749
0.7475 0.0167015871068529
0.7525 0.00579635965356813
0.7575 0.00438916708799925
0.7625 0.00245099068421994
0.7675 0.00142430349835415
0.7725 0.0016691646615766
0.7775 0.00221383891918055
0.7825 0.0022647340696426
0.7875 0.00112485616823951
0.7925 0.00121159903634666
0.7975 0.000618719086368582
0.8025 0.00195746239831248
0.8075 0.00211052270715228
0.8125 0.00226499978070228
0.8175 0.00204830467425802
0.8225 0.00213557941032228
0.8275 0.00215728376648924
0.8325 0.00215253516465513
0.8375 0.00149328793778869
0.8425 0.00210283045449903
0.8475 0.00190540611286073
0.8525 0.00128473917323556
0.8575 0.000803198318497643
0.8625 0.00101323873867382
0.8675 0.000237003719991305
0.8725 0.00221494467733775
0.8775 0.00183400172912735
0.8825 0.00137541812686063
0.8875 0.00195466124728841
0.8925 0.00177485548510151
0.8975 0.0018088387712864
0.9025 0.00146071593262078
0.9075 0.00274721246216447
0.9125 0.00167953047241935
0.9175 0.000550836205683706
0.9225 0.00199540733327809
0.9275 0.00287023866107769
0.9325 0.00141074914415169
0.9375 0.0013739437490657
0.9425 0.0017236385951483
0.9475 0.00093101545030355
0.9525 0.00180932305171964
0.9575 0.00075867152133363
0.9625 0.00141641223779588
0.9675 0.00112006187425264
0.9725 0.00176333489864244
0.9775 0.00194067231155268
0.9825 0.000818390347219144
0.9875 0.00369147800376881
0.9925 0.00273298329524285
0.9975 0.00306609981605222
};
\addlegendentry{CNN}

\nextgroupplot[
legend cell align={left},
legend style={at={(1,1)},anchor=north east,fill opacity=0.1, draw opacity=1, text opacity=1, draw=none},
tick align=outside,
tick pos=left,
x grid style={white!69.0196078431373!black},
xmajorgrids,
xlabel={\(x\)},
xmin=-0.04725, xmax=1.04725,
xtick style={color=black},
y grid style={white!69.0196078431373!black},
ymajorgrids,
ylabel={\(E\)},
ymin=-0.117868330334208, ymax=0.603154515104935,
ytick style={color=black},
width=.35\textwidth,
height=.4\textwidth,
clip=false,
%y label style={yshift=-.7em}
]
\addplot [semithick, black, dashed, mark=x,mark size=2, mark repeat=25, mark options={solid}]
table {%
0.0025 0.499999361504023
0.0075 0.499999183620267
0.0125 0.499998956656357
0.0175 0.499998657893175
0.0225 0.499998272625194
0.0275 0.499997777704083
0.0325 0.499997126285128
0.0375 0.499996272281854
0.0425 0.499995169976743
0.0475 0.499993730350822
0.0525 0.499991876410342
0.0575 0.499989476562542
0.0625 0.499986364006075
0.0675 0.499982355915585
0.0725 0.499977180296728
0.0775 0.499970525776456
0.0825 0.499961955650545
0.0875 0.499950973892023
0.0925 0.49993691017478
0.0975 0.499918981086569
0.1025 0.499896128536149
0.1075 0.499867124733552
0.1125 0.499830438144633
0.1175 0.499784161496544
0.1225 0.499725999880656
0.1275 0.499653169032844
0.1325 0.499562332427576
0.1375 0.499449480224796
0.1425 0.499309905902888
0.1475 0.499138042032686
0.1525 0.49892740118424
0.1575 0.498670486468602
0.1625 0.498358703749034
0.1675 0.497982270415046
0.1725 0.497530226757131
0.1775 0.496990335519934
0.1825 0.49634912991411
0.1875 0.495591999394454
0.1925 0.494703208363668
0.1975 0.493666077783127
0.2025 0.492463172397307
0.2075 0.491076549512914
0.2125 0.489488046580558
0.2175 0.487679600013574
0.2225 0.485633687799318
0.2275 0.483333679718255
0.2325 0.48076423294581
0.2375 0.477911799239919
0.2425 0.474764935511201
0.2475 0.471314719488362
0.2525 0.467555088721707
0.2575 0.463483090324178
0.2625 0.459099083657965
0.2675 0.454406855911048
0.2725 0.449413715086325
0.2775 0.444130393539245
0.2825 0.438571026778881
0.2875 0.432752851994598
0.2925 0.426696094320604
0.2975 0.420423598708122
0.3025 0.413960495202088
0.3075 0.407333857180233
0.3125 0.400572300712034
0.3175 0.393705637998278
0.3225 0.386764405139074
0.3275 0.379779558522867
0.3325 0.372782111535813
0.3375 0.365802766850548
0.3425 0.358871639307349
0.3475 0.352017958367442
0.3525 0.345269841759585
0.3575 0.338654021796531
0.3625 0.332195673534968
0.3675 0.325918213229631
0.3725 0.319843193919365
0.3775 0.313990117618382
0.3825 0.308376397774443
0.3875 0.303017309076339
0.3925 0.297925928448431
0.3975 0.293113196380939
0.4025 0.288587922654914
0.4075 0.284356778976805
0.4125 0.28042433520336
0.4175 0.276792931214978
0.4225 0.273462577948832
0.4275 0.270430830226565
0.4325 0.267692592262936
0.4375 0.265240081844257
0.4425 0.263062808250952
0.4475 0.26114780211783
0.4525 0.259479893722797
0.4575 0.258042107206346
0.4625 0.256815969293871
0.4675 0.255781716317408
0.4725 0.254918098730885
0.4775 0.254201785541831
0.4825 0.253606557360809
0.4875 0.253102949250893
0.4925 0.252659205893312
0.4975 0.252243789434365
0.5025 0.251827275987589
0.5075 0.251381014770881
0.5125 0.250873031148202
0.5175 0.250265601411364
0.5225 0.249517692496044
0.5275 0.248590686884939
0.5325 0.247453394320449
0.5375 0.246084020788545
0.5425 0.244469269449419
0.5475 0.242601861073284
0.5525 0.240478184700032
0.5575 0.238096661384523
0.5625 0.235457211563669
0.5675 0.23256160323498
0.5725 0.229414723541784
0.5775 0.226026208963633
0.5825 0.222411970099903
0.5875 0.21859481129097
0.5925 0.214603594912074
0.5975 0.21047092978495
0.6025 0.206229560100738
0.6075 0.201908555413813
0.6125 0.197530027012126
0.6175 0.19310734820258
0.6225 0.188645123357809
0.6275 0.184140898682197
0.6325 0.179587896810172
0.6375 0.174978196029754
0.6425 0.170305413552305
0.6475 0.165566403039851
0.6525 0.160761829825727
0.6575 0.155895508695644
0.6625 0.150972948924719
0.6675 0.14599961589413
0.6725 0.140979355280142
0.6775 0.135913453516059
0.6825 0.130800568473667
0.6875 0.125637492433925
0.6925 0.120420696567495
0.6975 0.115148229960878
0.7025 0.109821680117693
0.7075 0.104447879759985
0.7125 0.099040050456216
0.7175 0.0936183095804956
0.7225 0.0882094557276782
0.7275 0.0828461162686445
0.7325 0.0775653754621413
0.7375 0.0724069851934649
0.7425 0.0674113530543212
0.7475 0.0626174208730466
0.7525 0.0580606180437754
0.7575 0.0537710116311263
0.7625 0.0497718581564984
0.7675 0.0460786337784703
0.7725 0.0426986709975221
0.7775 0.0396313903583002
0.7825 0.0368690814934464
0.7875 0.0343980870855545
0.7925 0.0322002370953793
0.7975 0.0302543369596147
0.8025 0.0285375636073226
0.8075 0.0270266753831553
0.8125 0.0256989512538859
0.8175 0.0245328807748135
0.8225 0.0235086070033041
0.8275 0.0226081683524312
0.8325 0.0218155829214253
0.8375 0.0211168158856341
0.8425 0.0204996764296028
0.8475 0.0199536540088463
0.8525 0.0194697400800231
0.8575 0.0190402333117841
0.8625 0.0186585512793067
0.8675 0.0183190559120018
0.8725 0.0180168954550742
0.8775 0.017747867356557
0.8825 0.0175083087955527
0.8875 0.0172950032123299
0.8925 0.0171051056839067
0.8975 0.0169360900548237
0.9025 0.0167857058531322
0.9075 0.0166519413428778
0.9125 0.0165330009502372
0.9175 0.0164272791506258
0.9225 0.0163333443756653
0.9275 0.0162499199997569
0.9325 0.0161758679740341
0.9375 0.0161101745659652
0.9425 0.0160519363070998
0.9475 0.0160003481569746
0.9525 0.0159546901722751
0.9575 0.0159143188253597
0.9625 0.0158786592261423
0.9675 0.0158471966022788
0.9725 0.015819471627134
0.9775 0.0157950755882109
0.9825 0.0157736453028088
0.9875 0.0157548627175762
0.9925 0.0157384515158429
0.9975 0.0157241723861597
};
\addlegendentry{FOM}
\addplot [semithick, red, mark=o,mark size=2, mark repeat=25, mark options={solid}]
table {%
0.0025 0.496026834244761
0.0075 0.496026784168737
0.0125 0.49602671969299
0.0175 0.496026635691211
0.0225 0.496026525736975
0.0275 0.496026381621337
0.0325 0.496026192698993
0.0375 0.4960259450845
0.0425 0.496025620643744
0.0475 0.496025195709419
0.0525 0.496024639437358
0.0575 0.496023911704497
0.0625 0.496022960428407
0.0675 0.496021718162667
0.0725 0.49602009779221
0.0775 0.496017987118061
0.0825 0.496015242082117
0.0875 0.49601167834026
0.0925 0.496007060847421
0.0975 0.496001091072989
0.1025 0.495993391421833
0.1075 0.49598348639884
0.1125 0.495970780027982
0.1175 0.495954529026425
0.1225 0.495933811247129
0.1275 0.495907488947688
0.1325 0.495874166527358
0.1375 0.495832142506849
0.1425 0.495779355714125
0.1475 0.495713325889685
0.1525 0.495631089239132
0.1575 0.495529129836633
0.1625 0.495403308211861
0.1675 0.495248788919343
0.1725 0.4950599693693
0.1775 0.4948304126619
0.1825 0.494552787574625
0.1875 0.494218819163569
0.1925 0.493819253611612
0.1975 0.493343840951512
0.2025 0.492781339080651
0.2075 0.492119542051452
0.2125 0.491345334969818
0.2175 0.490444776986442
0.2225 0.489403212865514
0.2275 0.48820541252315
0.2325 0.486835736816765
0.2375 0.485278326815333
0.2425 0.483517312864128
0.2475 0.481537039040688
0.2525 0.479322298127445
0.2575 0.476858572025354
0.2625 0.474132272602276
0.2675 0.471130978289698
0.2725 0.467843662274004
0.2775 0.464260908826629
0.2825 0.460375115129565
0.2875 0.456180676829434
0.2925 0.451674156449146
0.2975 0.446854434658495
0.3025 0.441722845208988
0.3075 0.43628329501869
0.3125 0.430542371376156
0.3175 0.42450943842263
0.3225 0.41819672485441
0.3275 0.411619404049749
0.3325 0.404795666490254
0.3375 0.397746782426705
0.3425 0.390497150390604
0.3475 0.383074324728323
0.3525 0.375509013398015
0.3575 0.367835036545584
0.3625 0.360089237604699
0.3675 0.352311342335349
0.3725 0.344543767182771
0.3775 0.336831385399502
0.3825 0.329221264937432
0.3875 0.321762392261139
0.3925 0.314505386491973
0.3975 0.307502185474024
0.4025 0.300805650325001
0.4075 0.29446899563923
0.4125 0.288544924837368
0.4175 0.283084355835123
0.4225 0.278134680177799
0.4275 0.273737612933219
0.4325 0.269926838648716
0.4375 0.266725790115062
0.4425 0.264145944843835
0.4475 0.26218592775786
0.4525 0.260831433700222
0.4575 0.260055548831015
0.4625 0.259818585849822
0.4675 0.260066401147027
0.4725 0.260726916758803
0.4775 0.261706626314939
0.4825 0.26289137069062
0.4875 0.264155781204281
0.4925 0.265380662180395
0.4975 0.266469535232779
0.5025 0.267355228057877
0.5075 0.267993513015443
0.5125 0.268347498708088
0.5175 0.268368491860145
0.5225 0.2679821051476
0.5275 0.267089201906358
0.5325 0.265584916609639
0.5375 0.263384496046043
0.5425 0.260440887734434
0.5475 0.256746907668944
0.5525 0.252325328530938
0.5575 0.247215333450493
0.5625 0.24146244755571
0.5675 0.23511484113163
0.5725 0.228225222098359
0.5775 0.220855642520863
0.5825 0.213082150057144
0.5875 0.204996741427499
0.5925 0.196705161502968
0.5975 0.18832049028865
0.6025 0.17995383660725
0.6075 0.17170439747941
0.6125 0.163651330948107
0.6175 0.155849287584302
0.6225 0.148328318820406
0.6275 0.141097667981083
0.6325 0.134152057682715
0.6375 0.127478729864486
0.6425 0.121063659615039
0.6475 0.114895879850661
0.6525 0.108969498367532
0.6575 0.10328357404809
0.6625 0.097840430474844
0.6675 0.0926431812470018
0.6725 0.0876932349695655
0.6775 0.0829883867235971
0.6825 0.078521852113852
0.6875 0.0742823286931795
0.6925 0.0702549375263153
0.6975 0.0664227450843482
0.7025 0.0627685075233877
0.7075 0.0592763065346707
0.7125 0.05593283218737
0.7175 0.0527281805380399
0.7225 0.0496561421089895
0.7275 0.0467140407640087
0.7325 0.0439022317005707
0.7375 0.0412233832109181
0.7425 0.0386816575181199
0.7475 0.0362818824228978
0.7525 0.0340287783556193
0.7575 0.0319262824821773
0.7625 0.0299769963659542
0.7675 0.0281817757497395
0.7725 0.0265394766551976
0.7775 0.0250468665765341
0.7825 0.0236986997865967
0.7875 0.0224879414245178
0.7925 0.0214061090700217
0.7975 0.020443687675489
0.8025 0.0195905681785203
0.8075 0.0188364634737073
0.8125 0.0181712662333032
0.8175 0.0175853277785879
0.8225 0.017069651846122
0.8275 0.0166160087137515
0.8325 0.0162169824359945
0.8375 0.0158659670335385
0.8425 0.0155571273960618
0.8475 0.0152853386447708
0.8525 0.0150461148459901
0.8575 0.0148355350181559
0.8625 0.0146501717487152
0.8675 0.0144870256063356
0.8725 0.0143434669210281
0.8775 0.014217185360933
0.8825 0.0141061469833791
0.8875 0.0140085579990384
0.8925 0.0139228342861477
0.8975 0.0138475756591185
0.9025 0.013781543973461
0.9075 0.0137236442874623
0.9125 0.0136729084617042
0.9175 0.0136284807325308
0.9225 0.013589604927869
0.9275 0.0135556130956329
0.9325 0.0135259153859493
0.9375 0.0134999910732926
0.9425 0.0134773806308383
0.9475 0.0134576787855128
0.9525 0.0134405284968079
0.9575 0.0134256158233248
0.9625 0.0134126656754567
0.9675 0.0134014385076998
0.9725 0.0133917280890878
0.9775 0.0133833606280774
0.9825 0.0133761958024525
0.9875 0.0133701309602265
0.9925 0.0133651118920604
0.9975 0.0133611597771419
};
\addlegendentry{POD}
\addplot [semithick, color0, mark=pentagon,mark size=2, mark repeat=25, mark options={solid}]
table {%
0.0025 0.500140075687656
0.0075 0.500138569720573
0.0125 0.500139517339998
0.0175 0.500139966508182
0.0225 0.500139617444732
0.0275 0.500138978574465
0.0325 0.500137561965302
0.0375 0.500137891372051
0.0425 0.500137151996694
0.0475 0.500135950500453
0.0525 0.500134149494448
0.0575 0.50013480301304
0.0625 0.50013227644004
0.0675 0.500128712171409
0.0725 0.500126556371367
0.0775 0.50012183523443
0.0825 0.500116478500726
0.0875 0.500108110409841
0.0925 0.500100555359842
0.0975 0.50008712681842
0.1025 0.500072462342279
0.1075 0.50005143288653
0.1125 0.500025072417997
0.1175 0.499993383240689
0.1225 0.499952071574294
0.1275 0.499899056932826
0.1325 0.499833937676651
0.1375 0.499752629886652
0.1425 0.499648863699201
0.1475 0.499520221261541
0.1525 0.499364131411987
0.1575 0.49916976500376
0.1625 0.498929997942765
0.1675 0.498641273135382
0.1725 0.498286362495602
0.1775 0.497863030705589
0.1825 0.497352476264018
0.1875 0.49674421471504
0.1925 0.496023800388918
0.1975 0.495171691561157
0.2025 0.494173547647654
0.2075 0.493012143549371
0.2125 0.491663550133115
0.2175 0.490110957657565
0.2225 0.488335367971955
0.2275 0.486307187111791
0.2325 0.484015047494466
0.2375 0.481433638958292
0.2425 0.478543856838149
0.2475 0.475328681456947
0.2525 0.471765975824711
0.2575 0.467845790718573
0.2625 0.463552405181422
0.2675 0.458874772963551
0.2725 0.453802787195974
0.2775 0.448331265739512
0.2825 0.44245660179751
0.2875 0.43618238893917
0.2925 0.429509686820836
0.2975 0.422443585401646
0.3025 0.414998569884544
0.3075 0.407046027992034
0.3125 0.399572458871184
0.3175 0.393359932709707
0.3225 0.386936210855806
0.3275 0.38032201458913
0.3325 0.373537555422142
0.3375 0.366514761907843
0.3425 0.359369412256221
0.3475 0.352139880791348
0.3525 0.344852350357085
0.3575 0.337540151763602
0.3625 0.33024040984702
0.3675 0.322988159443128
0.3725 0.315820011220866
0.3775 0.309067613911723
0.3825 0.305252638261207
0.3875 0.301470778755394
0.3925 0.297735900818455
0.3975 0.294060506881717
0.4025 0.290467276731725
0.4075 0.286968448348876
0.4125 0.283576115118281
0.4175 0.280307067053909
0.4225 0.277093211811741
0.4275 0.274091854542008
0.4325 0.271228359319071
0.4375 0.268491574167801
0.4425 0.265886212245488
0.4475 0.263417712303893
0.4525 0.261083554784099
0.4575 0.258889325258665
0.4625 0.256871658033247
0.4675 0.255077939994129
0.4725 0.253439685098059
0.4775 0.252178161562171
0.4825 0.251203537504062
0.4875 0.250355023077618
0.4925 0.24961086561513
0.4975 0.248939584226531
0.5025 0.248307025224565
0.5075 0.247898040589204
0.5125 0.247589087674205
0.5175 0.247259400954689
0.5225 0.246882641620604
0.5275 0.24642810032311
0.5325 0.246407945928123
0.5375 0.246296296453594
0.5425 0.245794117179791
0.5475 0.244874819376307
0.5525 0.244878556155256
0.5575 0.246578866181811
0.5625 0.247233794988695
0.5675 0.248064861180355
0.5725 0.248387108505696
0.5775 0.248160905145571
0.5825 0.247361991845675
0.5875 0.245891200814941
0.5925 0.243324734959248
0.5975 0.240225771534656
0.6025 0.236624413495685
0.6075 0.232563868301012
0.6125 0.228080220049207
0.6175 0.223217221554655
0.6225 0.218011381509309
0.6275 0.212496395581232
0.6325 0.206702069633215
0.6375 0.200660723529083
0.6425 0.194402602655939
0.6475 0.187959083085118
0.6525 0.181371909801806
0.6575 0.174674502412807
0.6625 0.167904659596798
0.6675 0.161094638911197
0.6725 0.154270031999454
0.6775 0.147451874354061
0.6825 0.140652405505816
0.6875 0.133873729208463
0.6925 0.127121667213285
0.6975 0.120396667334319
0.7025 0.113704769881973
0.7075 0.107430020291282
0.7125 0.10119743837158
0.7175 0.0950233857125965
0.7225 0.0889374020070176
0.7275 0.0829754753150418
0.7325 0.0771778313762941
0.7375 0.0715822533419114
0.7425 0.0662310896868544
0.7475 0.0611603633611809
0.7525 0.0564013456174933
0.7575 0.0519827389056712
0.7625 0.0479214424584085
0.7675 0.0442280879134589
0.7725 0.0409012286467247
0.7775 0.0379353952752752
0.7825 0.0353156625537475
0.7875 0.0330195030899836
0.7925 0.0310227625308385
0.7975 0.0292979137542972
0.8025 0.0278166800774559
0.8075 0.0265473744857147
0.8125 0.025467407182384
0.8175 0.0245495167563215
0.8225 0.0237700109360146
0.8275 0.0231088655246475
0.8325 0.022548137113622
0.8375 0.0220720999297803
0.8425 0.0216695406139774
0.8475 0.0213286373442422
0.8525 0.0210385584028792
0.8575 0.0207929935020692
0.8625 0.0205818029161479
0.8675 0.0204047573195505
0.8725 0.0202528976133086
0.8775 0.0201238800497989
0.8825 0.0200144229372472
0.8875 0.0199204343023097
0.8925 0.0198409714397374
0.8975 0.0197748695722275
0.9025 0.0194386983628718
0.9075 0.0189553115588306
0.9125 0.0185324562397207
0.9175 0.0181694297491223
0.9225 0.0178544961414556
0.9275 0.0175845703861607
0.9325 0.0173515058952726
0.9375 0.017151094195737
0.9425 0.0169777655094346
0.9475 0.0168299780099334
0.9525 0.016702848274759
0.9575 0.0165933512992963
0.9625 0.0164995646126291
0.9675 0.0164194184912843
0.9725 0.0163503012120064
0.9775 0.0162894750767339
0.9825 0.0162413863425908
0.9875 0.0161963301884288
0.9925 0.0161580692750441
0.9975 0.0161276522725351
};
\addlegendentry{FCNN}
\addplot [semithick, green!50!black, mark=triangle, mark size=2, mark repeat=25, mark options={solid,rotate=180}, only marks]
table {%
0.0025 0.540835561662537
0.0075 0.562372304317069
0.0125 0.483907564200636
0.0175 0.457767292803346
0.0225 0.51470995927126
0.0275 0.473847052973453
0.0325 0.534276955996909
0.0375 0.491288420946959
0.0425 0.515868752391436
0.0475 0.56143257500371
0.0525 0.541937746317524
0.0575 0.527438171911746
0.0625 0.522615615964018
0.0675 0.478113660124478
0.0725 0.507392183159991
0.0775 0.460038010482089
0.0825 0.516525655689162
0.0875 0.492386635267604
0.0925 0.510942147458494
0.0975 0.515048093515374
0.1025 0.535655252686905
0.1075 0.536655297071117
0.1125 0.515821756565027
0.1175 0.433278222824374
0.1225 0.518693991180524
0.1275 0.475766934042409
0.1325 0.515051635122914
0.1375 0.513815867218078
0.1425 0.519967161841684
0.1475 0.52469247877082
0.1525 0.547397648421336
0.1575 0.529331671352966
0.1625 0.515243950294948
0.1675 0.462487580868375
0.1725 0.509411308624497
0.1775 0.477005882498008
0.1825 0.516264874780113
0.1875 0.492734387982292
0.1925 0.513785284293402
0.1975 0.526055971116259
0.2025 0.533226946816323
0.2075 0.570380749403156
0.2125 0.479142514787516
0.2175 0.448835549523125
0.2225 0.510543627106503
0.2275 0.481627288631868
0.2325 0.512759906795225
0.2375 0.487899997809227
0.2425 0.492551204481274
0.2475 0.509204558311209
0.2525 0.54509873391418
0.2575 0.542814448724928
0.2625 0.479764487968043
0.2675 0.420368944280517
0.2725 0.44830244630346
0.2775 0.419919389856223
0.2825 0.440432376199189
0.2875 0.392791632581989
0.2925 0.403022540951034
0.2975 0.382701462129915
0.3025 0.490342659851083
0.3075 0.486446848967694
0.3125 0.41353748909309
0.3175 0.32644804712946
0.3225 0.354234134479632
0.3275 0.306794339859329
0.3325 0.325794463568196
0.3375 0.294310611548813
0.3425 0.26230900490813
0.3475 0.249750247466072
0.3525 0.357291087341986
0.3575 0.380217514469521
0.3625 0.325692539883376
0.3675 0.258809749262796
0.3725 0.256510450260947
0.3775 0.24184143086567
0.3825 0.247076593402403
0.3875 0.213721906507143
0.3925 0.198700793306444
0.3975 0.226909198489349
0.4025 0.299187878335091
0.4075 0.312096233158667
0.4125 0.268826258298261
0.4175 0.256061881978995
0.4225 0.22992451728565
0.4275 0.221991503094745
0.4325 0.238327763547569
0.4375 0.203658882353152
0.4425 0.205188933648054
0.4475 0.191292131398734
0.4525 0.322939890147141
0.4575 0.273206309573891
0.4625 0.269655823144721
0.4675 0.23849276054424
0.4725 0.20566757207015
0.4775 0.209635324217316
0.4825 0.284326837035186
0.4875 0.201170555046328
0.4925 0.185655572179927
0.4975 0.186017410152496
0.5025 0.193435740089067
0.5075 0.189062989096581
0.5125 0.241557534829389
0.5175 0.214371684271138
0.5225 0.193369319165069
0.5275 0.194362872479937
0.5325 0.25682754289823
0.5375 0.21395566514719
0.5425 0.223067941297677
0.5475 0.305159778331963
0.5525 0.11052463829649
0.5575 0.037726287825739
0.5625 0.252894970040442
0.5675 0.25913883265
0.5725 0.145558679784774
0.5775 0.214837624835507
0.5825 0.21023289467547
0.5875 0.18875221391739
0.5925 0.260163439808786
0.5975 0.302784751177033
0.6025 0.219541584882384
0.6075 0.238422847456316
0.6125 0.231145352234404
0.6175 0.188995929904077
0.6225 0.2336674417095
0.6275 0.196290592292979
0.6325 0.269374006267635
0.6375 0.26475489656134
0.6425 0.248346528479347
0.6475 0.232284338599596
0.6525 0.252709700879148
0.6575 0.246958050388625
0.6625 0.300235943304961
0.6675 0.236172064325156
0.6725 0.268474100759644
0.6775 0.227413404635762
0.6825 0.222232455156302
0.6875 0.252738198692091
0.6925 0.242997613412264
0.6975 0.222262635987941
0.7025 0.274442195235224
0.7075 0.329702824116659
0.7125 0.287322740001272
0.7175 0.170264211683017
0.7225 0.166934719004126
0.7275 0.101441447449708
0.7325 0.029856793267574
0.7375 0.0160832618619325
0.7425 -0.0850945646324289
0.7475 -0.0180888654924372
0.7525 0.00794276096162504
0.7575 -5.22214564878203e-05
0.7625 0.0642889414200021
0.7675 0.0128788877571234
0.7725 0.0301322255798566
0.7775 0.0301881673002315
0.7825 -0.00317969349098216
0.7875 0.0346729031649259
0.7925 0.0210136826537895
0.7975 0.0159049082259858
0.8025 0.0153003539073819
0.8075 0.00686260852596309
0.8125 0.0852030032816548
0.8175 0.00635664329012142
0.8225 0.0504135567220288
0.8275 0.0239097275382885
0.8325 -0.00327520356726033
0.8375 0.0488823880799524
0.8425 0.0295404297362966
0.8475 0.0259471092714362
0.8525 0.00577727668384665
0.8575 0.00043477400994047
0.8625 0.0744034287237534
0.8675 0.00892546091535605
0.8725 0.0386614184715363
0.8775 0.0155194806224155
0.8825 -0.00939818035450396
0.8875 0.0464476761688731
0.8925 0.0335772337921785
0.8975 0.0292779511470415
0.9025 -0.0113625705386145
0.9075 0.01245426184423
0.9125 0.0806139203925452
0.9175 0.0240748950345743
0.9225 0.0340319409914282
0.9275 0.0215743667007377
0.9325 -0.00346177408640124
0.9375 0.0314157299573165
0.9425 0.0306417382387716
0.9475 0.0174471126382198
0.9525 0.0139400802928716
0.9575 0.00236124517937917
0.9625 0.0748931739596047
0.9675 -0.00165461394376987
0.9725 0.035573514366597
0.9775 0.0401613386810571
0.9825 -0.00486117144905422
0.9875 0.0545409534249203
0.9925 0.0270827195047629
0.9975 0.034283685895735
};
\addlegendentry{CNN}
\end{groupplot}

\end{tikzpicture}

	\caption{Matching of macroscopic quantities \(\rho\), \(\rho u\) and \(E\) reproduced by POD, the FCNN the and CNN with macroscopic quantities computed from the FOM. Top row shows results for \(\hy\), bottom row for \(\rare\) at time \(t_i=0.12s\). CNN is displayed with marks only because of trembles in the signal.}
	\label{Fig:ErrMacro}
\end{figure}
Loss of information described above can unfold in severe mistakes in \(\rho\), \(\rho u\) and \(E\), the macroscopic quantities, as displayed in \cref{Fig:ErrMacro}. Examining the macroscopic quantities enables a detailed look on the reconstruction errors. Features of the macroscopic quantities are expressed in terms of rarefaction wave, contact discontinuity and height as well as position of the shockfront. For a detailed elaboration of these terms see \cref{Ch:BGK}. Following the structure in the preceding figures, macroscopic quantities of \(\hy\) are displayed in the top row and for \(\rare\) in the bottom row of \cref{Fig:ErrMacro}. First the reproduction of the macroscopic quantities \(\rho\) and \(\rho u\)  obtained by the FCNN match the FOM solution exact for both levels of rarefaction \(\hy\) and \(\rare\). Interestingly, despite the overall impressive performance of the FCNN regarding the small number of parameters it uses, the total energy shows small deviations around the tail of the rarefaction wave for \(\hy\) and somewhat severe errors at the transition form rarefaction wave to shock front for \(\rare\). Second the CNN produces trembles in \(\rho u\) and especially in \(E\) which is why it's shown with marks only. The macroscopic quantities reproduced by the CNN show unmissable, it's inabillity to differentiate between \(\hy\) and \(\rare\). Specifically, seem the macroscopic quantities for \(\rare\) appear to be a copy of the the ones for \(\hy\). Additionally, considering \(\hy\), one can observe, that the momentum \(\rho u\) holds small errors for the tail of the rarefaction wave as well as the contact discontinuity. The value for the tip of the shockwave exceeds, comparable with POD, the exact solution. Third POD performs better on \(\rare\), which is not suprising considering the the difference in number of used parameters, holding only small deviations of the contact discontinuity and the shockwave for the momentum \(\rho u\) and the total energy \(E\). The density \(\rho\) matches the FOM solution exact. Pronounced deviations from the FOM solution occur using POD on \(\hy\). The density \(\rho\) undercuts the original shockwave. The momentum \(\rho u\) heavily exceeds the tail of the rarefaction wave and in the same extent undercuts the contact discontinuity. Hence in the total energy \(E\) the same is observable for the tail of the rarefaction wave and the contact discontinuity.
\begin{figure}[H]
	% This file was created by tikzplotlib v0.9.8.
\begin{tikzpicture}

\begin{groupplot}[
group style={group size=3 by 2,
	horizontal sep=.8cm,
	vertical sep=1.1cm
},
tick align=outside,
tick pos=left,
x grid style={white!69.0196078431373!black},
xlabel={\(t\)},
xmin=-0.006, xmax=0.126,
xtick style={color=black},
y grid style={white!69.0196078431373!black},
ymin=-0.031347233897592, ymax=0.112481710620415,
ytick style={color=black},
x tick label style={/pgf/number format/fixed},
y tick label style={/pgf/number format/fixed},
width=.55\textwidth,
height=.6\textwidth
]
\nextgroupplot[
legend cell align={left},
legend style={fill opacity=0,
	at={(1,1)},
	anchor=north east,
	draw opacity=1,
	text opacity=1,
	draw=none,
	nodes={
		scale=0.7,
		transform shape
	}
},
ylabel={\(\bar{\dot{\rho}}\)},
ymin=-0.0614423751831055, ymax=0.158164024353027,
y label style={yshift=-1em},
y label style={xshift=-.7em}
]
\addplot [semithick, black, mark=x, mark size=2.5, mark repeat=5, mark options={solid}]
table {%
0 -1.06764214677924e-05
0.005 -1.05043313283204e-05
0.01 -1.01943976815733e-05
0.015 -9.92951054712421e-06
0.02 -9.68121178601677e-06
0.025 -9.44233560318253e-06
0.03 -9.20958690642237e-06
0.035 -8.98114045355669e-06
0.04 -8.7558689685352e-06
0.045 -8.53302895365005e-06
0.05 -8.31210804363991e-06
0.055 -8.09274173718677e-06
0.06 -7.87466429130745e-06
0.065 -7.6576779619586e-06
0.07 -7.44163272514697e-06
0.075 -7.22641276951208e-06
0.08 -7.01192718111088e-06
0.085 -6.79810317905094e-06
0.09 -6.58488170302007e-06
0.095 -6.37221425847656e-06
0.1 -6.16006092712951e-06
0.105 -5.94838910217277e-06
0.11 -5.73717198193435e-06
0.115 -5.52638756801116e-06
0.12 -5.42110106493965e-06
};
\addlegendentry{FOM}
\addplot [semithick, red, mark=o, mark size=2.5, mark repeat=5, mark options={solid}]
table {%
0 0.00320090036619547
0.005 -0.00437489515648082
0.01 -0.0113141286949912
0.015 -0.00983423356615987
0.02 -0.00832201478583983
0.025 -0.00715067115115886
0.03 -0.00626926456104826
0.035 -0.00559929078165311
0.04 -0.00507985756838281
0.045 -0.00466825950748984
0.05 -0.00433514926056944
0.055 -0.00406028361365429
0.06 -0.00382951651894814
0.065 -0.00363279606843747
0.07 -0.00346284994708412
0.075 -0.00331432081637928
0.08 -0.0031831922008152
0.085 -0.00306640243334755
0.09 -0.00296158137682312
0.095 -0.00286686807133663
0.1 -0.00278078218987332
0.105 -0.00270213161850563
0.11 -0.00262994470215716
0.115 -0.0025634198784843
0.12 -0.00253148207930565
};
\addlegendentry{POD}
\addplot [semithick, color0, dashed, mark=pentagon, mark size=2.5, mark repeat=5, mark options={solid}]
table {%
0 -0.00746633686219411
0.005 -0.00191205168188446
0.01 0.00320702240208703
0.015 0.00214250930242343
0.02 0.00111506648273263
0.025 0.000475115820727012
0.03 0.00010403093992295
0.035 -7.12205054895776e-05
0.04 -0.000147954261507266
0.045 -0.00017581447766446
0.05 -0.000168761500205505
0.055 -0.000137564761239162
0.06 -0.000105405954101911
0.065 -7.944014357264e-05
0.07 -4.56097915417786e-05
0.075 -1.80734487642553e-05
0.08 1.01643112486727e-05
0.085 3.72892004847358e-05
0.09 5.28989428474347e-05
0.095 7.05332969062056e-05
0.1 9.31240780275289e-05
0.105 0.00010478244558243
0.11 0.000116572332629516
0.115 0.000132521768925642
0.12 0.000141214530785305
};
\addlegendentry{FCNN}
\addplot  [semithick, green!50!black, mark=triangle, mark size=2.5, mark repeat=5, mark options={solid,rotate=180}, only marks]
table {%
0 0.0842819213867188
0.005 0.0232086181640625
0.01 0.0284805297851562
0.015 0.148181915283203
0.02 0.046661376953125
0.025 -0.0418891906738281
0.03 0.0140495300292969
0.035 0.000995635986328125
0.04 0.027679443359375
0.045 0.0475044250488281
0.05 0.0190849304199219
0.055 -0.0116539001464844
0.06 -0.0280494689941406
0.065 -0.00231170654296875
0.07 0.0535316467285156
0.075 0.0304374694824219
0.08 -0.0279808044433594
0.085 -0.0463638305664062
0.09 -0.0435104370117188
0.095 0.0211029052734375
0.1 0.0492706298828125
0.105 -0.00998687744140625
0.11 -0.0488853454589844
0.115 -0.0514602661132812
0.12 -0.0494155883789062
};
\addlegendentry{CNN}

\nextgroupplot[
legend cell align={left},
legend style={fill opacity=0,
	draw opacity=1,
	text opacity=1, 
	at={(1,0)}, 
	anchor=south east, 
	draw=none,
	nodes={
		scale=0.7,
		transform shape
	}
},
ylabel={\(\bar{\dot{\rho u}}\)},
ymin=0.721508781709842, ymax=1.06553556938162,
width=.55\textwidth,
height=.6\textwidth,
y label style={yshift=-1.6em},
y label style={xshift=-1.4em}
]
\addplot  [semithick, black, mark=x, mark size=2.5, mark repeat=5, mark options={solid}]
table {%
0 0.968749498492742
0.005 0.968748999531741
0.01 0.968747998630038
0.015 0.968746993741499
0.02 0.968745987523224
0.025 0.968744980913048
0.03 0.968743974314931
0.035 0.968742967983464
0.04 0.968741962093837
0.045 0.968740956764444
0.05 0.968739952072257
0.055 0.968738948064989
0.06 0.968737944770258
0.065 0.968736942202105
0.07 0.968735940365403
0.075 0.968734939258829
0.08 0.968733938877292
0.085 0.968732939213847
0.09 0.968731940261284
0.095 0.968730942013192
0.1 0.968729944464235
0.105 0.968728947609632
0.11 0.968727951444045
0.115 0.968726955960737
0.12 0.968726458388709
};
\addlegendentry{FOM}
\addplot [semithick, red, mark=o, mark size=2.5, mark repeat=5, mark options={solid}]
table {%
0 0.776747752282815
0.005 0.799229546504149
0.01 0.83735394829241
0.015 0.861900922251396
0.02 0.876455586960931
0.025 0.885983570693113
0.03 0.892681143287458
0.035 0.897645432745145
0.04 0.901479309509238
0.045 0.904538271974328
0.05 0.90704386353466
0.055 0.909140538750041
0.06 0.910926147319329
0.065 0.912469178230276
0.07 0.91381896011011
0.075 0.91501192813239
0.08 0.916075602263358
0.085 0.917031185955863
0.09 0.917895307382834
0.095 0.918681213322456
0.1 0.919399605471469
0.105 0.920059238319471
0.11 0.920667354864786
0.115 0.921230009678775
0.12 0.921500662838827
};
\addlegendentry{POD}
\addplot  [semithick, color0, dashed, mark=pentagon, mark size=2.5, mark repeat=5, mark options={solid}]
table {%
0 0.981022530264175
0.005 0.969845267497955
0.01 0.959354321437879
0.015 0.961170023556659
0.02 0.963262696338745
0.025 0.964992540696261
0.03 0.966216851008718
0.035 0.967038764346892
0.04 0.967648145098694
0.045 0.96805996934129
0.05 0.96834140075498
0.055 0.968603158959088
0.06 0.968791679955521
0.065 0.968910933009226
0.07 0.969025615081555
0.075 0.969102313505783
0.08 0.969166812752425
0.085 0.969199221553223
0.09 0.969228125357358
0.095 0.969266452281406
0.1 0.969295363019938
0.105 0.96931896078058
0.11 0.96933353091193
0.115 0.969335049760666
0.12 0.969333696878397
};
\addlegendentry{FCNN}
\addplot [semithick, green!50!black, mark=triangle, mark size=2.5, mark repeat=5, mark options={solid,rotate=180}, only marks]
table {%
0 0.97954326764293
0.005 0.951945287849175
0.01 0.924409057007322
0.015 0.95474182480782
0.02 0.881638584425079
0.025 0.907850569358606
0.03 1.04989798812381
0.035 1.02313119627251
0.04 0.96617057868159
0.045 0.96673523863701
0.05 1.00865441136285
0.055 1.02940238470562
0.06 0.984113614670854
0.065 0.915812590632678
0.07 0.949019600881284
0.075 0.962200688930006
0.08 0.923812266239562
0.085 0.895938572593428
0.09 0.830964156112351
0.095 0.933549032946724
0.1 0.968627714754298
0.105 0.869239411782875
0.11 0.833597986548245
0.115 0.768703975596079
0.12 0.73714636296765
};
\addlegendentry{CNN}

\nextgroupplot[
legend cell align={left},
legend style={fill opacity=0,
	draw opacity=1, 
	text opacity=1, 
	at={(1,0)}, 
	anchor=south east, 
	draw=none,
	nodes={
		scale=0.7,
		transform shape
	}
},
ylabel={\(\bar{\dot{E}}\)},
ymin=-2.11101520609519, ymax=2.42930439772177,
width=.55\textwidth,
height=.6\textwidth,
y label style={yshift=-2em},
y label style={xshift=-3em}
]
\addplot [semithick, black, mark=x, mark size=2.5, mark repeat=5, mark options={solid}]
table {%
0 4.87748394917276e-05
0.005 4.79943104636504e-05
0.01 4.65877911999257e-05
0.015 4.53842673948657e-05
0.02 4.42552087491777e-05
0.025 4.31684471564608e-05
0.03 4.21090988496076e-05
0.035 4.1068895487939e-05
0.04 4.00427243398838e-05
0.045 3.90272160544214e-05
0.05 3.80200565288646e-05
0.055 3.70196084311658e-05
0.06 3.60246853468027e-05
0.065 3.50344095743083e-05
0.07 3.40481191756226e-05
0.075 3.30653053168817e-05
0.08 3.20855694617705e-05
0.085 3.11085940438716e-05
0.09 3.01341225821261e-05
0.095 2.91619465144777e-05
0.1 2.81918961952954e-05
0.105 2.72238339640296e-05
0.11 2.62576476686149e-05
0.115 2.52932443203235e-05
0.12 2.48114777576802e-05
};
\addlegendentry{FOM}
\addplot  [semithick, red, mark=o, mark size=2.5, mark repeat=5, mark options={solid}]
table {%
0 0.0143189698871922
0.005 -0.00810455140208077
0.01 -0.0298314956071479
0.015 -0.0275728368836106
0.02 -0.0246849481216564
0.025 -0.022325029104703
0.03 -0.0204887749707865
0.035 -0.0190569897727855
0.04 -0.0179250309049905
0.045 -0.0170152912259454
0.05 -0.0162722242749638
0.055 -0.015656108470349
0.06 -0.0151382259954005
0.065 -0.0146975051119611
0.07 -0.0143182441718253
0.075 -0.0139885690766448
0.08 -0.0136993767897806
0.085 -0.0134436012520851
0.09 -0.013215695339742
0.095 -0.0130112593925737
0.1 -0.0128267703619791
0.105 -0.0126593809300424
0.11 -0.0125067682475297
0.115 -0.0123670189805338
0.12 -0.0123001435222179
};
\addlegendentry{POD}
\addplot [semithick, color0, dashed, mark=pentagon, mark size=2.5, mark repeat=5, mark options={solid}]
table {%
0 -0.0144530996476604
0.005 -0.013301816060352
0.01 -0.002997673423355
0.015 0.00688496029257024
0.02 0.00809063479628591
0.025 0.00799179979590647
0.03 0.00721117010957428
0.035 0.0069053952409277
0.04 0.00646577613554911
0.045 0.00591729384263573
0.05 0.00574836602891438
0.055 0.00569547312151641
0.06 0.00551190072786412
0.065 0.00528559237558568
0.07 0.00511527266292688
0.075 0.00491798123422171
0.08 0.00473666708664311
0.085 0.00470337758398642
0.09 0.00442858501268262
0.095 0.0041957603795808
0.1 0.00419847093976955
0.105 0.00404308396729647
0.11 0.0038131760915121
0.115 0.00370871306318676
0.12 0.00370363698505116
};
\addlegendentry{FCNN}
\addplot [semithick, green!50!black, mark=triangle, mark size=2.5, mark repeat=5, mark options={solid,rotate=180}, only marks]
table {%
0 2.22292623391191
0.005 1.77012520595716
0.01 1.42936709093432
0.015 1.82681980799178
0.02 -1.3737616654558
0.025 -1.90463704228533
0.03 0.617198420515578
0.035 0.257909501820961
0.04 0.769456099482134
0.045 -0.134924309656782
0.05 -0.541436050787752
0.055 -0.213320312091664
0.06 -0.815704385410292
0.065 -0.232429535139634
0.07 1.1558903965101
0.075 0.825456903239349
0.08 -0.689314545307141
0.085 -1.07111149054275
0.09 -0.773864974842621
0.095 0.995079248783416
0.1 1.47938049120881
0.105 -0.469208151587615
0.11 -1.08921418156228
0.115 -0.206365288559116
0.12 0.327869451377538
};
\addlegendentry{CNN}

\nextgroupplot[
legend cell align={left},
legend style={fill opacity=0,
	text opacity=1, 
	at={(1,1)}, 
	anchor=north east, 
	draw=none,
	nodes={
		scale=0.7,
		transform shape
	}
},
ylabel={\(\bar{\dot{\rho}}\)},
ymin=-0.0869522094726562, ymax=0.153236389160156,
width=.55\textwidth,
height=.6\textwidth,
y label style={yshift=-1em},
y label style={xshift=-.7em}
]
\addplot [semithick, black, mark=x, mark size=2.5, mark repeat=5, mark options={solid}]
table {%
0 -1.0566054697847e-08
0.005 -1.02966097870194e-08
0.01 -9.79558478775289e-09
0.015 -9.34416988229714e-09
0.02 -8.91122198254379e-09
0.025 -8.48964276656261e-09
0.03 -8.07637690058982e-09
0.035 -7.66986119060675e-09
0.04 -7.26916482562956e-09
0.045 -6.87370516061492e-09
0.05 -6.48320508389588e-09
0.055 -6.1007909835098e-09
0.06 -5.7680509257807e-09
0.065 -5.8217608511768e-09
0.07 -8.19480305835896e-09
0.075 -2.12261852539086e-08
0.08 -7.32128810909671e-08
0.085 -2.42511106307575e-07
0.09 -7.11676761966373e-07
0.095 -1.84790528123813e-06
0.1 -4.3030998000404e-06
0.105 -9.11806750991673e-06
0.11 -1.78100265486592e-05
0.115 -3.24263887492293e-05
0.12 -4.15057958917942e-05
};
\addlegendentry{FOM}
\addplot [semithick, red, mark=o, mark size=2.5, mark repeat=5, mark options={solid}]
table {%
0 -0.00850277287727863
0.005 -0.00772076083175932
0.01 -0.00627892573997002
0.015 -0.00505141716242008
0.02 -0.00399637070279368
0.025 -0.00309606729447864
0.03 -0.00233937069619117
0.035 -0.00171506938696098
0.04 -0.00121061889818463
0.045 -0.000812734404206594
0.05 -0.000508152395383377
0.055 -0.000284182881259198
0.06 -0.000129037084739991
0.065 -3.19895342002496e-05
0.07 1.65679465879975e-05
0.075 2.51385344043342e-05
0.08 1.17137958000058e-06
0.085 -4.88665757245599e-05
0.09 -0.000119410074333359
0.095 -0.000205715696424136
0.1 -0.000303790813717342
0.105 -0.000410338307908376
0.11 -0.000522721868449594
0.115 -0.000638953255311492
0.12 -0.000697822820029614
};
\addlegendentry{POD}
\addplot[semithick, color0, dashed, mark=pentagon, mark size=2.5, mark repeat=5, mark options={solid}]
table {%
0 0.000326463574765512
0.005 -0.00131532022280823
0.01 -0.00240714307327039
0.015 -0.00113891668360822
0.02 -0.000588241155391245
0.025 -0.000590722846460778
0.03 -0.000656246120463777
0.035 -0.000607481205086913
0.04 -6.6451880215368e-05
0.045 4.51196531940923e-05
0.05 -8.27034739785404e-05
0.055 -0.000183065868625931
0.06 -0.000287167389060983
0.065 -0.000487041868908022
0.07 -0.000788421822448981
0.075 -0.000905088287311173
0.08 -0.00105589884350366
0.085 -0.00115068076524949
0.09 -0.0012397377247737
0.095 -0.00139431988364436
0.1 -0.00165710132255015
0.105 -0.0019471158124631
0.11 -0.00238798052932054
0.115 -0.00286884322223102
0.12 -0.0029792309312171
};
\addlegendentry{FCNN}
\addplot [semithick, green!50!black, mark=triangle, mark size=2.5, mark repeat=5, mark options={solid,rotate=180}, only marks]
table {%
0 0.0805206298828125
0.005 0.0176162719726562
0.01 0.0207061767578125
0.015 0.142318725585938
0.02 0.0116500854492188
0.025 -0.0760345458984375
0.03 0.00912857055664062
0.035 -0.0053253173828125
0.04 0.0281791687011719
0.045 0.009979248046875
0.05 -0.0237503051757812
0.055 -0.0180511474609375
0.06 -0.0328407287597656
0.065 0.0026397705078125
0.07 0.0280723571777344
0.075 -0.00083160400390625
0.08 -0.0287628173828125
0.085 -0.0436363220214844
0.09 -0.0279083251953125
0.095 0.0157318115234375
0.1 0.0327720642089844
0.105 -0.00761795043945312
0.11 -0.0422439575195312
0.115 -0.0342559814453125
0.12 -0.0254592895507812
};
\addlegendentry{CNN}

\nextgroupplot[
legend cell align={left},
legend style={fill opacity=0,
	text opacity=1, 
	at={(1,0)}, 
	anchor=south east, 
	draw=none,
	nodes={
		scale=0.7,
		transform shape
	}
},
ylabel={\(\bar{\dot{\rho u}}\)},
ytick={0.8,0.9,1},
ymin=0.701558316660603, ymax=1.04570506859874,
width=.55\textwidth,
height=.6\textwidth,
y label style={yshift=-1.6em},
y label style={xshift=-.9em}
]
\addplot [semithick, black, mark=x, mark size=2.5, mark repeat=5, mark options={solid}]
table {%
0 0.968749977995855
0.005 0.968749977493689
0.01 0.968749976488528
0.015 0.968749975482364
0.02 0.968749974475969
0.025 0.968749973469496
0.03 0.968749972463012
0.035 0.968749971456551
0.04 0.968749970450135
0.045 0.968749969443747
0.05 0.968749968436312
0.055 0.96874996740731
0.06 0.968749966119898
0.065 0.968749962757677
0.07 0.968749947403054
0.075 0.968749879330216
0.08 0.968749627063247
0.085 0.968748844536202
0.09 0.968746766180359
0.095 0.968741931728314
0.1 0.968731885809511
0.105 0.96871291830678
0.11 0.968679910009525
0.115 0.968626314303693
0.12 0.96859346363712
};
\addlegendentry{FOM}
\addplot  [semithick, red, mark=o, mark size=2.5, mark repeat=5, mark options={solid}]
table {%
0 0.874450523447081
0.005 0.882012675145708
0.01 0.896373906746789
0.015 0.908519731859784
0.02 0.918013001095483
0.025 0.92537536910019
0.03 0.931091478463895
0.035 0.935538828032017
0.04 0.939001516793107
0.045 0.941693218607178
0.05 0.943776372159542
0.055 0.945376223453617
0.06 0.946590699787999
0.065 0.947497302172493
0.07 0.948157958205335
0.075 0.948622499149348
0.08 0.948931205690461
0.085 0.949116712127438
0.09 0.949205454769583
0.095 0.949218783052661
0.1 0.949173810629732
0.105 0.949084060185012
0.11 0.948959943099316
0.115 0.948809107113092
0.12 0.948727801667509
};
\addlegendentry{POD}
\addplot [semithick, color0, dashed, mark=pentagon, mark size=2.5, mark repeat=5, mark options={solid}]
table {%
0 0.970399317976372
0.005 0.963503522566775
0.01 0.957508562394808
0.015 0.961566075034872
0.02 0.966404637762006
0.025 0.968473543644464
0.03 0.969027915221748
0.035 0.970405037818537
0.04 0.971588561357327
0.045 0.971385233690603
0.05 0.97084447858429
0.055 0.970142786797226
0.06 0.970058801045607
0.065 0.969591850053609
0.07 0.968916380492354
0.075 0.968742961697338
0.08 0.968382479535162
0.085 0.968240874778057
0.09 0.967734768725052
0.095 0.967435535078145
0.1 0.967173210304168
0.105 0.966540158655194
0.11 0.965694011500871
0.115 0.964573306518972
0.12 0.96413484084443
};
\addlegendentry{FCNN}
\addplot  [semithick, green!50!black, mark=triangle, mark size=2.5, mark repeat=5, mark options={solid,rotate=180}, only marks]
table {%
0 0.961906278822966
0.005 0.933425723879134
0.01 0.905878353170239
0.015 0.940320663551927
0.02 0.857029422900776
0.025 0.879651316447493
0.03 1.03006203441973
0.035 1.00402846909231
0.04 0.953349925570686
0.045 0.935897540926501
0.05 0.970320240998085
0.055 1.00498821285898
0.06 0.961259257132621
0.065 0.898568595199741
0.07 0.907571697214435
0.075 0.911803760208734
0.08 0.892125015958701
0.085 0.867297804450649
0.09 0.809100371138463
0.095 0.88698271808509
0.1 0.912923489640189
0.105 0.833686946039219
0.11 0.802155935069418
0.115 0.744677807304853
0.12 0.717201350839609
};
\addlegendentry{CNN}

\nextgroupplot[
legend cell align={left},
legend style={fill opacity=0, 
	text opacity=1, 
	at={(1,0)},
	 anchor=south east, 
	 draw=none,
	 nodes={
	 	scale=0.7,
	 	transform shape
	 }
},
ylabel={\(\bar{\dot{E}}\)},
ymin=-2.1075135100065, ymax=2.44711060492608,
width=.55\textwidth,
height=.6\textwidth,
y label style={yshift=-2em},
y label style={xshift=-3em}
]
\addplot [semithick, black, mark=x, mark size=2.5, mark repeat=5, mark options={solid}]
table {%
0 4.82813575786167e-08
0.005 4.7063597463648e-08
0.01 4.47984191964679e-08
0.015 4.27564792460089e-08
0.02 4.07976692429202e-08
0.025 3.88903416137509e-08
0.03 3.70209036759661e-08
0.035 3.51821611843661e-08
0.04 3.33698650933911e-08
0.045 3.15811483631023e-08
0.05 2.98109235075117e-08
0.055 2.79997891539097e-08
0.06 2.55037626573085e-08
0.065 1.76024457232415e-08
0.07 -2.03018117872489e-08
0.075 -1.84349591592081e-07
0.08 -7.69065788830403e-07
0.085 -2.50761453557402e-06
0.09 -6.92790824885492e-06
0.095 -1.676259263661e-05
0.1 -3.6295084708371e-05
0.105 -7.1516663975757e-05
0.11 -0.000130010286365945
0.115 -0.000220553638243359
0.12 -0.000274993156637038
};
\addlegendentry{FOM}
\addplot [semithick, red, mark=o, mark size=2.5, mark repeat=5, mark options={solid}]
table {%
0 -0.0369185874576559
0.005 -0.0334847636942506
0.01 -0.0271861464861836
0.015 -0.0218886278141817
0.02 -0.0173890616645664
0.025 -0.0135858142109271
0.03 -0.0104139688794262
0.035 -0.00781383933774293
0.04 -0.00572432983522475
0.045 -0.00408429540355471
0.05 -0.00283483794892092
0.055 -0.00192094953581545
0.06 -0.00129239866497599
0.065 -0.000904061088800034
0.07 -0.000715899060683256
0.075 -0.000692736728687748
0.08 -0.000803927287726935
0.085 -0.00102297077507174
0.09 -0.00132711814463704
0.095 -0.00169698339989566
0.1 -0.00211617764014349
0.105 -0.00257097444239918
0.11 -0.00305001321775222
0.115 -0.00354404418175491
0.12 -0.00379377610802578
};
\addlegendentry{POD}
\addplot  [semithick, color0, dashed, mark=pentagon, mark size=2.5, mark repeat=5, mark options={solid}]
table {%
0 -0.0220373594494561
0.005 -0.0187375871121702
0.01 -0.0137267475558431
0.015 -0.00708892810284567
0.02 -0.000805795336006554
0.025 0.00385607701109691
0.03 0.00688734023708193
0.035 0.00506515072139635
0.04 -0.0011157740152008
0.045 -0.00625783112414524
0.05 -0.00777156337428764
0.055 -0.0101094557751722
0.06 -0.0125639515699199
0.065 -0.0137910901982785
0.07 -0.0135298759123259
0.075 -0.0147397060042884
0.08 -0.0167050544650884
0.085 -0.0165941921786654
0.09 -0.0155688464916217
0.095 -0.0160095281063555
0.1 -0.0163660479894041
0.105 -0.0146851150221678
0.11 -0.0137112109201389
0.115 -0.0133200208008226
0.12 -0.0128320391498704
};
\addlegendentry{FCNN}
\addplot [semithick, green!50!black, mark=triangle, mark size=2.5, mark repeat=5, mark options={solid,rotate=180}, only marks]
table {%
0 2.2400822360655
0.005 1.78518401371828
0.01 1.44243569371781
0.015 1.84259146740263
0.02 -1.36916045778383
0.025 -1.90048514114593
0.03 0.63336327700209
0.035 0.272824821472579
0.04 0.786352591134989
0.045 -0.131427471443889
0.05 -0.540336303919108
0.055 -0.200548516735186
0.06 -0.803961333714966
0.065 -0.218569256769491
0.07 1.1578414467834
0.075 0.82358957236357
0.08 -0.681481865767363
0.085 -1.06250963505425
0.09 -0.761910306073361
0.095 0.997457794438301
0.1 1.47727023886346
0.105 -0.464125495002811
0.11 -1.08283991448095
0.115 -0.195796045692205
0.12 0.341143847513344
};
\addlegendentry{CNN}
\end{groupplot}

\end{tikzpicture}

	\caption{Comparison of the conservative properties of reconstructions obatined from POD, the FCNN and the CNN against the conservative properties of the FOM solution using the temporal mean.}
	\label{Fig:Conservation}
\end{figure}
The physical consistency of \(\tilde{f}\), in terms of conservation of mass momentum and energy is a critical criteria for it's validity. Hence conservation properties are analyzed in the following. In that respect, the temporal mean over the time derivative, which can be calculated exemplary for \(\rho\) with
\begin{equation}
	\frac{\mathrm{d}}{\mathrm{d}t}\int \rho(x,t)\, \mathrm{d}x\Delta t  =\overline{\dot{\rho}}\mathrm{,}
\end{equation}
of the macroscopic quantities is employed. \Cref{Fig:Conservation} shows the conservation of mass, momentum and total energy as a temporal mean for \(\hy\) in the top row and for \(\rare\) in the bottom row.\\
Conservation of mass is met using the FCNN, except for small deviations at the outset for both cases \(\hy\) and \(\rare\). Similarly, does POD meet conservation of mass for \(\rare\). The erroneous \(\hy\) case losses mass at the onset and gains mass towards the end with POD. Conservation of momentum meets the FOM solution, except for minor gains and losses, after \(t\approx 0.03s\) using the FCNN for both cases \(\hy\) and \(\rare\). POD gains momentum of 0.13 for \(\hy\) and 0.07 for \(\rare\). The conservation of total energy is met for \(\hy\) and \(\rare\) using POD and the FCNN. Finally the reconstructions of the CNN do not conserve mass, momentum nor total energy. All conservative properties behave comparable to a sawtooth wave. A gain and loss of either of the quantities can be observed.\\

Because of the black-box nature of neural networks the question of interpretability often arises when working with them \cite{fan2021interpretability}. Especially benchmarking neural networks for model order reduction with POD asks for the evaluation of the interpretability of the intrinsic variables. Owing to the tremendous quality of POD, which lies in "the physically interpretable decomposition"\cite{Kutz}[p.375] of the input data, as stated in \cref{Ch:DimRedAl}.\\
Following the deduction, that \(\hy\) can be completely described in terms of three macroscopic quantities and that \(\rare\) is describable in a similar way, it is put to test if the intrinsic variables \(\idhy\) and \(\idrare\) show any similarities, if not match any macroscopic quantity. To this end two supplementary macroscopic quantities namely the temperature \(T\) and macroscopic velocity \(u\) are added to the three macroscopic variables. In \cref{Fig: Macro_hy} and \cref{Fig: Macro_rare} those are depicted first over the whole domain of \(x\) and \(t\) and for two specific timesteps \(t=0.055s\) and \(t=0.12s\) for \(\hy\) and \(\rare\) respectively. Similarily are the intrinsic variables \(\idhy\) and \(\idrare\) with\\\\
\begin{minipage}{0.45\textwidth}
	\begin{equation}
		[h_0(x,t),\dots,h_p(x,t)] = \idhy
	\end{equation}
\end{minipage}%
\begin{minipage}{0.45\textwidth}
	\begin{equation}
		\mathrm{and}\quad[r_0(x,t),\dots,r_p(x,t)] = \idrare
	\end{equation}
\end{minipage},\\\\\
depicted in \cref{Fig: Code_hy} and \cref{Fig: Code_rare} respectively.\\
Strikingly, most intrinsic variables appear to be a sort of linear combination of the five intrinsic variables. In particular \(\idhy\). For example do the rarefaction wave, shock wave and contact discontinutity, that can be identified in \(h_0\) reflect a combination of those found in the density \(\rho\) and the total energy \(E\). Furthermore does \(h_1\) seem to be a mirror image of the momentum \(\rho u\), where beginning end values are inspired by the temperature \(T\). Then again \(T\) plays a more pronounced role in \(h_2\), where it's fluctuation appears. The peak of \(h_2\) could be guessed from the peak of the macroscopic velocity \(u\) with it's bump at \(t=0.12\).\\
A clear identification of macroscopic quantities that can be seen in \(\idrare\) is possible for \(r_3\), which reflects the shape of the density \(\rho\). Moreover, the peak of the velocity \(u\) can be rediscovered in the through of \(r_0\). For other intrinsic variables of \(\idrare\) namely \(r_1\), \(r_2\) and \(r_3\) a clear discernability of macroscopic quantities is not observed. It seems rather that those constitute abstract information about the rarefied flow. Nonetheless it can't be verified, that also those can be created through linear combinations of the intrinsic variables.
\clearpage
\begin{figure}[htp!]
	% This file was created by tikzplotlib v0.9.8.
\begin{tikzpicture}

\begin{groupplot}[
group style={group size=5 by 3,
	horizontal sep=1.5cm,
	%vertical sep=1cm
},
x tick label style={/pgf/number format/fixed},
y tick label style={/pgf/number format/fixed},
width=0.2\textwidth,
height=0.2\textwidth,
y label style={yshift=-1cm},
x label style={yshift=.5cm}
]
\nextgroupplot[
colorbar horizontal,
colorbar style={xtick={0.124997358609037,0.999999994500002},
	minor xtick={},
	at={(0.5,1.03)},
	xlabel={\(\rho\)},
	x label style={yshift=-1cm},
	anchor=south,
	xticklabel pos=upper,
	height=0.1*\pgfkeysvalueof{/pgfplots/parent axis width},
	tick pos=right
},
colormap/blackwhite,
minor xtick={},
minor ytick={},
point meta max=0.999999994500002,
point meta min=0.124997358609037,
tick align=outside,
tick pos=left,
x grid style={white!69.0196078431373!black},
xlabel={\(x\)},
xmin=0, xmax=1,
xtick style={color=black},
xtick={0,1},
y grid style={white!69.0196078431373!black},
ylabel={\(t\)},
ymin=0, ymax=0.12,
ytick style={color=black},
ytick={0,0.12}
]
\addplot graphics [includegraphics cmd=\pgfimage,xmin=0, xmax=1, ymin=0, ymax=0.12] {Figures/Chapter_5/fom_mac_hy-000.png};

\nextgroupplot[
colorbar horizontal,
colorbar style={xtick={-2.90480104631256e-17,0.415611531428894},
	minor xtick={},
	at={(0.5,1.03)},
	xlabel={\(\rho u\)},
	x label style={yshift=-1cm},
	anchor=south,
	xticklabel pos=upper,
	height=0.1*\pgfkeysvalueof{/pgfplots/parent axis width},
	tick pos=right
},
colormap/blackwhite,
minor xtick={},
minor ytick={},
point meta max=0.415611531428894,
point meta min=-2.90480104631256e-17,
tick align=outside,
tick pos=left,
x grid style={white!69.0196078431373!black},
xlabel={\(x\)},
xmin=0, xmax=1,
xtick style={color=black},
xtick={0,1},
y grid style={white!69.0196078431373!black},
ylabel={\(t\)},
ymin=0, ymax=0.12,
ytick style={color=black},
ytick={0,0.12}
]
\addplot graphics [includegraphics cmd=\pgfimage,xmin=0, xmax=1, ymin=0, ymax=0.12] {Figures/Chapter_5/fom_mac_hy-001.png};

\nextgroupplot[
colorbar horizontal,
colorbar style={xtick={0.0156250080075174,0.49999999725},
	minor xtick={},
	at={(0.5,1.03)},
	xlabel={\(E\)},
	x label style={yshift=-1cm},
	anchor=south,
	xticklabel pos=upper,
	height=0.1*\pgfkeysvalueof{/pgfplots/parent axis width},
	tick pos=right
},
colormap/blackwhite,
minor xtick={},
minor ytick={},
point meta max=0.49999999725,
point meta min=0.0156250080075174,
tick align=outside,
tick pos=left,
x grid style={white!69.0196078431373!black},
xlabel={\(x\)},
xmin=0, xmax=1,
xtick style={color=black},
xtick={0,1},
y grid style={white!69.0196078431373!black},
ylabel={\(t\)},
ymin=0, ymax=0.12,
ytick style={color=black},
ytick={0,0.12}
]
\addplot graphics [includegraphics cmd=\pgfimage,xmin=0, xmax=1, ymin=0, ymax=0.12] {Figures/Chapter_5/fom_mac_hy-002.png};

\nextgroupplot[
colorbar horizontal,
colorbar style={xtick={0.083333377680088,0.333911537501331},
	minor xtick={},
	at={(0.5,1.03)},
	xlabel={\(T\)},
	x label style={yshift=-1cm},
	anchor=south,
	xticklabel pos=upper,
	height=0.1*\pgfkeysvalueof{/pgfplots/parent axis width},
	tick pos=right
},
colormap/blackwhite,
minor xtick={},
minor ytick={},
point meta max=0.333911537501331,
point meta min=0.083333377680088,
tick align=outside,
tick pos=left,
x grid style={white!69.0196078431373!black},
xlabel={\(x\)},
xmin=0, xmax=1,
xtick style={color=black},
xtick={0,1},
y grid style={white!69.0196078431373!black},
ylabel={\(t\)},
ymin=0, ymax=0.12,
ytick style={color=black},
ytick={0,0.12}
]
\addplot graphics [includegraphics cmd=\pgfimage,xmin=0, xmax=1, ymin=0, ymax=0.12] {Figures/Chapter_5/fom_mac_hy-003.png};

\nextgroupplot[
colorbar horizontal,
colorbar style={xtick={-2.90480106228896e-17,0.762304583570471},
	minor xtick={},
	at={(0.5,1.03)},
	xlabel={\(u\)},
	x label style={yshift=-1cm},
	anchor=south,
	xticklabel pos=upper,
	height=0.1*\pgfkeysvalueof{/pgfplots/parent axis width},
	tick pos=right
},
colormap/blackwhite,
minor xtick={},
minor ytick={},
point meta max=0.762304583570471,
point meta min=-2.90480106228896e-17,
tick align=outside,
tick pos=left,
x grid style={white!69.0196078431373!black},
xlabel={\(x\)},
xmin=0, xmax=1,
xtick style={color=black},
xtick={0,1},
y grid style={white!69.0196078431373!black},
ylabel={\(t\)},
ymin=0, ymax=0.12,
ytick style={color=black},
ytick={0,0.12}
]
\addplot graphics [includegraphics cmd=\pgfimage,xmin=0, xmax=1, ymin=0, ymax=0.12] {Figures/Chapter_5/fom_mac_hy-004.png};

\nextgroupplot[
minor xtick={},
minor ytick={},
tick align=outside,
tick pos=left,
x grid style={white!69.0196078431373!black},
xlabel={\(x\)},
xmin=-0.04725, xmax=1.04725,
xtick style={color=black},
xtick={0,1},
y grid style={white!69.0196078431373!black},
ylabel={\(\rho\)},
ymin=0.0812487225722646, ymax=1.04375005506799,
ytick style={color=black},
ytick={0.124998783140252,0.999999994500001}
]
\addplot [thick, black]
table {%
0.0025 0.999999994500001
0.0075 0.999999994500001
0.0125 0.999999994500001
0.0175 0.999999994500001
0.0225 0.999999994500001
0.0275 0.999999994500001
0.0325 0.999999994500001
0.0375 0.999999994500001
0.0425 0.999999994500001
0.0475 0.999999994500001
0.0525 0.999999994500001
0.0575 0.999999994500001
0.0625 0.999999994500001
0.0675 0.999999994500001
0.0725 0.999999994500001
0.0775 0.999999994500001
0.0825 0.999999994500001
0.0875 0.999999994500001
0.0925 0.999999994500001
0.0975 0.999999994500001
0.1025 0.999999994500001
0.1075 0.999999994500001
0.1125 0.999999994500001
0.1175 0.999999994500001
0.1225 0.999999994500001
0.1275 0.999999994500001
0.1325 0.999999994500001
0.1375 0.999999994500001
0.1425 0.999999994500001
0.1475 0.999999994500001
0.1525 0.999999994500001
0.1575 0.999999994500001
0.1625 0.999999994500001
0.1675 0.999999994500001
0.1725 0.999999994500001
0.1775 0.999999994500001
0.1825 0.999999994500001
0.1875 0.999999994500001
0.1925 0.999999994500001
0.1975 0.999999994499995
0.2025 0.999999994499979
0.2075 0.999999994499945
0.2125 0.999999994499812
0.2175 0.999999994499411
0.2225 0.999999994498286
0.2275 0.999999994495131
0.2325 0.999999994486356
0.2375 0.999999994462451
0.2425 0.999999994398538
0.2475 0.999999994230879
0.2525 0.999999993799537
0.2575 0.999999992711538
0.2625 0.999999990022083
0.2675 0.999999983509605
0.2725 0.999999968068196
0.2775 0.999999932235148
0.2825 0.999999850891149
0.2875 0.999999670344097
0.2925 0.999999278742036
0.2975 0.999998449190314
0.3025 0.999996733943182
0.3075 0.999993274379751
0.3125 0.999986472339519
0.3175 0.999973444340702
0.3225 0.999949154727045
0.3275 0.999905105245315
0.3325 0.999827461686922
0.3375 0.999694543592396
0.3425 0.999473712015224
0.3475 0.999117873982564
0.3525 0.998562064831936
0.3575 0.997720814030668
0.3625 0.996487148463013
0.3675 0.994734028691004
0.3725 0.992318679960137
0.3775 0.989089702945492
0.3825 0.98489618194516
0.3875 0.97959747529628
0.3925 0.973072170233571
0.3975 0.965224882968448
0.4025 0.95599009910233
0.4075 0.945332889335659
0.4125 0.933246900459912
0.4175 0.919750381408683
0.4225 0.904881127202527
0.4275 0.888691157798259
0.4325 0.871241779598658
0.4375 0.852599490799377
0.4425 0.832833056283762
0.4475 0.812012045679437
0.4525 0.790207252715807
0.4575 0.767493780228852
0.4625 0.743958347696898
0.4675 0.719713868526402
0.4725 0.694927054753327
0.4775 0.669869154074098
0.4825 0.645004268397664
0.4875 0.621121872204297
0.4925 0.599452993335406
0.4975 0.581482145599878
0.5025 0.567860282329101
0.5075 0.556764870941211
0.5125 0.543690256178222
0.5175 0.524252207006042
0.5225 0.495854068201187
0.5275 0.458447711696402
0.5325 0.414606861545587
0.5375 0.368673803400349
0.5425 0.325265822705033
0.5475 0.287905123346416
0.5525 0.258361258060959
0.5575 0.236757209544028
0.5625 0.222103364648037
0.5675 0.212886919989711
0.5725 0.207510745499852
0.5775 0.204535986651624
0.5825 0.202758707648645
0.5875 0.201169115741411
0.5925 0.198843150352639
0.5975 0.1948258339556
0.6025 0.188107095419277
0.6075 0.177876519405375
0.6125 0.16426613933159
0.6175 0.149322866520476
0.6225 0.136785563900149
0.6275 0.129358806148532
0.6325 0.126302969973132
0.6375 0.125346182811746
0.6425 0.125086839254276
0.6475 0.125020631741026
0.6525 0.125004137887791
0.6575 0.125000082198797
0.6625 0.124999095112264
0.6675 0.124998857268174
0.6725 0.124998800556639
0.6775 0.124998787183934
0.6825 0.124998784067485
0.6875 0.12499878335013
0.6925 0.124998783187123
0.6975 0.124998783150575
0.7025 0.124998783142493
0.7075 0.124998783140731
0.7125 0.124998783140353
0.7175 0.124998783140273
0.7225 0.124998783140256
0.7275 0.124998783140253
0.7325 0.124998783140252
0.7375 0.124998783140252
0.7425 0.124998783140252
0.7475 0.124998783140252
0.7525 0.124998783140252
0.7575 0.124998783140252
0.7625 0.124998783140252
0.7675 0.124998783140252
0.7725 0.124998783140252
0.7775 0.124998783140252
0.7825 0.124998783140252
0.7875 0.124998783140252
0.7925 0.124998783140252
0.7975 0.124998783140252
0.8025 0.124998783140252
0.8075 0.124998783140252
0.8125 0.124998783140252
0.8175 0.124998783140252
0.8225 0.124998783140252
0.8275 0.124998783140252
0.8325 0.124998783140252
0.8375 0.124998783140252
0.8425 0.124998783140252
0.8475 0.124998783140252
0.8525 0.124998783140252
0.8575 0.124998783140252
0.8625 0.124998783140252
0.8675 0.124998783140252
0.8725 0.124998783140252
0.8775 0.124998783140252
0.8825 0.124998783140252
0.8875 0.124998783140252
0.8925 0.124998783140252
0.8975 0.124998783140252
0.9025 0.124998783140252
0.9075 0.124998783140252
0.9125 0.124998783140252
0.9175 0.124998783140252
0.9225 0.124998783140252
0.9275 0.124998783140252
0.9325 0.124998783140252
0.9375 0.124998783140252
0.9425 0.124998783140252
0.9475 0.124998783140252
0.9525 0.124998783140252
0.9575 0.124998783140252
0.9625 0.124998783140252
0.9675 0.124998783140252
0.9725 0.124998783140252
0.9775 0.124998783140252
0.9825 0.124998783140252
0.9875 0.124998783140252
0.9925 0.124998783140252
0.9975 0.124998783140252
};

\nextgroupplot[
minor xtick={},
minor ytick={},
tick align=outside,
tick pos=left,
x grid style={white!69.0196078431373!black},
xlabel={\(x\)},
xmin=-0.04725, xmax=1.04725,
xtick style={color=black},
xtick={0,1},
y grid style={white!69.0196078431373!black},
ylabel={\(\rho u\)},
ymin=-0.0203923530477917, ymax=0.428239414003627,
ytick style={color=black},
ytick={-6.8735992454471e-19,0.407847060955835}
]
\addplot [thick, black]
table {%
0.0025 1.15404067964358e-17
0.0075 1.15404067964358e-17
0.0125 1.15404067964358e-17
0.0175 1.15404067964358e-17
0.0225 1.15404067964358e-17
0.0275 1.15404067964358e-17
0.0325 1.15404067964358e-17
0.0375 1.15404067964358e-17
0.0425 1.15404067964358e-17
0.0475 1.15404067964358e-17
0.0525 1.15404067964358e-17
0.0575 1.15404067964358e-17
0.0625 1.15404067964358e-17
0.0675 1.15404067964358e-17
0.0725 1.15404067964358e-17
0.0775 1.15404067964358e-17
0.0825 1.15404067964358e-17
0.0875 1.15404067964358e-17
0.0925 1.15404067964358e-17
0.0975 1.15404067964358e-17
0.1025 1.15404067964358e-17
0.1075 1.15404067964358e-17
0.1125 1.15404067964358e-17
0.1175 1.15404067964358e-17
0.1225 1.15404067964358e-17
0.1275 1.15404067964358e-17
0.1325 1.15404067964358e-17
0.1375 1.15404067964358e-17
0.1425 1.15404067964358e-17
0.1475 1.15404067964358e-17
0.1525 1.15404067964358e-17
0.1575 1.15404067964358e-17
0.1625 1.15404067964358e-17
0.1675 1.15404067964358e-17
0.1725 1.15404067964358e-17
0.1775 1.15404067964358e-17
0.1825 1.15404067964358e-17
0.1875 1.15404067964358e-17
0.1925 6.84749205666374e-17
0.1975 1.18043119727734e-14
0.2025 4.63578187084172e-14
0.2075 1.34998906499146e-13
0.2125 3.80769725895451e-13
0.2175 1.09080411681393e-12
0.2225 3.11942683901988e-12
0.2275 8.83274132558646e-12
0.2325 2.46723843574641e-11
0.2375 6.77947191162207e-11
0.2425 1.83048261678207e-10
0.2475 4.85286541054676e-10
0.2525 1.26264976834725e-09
0.2575 3.22284613092963e-09
0.2625 8.06676516077416e-09
0.2675 1.97923436391361e-08
0.2725 4.75842496414979e-08
0.2775 1.12052611049674e-07
0.2825 2.58340052684623e-07
0.2875 5.82886313493648e-07
0.2925 1.28647783519687e-06
0.2975 2.77615291651835e-06
0.3025 5.85457555370196e-06
0.3075 1.20597590902295e-05
0.3125 2.42518729365704e-05
0.3175 4.75861049234438e-05
0.3225 9.10545743245009e-05
0.3275 0.000169810507515113
0.3325 0.000308476836421227
0.3375 0.000545550516768451
0.3425 0.000938802432327607
0.3475 0.00157121856341669
0.3525 0.00255655905424123
0.3575 0.00404313619003247
0.3625 0.00621412296938442
0.3675 0.00928282881830269
0.3725 0.0134820716933305
0.3775 0.0190479888885201
0.3825 0.0262000546187554
0.3875 0.0351202259882764
0.3925 0.0459345717502538
0.3975 0.0587002656461379
0.4025 0.0733996222893122
0.4075 0.0899413540120781
0.4125 0.108167910183847
0.4175 0.127866947388986
0.4225 0.148784751256874
0.4275 0.170639672837419
0.4325 0.193134145654113
0.4375 0.215964411396215
0.4425 0.238827561560657
0.4475 0.261425830380349
0.4525 0.283468238294613
0.4575 0.304669705539696
0.4625 0.324747672927422
0.4675 0.343416156836589
0.4725 0.360377217561688
0.4775 0.375310584459639
0.4825 0.387865184387792
0.4875 0.397665013945785
0.4925 0.404359157811206
0.4975 0.407745648107646
0.5025 0.407847060955835
0.5075 0.404658048637106
0.5125 0.397426748388926
0.5175 0.384682475098819
0.5225 0.365179090051594
0.5275 0.338953509072129
0.5325 0.307770367708183
0.5375 0.274681756796444
0.5425 0.243014783462431
0.5475 0.215393813940533
0.5525 0.193238817894406
0.5575 0.176791731767469
0.5625 0.165449433511321
0.5675 0.158150296002077
0.5725 0.153667721457418
0.5775 0.150766738526315
0.5825 0.148228993601284
0.5875 0.144770170630586
0.5925 0.138896823244829
0.5975 0.128811796292169
0.6025 0.112623780654813
0.6075 0.0893187844698495
0.6125 0.0607490005920566
0.6175 0.0331061543958516
0.6225 0.0137788594184931
0.6275 0.00448021636947002
0.6325 0.00124492690667728
0.6375 0.000321774652458812
0.6425 8.07607856890546e-05
0.6475 1.99826545218078e-05
0.6525 4.89416562545134e-06
0.6575 1.18723187355883e-06
0.6625 2.85141295987942e-07
0.6675 6.776175580402e-08
0.6725 1.59230718527975e-08
0.6775 3.69756609747147e-09
0.6825 8.48010666426062e-10
0.6875 1.91980051645153e-10
0.6925 4.2882373432611e-11
0.6975 9.44703081567311e-12
0.7025 2.05191616785857e-12
0.7075 4.39326762502542e-13
0.7125 9.27173685588609e-14
0.7175 1.92686137304128e-14
0.7225 3.93338716569965e-15
0.7275 7.70936989542292e-16
0.7325 1.36280404839347e-16
0.7375 4.86635001234055e-18
0.7425 -6.8735992454471e-19
0.7475 -2.76178848890783e-20
0.7525 -5.62934991969684e-20
0.7575 -5.62968927649107e-20
0.7625 -5.62968927649107e-20
0.7675 -5.62968927649107e-20
0.7725 -5.62968927649107e-20
0.7775 -5.62968927649107e-20
0.7825 -5.62968927649107e-20
0.7875 -5.62968927649107e-20
0.7925 -5.62968927649107e-20
0.7975 -5.62968927649107e-20
0.8025 -5.62968927649107e-20
0.8075 -5.62968927649107e-20
0.8125 -5.62968927649107e-20
0.8175 -5.62968927649107e-20
0.8225 -5.62968927649107e-20
0.8275 -5.62968927649107e-20
0.8325 -5.62968927649107e-20
0.8375 -5.62968927649107e-20
0.8425 -5.62968927649107e-20
0.8475 -5.62968927649107e-20
0.8525 -5.62968927649107e-20
0.8575 -5.62968927649107e-20
0.8625 -5.62968927649107e-20
0.8675 -5.62968927649107e-20
0.8725 -5.62968927649107e-20
0.8775 -5.62968927649107e-20
0.8825 -5.62968927649107e-20
0.8875 -5.62968927649107e-20
0.8925 -5.62968927649107e-20
0.8975 -5.62968927649107e-20
0.9025 -5.62968927649107e-20
0.9075 -5.62968927649107e-20
0.9125 -5.62968927649107e-20
0.9175 -5.62968927649107e-20
0.9225 -5.62968927649107e-20
0.9275 -5.62968927649107e-20
0.9325 -5.62968927649107e-20
0.9375 -5.62968927649107e-20
0.9425 -5.62968927649107e-20
0.9475 -5.62968927649107e-20
0.9525 -5.62968927649107e-20
0.9575 -5.62968927649107e-20
0.9625 -5.62968927649107e-20
0.9675 -5.62968927649107e-20
0.9725 -5.62968927649107e-20
0.9775 -5.62968927649107e-20
0.9825 -5.62968927649107e-20
0.9875 -5.62968927649107e-20
0.9925 -5.62968927649107e-20
0.9975 -5.62968927649107e-20
};

\nextgroupplot[
minor xtick={},
minor ytick={},
tick align=outside,
tick pos=left,
x grid style={white!69.0196078431373!black},
xlabel={\(x\)},
xmin=-0.04725, xmax=1.04725,
xtick style={color=black},
xtick={0,1},
y grid style={white!69.0196078431373!black},
ylabel={\(E\)},
ymin=-0.00858791691266637, ymax=0.524218469352973,
ytick style={color=black},
ytick={0.0156305551903172,0.499999997249989}
]
\addplot [thick, black]
table {%
0.0025 0.499999997249989
0.0075 0.499999997249989
0.0125 0.499999997249989
0.0175 0.499999997249989
0.0225 0.499999997249989
0.0275 0.499999997249989
0.0325 0.499999997249989
0.0375 0.499999997249989
0.0425 0.499999997249989
0.0475 0.499999997249989
0.0525 0.499999997249989
0.0575 0.499999997249989
0.0625 0.499999997249989
0.0675 0.499999997249989
0.0725 0.499999997249989
0.0775 0.499999997249989
0.0825 0.499999997249989
0.0875 0.499999997249989
0.0925 0.499999997249989
0.0975 0.499999997249989
0.1025 0.499999997249989
0.1075 0.499999997249989
0.1125 0.499999997249989
0.1175 0.499999997249989
0.1225 0.499999997249989
0.1275 0.499999997249989
0.1325 0.499999997249989
0.1375 0.499999997249989
0.1425 0.499999997249989
0.1475 0.499999997249989
0.1525 0.499999997249989
0.1575 0.499999997249989
0.1625 0.499999997249989
0.1675 0.499999997249989
0.1725 0.499999997249989
0.1775 0.499999997249989
0.1825 0.499999997249989
0.1875 0.499999997249989
0.1925 0.499999997249989
0.1975 0.499999997249969
0.2025 0.499999997249932
0.2075 0.499999997249845
0.2125 0.499999997249607
0.2175 0.499999997248935
0.2225 0.499999997247024
0.2275 0.499999997241669
0.2325 0.499999997226849
0.2375 0.499999997186552
0.2425 0.499999997079003
0.2475 0.499999996797344
0.2525 0.499999996073909
0.2575 0.499999994252255
0.2625 0.499999989757138
0.2675 0.499999978891962
0.2725 0.499999953178504
0.2775 0.499999893624765
0.2825 0.49999975870707
0.2875 0.49999945988261
0.2925 0.499998813164951
0.2975 0.499997446331006
0.3025 0.499994626933664
0.3075 0.49998895464804
0.3125 0.499977831529625
0.3175 0.499956586870664
0.3225 0.499917095880319
0.3275 0.499845708012845
0.3325 0.499720316905924
0.3375 0.499506495144049
0.3425 0.499152820937251
0.3475 0.498585860485721
0.3525 0.497705712388177
0.3575 0.496383465221192
0.3625 0.494462179217272
0.3675 0.491762851856994
0.3725 0.488096106256415
0.3775 0.483279089715762
0.3825 0.477155597032512
0.3875 0.469616247264005
0.3925 0.460615131858745
0.3975 0.450179931930038
0.4025 0.438413891279307
0.4075 0.42548974132439
0.4125 0.411637152170771
0.4175 0.397126153333824
0.4225 0.382249133561768
0.4275 0.367303637050045
0.4325 0.352577481687938
0.4375 0.338336982087429
0.4425 0.324818432825832
0.4475 0.312222566916592
0.4525 0.300711448044882
0.4575 0.290407136555843
0.4625 0.28139142437506
0.4675 0.273705893267251
0.4725 0.267351443052213
0.4775 0.262286206427164
0.4825 0.25842048793501
0.4875 0.255607685589313
0.4925 0.253633155420418
0.4975 0.252208500243495
0.5025 0.2509584190308
0.5075 0.249381041733641
0.5125 0.246751182199866
0.5175 0.242279053307741
0.5225 0.235398034701323
0.5275 0.226067578213702
0.5325 0.214897854207237
0.5375 0.2029859282194
0.5425 0.191550708950773
0.5475 0.18157228637848
0.5525 0.17359424832826
0.5575 0.167714811772072
0.5625 0.163693860939902
0.5675 0.161083275207547
0.5725 0.159317103076181
0.5775 0.157733640043597
0.5825 0.155525410369482
0.5875 0.151631296288575
0.5925 0.144620826134798
0.5975 0.132715809070979
0.6025 0.114285187716012
0.6075 0.0892800957304926
0.6125 0.0613309600639335
0.6175 0.0376657006714086
0.6225 0.0236652446888485
0.6275 0.0179859189338998
0.6325 0.0162501979253571
0.6375 0.0157874428881144
0.6425 0.0156696910946927
0.6475 0.0156402256376817
0.6525 0.0156329238674359
0.6575 0.015631130070488
0.6625 0.0156306933488
0.6675 0.0156305880456963
0.6725 0.015630562916633
0.6775 0.0156305569858821
0.6825 0.0156305556024524
0.6875 0.0156305552836985
0.6925 0.0156305552111938
0.6975 0.0156305551949205
0.7025 0.0156305551913181
0.7075 0.0156305551905318
0.7125 0.0156305551903626
0.7175 0.0156305551903268
0.7225 0.0156305551903192
0.7275 0.0156305551903177
0.7325 0.0156305551903173
0.7375 0.0156305551903172
0.7425 0.0156305551903172
0.7475 0.0156305551903172
0.7525 0.0156305551903172
0.7575 0.0156305551903172
0.7625 0.0156305551903172
0.7675 0.0156305551903172
0.7725 0.0156305551903172
0.7775 0.0156305551903172
0.7825 0.0156305551903172
0.7875 0.0156305551903172
0.7925 0.0156305551903172
0.7975 0.0156305551903172
0.8025 0.0156305551903172
0.8075 0.0156305551903172
0.8125 0.0156305551903172
0.8175 0.0156305551903172
0.8225 0.0156305551903172
0.8275 0.0156305551903172
0.8325 0.0156305551903172
0.8375 0.0156305551903172
0.8425 0.0156305551903172
0.8475 0.0156305551903172
0.8525 0.0156305551903172
0.8575 0.0156305551903172
0.8625 0.0156305551903172
0.8675 0.0156305551903172
0.8725 0.0156305551903172
0.8775 0.0156305551903172
0.8825 0.0156305551903172
0.8875 0.0156305551903172
0.8925 0.0156305551903172
0.8975 0.0156305551903172
0.9025 0.0156305551903172
0.9075 0.0156305551903172
0.9125 0.0156305551903172
0.9175 0.0156305551903172
0.9225 0.0156305551903172
0.9275 0.0156305551903172
0.9325 0.0156305551903172
0.9375 0.0156305551903172
0.9425 0.0156305551903172
0.9475 0.0156305551903172
0.9525 0.0156305551903172
0.9575 0.0156305551903172
0.9625 0.0156305551903172
0.9675 0.0156305551903172
0.9725 0.0156305551903172
0.9775 0.0156305551903172
0.9825 0.0156305551903172
0.9875 0.0156305551903172
0.9925 0.0156305551903172
0.9975 0.0156305551903172
};

\nextgroupplot[
minor xtick={},
minor ytick={},
tick align=outside,
tick pos=left,
x grid style={white!69.0196078431373!black},
xlabel={\(x\)},
xmin=-0.04725, xmax=1.04725,
xtick style={color=black},
xtick={0,1},
y grid style={white!69.0196078431373!black},
ylabel={\(T\)},
ymin=0.0708652945120719, ymax=0.345831811372433,
ytick style={color=black},
ytick={0.0833637725511792,0.333333333333326}
]
\addplot [thick, black]
table {%
0.0025 0.333333333333326
0.0075 0.333333333333326
0.0125 0.333333333333326
0.0175 0.333333333333326
0.0225 0.333333333333326
0.0275 0.333333333333326
0.0325 0.333333333333326
0.0375 0.333333333333326
0.0425 0.333333333333326
0.0475 0.333333333333326
0.0525 0.333333333333326
0.0575 0.333333333333326
0.0625 0.333333333333326
0.0675 0.333333333333326
0.0725 0.333333333333326
0.0775 0.333333333333326
0.0825 0.333333333333326
0.0875 0.333333333333326
0.0925 0.333333333333326
0.0975 0.333333333333326
0.1025 0.333333333333326
0.1075 0.333333333333326
0.1125 0.333333333333326
0.1175 0.333333333333326
0.1225 0.333333333333326
0.1275 0.333333333333326
0.1325 0.333333333333326
0.1375 0.333333333333326
0.1425 0.333333333333326
0.1475 0.333333333333326
0.1525 0.333333333333326
0.1575 0.333333333333326
0.1625 0.333333333333326
0.1675 0.333333333333326
0.1725 0.333333333333326
0.1775 0.333333333333326
0.1825 0.333333333333326
0.1875 0.333333333333326
0.1925 0.333333333333326
0.1975 0.333333333333314
0.2025 0.333333333333295
0.2075 0.333333333333248
0.2125 0.333333333333134
0.2175 0.33333333333282
0.2225 0.333333333331921
0.2275 0.333333333329402
0.2325 0.333333333322447
0.2375 0.333333333303551
0.2425 0.333333333253156
0.2475 0.333333333121269
0.2525 0.33333333278276
0.2575 0.33333333193099
0.2625 0.333333329830731
0.2675 0.333333324758106
0.2725 0.333333312762936
0.2775 0.333333285004787
0.2825 0.333333222174291
0.2875 0.333333083140179
0.2925 0.333332782528339
0.2975 0.333332147819492
0.3025 0.333330839955146
0.3075 0.333328211555849
0.3125 0.333323063238259
0.3175 0.333313241845167
0.3225 0.333295007632065
0.3275 0.333262087219794
0.3325 0.333204336726658
0.3375 0.333105980213332
0.3425 0.33294347750529
0.3475 0.332683218543981
0.3525 0.332279422724322
0.3575 0.33167279269194
0.3625 0.33079055376594
0.3675 0.329548418056868
0.3725 0.327854703317146
0.3775 0.325616354940697
0.3825 0.322746106571042
0.3875 0.319169658522021
0.3925 0.314831697425404
0.3975 0.309699848561121
0.4025 0.303766137326838
0.4075 0.297046061768079
0.4125 0.289575784870269
0.4175 0.281408161605075
0.4225 0.272608326145911
0.4275 0.263249440587433
0.4325 0.253409025718083
0.4375 0.243166122044222
0.4425 0.232599407144173
0.4475 0.221786346113873
0.4525 0.210803491143528
0.4575 0.199728199816146
0.4625 0.188642365687236
0.4675 0.177639359588024
0.4725 0.166836436205962
0.4775 0.156396450760727
0.4825 0.146563959060851
0.4875 0.137716202927281
0.4925 0.130400602673548
0.4975 0.125253763870029
0.5025 0.122679258973719
0.5075 0.122526671590588
0.5125 0.124452801764974
0.5175 0.128620172793421
0.5225 0.135694834411105
0.5275 0.146530513979517
0.5325 0.161865831188402
0.5375 0.182021083522491
0.5425 0.206537308441314
0.5475 0.233872750752188
0.5525 0.261465132220015
0.5575 0.286391013633446
0.5625 0.306375344417972
0.5675 0.320481908105421
0.5725 0.329041281004936
0.5775 0.33300531514951
0.5825 0.333214422644781
0.5875 0.329871031241723
0.5925 0.322228534047938
0.5975 0.308422335908426
0.6025 0.285546708046136
0.6075 0.250566560155826
0.6125 0.203319826562274
0.6175 0.151777344050922
0.6225 0.111957476291541
0.6275 0.0922928274721923
0.6325 0.0857412538064779
0.6375 0.0839649548798654
0.6425 0.0835135287719784
0.6475 0.0834007625936885
0.6525 0.0833728335492405
0.6575 0.083365972191872
0.6625 0.0833643013412035
0.6675 0.083363898344254
0.6725 0.0833638021433095
0.6775 0.083363779430819
0.6825 0.0833637741308677
0.6875 0.0833637729092466
0.6925 0.0833637726312631
0.6975 0.0833637725688457
0.7025 0.0833637725550225
0.7075 0.0833637725520039
0.7125 0.0833637725513542
0.7175 0.0833637725512163
0.7225 0.0833637725511871
0.7275 0.083363772551181
0.7325 0.0833637725511798
0.7375 0.0833637725511793
0.7425 0.0833637725511793
0.7475 0.0833637725511792
0.7525 0.0833637725511792
0.7575 0.0833637725511792
0.7625 0.0833637725511792
0.7675 0.0833637725511792
0.7725 0.0833637725511792
0.7775 0.0833637725511792
0.7825 0.0833637725511792
0.7875 0.0833637725511792
0.7925 0.0833637725511792
0.7975 0.0833637725511792
0.8025 0.0833637725511792
0.8075 0.0833637725511792
0.8125 0.0833637725511792
0.8175 0.0833637725511792
0.8225 0.0833637725511792
0.8275 0.0833637725511792
0.8325 0.0833637725511792
0.8375 0.0833637725511792
0.8425 0.0833637725511792
0.8475 0.0833637725511792
0.8525 0.0833637725511792
0.8575 0.0833637725511792
0.8625 0.0833637725511792
0.8675 0.0833637725511792
0.8725 0.0833637725511792
0.8775 0.0833637725511792
0.8825 0.0833637725511792
0.8875 0.0833637725511792
0.8925 0.0833637725511792
0.8975 0.0833637725511792
0.9025 0.0833637725511792
0.9075 0.0833637725511792
0.9125 0.0833637725511792
0.9175 0.0833637725511792
0.9225 0.0833637725511792
0.9275 0.0833637725511792
0.9325 0.0833637725511792
0.9375 0.0833637725511792
0.9425 0.0833637725511792
0.9475 0.0833637725511792
0.9525 0.0833637725511792
0.9575 0.0833637725511792
0.9625 0.0833637725511792
0.9675 0.0833637725511792
0.9725 0.0833637725511792
0.9775 0.0833637725511792
0.9825 0.0833637725511792
0.9875 0.0833637725511792
0.9925 0.0833637725511792
0.9975 0.0833637725511792
};

\nextgroupplot[
minor xtick={},
minor ytick={},
tick align=outside,
tick pos=left,
x grid style={white!69.0196078431373!black},
xlabel={\(x\)},
xmin=-0.04725, xmax=1.04725,
xtick style={color=black},
xtick={0,1},
y grid style={white!69.0196078431373!black},
ylabel={\(u\)},
ymin=-0.0374070824855321, ymax=0.785548732196174,
ytick style={color=black},
ytick={-5.49893292779877e-18,0.748141649710642}
]
\addplot [thick, black]
table {%
0.0025 1.15404068599081e-17
0.0075 1.15404068599081e-17
0.0125 1.15404068599081e-17
0.0175 1.15404068599081e-17
0.0225 1.15404068599081e-17
0.0275 1.15404068599081e-17
0.0325 1.15404068599081e-17
0.0375 1.15404068599081e-17
0.0425 1.15404068599081e-17
0.0475 1.15404068599081e-17
0.0525 1.15404068599081e-17
0.0575 1.15404068599081e-17
0.0625 1.15404068599081e-17
0.0675 1.15404068599081e-17
0.0725 1.15404068599081e-17
0.0775 1.15404068599081e-17
0.0825 1.15404068599081e-17
0.0875 1.15404068599081e-17
0.0925 1.15404068599081e-17
0.0975 1.15404068599081e-17
0.1025 1.15404068599081e-17
0.1075 1.15404068599081e-17
0.1125 1.15404068599081e-17
0.1175 1.15404068599081e-17
0.1225 1.15404068599081e-17
0.1275 1.15404068599081e-17
0.1325 1.15404068599081e-17
0.1375 1.15404068599081e-17
0.1425 1.15404068599081e-17
0.1475 1.15404068599081e-17
0.1525 1.15404068599081e-17
0.1575 1.15404068599081e-17
0.1625 1.15404068599081e-17
0.1675 1.15404068599081e-17
0.1725 1.15404068599081e-17
0.1775 1.15404068599081e-17
0.1825 1.15404068599081e-17
0.1875 1.15404068599081e-17
0.1925 6.84749209432494e-17
0.1975 1.18043120376972e-14
0.2025 4.63578189633862e-14
0.2075 1.34998907241647e-13
0.2125 3.80769727989756e-13
0.2175 1.09080412281399e-12
0.2225 3.11942685618207e-12
0.2275 8.83274137420954e-12
0.2325 2.46723844934988e-11
0.2375 6.77947194916373e-11
0.2425 1.83048262703545e-10
0.2475 4.85286543854353e-10
0.2525 1.26264977617627e-09
0.2575 3.22284615441922e-09
0.2625 8.06676524126367e-09
0.2675 1.97923439655197e-08
0.2725 4.75842511609489e-08
0.2775 1.12052618642903e-07
0.2825 2.58340091205417e-07
0.2875 5.82886505645626e-07
0.2925 1.28647876307992e-06
0.2975 2.77615722180986e-06
0.3025 5.85459467514082e-06
0.3075 1.20598402001349e-05
0.3125 2.42522010121116e-05
0.3175 4.75873686373922e-05
0.3225 9.10592042545963e-05
0.3275 0.000169826623170857
0.3325 0.000308530069678983
0.3375 0.00054571720958686
0.3425 0.000939296772933341
0.3475 0.00157260579990796
0.3525 0.00256024051411518
0.3575 0.00405237230012142
0.3625 0.0062360292141942
0.3675 0.00933197070830903
0.3725 0.0135864334367586
0.3775 0.0192581004855227
0.3825 0.0266018440309217
0.3875 0.0358516909995652
0.3925 0.0472057193242181
0.3975 0.0608151185095969
0.4025 0.0767786427476958
0.4075 0.0951425207212299
0.4125 0.115904923049347
0.4175 0.139023532877633
0.4225 0.16442463742928
0.4275 0.192012344603698
0.4325 0.221676864191569
0.4375 0.253301126410164
0.4425 0.286765228347617
0.4475 0.321948217112475
0.4525 0.358726444638899
0.4575 0.396967002714797
0.4625 0.436513245577197
0.4675 0.47715650879387
0.4725 0.518582799585503
0.4775 0.560274468792936
0.4825 0.601337391690967
0.4875 0.640236693862531
0.4925 0.674546899100992
0.4975 0.701217829632587
0.5025 0.718217268661641
0.5075 0.726802407545993
0.5125 0.730980082634863
0.5175 0.733773687469447
0.5225 0.736464846151926
0.5275 0.739350421922478
0.5325 0.742318558262313
0.5375 0.745053633491184
0.5425 0.747126708368648
0.5475 0.748141649710642
0.5525 0.747940381405063
0.5575 0.746721639894103
0.5625 0.744920878499547
0.5675 0.742884043837545
0.5725 0.740528983630524
0.5775 0.737115952035906
0.5825 0.731061049462526
0.5875 0.719644116826952
0.5925 0.698524555653549
0.5975 0.661163838885577
0.6025 0.598721597416531
0.6075 0.502139263621944
0.6125 0.369820590166959
0.6175 0.2217085377966
0.6225 0.100733286654076
0.6275 0.034634026881214
0.6325 0.00985667167559168
0.6375 0.00256708776638277
0.6425 0.000645637751905175
0.6475 0.000159834854803812
0.6525 3.915202894999e-05
0.6575 9.49784874277672e-06
0.6625 2.28114688135824e-06
0.6675 5.42099002222421e-07
0.6725 1.2738579715877e-07
0.6775 2.95808157884807e-08
0.6825 6.78415132396993e-09
0.6875 1.53585536194704e-09
0.6925 3.43062327002145e-10
0.6975 7.55769822518438e-11
0.7025 1.64154891453581e-11
0.7075 3.51464831467935e-12
0.7125 7.41746169278742e-13
0.7175 1.54150410478714e-13
0.7225 3.14674036569312e-14
0.7275 6.16755595674299e-15
0.7325 1.090253852203e-15
0.7375 3.89311790890025e-17
0.7425 -5.49893292779877e-18
0.7475 -2.20945229987481e-19
0.7525 -4.50352377701194e-19
0.7575 -4.50379526509022e-19
0.7625 -4.50379526509022e-19
0.7675 -4.50379526509022e-19
0.7725 -4.50379526509022e-19
0.7775 -4.50379526509022e-19
0.7825 -4.50379526509022e-19
0.7875 -4.50379526509022e-19
0.7925 -4.50379526509022e-19
0.7975 -4.50379526509022e-19
0.8025 -4.50379526509022e-19
0.8075 -4.50379526509022e-19
0.8125 -4.50379526509022e-19
0.8175 -4.50379526509022e-19
0.8225 -4.50379526509022e-19
0.8275 -4.50379526509022e-19
0.8325 -4.50379526509022e-19
0.8375 -4.50379526509022e-19
0.8425 -4.50379526509022e-19
0.8475 -4.50379526509022e-19
0.8525 -4.50379526509022e-19
0.8575 -4.50379526509022e-19
0.8625 -4.50379526509022e-19
0.8675 -4.50379526509022e-19
0.8725 -4.50379526509022e-19
0.8775 -4.50379526509022e-19
0.8825 -4.50379526509022e-19
0.8875 -4.50379526509022e-19
0.8925 -4.50379526509022e-19
0.8975 -4.50379526509022e-19
0.9025 -4.50379526509022e-19
0.9075 -4.50379526509022e-19
0.9125 -4.50379526509022e-19
0.9175 -4.50379526509022e-19
0.9225 -4.50379526509022e-19
0.9275 -4.50379526509022e-19
0.9325 -4.50379526509022e-19
0.9375 -4.50379526509022e-19
0.9425 -4.50379526509022e-19
0.9475 -4.50379526509022e-19
0.9525 -4.50379526509022e-19
0.9575 -4.50379526509022e-19
0.9625 -4.50379526509022e-19
0.9675 -4.50379526509022e-19
0.9725 -4.50379526509022e-19
0.9775 -4.50379526509022e-19
0.9825 -4.50379526509022e-19
0.9875 -4.50379526509022e-19
0.9925 -4.50379526509022e-19
0.9975 -4.50379526509022e-19
};

\nextgroupplot[
minor xtick={},
minor ytick={},
tick align=outside,
tick pos=left,
x grid style={white!69.0196078431373!black},
xlabel={\(x\)},
xmin=-0.04725, xmax=1.04725,
xtick style={color=black},
xtick={0,1},
y grid style={white!69.0196078431373!black},
ylabel={\(\rho\)},
ymin=0.0812472268144891, ymax=1.04375012629455,
ytick style={color=black},
ytick={0.124997358609037,0.999999994499998}
]
\addplot [thick, black]
table {%
0.0025 0.999999994499998
0.0075 0.999999994499989
0.0125 0.999999994499973
0.0175 0.999999994499938
0.0225 0.999999994499854
0.0275 0.999999994499657
0.0325 0.999999994499225
0.0375 0.999999994498307
0.0425 0.999999994496346
0.0475 0.999999994492153
0.0525 0.999999994483302
0.0575 0.999999994464812
0.0625 0.999999994426618
0.0675 0.999999994348593
0.0725 0.999999994191032
0.0775 0.999999993876523
0.0825 0.999999993255951
0.0875 0.999999992045765
0.0925 0.999999989713722
0.0975 0.999999985273699
0.1025 0.999999976922728
0.1075 0.999999961408873
0.1125 0.999999932947061
0.1175 0.999999881389466
0.1225 0.999999789188686
0.1275 0.999999626442624
0.1325 0.99999934295137
0.1375 0.999998855715861
0.1425 0.99999802963589
0.1475 0.99999664829786
0.1525 0.999994370682796
0.1575 0.999990668414943
0.1625 0.999984736919399
0.1675 0.999975372768066
0.1725 0.999960808885077
0.1775 0.999938499600449
0.1825 0.999904849332842
0.1875 0.999854882524753
0.1925 0.99978185882593
0.1975 0.999676846595887
0.2025 0.999528279209094
0.2075 0.999321531266986
0.2125 0.999038563648431
0.2175 0.998657694619451
0.2225 0.99815355591569
0.2275 0.997497285204109
0.2325 0.996656988391704
0.2375 0.995598477836268
0.2425 0.994286259099913
0.2475 0.992684705039226
0.2525 0.990759328214853
0.2575 0.988478046531036
0.2625 0.985812336037212
0.2675 0.982738179020057
0.2725 0.979236741747293
0.2775 0.975294749126552
0.2825 0.970904557028253
0.2875 0.96606395178504
0.2925 0.960775726924111
0.2975 0.955047098184107
0.3025 0.948889019907117
0.3075 0.942315460872332
0.3125 0.9353426880196
0.3175 0.927988594730014
0.3225 0.920272098375855
0.3275 0.912212621103049
0.3325 0.903829659028516
0.3375 0.895142438489912
0.3425 0.886169653587621
0.3475 0.876929276719067
0.3525 0.867438432750232
0.3575 0.857713327521522
0.3625 0.847769222220737
0.3675 0.837620446519981
0.3725 0.82728044509819
0.3775 0.816761854164244
0.3825 0.806076606850691
0.3875 0.795236068929107
0.3925 0.784251209358396
0.3975 0.773132813964529
0.4025 0.761891755435828
0.4075 0.750539339336509
0.4125 0.739087754739222
0.4175 0.727550670369378
0.4225 0.715944034148932
0.4275 0.704287157239119
0.4325 0.692604194420389
0.4375 0.680926170788684
0.4425 0.669293745622052
0.4475 0.657760931521731
0.4525 0.646399957544768
0.4575 0.635307281677215
0.4625 0.624610228710984
0.4675 0.614472544087769
0.4725 0.605094996557836
0.4775 0.596704251904308
0.4825 0.589521611837022
0.4875 0.583707948506557
0.4925 0.57929759656908
0.4975 0.576156313265513
0.5025 0.574000965899689
0.5075 0.57248017604117
0.5125 0.571264200514682
0.5175 0.570086379629996
0.5225 0.568721456981151
0.5275 0.566925162056573
0.5325 0.564365789892603
0.5375 0.560570505963129
0.5425 0.554906052745071
0.5475 0.546613496573865
0.5525 0.534908968811207
0.5575 0.519142277507728
0.5625 0.49897861904467
0.5675 0.474548678169257
0.5725 0.446512146926428
0.5775 0.41600335371414
0.5825 0.384466860249091
0.5875 0.353428287708994
0.5925 0.324264601694908
0.5975 0.298031573575595
0.6025 0.275379773169268
0.6075 0.256558296419699
0.6125 0.241481616612008
0.6175 0.229826468110973
0.6225 0.221130573161533
0.6275 0.214876144173828
0.6325 0.210551636151645
0.6375 0.207691966758486
0.6425 0.205900478048754
0.6475 0.20485677362768
0.6525 0.204314507250902
0.6575 0.204092812694663
0.6625 0.204064483070227
0.6675 0.204143251128216
0.6725 0.204271659470887
0.6775 0.204410148437548
0.6825 0.204527208694704
0.6875 0.204589750704457
0.6925 0.204552164518092
0.6975 0.204341776813738
0.7025 0.203837533803555
0.7075 0.202838125078778
0.7125 0.201017075159804
0.7175 0.197870633082619
0.7225 0.192691646081915
0.7275 0.18466735979763
0.7325 0.173289805104828
0.7375 0.159203830364094
0.7425 0.144954928763386
0.7475 0.134090935533034
0.7525 0.128237750573878
0.7575 0.125970038823052
0.7625 0.125265821320929
0.7675 0.125069121636156
0.7725 0.125016331920844
0.7775 0.125002354274959
0.7825 0.124998671121747
0.7875 0.124997702861611
0.7925 0.124997448751868
0.7975 0.124997382171878
0.8025 0.124997364756704
0.8075 0.124997360209753
0.8125 0.124997359024918
0.8175 0.124997358716831
0.8225 0.124997358636905
0.8275 0.124997358616222
0.8325 0.124997358610884
0.8375 0.12499735860951
0.8425 0.124997358609158
0.8475 0.124997358609067
0.8525 0.124997358609045
0.8575 0.124997358609039
0.8625 0.124997358609038
0.8675 0.124997358609038
0.8725 0.124997358609037
0.8775 0.124997358609037
0.8825 0.124997358609037
0.8875 0.124997358609037
0.8925 0.124997358609037
0.8975 0.124997358609037
0.9025 0.124997358609037
0.9075 0.124997358609037
0.9125 0.124997358609037
0.9175 0.124997358609037
0.9225 0.124997358609037
0.9275 0.124997358609037
0.9325 0.124997358609037
0.9375 0.124997358609037
0.9425 0.124997358609037
0.9475 0.124997358609037
0.9525 0.124997358609037
0.9575 0.124997358609037
0.9625 0.124997358609037
0.9675 0.124997358609037
0.9725 0.124997358609037
0.9775 0.124997358609037
0.9825 0.124997358609037
0.9875 0.124997358609037
0.9925 0.124997358609037
0.9975 0.124997358609037
};

\nextgroupplot[
minor xtick={},
minor ytick={},
tick align=outside,
tick pos=left,
x grid style={white!69.0196078431373!black},
xlabel={\(x\)},
xmin=-0.04725, xmax=1.04725,
xtick style={color=black},
xtick={0,1},
y grid style={white!69.0196078431373!black},
ylabel={\(\rho u\)},
ymin=-0.0207805765714447, ymax=0.436392108000339,
ytick style={color=black},
ytick={-2.60371400131214e-18,0.415611531428894}
]
\addplot [thick, black]
table {%
0.0025 6.30909939231123e-15
0.0075 2.52743809394443e-14
0.0125 6.71665948905054e-14
0.0175 1.47797315006362e-13
0.0225 3.07924882400649e-13
0.0275 6.6172472758689e-13
0.0325 1.42564590009118e-12
0.0375 3.07415834705836e-12
0.0425 6.61513596957766e-12
0.0475 1.41586575371753e-11
0.0525 3.00685203765747e-11
0.0575 6.32804197847002e-11
0.0625 1.31840368981028e-10
0.0675 2.71816268178485e-10
0.0725 5.54376848420904e-10
0.0775 1.11825149293422e-09
0.0825 2.23051508115254e-09
0.0875 4.3988436607807e-09
0.0925 8.57587718345818e-09
0.0975 1.6525966137083e-08
0.1025 3.14735316685944e-08
0.1075 5.92319220162265e-08
0.1125 1.1013832102345e-07
0.1175 2.02317177002482e-07
0.1225 3.67093736874415e-07
0.1275 6.57821440259511e-07
0.1325 1.16402287545304e-06
0.1375 2.03362895936117e-06
0.1425 3.50728899710936e-06
0.1475 5.97025139178314e-06
0.1525 1.00291758876392e-05
0.1575 1.66233526126904e-05
0.1625 2.71819698904176e-05
0.1675 4.38409222097891e-05
0.1725 6.97336015392367e-05
0.1775 0.000109369362893115
0.1825 0.000169109904238073
0.1875 0.000257746595168556
0.1925 0.000387169949609523
0.1975 0.000573105626212421
0.2025 0.000835870200910254
0.2075 0.00120107654392011
0.2125 0.0017001966437239
0.2175 0.00237087431188884
0.2225 0.00325687722538598
0.2275 0.00440759236330102
0.2325 0.00587700372294412
0.2375 0.00772214475261246
0.2425 0.0100010837355784
0.2475 0.0127705675659017
0.2525 0.0160835049890066
0.2575 0.0199865026149812
0.2625 0.024517668517973
0.2675 0.0297048683221924
0.2725 0.0355645634604718
0.2775 0.0421012918433731
0.2825 0.0493077803908931
0.2875 0.0571656182273979
0.2925 0.0656463765219179
0.2975 0.0747130389685672
0.3025 0.0843216044817968
0.3075 0.0944227366293171
0.3125 0.104963357117122
0.3175 0.115888107826333
0.3225 0.127140633030916
0.3275 0.138664657456851
0.3325 0.150404855140034
0.3375 0.162307518113764
0.3425 0.174321043116104
0.3475 0.186396259514057
0.3525 0.198486623435526
0.3575 0.210548302610127
0.3625 0.222540174456852
0.3675 0.234423757164425
0.3725 0.2461630903626
0.3775 0.257724578799727
0.3825 0.269076809419216
0.3875 0.280190349466021
0.3925 0.291037530789031
0.3975 0.301592223328621
0.4025 0.311829598863364
0.4075 0.321725884407147
0.4125 0.33125810319655
0.4175 0.340403800052193
0.4225 0.34914074723873
0.4275 0.357446627257301
0.4325 0.36529869126666
0.4375 0.37267339798658
0.4425 0.379546051649685
0.4475 0.385890485406314
0.4525 0.391678889528999
0.4575 0.396881977899246
0.4625 0.401469838716726
0.4675 0.405414024841958
0.4725 0.408691630863368
0.4775 0.411292024184161
0.4825 0.413226029159821
0.4875 0.414535185870344
0.4925 0.415295776454039
0.4975 0.415611531428894
0.5025 0.41559377314098
0.5075 0.415336270392398
0.5125 0.414895446220246
0.5175 0.41427973531504
0.5225 0.413441913351203
0.5275 0.412265556853831
0.5325 0.410543069713507
0.5375 0.407952684100831
0.5425 0.404050220477402
0.5475 0.39829414322411
0.5525 0.390116259686097
0.5575 0.379034706827778
0.5625 0.364785475907473
0.5675 0.347433283903082
0.5725 0.327421672581876
0.5775 0.305538966019578
0.5825 0.282804899447141
0.5875 0.260309643125283
0.5925 0.239050561287535
0.5975 0.219807541815168
0.6025 0.20307934526991
0.6075 0.189081107060286
0.6125 0.177786616825983
0.6175 0.168992999905332
0.6225 0.162388541129918
0.6275 0.15761173233599
0.6325 0.154296586879928
0.6375 0.152103711892583
0.6425 0.150738704005009
0.6475 0.149960167756696
0.6525 0.149579858547135
0.6575 0.149457421963541
0.6625 0.149491943424773
0.6675 0.1496120131206
0.6725 0.149765299386181
0.6775 0.149907790936039
0.6825 0.149991956492704
0.6875 0.149952035270804
0.6925 0.149683390951819
0.6975 0.149011228649705
0.7025 0.147642219487674
0.7075 0.145092233973356
0.7125 0.140589880124687
0.7175 0.132983873988083
0.7225 0.120763965024808
0.7275 0.10247138733323
0.7325 0.0779153045242086
0.7375 0.0500927744393442
0.7425 0.0255869633865413
0.7475 0.0101029770077908
0.7525 0.00323067177688417
0.7575 0.000916685479303123
0.7625 0.000247535672051547
0.7675 6.57015394421608e-05
0.7725 1.733476779243e-05
0.7775 4.56166541654529e-06
0.7825 1.19832317168437e-06
0.7875 3.14296210643595e-07
0.7925 8.22998342934205e-08
0.7975 2.15132336229347e-08
0.8025 5.61307170845078e-09
0.8075 1.46156535792315e-09
0.8125 3.79741250295264e-10
0.8175 9.8431384383556e-11
0.8225 2.54493088990962e-11
0.8275 6.56196969727746e-12
0.8325 1.68704996773948e-12
0.8375 4.3243891523643e-13
0.8425 1.10497770422186e-13
0.8475 2.8177175461474e-14
0.8525 7.14463267733202e-15
0.8575 1.78835622885588e-15
0.8625 4.08881035414389e-16
0.8675 5.23823250656242e-17
0.8725 2.23070011976512e-18
0.8775 -1.70270269849063e-18
0.8825 1.91820139524602e-19
0.8875 -2.40410875285755e-18
0.8925 -2.60371400131214e-18
0.8975 1.34734437976866e-19
0.9025 1.3517563413719e-19
0.9075 1.3561001001166e-19
0.9125 1.3561001001166e-19
0.9175 1.3561001001166e-19
0.9225 1.3561001001166e-19
0.9275 1.3561001001166e-19
0.9325 1.3561001001166e-19
0.9375 1.3561001001166e-19
0.9425 1.3561001001166e-19
0.9475 1.3561001001166e-19
0.9525 1.3561001001166e-19
0.9575 1.3561001001166e-19
0.9625 1.3561001001166e-19
0.9675 1.3561001001166e-19
0.9725 1.3561001001166e-19
0.9775 1.3561001001166e-19
0.9825 1.3561001001166e-19
0.9875 1.3561001001166e-19
0.9925 1.3561001001166e-19
0.9975 1.3561001001166e-19
};

\nextgroupplot[
minor xtick={},
minor ytick={},
tick align=outside,
tick pos=left,
x grid style={white!69.0196078431373!black},
xlabel={\(x\)},
xmin=-0.04725, xmax=1.04725,
xtick style={color=black},
xtick={0,1},
y grid style={white!69.0196078431373!black},
ylabel={\(E\)},
ymin=-0.00858107903990665, ymax=0.524218143739969,
ytick style={color=black},
ytick={0.0156370674500877,0.499999997249975}
]
\addplot [thick, black]
table {%
0.0025 0.499999997249975
0.0075 0.499999997249952
0.0125 0.49999999724991
0.0175 0.49999999724983
0.0225 0.499999997249673
0.0275 0.499999997249343
0.0325 0.499999997248633
0.0375 0.499999997247108
0.0425 0.499999997243849
0.0475 0.499999997236918
0.0525 0.49999999722231
0.0575 0.499999997191854
0.0625 0.499999997129034
0.0675 0.499999997000896
0.0725 0.499999996742452
0.0775 0.499999996227163
0.0825 0.499999995211631
0.0875 0.499999993233655
0.0925 0.499999989426775
0.0975 0.499999982187882
0.1025 0.499999968590229
0.1075 0.499999943362711
0.1125 0.49999989714242
0.1175 0.499999813531151
0.1225 0.499999664219035
0.1275 0.499999401043895
0.1325 0.499998943288962
0.1375 0.499998157735899
0.1425 0.499996827939107
0.1475 0.499994607841606
0.1525 0.499990953221392
0.1575 0.499985022618125
0.1625 0.499975537529369
0.1675 0.499960590121808
0.1725 0.499937386002681
0.1775 0.499901910468537
0.1825 0.499848509989294
0.1875 0.499769387444734
0.1925 0.499654020597409
0.1975 0.499488528760043
0.2025 0.499255032002629
0.2075 0.498931068658278
0.2125 0.498489156918177
0.2175 0.497896600156329
0.2225 0.497115637784343
0.2275 0.496104028964957
0.2325 0.494816122733982
0.2375 0.493204416224345
0.2425 0.491221538805241
0.2475 0.488822534483274
0.2525 0.48596726066466
0.2575 0.482622690300756
0.2625 0.478764904226945
0.2675 0.474380592224794
0.2725 0.469467939022164
0.2775 0.46403684380784
0.2825 0.458108495428486
0.2875 0.451714388097888
0.2925 0.444894905918178
0.2975 0.437697625542276
0.3025 0.430175486289908
0.3075 0.422384960603958
0.3125 0.414384331017942
0.3175 0.406232148833711
0.3225 0.397985919383409
0.3275 0.389701032374154
0.3325 0.381429935098179
0.3375 0.373221531647741
0.3425 0.365120782228682
0.3475 0.357168472254166
0.3525 0.349401119983176
0.3575 0.341850992947928
0.3625 0.334546206351425
0.3675 0.327510880277021
0.3725 0.320765336408751
0.3775 0.314326318665062
0.3825 0.308207225493483
0.3875 0.302418344457086
0.3925 0.296967082133645
0.3975 0.291858184256781
0.4025 0.287093942489105
0.4075 0.28267438527183
0.4125 0.278597450881622
0.4175 0.27485914117029
0.4225 0.271453654478241
0.4275 0.268373495894285
0.4325 0.265609562373013
0.4375 0.263151199234399
0.4425 0.260986223393296
0.4475 0.25910090775627
0.4525 0.257479921790285
0.4575 0.256106228028066
0.4625 0.254960948347607
0.4675 0.254023244450158
0.4725 0.253270307401023
0.4775 0.252677601550096
0.4825 0.252219485718782
0.4875 0.251870132202037
0.4925 0.251604330194165
0.4975 0.251397729084703
0.5025 0.25122669129834
0.5075 0.251068332781992
0.5125 0.250900333775037
0.5175 0.250698706378175
0.5225 0.250431897599141
0.5275 0.250051148444404
0.5325 0.249478662542547
0.5375 0.248596912673306
0.5425 0.247244381726547
0.5475 0.245223917214396
0.5525 0.242327893303772
0.5575 0.238378924611544
0.5625 0.233277606118821
0.5675 0.227043160220145
0.5725 0.219832439446087
0.5775 0.211928643264648
0.5825 0.203701164573315
0.5875 0.195547680233291
0.5925 0.187834659876278
0.5975 0.18085111996356
0.6025 0.174784098494021
0.6075 0.169716410533201
0.6125 0.165641200169253
0.6175 0.162485367499619
0.6225 0.160134752274621
0.6275 0.158456399070025
0.6325 0.157315740975606
0.6375 0.156588304507358
0.6425 0.156166477645325
0.6475 0.155962258709567
0.6525 0.155906982680582
0.6575 0.155948945374948
0.6625 0.156049656989133
0.6675 0.156179160619016
0.6725 0.156310449279297
0.6775 0.156412496184095
0.6825 0.156440733162282
0.6875 0.156322878793174
0.6925 0.155936716265043
0.6975 0.155074755703434
0.7025 0.15338930096658
0.7075 0.150312941539415
0.7125 0.144962247510296
0.7175 0.136077279136682
0.7225 0.122162286306937
0.7275 0.102173607088272
0.7325 0.0770700772857907
0.7375 0.0513960997098126
0.7425 0.0317715372289992
0.7475 0.0212761317050686
0.7525 0.0172929078157419
0.7575 0.0160879851700508
0.7625 0.0157570212643047
0.7675 0.0156687593965459
0.7725 0.0156454185249256
0.7775 0.0156392644262287
0.7825 0.015637644580833
0.7875 0.0156372188316351
0.7925 0.0156371070941755
0.7975 0.0156370778143173
0.8025 0.0156370701545981
0.8075 0.0156370681544042
0.8125 0.0156370676331101
0.8175 0.0156370674975363
0.8225 0.0156370674623578
0.8275 0.0156370674532522
0.8325 0.0156370674509016
0.8375 0.0156370674502965
0.8425 0.0156370674501412
0.8475 0.0156370674501014
0.8525 0.0156370674500913
0.8575 0.0156370674500887
0.8625 0.015637067450088
0.8675 0.0156370674500878
0.8725 0.0156370674500877
0.8775 0.0156370674500877
0.8825 0.0156370674500877
0.8875 0.0156370674500877
0.8925 0.0156370674500877
0.8975 0.0156370674500877
0.9025 0.0156370674500877
0.9075 0.0156370674500877
0.9125 0.0156370674500877
0.9175 0.0156370674500877
0.9225 0.0156370674500877
0.9275 0.0156370674500877
0.9325 0.0156370674500877
0.9375 0.0156370674500877
0.9425 0.0156370674500877
0.9475 0.0156370674500877
0.9525 0.0156370674500877
0.9575 0.0156370674500877
0.9625 0.0156370674500877
0.9675 0.0156370674500877
0.9725 0.0156370674500877
0.9775 0.0156370674500877
0.9825 0.0156370674500877
0.9875 0.0156370674500877
0.9925 0.0156370674500877
0.9975 0.0156370674500877
};

\nextgroupplot[
minor xtick={},
minor ytick={},
tick align=outside,
tick pos=left,
x grid style={white!69.0196078431373!black},
xlabel={\(x\)},
xmin=-0.04725, xmax=1.04725,
xtick style={color=black},
xtick={0,1},
y grid style={white!69.0196078431373!black},
ylabel={\(T\)},
ymin=0.0709027614945945, ymax=0.345830027230399,
ytick style={color=black},
ytick={0.0833994553916766,0.333333333333317}
]
\addplot [thick, black]
table {%
0.0025 0.333333333333317
0.0075 0.333333333333305
0.0125 0.333333333333282
0.0175 0.333333333333241
0.0225 0.333333333333164
0.0275 0.33333333333301
0.0325 0.33333333333268
0.0375 0.33333333333197
0.0425 0.333333333330451
0.0475 0.333333333327227
0.0525 0.33333333332044
0.0575 0.333333333306298
0.0625 0.33333333327715
0.0675 0.333333333217733
0.0725 0.333333333097958
0.0775 0.333333332859267
0.0825 0.333333332389103
0.0875 0.333333331473848
0.0925 0.333333329713276
0.0975 0.333333326367355
0.1025 0.33333332008591
0.1075 0.333333308438848
0.1125 0.333333287112586
0.1175 0.333333248557588
0.1225 0.333333179749718
0.1275 0.333333058548142
0.1325 0.333332847874748
0.1375 0.333332486582965
0.1425 0.333331875407135
0.1475 0.333330855774932
0.1525 0.333329178529746
0.1575 0.333326458784172
0.1625 0.333322112295555
0.1675 0.333315268072928
0.1725 0.333304651628372
0.1775 0.333288433696732
0.1825 0.333264040755053
0.1875 0.333227926677574
0.1925 0.333175309673737
0.1975 0.333099885349812
0.2025 0.332993534972526
0.2075 0.332846056893268
0.2125 0.332644957082525
0.2175 0.33237533977773
0.2225 0.332019939151686
0.2275 0.33155932586976
0.2325 0.330972307747844
0.2375 0.330236522472698
0.2425 0.329329195410912
0.2475 0.328228011299205
0.2525 0.32691202994904
0.2575 0.325362567059281
0.2625 0.323563963860251
0.2675 0.321504183063864
0.2725 0.319175190528286
0.2775 0.316573107732801
0.2825 0.313698144918646
0.2875 0.310554344784474
0.2925 0.307149179704092
0.2975 0.30349305104708
0.3025 0.299598738221838
0.3075 0.295480839267284
0.3125 0.291155236209172
0.3175 0.286638608809582
0.3225 0.281948011180238
0.3275 0.277100517891027
0.3325 0.272112940105169
0.3375 0.267001607969873
0.3425 0.261782212814345
0.3475 0.256469701363959
0.3525 0.251078213846156
0.3575 0.245621058237594
0.3625 0.240110713731469
0.3675 0.234558857599005
0.3725 0.228976410851061
0.3775 0.223373599401614
0.3825 0.217760028768635
0.3875 0.212144771736798
0.3925 0.206536469905743
0.3975 0.200943451753608
0.4025 0.195373871900272
0.4075 0.189835878857714
0.4125 0.184337821975768
0.4175 0.178888512882353
0.4225 0.173497562906049
0.4275 0.168175826196188
0.4325 0.162935988764389
0.4375 0.157793355941583
0.4425 0.152766902145897
0.4475 0.147880649680732
0.4525 0.143165418782389
0.4575 0.138660899113977
0.4625 0.134417758330358
0.4675 0.130499015077337
0.4725 0.126979078702183
0.4775 0.123937884817884
0.4825 0.121447391937
0.4875 0.119550344672598
0.4925 0.118238037440193
0.4975 0.117441064641376
0.5025 0.117044593334553
0.5075 0.11692330401508
0.5125 0.116975698056654
0.5175 0.117141200801105
0.5225 0.117400784429857
0.5275 0.117772478560407
0.5325 0.118310548401798
0.5375 0.119109655788011
0.5425 0.120310431989898
0.5475 0.122101956901485
0.5525 0.124718283185903
0.5575 0.128428562825848
0.5625 0.133521792062076
0.5675 0.14028666561698
0.5725 0.148984924093132
0.5775 0.159814655646866
0.5825 0.1728606799388
0.5875 0.188034521783107
0.5925 0.205016989971065
0.5975 0.223228419093142
0.6025 0.241856230675465
0.6075 0.259956230010923
0.6125 0.276611964531891
0.6175 0.29110225187017
0.6225 0.303016325545045
0.6275 0.312279947962335
0.6325 0.319097903223546
0.6375 0.323849934362205
0.6425 0.326983065531404
0.6475 0.328929010568965
0.6525 0.330055966838564
0.6575 0.330650274695768
0.6625 0.330917767670028
0.6675 0.330994745240975
0.6725 0.330961035111101
0.6775 0.330850284783928
0.6825 0.330654378311617
0.6875 0.330319407519665
0.6925 0.329729902092802
0.6975 0.328676023409994
0.7025 0.326795213664925
0.7075 0.323476153480128
0.7125 0.317712571416376
0.7175 0.307910678217011
0.7225 0.291725533290524
0.7275 0.266219462471381
0.7325 0.229110499796242
0.7375 0.182220765644661
0.7425 0.135735406458926
0.7475 0.103887366477796
0.7525 0.0896886855704828
0.7575 0.0851242086596257
0.7625 0.0838578127836245
0.7675 0.0835204401872121
0.7725 0.0834313249728687
0.7775 0.0834078389094148
0.7825 0.0834016577123346
0.7875 0.0834000330846204
0.7925 0.0833996066868509
0.7975 0.0833994949473618
0.8025 0.0833994657142525
0.8075 0.0833994580800972
0.8125 0.0833994560903372
0.8175 0.0833994555728198
0.8225 0.0833994554385246
0.8275 0.0833994554037608
0.8325 0.0833994553947854
0.8375 0.0833994553924746
0.8425 0.0833994553918815
0.8475 0.0833994553917296
0.8525 0.0833994553916906
0.8575 0.0833994553916804
0.8625 0.0833994553916777
0.8675 0.0833994553916768
0.8725 0.0833994553916767
0.8775 0.0833994553916766
0.8825 0.0833994553916766
0.8875 0.0833994553916766
0.8925 0.0833994553916766
0.8975 0.0833994553916767
0.9025 0.0833994553916767
0.9075 0.0833994553916767
0.9125 0.0833994553916767
0.9175 0.0833994553916767
0.9225 0.0833994553916767
0.9275 0.0833994553916767
0.9325 0.0833994553916767
0.9375 0.0833994553916767
0.9425 0.0833994553916767
0.9475 0.0833994553916767
0.9525 0.0833994553916767
0.9575 0.0833994553916767
0.9625 0.0833994553916767
0.9675 0.0833994553916767
0.9725 0.0833994553916767
0.9775 0.0833994553916767
0.9825 0.0833994553916767
0.9875 0.0833994553916767
0.9925 0.0833994553916767
0.9975 0.0833994553916767
};

\nextgroupplot[
minor xtick={},
minor ytick={},
tick align=outside,
tick pos=left,
x grid style={white!69.0196078431373!black},
xlabel={\(x\)},
xmin=-0.04725, xmax=1.04725,
xtick style={color=black},
xtick={0,1},
y grid style={white!69.0196078431373!black},
ylabel={\(u\)},
ymin=-0.0368765529064683, ymax=0.774407611035834,
ytick style={color=black},
ytick={-2.08301521751028e-17,0.737531058129366}
]
\addplot [thick, black]
table {%
0.0025 6.30909942701129e-15
0.0075 2.52743810784537e-14
0.0125 6.71665952599235e-14
0.0175 1.47797315819256e-13
0.0225 3.07924884094281e-13
0.0275 6.61724731226603e-13
0.0325 1.42564590793334e-12
0.0375 3.07415836397144e-12
0.0425 6.61513600598509e-12
0.0475 1.4158657615159e-11
0.0525 3.00685205424536e-11
0.0575 6.32804201349692e-11
0.0625 1.31840369715825e-10
0.0675 2.7181626971463e-10
0.0725 5.54376851641261e-10
0.0775 1.11825149978181e-09
0.0825 2.23051509619524e-09
0.0875 4.39884369577014e-09
0.0925 8.57587727167203e-09
0.0975 1.65259663804494e-08
0.1025 3.14735323949177e-08
0.1075 5.92319243020533e-08
0.1125 1.10138328408548e-07
0.1175 2.02317200999434e-07
0.1225 3.67093814261944e-07
0.1275 6.57821685993654e-07
0.1325 1.16402364027318e-06
0.1375 2.0336312864132e-06
0.1425 3.50729590775934e-06
0.1475 5.97027140235457e-06
0.1525 1.00292323453694e-05
0.1575 1.66235077363668e-05
0.1625 2.71823847773473e-05
0.1675 4.38420019169388e-05
0.1725 6.97363345839396e-05
0.1775 0.000109376089566325
0.1825 0.000169125996689492
0.1875 0.000257784004132395
0.1925 0.000387254425744619
0.1975 0.000573290887114139
0.2025 0.000836264684348561
0.2075 0.00120189199005582
0.2125 0.00170183284768796
0.2175 0.00237406102677883
0.2225 0.0032629019914658
0.2275 0.00441865098650282
0.2325 0.00589671651470361
0.2375 0.00775628420946864
0.2425 0.0100585557167732
0.2475 0.0128646764688463
0.2525 0.0162335135597319
0.2575 0.0202194704122381
0.2625 0.0248705231429032
0.2675 0.0302266350858707
0.2725 0.036318657117596
0.2775 0.0431677622391363
0.2825 0.0507854042232683
0.2875 0.0591737411604795
0.2925 0.0683264311142439
0.2975 0.0782296905677469
0.3025 0.0888635053339017
0.3075 0.100202894412776
0.3125 0.112219145412213
0.3175 0.124880961344195
0.3225 0.138155479510136
0.3275 0.152009141563046
0.3325 0.166408408528767
0.3375 0.181320325274237
0.3425 0.196712945890634
0.3475 0.212555635286164
0.3525 0.228819263640672
0.3575 0.245476309921095
0.3625 0.262500888949361
0.3675 0.279868713972269
0.3725 0.297557003578614
0.3775 0.31554433827403
0.3825 0.333810468052768
0.3875 0.352336067758264
0.3925 0.371102431613885
0.3975 0.390091091570792
0.4025 0.409283335380085
0.4075 0.428659588572078
0.4125 0.448198608449994
0.4175 0.467876415919416
0.4225 0.487664860080531
0.4275 0.507529668237214
0.4325 0.527427777956735
0.4375 0.547303678392814
0.4425 0.567084414178903
0.4475 0.586672857741118
0.4525 0.605938915925551
0.4575 0.624708687190669
0.4625 0.642752584352075
0.4675 0.659775654327771
0.4725 0.675417303379247
0.4775 0.68927282296309
0.4825 0.700951450909761
0.4875 0.710175674206512
0.4925 0.71689539006143
0.4975 0.721352039125129
0.5025 0.724029745297691
0.5075 0.725503323564045
0.5125 0.726275943506429
0.5175 0.726696427274618
0.5225 0.726967319899988
0.5275 0.72719572960618
0.5325 0.727441452097287
0.5375 0.727745537378778
0.5425 0.728141670970431
0.5475 0.728657718334051
0.5525 0.729313364390018
0.5575 0.730117201487479
0.5625 0.73106434220745
0.5675 0.732134130566819
0.5725 0.733287268522676
0.5775 0.734462747215089
0.5825 0.73557679136224
0.5875 0.736527471563388
0.5925 0.737208317028855
0.5975 0.737531058129366
0.6025 0.737451930229758
0.6075 0.736990811441046
0.6125 0.736232510450827
0.6175 0.73530695265148
0.6225 0.73435590026394
0.6275 0.73350037502761
0.6325 0.732820650079393
0.6375 0.732352407589534
0.6425 0.732094968566886
0.6475 0.732024453481061
0.6525 0.732105911419439
0.6575 0.732301250544965
0.6625 0.732572082978921
0.6675 0.732877586174198
0.6725 0.733167291899958
0.6775 0.733367653621364
0.6825 0.733359426601261
0.6875 0.73294011432381
0.6925 0.731761461945225
0.6975 0.729225472016583
0.7025 0.724313215199915
0.7075 0.715310466989405
0.7125 0.699392725781732
0.7175 0.672074839587525
0.7225 0.626721331621559
0.7275 0.554897126625539
0.7325 0.449624283881416
0.7375 0.31464553537929
0.7425 0.176516684219187
0.7475 0.0753442204547963
0.7525 0.0251928294314783
0.7575 0.00727701196147742
0.7625 0.00197608309626107
0.7675 0.000525321826703926
0.7725 0.000138660025662933
0.7775 3.64926360227689e-05
0.7825 9.58668728979697e-06
0.7875 2.51441589283894e-06
0.7925 6.5841211252874e-07
0.7975 1.72109473407634e-07
0.8025 4.4905520363378e-08
0.8075 1.16927697950624e-08
0.8125 3.03799418849773e-09
0.8175 7.87467714470209e-10
0.8225 2.03598773419059e-10
0.8275 5.24968668932007e-11
0.8325 1.34966849418895e-11
0.8375 3.45958442679868e-12
0.8425 8.84000843311345e-13
0.8475 2.25422167116338e-13
0.8525 5.71582692373392e-14
0.8575 1.43071521571061e-14
0.8625 3.27111740571472e-15
0.8675 4.19067455892919e-16
0.8725 1.78459780637624e-17
0.8775 -1.36219094342328e-17
0.8825 1.53459354388896e-18
0.8875 -1.92332764436811e-17
0.8925 -2.08301521751028e-17
0.8975 1.07789828102115e-18
0.9025 1.0814279248891e-18
0.9075 1.08490300531723e-18
0.9125 1.08490300531723e-18
0.9175 1.08490300531723e-18
0.9225 1.08490300531723e-18
0.9275 1.08490300531723e-18
0.9325 1.08490300531723e-18
0.9375 1.08490300531723e-18
0.9425 1.08490300531723e-18
0.9475 1.08490300531723e-18
0.9525 1.08490300531723e-18
0.9575 1.08490300531723e-18
0.9625 1.08490300531723e-18
0.9675 1.08490300531723e-18
0.9725 1.08490300531723e-18
0.9775 1.08490300531723e-18
0.9825 1.08490300531723e-18
0.9875 1.08490300531723e-18
0.9925 1.08490300531723e-18
0.9975 1.08490300531723e-18
};
\end{groupplot}

\end{tikzpicture}

	\caption{Macroscopic quantities of \(\hy\). Density \(\rho\), momentum \(\rho u\), total energy \(E\), teperature \(T\) and velocity \(u\) over time \(t\) and space \(x\) in the top row and specifically at \(t_i=0.055s\) and \(t_i=0.12\) in the middle and bottom row respectively.}
	\label{Fig: Macro_hy}
\end{figure}
\begin{figure}[hbp!]
	% This file was created by tikzplotlib v0.9.8.
\begin{tikzpicture}

\begin{groupplot}[
group style={group size=5 by 3,
	horizontal sep=1.5cm,
	%vertical sep=1cm
},
x tick label style={/pgf/number format/fixed},
y tick label style={/pgf/number format/fixed},
width=0.2\textwidth,
height=0.2\textwidth,
y label style={yshift=-1cm},
x label style={yshift=.5cm}
]
\nextgroupplot[
colorbar horizontal,
colorbar style={xtick={0.124999996149446,0.9999999945},
	minor xtick={},
	at={(0.5,1.03)},
	xlabel={\(\rho\)},
	x label style={yshift=-1cm},
	anchor=south,
	xticklabel pos=upper,
	height=0.1*\pgfkeysvalueof{/pgfplots/parent axis width},
	tick pos=right
},
colormap/blackwhite,
minor xtick={},
minor ytick={},
point meta max=0.9999999945,
point meta min=0.124999996149446,
tick align=outside,
tick pos=left,
x grid style={white!69.0196078431373!black},
xlabel={\(x\)},
xmin=0, xmax=1,
xtick style={color=black},
xtick={0,1},
y grid style={white!69.0196078431373!black},
ylabel={\(t\)},
ymin=0, ymax=0.12,
ytick style={color=black},
ytick={0,0.12}
]
\addplot graphics [includegraphics cmd=\pgfimage,xmin=0, xmax=1, ymin=0, ymax=0.12] {Figures/Chapter_5/fom_mac_rare-000.png};

\nextgroupplot[
colorbar horizontal,
colorbar style={xtick={-2.81647291408234e-17,0.396099553510851},
	minor xtick={},
	at={(0.5,1.03)},
	xlabel={\(\rho u\)},
	x label style={yshift=-1cm},
	anchor=south,
	xticklabel pos=upper,
	height=0.1*\pgfkeysvalueof{/pgfplots/parent axis width},
	tick pos=right
},
colormap/blackwhite,
minor xtick={},
minor ytick={},
point meta max=0.396099553510851,
point meta min=-2.81647291408234e-17,
tick align=outside,
tick pos=left,
x grid style={white!69.0196078431373!black},
xlabel={\(x\)},
xmin=0, xmax=1,
xtick style={color=black},
xtick={0,1},
y grid style={white!69.0196078431373!black},
ylabel={\(t\)},
ymin=0, ymax=0.12,
ytick style={color=black},
ytick={0,0.12}
]
\addplot graphics [includegraphics cmd=\pgfimage,xmin=0, xmax=1, ymin=0, ymax=0.12] {Figures/Chapter_5/fom_mac_rare-001.png};

\nextgroupplot[
colorbar horizontal,
colorbar style={xtick={0.0156250080075174,0.49999999725},
	minor xtick={},
	at={(0.5,1.03)},
	xlabel={\(E\)},
	x label style={yshift=-1cm},
	anchor=south,
	xticklabel pos=upper,
	height=0.1*\pgfkeysvalueof{/pgfplots/parent axis width},
	tick pos=right
},
colormap/blackwhite,
minor xtick={},
minor ytick={},
point meta max=0.49999999725,
point meta min=0.0156250080075174,
tick align=outside,
tick pos=left,
x grid style={white!69.0196078431373!black},
xlabel={\(x\)},
xmin=0, xmax=1,
xtick style={color=black},
xtick={0,1},
y grid style={white!69.0196078431373!black},
ylabel={\(t\)},
ymin=0, ymax=0.12,
ytick style={color=black},
ytick={0,0.12}
]
\addplot graphics [includegraphics cmd=\pgfimage,xmin=0, xmax=1, ymin=0, ymax=0.12] {Figures/Chapter_5/fom_mac_rare-002.png};

\nextgroupplot[
colorbar horizontal,
colorbar style={xtick={0.083333377680088,0.333333333333333},
	minor xtick={},
	at={(0.5,1.03)},
	xlabel={\(T\)},
	x label style={yshift=-1cm},
	anchor=south,
	xticklabel pos=upper,
	height=0.1*\pgfkeysvalueof{/pgfplots/parent axis width},
	tick pos=right
},
colormap/blackwhite,
minor xtick={},
minor ytick={},
point meta max=0.333333333333333,
point meta min=0.083333377680088,
tick align=outside,
tick pos=left,
x grid style={white!69.0196078431373!black},
xlabel={\(x\)},
xmin=0, xmax=1,
xtick style={color=black},
xtick={0,1},
y grid style={white!69.0196078431373!black},
ylabel={\(t\)},
ymin=0, ymax=0.12,
ytick style={color=black},
ytick={0,0.12}
]
\addplot graphics [includegraphics cmd=\pgfimage,xmin=0, xmax=1, ymin=0, ymax=0.12] {Figures/Chapter_5/fom_mac_rare-003.png};

\nextgroupplot[
colorbar horizontal,
colorbar style={xtick={-2.81647292957294e-17,0.815300118278757},
	minor xtick={},
	at={(0.5,1.03)},
	xlabel={\(u\)},
	x label style={yshift=-1cm},
	anchor=south,
	xticklabel pos=upper,
	height=0.1*\pgfkeysvalueof{/pgfplots/parent axis width},
	tick pos=right
},
colormap/blackwhite,
minor xtick={},
minor ytick={},
point meta max=0.815300118278757,
point meta min=-2.81647292957294e-17,
tick align=outside,
tick pos=left,
x grid style={white!69.0196078431373!black},
xlabel={\(x\)},
xmin=0, xmax=1,
xtick style={color=black},
xtick={0,1},
y grid style={white!69.0196078431373!black},
ylabel={\(t\)},
ymin=0, ymax=0.12,
ytick style={color=black},
ytick={0,0.12}
]
\addplot graphics [includegraphics cmd=\pgfimage,xmin=0, xmax=1, ymin=0, ymax=0.12] {Figures/Chapter_5/fom_mac_rare-004.png};

\nextgroupplot[
minor xtick={},
minor ytick={},
tick align=outside,
tick pos=left,
x grid style={white!69.0196078431373!black},
xlabel={\(x\)},
xmin=-0.04725, xmax=1.04725,
xtick style={color=black},
xtick={0,1},
y grid style={white!69.0196078431373!black},
ylabel={\(\rho\)},
ymin=0.0812499964127393, ymax=1.04374999440891,
ytick style={color=black},
ytick={0.124999996321656,0.999999994499992}
]
\addplot [thick, black]
table {%
0.0025 0.999999994499992
0.0075 0.999999994499987
0.0125 0.999999994499984
0.0175 0.999999994499979
0.0225 0.99999999449997
0.0275 0.999999994499955
0.0325 0.999999994499931
0.0375 0.999999994499895
0.0425 0.999999994499839
0.0475 0.999999994499752
0.0525 0.999999994499619
0.0575 0.999999994499412
0.0625 0.999999994499096
0.0675 0.999999994498612
0.0725 0.999999994497872
0.0775 0.999999994496743
0.0825 0.999999994495024
0.0875 0.999999994492411
0.0925 0.999999994488446
0.0975 0.999999994482439
0.1025 0.999999994473355
0.1075 0.999999994459637
0.1125 0.999999994438956
0.1175 0.999999994407823
0.1225 0.999999994361019
0.1275 0.999999994290753
0.1325 0.999999994185393
0.1375 0.999999994027595
0.1425 0.999999993791515
0.1475 0.999999993438671
0.1525 0.999999992911786
0.1575 0.99999999212565
0.1625 0.999999990953536
0.1675 0.999999989207027
0.1725 0.999999986606033
0.1775 0.999999982734243
0.1825 0.999999976972936
0.1875 0.999999968402652
0.1925 0.999999955657141
0.1975 0.999999936706554
0.2025 0.999999908535798
0.2075 0.999999866667832
0.2125 0.999999804458105
0.2175 0.999999712052017
0.2225 0.999999574847832
0.2275 0.999999371236432
0.2325 0.999999069288409
0.2375 0.999998621916753
0.2425 0.999997959845388
0.2475 0.999996981441451
0.2525 0.999995538100307
0.2575 0.999993413381166
0.2625 0.999990293450628
0.2675 0.999985725576187
0.2725 0.999979060404302
0.2775 0.999969372558097
0.2825 0.999955352729003
0.2875 0.999935162994134
0.2925 0.999906245715912
0.2975 0.999865075311847
0.3025 0.999806841766291
0.3075 0.99972505544538
0.3125 0.999611065106376
0.3175 0.999453485514922
0.3225 0.999237538254871
0.3275 0.998944319335616
0.3325 0.998550019831539
0.3375 0.998025140175694
0.3425 0.997333753318997
0.3475 0.996432884539364
0.3525 0.995272083586359
0.3575 0.993793265376366
0.3625 0.991930886383997
0.3675 0.989612504018413
0.3725 0.986759736052723
0.3775 0.983289599224589
0.3825 0.979116165874536
0.3875 0.974152443427643
0.3925 0.96831236431313
0.3975 0.961512782936456
0.4025 0.953675412637978
0.4075 0.944728682663905
0.4125 0.934609513922606
0.4175 0.923264953132502
0.4225 0.910653451119698
0.4275 0.896745411450282
0.4325 0.881522723693477
0.4375 0.864977688679252
0.4425 0.847113188914092
0.4475 0.827947501686763
0.4525 0.807526777135365
0.4575 0.785942810616271
0.4625 0.763341461137397
0.4675 0.739892554457667
0.4725 0.715694148416868
0.4775 0.690638608915494
0.4825 0.664400395330816
0.4875 0.636822434077194
0.4925 0.60872390459478
0.4975 0.58221724504507
0.5025 0.556935924556554
0.5075 0.530416603262094
0.5125 0.500709462416212
0.5175 0.470079777112704
0.5225 0.440410314458697
0.5275 0.412223021598882
0.5325 0.385227892952073
0.5375 0.359159569078226
0.5425 0.334126363447275
0.5475 0.310501889986595
0.5525 0.288671353450243
0.5575 0.26885474172455
0.5625 0.251068395148177
0.5675 0.235183482529519
0.5725 0.221011059845551
0.5775 0.208364130803884
0.5825 0.197082827087874
0.5875 0.187033728876208
0.5925 0.178100725398863
0.5975 0.170178768041583
0.6025 0.163173106794902
0.6075 0.157001237141418
0.6125 0.151593904517918
0.6175 0.146893284417612
0.6225 0.14284865185256
0.6275 0.139411178274656
0.6325 0.136529682971027
0.6375 0.134148544226457
0.6425 0.132208035214824
0.6475 0.130646522868092
0.6525 0.129403525742688
0.6575 0.128422632356991
0.6625 0.127653608867608
0.6675 0.127053455937797
0.6725 0.126586521264851
0.6775 0.126223952386499
0.6825 0.125942799354819
0.6875 0.12572501246365
0.6925 0.125556489345038
0.6975 0.125426246802703
0.7025 0.12532573928966
0.7075 0.125248316442011
0.7125 0.125188799475702
0.7175 0.125143153459574
0.7225 0.125108234411298
0.7275 0.12508159379698
0.7325 0.125061326804513
0.7375 0.125045954048703
0.7425 0.125034328968261
0.7475 0.125025565128243
0.7525 0.125018979067014
0.7575 0.125014045358019
0.7625 0.125010361308271
0.7675 0.125007619273856
0.7725 0.125005584997219
0.7775 0.125004080700578
0.7825 0.12500297192942
0.7875 0.125002157346325
0.7925 0.125001560839928
0.7975 0.125001125445227
0.8025 0.125000808676501
0.8075 0.125000578957971
0.8125 0.125000412904346
0.8175 0.125000293256776
0.8225 0.12500020732225
0.8275 0.125000145798149
0.8325 0.125000101890262
0.8375 0.125000070653526
0.8425 0.125000048501115
0.8475 0.125000032840272
0.8525 0.125000021803176
0.8575 0.125000014048799
0.8625 0.125000008617566
0.8675 0.125000004825148
0.8725 0.125000002185124
0.8775 0.125000000352893
0.8825 0.124999999085122
0.8875 0.124999998210547
0.8925 0.124999997609017
0.8975 0.124999997196515
0.9025 0.124999996914473
0.9075 0.124999996722195
0.9125 0.124999996591496
0.9175 0.124999996502911
0.9225 0.124999996443044
0.9275 0.124999996402699
0.9325 0.124999996375589
0.9375 0.124999996357424
0.9425 0.124999996345286
0.9475 0.124999996337198
0.9525 0.124999996331825
0.9575 0.124999996328264
0.9625 0.124999996325912
0.9675 0.124999996324362
0.9725 0.124999996323344
0.9775 0.124999996322677
0.9825 0.124999996322241
0.9875 0.124999996321957
0.9925 0.124999996321773
0.9975 0.124999996321656
};

\nextgroupplot[
minor xtick={},
minor ytick={},
tick align=outside,
tick pos=left,
x grid style={white!69.0196078431373!black},
xlabel={\(x\)},
xmin=-0.04725, xmax=1.04725,
xtick style={color=black},
xtick={0,1},
y grid style={white!69.0196078431373!black},
ylabel={\(\rho u\)},
ymin=-0.0192124383402218, ymax=0.403461205145173,
ytick style={color=black},
ytick={2.33677247332681e-14,0.384248766804927}
]
\addplot [thick, black]
table {%
0.0025 2.33677247332681e-14
0.0075 3.6255441452383e-14
0.0125 5.54886342850639e-14
0.0175 8.67135124958979e-14
0.0225 1.34698818963518e-13
0.0275 2.07151205084477e-13
0.0325 3.18467443066487e-13
0.0375 4.899035474674e-13
0.0425 7.52673887838116e-13
0.0475 1.15380247952838e-12
0.0525 1.76579034835247e-12
0.0575 2.69821817976218e-12
0.0625 4.11279840960856e-12
0.0675 6.25668492396008e-12
0.0725 9.49783552715481e-12
0.0775 1.43888758925938e-11
0.0825 2.17545590394679e-11
0.0875 3.28258150054568e-11
0.0925 4.94346526676876e-11
0.0975 7.43049793025304e-11
0.1025 1.11477423984032e-10
0.1075 1.669407200739e-10
0.1125 2.49554553783906e-10
0.1175 3.72408321572302e-10
0.1225 5.54817077636274e-10
0.1275 8.25247105671183e-10
0.1325 1.22560776211392e-09
0.1375 1.81753994868377e-09
0.1425 2.69162632107978e-09
0.1475 3.98088138077273e-09
0.1525 5.88050417672937e-09
0.1575 8.6767969661497e-09
0.1625 1.27895072146836e-08
0.1675 1.88338217731321e-08
0.1725 2.77111506739706e-08
0.1775 4.07420647526143e-08
0.1825 5.98609781006711e-08
0.1875 8.79012267775106e-08
0.1925 1.29012441039991e-07
0.1975 1.89271369358607e-07
0.2025 2.7757527924223e-07
0.2075 4.06947481064229e-07
0.2125 5.96442716827283e-07
0.2175 8.73923452141882e-07
0.2225 1.28009660270306e-06
0.2275 1.87436745934427e-06
0.2325 2.74330157858636e-06
0.2375 4.01280936773188e-06
0.2425 5.86561120293073e-06
0.2475 8.56613844476741e-06
0.2525 1.24958180254762e-05
0.2575 1.82027181929906e-05
0.2625 2.64708401827716e-05
0.2675 3.84159516238698e-05
0.2725 5.56167708281105e-05
0.2775 8.02924744729634e-05
0.2825 0.000115539781981149
0.2875 0.000165645018629672
0.2925 0.000236488170236288
0.2975 0.000336056418448237
0.3025 0.000475083184338223
0.3075 0.000667824323855075
0.3125 0.000932974732867279
0.3175 0.001294715237788
0.3225 0.00178386064095368
0.3275 0.00243905526371887
0.3325 0.0033079335267503
0.3375 0.00444813273189675
0.3425 0.0059280175377324
0.3475 0.00782695623050905
0.3525 0.0102349839466134
0.3575 0.0132517031060365
0.3625 0.0169843101358
0.3675 0.0215447004974224
0.3725 0.0270456872141374
0.3775 0.0335964628816897
0.3825 0.0412975282210396
0.3875 0.0502353845229798
0.3925 0.060477324503732
0.3975 0.0720666416937342
0.4025 0.0850185104545335
0.4075 0.0993166871705487
0.4125 0.11491109445894
0.4175 0.131716333357889
0.4225 0.149611253207777
0.4275 0.16843983307205
0.4325 0.188013597690503
0.4375 0.208115345791593
0.4425 0.22850303583486
0.4475 0.248911749322344
0.4525 0.269051987766788
0.4575 0.288605578393918
0.4625 0.307225976352533
0.4675 0.324553627337234
0.4725 0.340250878953999
0.4775 0.354037836210132
0.4825 0.365680651926565
0.4875 0.374893460640168
0.4925 0.381232792925309
0.4975 0.384248766804927
0.5025 0.383251889898417
0.5075 0.378131661279419
0.5125 0.369415182031432
0.5175 0.357698542382117
0.5225 0.343724641687231
0.5275 0.327934664469581
0.5325 0.310550391138711
0.5375 0.291838991326971
0.5425 0.272234054005252
0.5475 0.252267749912733
0.5525 0.232429602412833
0.5575 0.213067672272857
0.5625 0.194375260866561
0.5675 0.176438926599236
0.5725 0.159299000744419
0.5775 0.142988320706212
0.5825 0.127542864873256
0.5875 0.112996588382993
0.5925 0.0993752372533647
0.5975 0.0866963731968018
0.6025 0.0749744462369091
0.6075 0.0642258185442044
0.6125 0.0544694604806122
0.6175 0.0457221989155314
0.6225 0.0379903961849472
0.6275 0.0312614368590754
0.6325 0.0254982158728871
0.6375 0.0206384457864232
0.6425 0.0165987864199386
0.6475 0.0132822926881413
0.6525 0.0105869770455468
0.6575 0.0084134762303586
0.6625 0.00667059270370915
0.6675 0.00527838476107459
0.6725 0.00416914713247937
0.6775 0.00328692934938171
0.6825 0.00258624658806784
0.6875 0.00203048678526661
0.6925 0.00159033023282016
0.6975 0.00124234149109833
0.7025 0.000967788993373826
0.7075 0.000751689564609087
0.7125 0.00058204906092609
0.7175 0.000449263086628851
0.7225 0.000345643784160341
0.7275 0.000265044312757706
0.7325 0.000202558830200726
0.7375 0.000154281230257167
0.7425 0.000117110152017838
0.7475 8.85909089523522e-05
0.7525 6.67872085796664e-05
0.7575 5.0177101561982e-05
0.7625 3.75687254278601e-05
0.7675 2.80322464726286e-05
0.7725 2.08450524287301e-05
0.7775 1.54477674673361e-05
0.7825 1.14090847444372e-05
0.7875 8.39776138536976e-06
0.7925 6.16041097990705e-06
0.7975 4.50397015155016e-06
0.8025 3.28191709545028e-06
0.8075 2.38348796121288e-06
0.8125 1.72527701922391e-06
0.8175 1.2447230638381e-06
0.8225 8.95081040141598e-07
0.8275 6.41557439338186e-07
0.8325 4.58353188021335e-07
0.8375 3.26410812423989e-07
0.8425 2.31705584948482e-07
0.8475 1.63954873217376e-07
0.8525 1.15647499161798e-07
0.8575 8.13168441806091e-08
0.8625 5.69987632743018e-08
0.8675 3.98289908890664e-08
0.8725 2.77453645902587e-08
0.8775 1.92684653138386e-08
0.8825 1.33406700738687e-08
0.8875 9.20852838226625e-09
0.8925 6.337136985287e-09
0.8975 4.34804863781072e-09
0.9025 2.97441974011211e-09
0.9075 2.02873223232384e-09
0.9125 1.37965294046396e-09
0.9175 9.35503832682133e-10
0.9225 6.32498932543836e-10
0.9275 4.26403118651723e-10
0.9325 2.86638795500749e-10
0.9375 1.92137118715233e-10
0.9425 1.28427073897921e-10
0.9475 8.56009266982616e-11
0.9525 5.68963561084084e-11
0.9575 3.77122456770427e-11
0.9625 2.4927601519356e-11
0.9675 1.64318723405357e-11
0.9725 1.08021613059092e-11
0.9775 7.08206965585488e-12
0.9825 4.63060759458027e-12
0.9875 3.01943560276007e-12
0.9925 1.96261694576357e-12
0.9975 1.26837245145557e-12
};

\nextgroupplot[
minor xtick={},
minor ytick={},
tick align=outside,
tick pos=left,
x grid style={white!69.0196078431373!black},
xlabel={\(x\)},
xmin=-0.04725, xmax=1.04725,
xtick style={color=black},
xtick={0,1},
y grid style={white!69.0196078431373!black},
ylabel={\(E\)},
ymin=-0.00859373560820462, ymax=0.52421874643366,
ytick style={color=black},
ytick={0.0156250135755165,0.499999997249939}
]
\addplot [thick, black]
table {%
0.0025 0.499999997249939
0.0075 0.499999997249906
0.0125 0.499999997249855
0.0175 0.499999997249778
0.0225 0.499999997249659
0.0275 0.499999997249478
0.0325 0.499999997249201
0.0375 0.499999997248782
0.0425 0.499999997248146
0.0475 0.499999997247184
0.0525 0.499999997245733
0.0575 0.499999997243549
0.0625 0.499999997240268
0.0675 0.499999997235353
0.0725 0.499999997228004
0.0775 0.49999999721704
0.0825 0.499999997200719
0.0875 0.499999997176474
0.0925 0.499999997140533
0.0975 0.499999997087364
0.1025 0.499999997008862
0.1075 0.499999996893183
0.1125 0.499999996723042
0.1175 0.499999996473249
0.1225 0.499999996107155
0.1275 0.499999995571507
0.1325 0.499999994789024
0.1375 0.499999993647674
0.1425 0.499999991985222
0.1475 0.499999989566935
0.1525 0.499999986053435
0.1575 0.49999998095437
0.1625 0.499999973561606
0.1675 0.499999962852912
0.1725 0.499999947353072
0.1775 0.49999992493358
0.1825 0.499999892523735
0.1875 0.499999845693909
0.1925 0.499999778054517
0.1975 0.499999680389368
0.2025 0.499999539406662
0.2075 0.499999335940331
0.2125 0.499999042362839
0.2175 0.499998618869528
0.2225 0.499998008153291
0.2275 0.499997127792131
0.2325 0.499995859402314
0.2375 0.499994033243203
0.2425 0.49999140646812
0.2475 0.499987632566777
0.2525 0.499982218705048
0.2575 0.499974466605868
0.2625 0.499963391310278
0.2675 0.499947610612639
0.2725 0.499925196222446
0.2775 0.499893475876635
0.2825 0.499848773917252
0.2875 0.499786076597365
0.2925 0.499698608084711
0.2975 0.499577304483748
0.3025 0.499410177054769
0.3075 0.499181563153822
0.3125 0.498871275214123
0.3175 0.4984536750574
0.3225 0.4978967231054
0.3275 0.49716107882136
0.3325 0.496199357736814
0.3375 0.494955677862722
0.3425 0.493365648711269
0.3475 0.491356963061063
0.3525 0.48885073846876
0.3575 0.485763717351832
0.3625 0.482011369413189
0.3675 0.47751185102333
0.3725 0.472190671003142
0.3775 0.465985804159949
0.3825 0.458852899688908
0.3875 0.450770169278394
0.3925 0.441742525630911
0.3975 0.431804585978666
0.4025 0.421022255192264
0.4075 0.409492740858449
0.4125 0.39734299274602
0.4175 0.384726658216464
0.4225 0.37181967555793
0.4275 0.35881460737834
0.4325 0.345913827410857
0.4375 0.333321827708354
0.4425 0.32123724557621
0.4475 0.30984554275428
0.4525 0.299313174140482
0.4575 0.289783134797104
0.4625 0.281370145039394
0.4675 0.27415274335285
0.4725 0.268161301126445
0.4775 0.263366281355939
0.4825 0.259676507428182
0.4875 0.256954510963615
0.4925 0.255038810695215
0.4975 0.253743499936003
0.5025 0.252647869890552
0.5075 0.25121143023096
0.5125 0.249026656754948
0.5175 0.245986744946308
0.5225 0.242030859928761
0.5275 0.237079905049926
0.5325 0.231068076258586
0.5375 0.224003172509808
0.5425 0.215985091586222
0.5475 0.207171733495895
0.5525 0.197725898856536
0.5575 0.187780364077212
0.5625 0.1774346870351
0.5675 0.166772630190438
0.5725 0.155880713362534
0.5775 0.144855974967016
0.5825 0.13380322100094
0.5875 0.122828881627308
0.5925 0.112037764596017
0.5975 0.101534149812424
0.6025 0.0914244891106764
0.6075 0.0818178151369991
0.6125 0.0728214986333684
0.6175 0.0645326025548811
0.6225 0.0570272464005408
0.6275 0.050351307694589
0.6325 0.044515326721939
0.6375 0.039495005956397
0.6425 0.0352368728806966
0.6475 0.0316672407354555
0.6525 0.0287020478765973
0.6575 0.0262555118954736
0.6625 0.024246426276715
0.6675 0.0226018634853194
0.6725 0.0212586928478139
0.6775 0.0201635882407673
0.6825 0.0192721810556539
0.6875 0.0185478539389654
0.6925 0.0179604857323764
0.6975 0.0174853076635305
0.7025 0.0171019311301252
0.7075 0.0167935515965868
0.7125 0.0165463075843674
0.7175 0.0163487660395294
0.7225 0.0161915066435554
0.7275 0.0160667824520404
0.7325 0.0159682396582423
0.7375 0.0158906839467309
0.7425 0.0158298844207152
0.7475 0.0157824085225259
0.7525 0.0157454829709455
0.7575 0.0157168767820556
0.7625 0.0156948031382684
0.7675 0.0156778373681749
0.7725 0.0156648486827806
0.7775 0.0156549436266791
0.7825 0.015647419468543
0.7875 0.0156417259860658
0.7925 0.0156374343035168
0.7975 0.0156342116203985
0.8025 0.0156318008308154
0.8075 0.0156300041773599
0.8125 0.0156286702120174
0.8175 0.0156276834506737
0.8225 0.0156269562080186
0.8275 0.0156264221867492
0.8325 0.0156260314699019
0.8375 0.01562574662894
0.8425 0.015625539714038
0.8475 0.0156253899379965
0.8525 0.0156252819025322
0.8575 0.0156252042463844
0.8625 0.0156251486197397
0.8675 0.0156251089097902
0.8725 0.0156250806585782
0.8775 0.0156250606273381
0.8825 0.0156250464719044
0.8875 0.0156250365019175
0.8925 0.0156250295029597
0.8975 0.0156250246057271
0.9025 0.0156250211902003
0.9075 0.0156250188157387
0.9125 0.0156250171702917
0.9175 0.0156250160336474
0.9225 0.0156250152509434
0.9275 0.0156250147136478
0.9325 0.0156250143459576
0.9375 0.0156250140951078
0.9425 0.0156250139244924
0.9475 0.0156250138087997
0.9525 0.0156250137305855
0.9575 0.0156250136778665
0.9625 0.0156250136424375
0.9675 0.0156250136186979
0.9725 0.0156250136028375
0.9775 0.0156250135922718
0.9825 0.0156250135852537
0.9875 0.0156250135806054
0.9925 0.0156250135775359
0.9975 0.0156250135755165
};

\nextgroupplot[
minor xtick={},
minor ytick={},
tick align=outside,
tick pos=left,
x grid style={white!69.0196078431373!black},
xlabel={\(x\)},
xmin=-0.04725, xmax=1.04725,
xtick style={color=black},
xtick={0,1},
y grid style={white!69.0196078431373!black},
ylabel={\(T\)},
ymin=0.0708334119310708, ymax=0.345833329590544,
ytick style={color=black},
ytick={0.0833334081883196,0.333333333333296}
]
\addplot [thick, black]
table {%
0.0025 0.333333333333296
0.0075 0.333333333333275
0.0125 0.333333333333242
0.0175 0.333333333333192
0.0225 0.333333333333116
0.0275 0.333333333333
0.0325 0.333333333332824
0.0375 0.333333333332556
0.0425 0.333333333332151
0.0475 0.333333333331539
0.0525 0.333333333330616
0.0575 0.333333333329228
0.0625 0.333333333327147
0.0675 0.333333333324031
0.0725 0.333333333319378
0.0775 0.333333333312446
0.0825 0.333333333302138
0.0875 0.333333333286846
0.0925 0.333333333264207
0.0975 0.333333333230763
0.1025 0.333333333181456
0.1075 0.33333333310891
0.1125 0.333333333002376
0.1175 0.333333332846225
0.1225 0.333333332617763
0.1275 0.333333332284087
0.1325 0.333333331797551
0.1375 0.333333331089251
0.1425 0.333333330059643
0.1475 0.333333328565066
0.1525 0.333333326398361
0.1575 0.33333332326103
0.1625 0.333333318723225
0.1675 0.333333312166265
0.1725 0.333333302700036
0.1775 0.333333289044305
0.1825 0.333333269358175
0.1875 0.33333324099505
0.1925 0.33333320015062
0.1975 0.33333314135737
0.2025 0.333333056759124
0.2075 0.333332935070835
0.2125 0.333332760088961
0.2175 0.333332508561854
0.2225 0.333332147151866
0.2275 0.3333316281137
0.2325 0.333330883167284
0.2375 0.333329814846335
0.2425 0.333328284341848
0.2475 0.333326094517724
0.2525 0.333322966338291
0.2575 0.333318506402084
0.2625 0.333312162617577
0.2675 0.333303164293782
0.2725 0.333290442084308
0.2775 0.333272522386881
0.2825 0.333247390081356
0.2875 0.33321231307359
0.2925 0.333163622262731
0.2975 0.333096441606289
0.3025 0.333004365322323
0.3075 0.332879083356536
0.3125 0.332709962396404
0.3175 0.33248359807261
0.3225 0.332183364315622
0.3275 0.331788997350915
0.3325 0.331276263065084
0.3375 0.330616765330084
0.3425 0.32977795674414
0.3475 0.328723409546488
0.3525 0.327413391285373
0.3575 0.32580576670317
0.3625 0.323857215812838
0.3675 0.321524721994431
0.3725 0.318767248512614
0.3775 0.315547493093783
0.3825 0.311833593294066
0.3875 0.307600653706648
0.3925 0.302831980827112
0.3975 0.297519942205642
0.4025 0.291666412018219
0.4075 0.28528282270788
0.4125 0.27838990498588
0.4175 0.271017249928156
0.4225 0.263202838143777
0.4275 0.254992618885296
0.4325 0.246440081713227
0.4375 0.237605624176689
0.4425 0.228555583448065
0.4475 0.219361321774169
0.4525 0.210099755363275
0.4575 0.200857528655088
0.4625 0.191740083386412
0.4675 0.182882474687019
0.4725 0.174451911597328
0.4775 0.166630474397528
0.4825 0.159585152504369
0.4875 0.153476346295668
0.4925 0.148572230665143
0.4975 0.145359520823201
0.5025 0.14457881383907
0.5075 0.146334323151314
0.5125 0.150123576470322
0.5175 0.155852536447474
0.5225 0.163330127996903
0.5275 0.172462334731979
0.5325 0.183256737975907
0.5375 0.195706342634408
0.5425 0.209664928451833
0.5475 0.224784607028763
0.5525 0.240534643503173
0.5575 0.256277630512773
0.5625 0.271354359727644
0.5675 0.285135625519204
0.5725 0.297033199531276
0.5775 0.306494352672037
0.5825 0.313009878645008
0.5875 0.316147570691825
0.5925 0.315602155585783
0.5975 0.311244161169694
0.6025 0.303154270797109
0.6075 0.291637263906342
0.6125 0.277213177588429
0.6175 0.260582942167813
0.6225 0.242566752795993
0.6275 0.224019766915683
0.6325 0.205739434484896
0.6375 0.188385282924616
0.6425 0.172429659887137
0.6475 0.15814714962721
0.6525 0.145637289781381
0.6575 0.134866735518875
0.6625 0.125715933845723
0.6675 0.118019716622989
0.6725 0.111597121480908
0.6775 0.106270330313022
0.6825 0.10187496097367
0.6875 0.0982644963017722
0.6925 0.0953112237296221
0.6975 0.0929053550859391
0.7025 0.0909533535168867
0.7075 0.0893760299844574
0.7125 0.088106681390608
0.7175 0.0870893807709982
0.7225 0.0862774501063877
0.7275 0.0856321123605462
0.7325 0.0851213082221867
0.7375 0.0847186614543294
0.7425 0.0844025781146052
0.7475 0.0841554664939134
0.7525 0.0839630657096158
0.7575 0.0838138715651808
0.7625 0.083698648809848
0.7675 0.0836100194863242
0.7725 0.0835421177122709
0.7775 0.0834903019935507
0.7825 0.0834509169735129
0.7875 0.0834210973423168
0.7925 0.0833986074341405
0.7975 0.0833817108103125
0.8025 0.083369064852407
0.8075 0.0833596360642518
0.8125 0.0833526324006123
0.8175 0.0833474494995728
0.8225 0.0833436281937656
0.8275 0.0833408211130484
0.8325 0.0833387665703914
0.8375 0.0833372682474989
0.8425 0.0833361794718759
0.8475 0.0833353911080189
0.8525 0.0833348222775095
0.8575 0.0833344132812551
0.8625 0.0833341202267773
0.8675 0.0833339109687264
0.8725 0.0833337620556433
0.8775 0.0833336564438657
0.8825 0.0833335817934067
0.8875 0.0833335292031965
0.8925 0.0833334922764421
0.8975 0.0833334664328704
0.9025 0.0833334484047559
0.9075 0.0833334358691452
0.9125 0.0833334271805606
0.9175 0.0833334211775146
0.9225 0.083333417043005
0.9275 0.0833334142043244
0.9325 0.0833334122613835
0.9375 0.0833334109356283
0.9425 0.083333410033771
0.9475 0.0833334094221351
0.9525 0.0833334090085752
0.9575 0.0833334087297808
0.9625 0.083333408542394
0.9675 0.0833334084168162
0.9725 0.083333408332906
0.9775 0.0833334082770008
0.9825 0.0833334082398613
0.9875 0.0833334082152596
0.9925 0.0833334081990116
0.9975 0.0833334081883196
};

\nextgroupplot[
minor xtick={},
minor ytick={},
tick align=outside,
tick pos=left,
x grid style={white!69.0196078431373!black},
xlabel={\(x\)},
xmin=-0.04725, xmax=1.04725,
xtick style={color=black},
xtick={0,1},
y grid style={white!69.0196078431373!black},
ylabel={\(u\)},
ymin=-0.0407381882705054, ymax=0.855501953681128,
ytick style={color=black},
ytick={2.33677248617908e-14,0.814763765410599}
]
\addplot [thick, black]
table {%
0.0025 2.33677248617908e-14
0.0075 3.62554416517884e-14
0.0125 5.54886345902523e-14
0.0175 8.6713512972824e-14
0.0225 1.34698819704366e-13
0.0275 2.07151206223818e-13
0.0325 3.18467444818079e-13
0.0375 4.89903550161921e-13
0.0425 7.52673891977944e-13
0.0475 1.15380248587458e-12
0.0525 1.76579035806499e-12
0.0575 2.69821819460397e-12
0.0625 4.11279843223267e-12
0.0675 6.25668495838053e-12
0.0725 9.49783557941311e-12
0.0775 1.43888759717795e-11
0.0825 2.17545591592262e-11
0.0875 3.28258151862479e-11
0.0925 4.94346529401494e-11
0.0975 7.43049797125127e-11
0.1025 1.11477424600128e-10
0.1075 1.66940720998812e-10
0.1125 2.4955455517169e-10
0.1175 3.72408323654875e-10
0.1225 5.54817080764877e-10
0.1275 8.25247110382722e-10
0.1325 1.22560776924034e-09
0.1375 1.81753995953885e-09
0.1425 2.6916263377907e-09
0.1475 3.98088140689261e-09
0.1525 5.88050421841164e-09
0.1575 8.67679703447384e-09
0.1625 1.27895073303834e-08
0.1675 1.8833821976405e-08
0.1725 2.77111510451328e-08
0.1775 4.07420654560569e-08
0.1825 5.98609794790937e-08
0.1875 8.79012295549564e-08
0.1925 1.29012446760772e-07
0.1975 1.89271381338245e-07
0.2025 2.77575304630434e-07
0.2075 4.06947535323426e-07
0.2125 5.96442833456845e-07
0.2175 8.7392370378645e-07
0.2225 1.28009714693913e-06
0.2275 1.87436863787898e-06
0.2325 2.74330413181131e-06
0.2375 4.01281489772486e-06
0.2425 5.86562316970889e-06
0.2475 8.56616430223589e-06
0.2525 1.24958737808116e-05
0.2575 1.82028380881467e-05
0.2625 2.64710971257827e-05
0.2675 3.84164999972722e-05
0.2725 5.56179354451923e-05
0.2775 8.02949337013804e-05
0.2825 0.000115544940747431
0.2875 0.000165655759253107
0.2925 0.000236510344094278
0.2975 0.00033610176687432
0.3025 0.000475174968295802
0.3075 0.000668007989014097
0.3125 0.00093333774048209
0.3175 0.00129542320533403
0.3225 0.00178522180428602
0.3275 0.00244163284830635
0.3325 0.00331273692960155
0.3375 0.0044569345528848
0.3425 0.00594386534899147
0.3475 0.00785497583625749
0.3525 0.0102836039665985
0.3575 0.0133344666015802
0.3625 0.0171224733183931
0.3675 0.0217708450630304
0.3725 0.02740858410207
0.3775 0.0341674140641612
0.3825 0.042178374395599
0.3875 0.0515682990500153
0.3925 0.0624564208127523
0.3975 0.0749513089921102
0.4025 0.0891482671440195
0.4075 0.105127206353574
0.4125 0.122950914523278
0.4175 0.142663633999098
0.4225 0.164289997499951
0.4275 0.187834619415154
0.4325 0.213282757933622
0.4375 0.240601981433033
0.4425 0.269743215930537
0.4475 0.300637116260681
0.4525 0.33318026768255
0.4575 0.367209387878511
0.4625 0.402475159537068
0.4675 0.438649673363895
0.4725 0.475413805892702
0.4775 0.512623869618404
0.4825 0.55039198425595
0.4875 0.588693865949334
0.4925 0.626281948265349
0.4975 0.659974897815303
0.5025 0.688143596058328
0.5075 0.712895597449036
0.5125 0.737783504727054
0.5175 0.76093156906079
0.5225 0.78046455862347
0.5275 0.795527292962985
0.5325 0.806147210055082
0.5375 0.812560812665993
0.5425 0.814763765410599
0.5475 0.812451576135731
0.5525 0.805170307461408
0.5575 0.792501076626544
0.5625 0.774192469553339
0.5675 0.750218190076722
0.5725 0.720773887314699
0.5775 0.686242493631474
0.5825 0.647153619408898
0.5875 0.604150861248039
0.5925 0.557972108371881
0.5975 0.509442947522207
0.6025 0.459477959999542
0.6075 0.409078423288814
0.6125 0.35931167980553
0.6175 0.311261328908304
0.6225 0.265948580488941
0.6275 0.2242390979401
0.6325 0.186759503999567
0.6375 0.15384770595485
0.6425 0.125550511305665
0.6475 0.101665872130036
0.6525 0.0818136676321976
0.6575 0.065513968028398
0.6625 0.0522554181028078
0.6675 0.04154459807578
0.6725 0.0329351584261998
0.6775 0.0260404565634033
0.6825 0.0205350889555949
0.6875 0.0161502213877582
0.6925 0.0126662527848308
0.6975 0.00990495628122036
0.7025 0.0077221885851957
0.7075 0.00600159416080546
0.7125 0.00464937009831347
0.7175 0.00358999333330672
0.7225 0.00276275806933718
0.7275 0.0021189713427213
0.7325 0.00161967600517586
0.7375 0.00123379625859048
0.7425 0.000936623989461055
0.7475 0.000708582351629299
0.7525 0.00053421655718262
0.7575 0.000401371713220569
0.7625 0.000300524892774423
0.7675 0.000224244303150978
0.7725 0.0001667529689109
0.7775 0.000123578105456718
0.7825 9.12705079594357e-05
0.7875 6.71809316226704e-05
0.7925 4.92826724603529e-05
0.7975 3.60314368011327e-05
0.8025 2.62551669081102e-05
0.8075 1.90678153739934e-05
0.8125 1.38021705619816e-05
0.8175 9.95776114925735e-06
0.8225 7.16063644465869e-06
0.8275 5.13245352828768e-06
0.8325 3.66682251526263e-06
0.8375 2.61128502341996e-06
0.8425 1.85364396035746e-06
0.8475 1.31163864114245e-06
0.8525 9.25179831919514e-07
0.8575 6.50534680331028e-07
0.8625 4.55990074758216e-07
0.8675 3.18631914812963e-07
0.8725 2.21962912841937e-07
0.8775 1.54147722075528e-07
0.8825 1.06725361372076e-07
0.8875 7.36682281127369e-08
0.8925 5.0697096852023e-08
0.8975 3.47843898826259e-08
0.9025 2.37953585082667e-08
0.9075 1.62298582841772e-08
0.9125 1.10372238246751e-08
0.9175 7.48403087083562e-09
0.9225 5.05999160433605e-09
0.9275 3.41122504738341e-09
0.9325 2.29311043049539e-09
0.9375 1.53709699451381e-09
0.9425 1.02741662122268e-09
0.9475 6.84807433652603e-10
0.9525 4.55170862224439e-10
0.9575 3.01697974278383e-10
0.9625 1.99420818016365e-10
0.9675 1.31454982589733e-10
0.9725 8.64172929890866e-11
0.9775 5.66565589135949e-11
0.9825 3.70448618465788e-11
0.9875 2.41554855328399e-11
0.9925 1.57009360281214e-11
0.9975 1.01469799102372e-11
};

\nextgroupplot[
minor xtick={},
minor ytick={},
tick align=outside,
tick pos=left,
x grid style={white!69.0196078431373!black},
xlabel={\(x\)},
xmin=-0.04725, xmax=1.04725,
xtick style={color=black},
xtick={0,1},
y grid style={white!69.0196078431373!black},
ylabel={\(\rho\)},
ymin=0.081267564417733, ymax=1.04374896210648,
ytick style={color=black},
ytick={0.12501671885813,0.999999807666081}
]
\addplot [thick, black]
table {%
0.0025 0.999999807666081
0.0075 0.999999748719081
0.0125 0.999999674796829
0.0175 0.999999578515261
0.0225 0.999999452070285
0.0275 0.999999285816446
0.0325 0.999999067316936
0.0375 0.999998780365606
0.0425 0.999998403807108
0.0475 0.999997910049692
0.0525 0.999997263171242
0.0575 0.999996416506449
0.0625 0.999995309581959
0.0675 0.999993864239697
0.0725 0.999991979757303
0.0775 0.999989526739214
0.0825 0.999986339513002
0.0875 0.999982206724206
0.0925 0.999976859780583
0.0975 0.999969958755766
0.1025 0.999961075326019
0.1075 0.999949672286213
0.1125 0.999935079177682
0.1175 0.999916463567656
0.1225 0.999892797554757
0.1275 0.999862819145735
0.1325 0.99982498826325
0.1375 0.999777437310758
0.1425 0.999717916444541
0.1475 0.999643733987731
0.1525 0.999551692766091
0.1575 0.999438023543809
0.1625 0.999298317176075
0.1675 0.999127457552945
0.1725 0.998919557856471
0.1775 0.998667903054044
0.1825 0.998364901863486
0.1875 0.99800205160545
0.1925 0.997569919364129
0.1975 0.997058142673286
0.2025 0.996455452509002
0.2075 0.99574972069893
0.2125 0.994928032967131
0.2175 0.993976787763627
0.2225 0.992881819839973
0.2275 0.991628546305134
0.2325 0.990202131718563
0.2375 0.988587667739745
0.2425 0.986770362036876
0.2475 0.98473573062587
0.2525 0.982469787603312
0.2575 0.979959226364307
0.2625 0.977191586840609
0.2675 0.97415540401367
0.2725 0.970840333891215
0.2775 0.967237254215386
0.2825 0.963338338325584
0.2875 0.959137101767173
0.2925 0.954628422367118
0.2975 0.949808535550069
0.3025 0.944675007613097
0.3075 0.93922669048504
0.3125 0.93346366213093
0.3175 0.927387157171918
0.3225 0.920999492407739
0.3275 0.914303991672197
0.3325 0.907304913751212
0.3375 0.90000738591857
0.3425 0.892417344055854
0.3475 0.884541478519005
0.3525 0.876387183275967
0.3575 0.867962504945375
0.3625 0.859276088942573
0.3675 0.850337122723488
0.3725 0.841155281587692
0.3775 0.831740690478113
0.3825 0.822103924376787
0.3875 0.812256077407175
0.3925 0.802208932296694
0.3975 0.791975252482376
0.4025 0.781569195218389
0.4075 0.771006805995516
0.4125 0.760306509342156
0.4175 0.749489472124117
0.4225 0.738579698452171
0.4275 0.727603729465644
0.4325 0.71658986174401
0.4375 0.705566850669331
0.4425 0.694562135134672
0.4475 0.68359978054784
0.4525 0.672698740784772
0.4575 0.661872775215992
0.4625 0.651134050812332
0.4675 0.640501842836554
0.4725 0.630014074440158
0.4775 0.619732881873294
0.4825 0.609731506896937
0.4875 0.600058305346286
0.4925 0.590696145471375
0.4975 0.581551383185598
0.5025 0.572476507211133
0.5075 0.563293923280107
0.5125 0.553800233185305
0.5175 0.543781521373741
0.5225 0.533063507520285
0.5275 0.521566254711062
0.5325 0.509313959487317
0.5375 0.496400819506339
0.5425 0.482943369363496
0.5475 0.469048299785303
0.5525 0.454804358117032
0.5575 0.440289846649872
0.5625 0.425583489588254
0.5675 0.410771644257896
0.5725 0.395950812291248
0.5775 0.381227048515738
0.5825 0.366713602378983
0.5875 0.352527020018551
0.5925 0.338781437380338
0.5975 0.325581202776111
0.6025 0.313012826599636
0.6075 0.301137930363013
0.6125 0.28998893052361
0.6175 0.279568585432994
0.6225 0.269853519370675
0.6275 0.260800826173275
0.6325 0.25235619826391
0.6375 0.244461883155795
0.6425 0.237063085501019
0.6475 0.230112020114516
0.6525 0.223569462915682
0.6575 0.217404175242648
0.6625 0.211590905285989
0.6675 0.206107782796998
0.6725 0.200933851635925
0.6775 0.19604728570937
0.6825 0.191424572450542
0.6875 0.187040686463892
0.6925 0.182870064680001
0.6975 0.178888064771805
0.7025 0.175072549621457
0.7075 0.171405280409886
0.7125 0.167872893146608
0.7175 0.164467346526728
0.7225 0.161185834559035
0.7275 0.158030236348808
0.7325 0.155006220081722
0.7375 0.152122131241113
0.7425 0.149387785783819
0.7475 0.146813269140009
0.7525 0.14440782144993
0.7575 0.142178873756478
0.7625 0.140131288852554
0.7675 0.138266849981738
0.7725 0.136584025400125
0.7775 0.135078014041113
0.7825 0.133741048757712
0.7875 0.132562904652122
0.7925 0.131531538365762
0.7975 0.13063377571739
0.8025 0.129855971357034
0.8075 0.129184582173803
0.8125 0.128606620139977
0.8175 0.128109973780338
0.8225 0.12768360592339
0.8275 0.127317646828479
0.8325 0.12700340652163
0.8375 0.126733329925724
0.8425 0.126500915227678
0.8475 0.126300611662415
0.8525 0.126127708680483
0.8575 0.125978224881435
0.8625 0.125848802292361
0.8675 0.125736609473238
0.8725 0.125639255381861
0.8775 0.125554714782859
0.8825 0.125481265132884
0.8875 0.125417434255349
0.8925 0.125361957699734
0.8975 0.125313744440518
0.9025 0.125271849487081
0.9075 0.125235452020281
0.9125 0.125203837809891
0.9175 0.125176384863169
0.9225 0.12515255147411
0.9275 0.12513186605675
0.9325 0.125113918334066
0.9375 0.125098351605385
0.9425 0.125084855926945
0.9475 0.125073162116278
0.9525 0.125063036539648
0.9575 0.125054276673823
0.9625 0.125046707458714
0.9675 0.125040178477778
0.9725 0.125034561985197
0.9775 0.125029751574714
0.9825 0.125025660209438
0.9875 0.125022212226635
0.9925 0.125019310312244
0.9975 0.12501671885813
};

\nextgroupplot[
minor xtick={},
minor ytick={},
tick align=outside,
tick pos=left,
x grid style={white!69.0196078431373!black},
xlabel={\(x\)},
xmin=-0.04725, xmax=1.04725,
xtick style={color=black},
xtick={0,1},
y grid style={white!69.0196078431373!black},
ylabel={\(\rho u\)},
ymin=-0.0198045004974811, ymax=0.415904508463629,
ytick style={color=black},
ytick={4.54455296600542e-07,0.396099553510851}
]
\addplot [thick, black]
table {%
0.0025 4.54455296600542e-07
0.0075 5.9985472300685e-07
0.0125 7.88339375095883e-07
0.0175 1.03258648852749e-06
0.0225 1.34970551615619e-06
0.0275 1.76218161750585e-06
0.0325 2.29933729248087e-06
0.0375 2.99930283451103e-06
0.0425 3.91157900605425e-06
0.0475 5.10032656535721e-06
0.0525 6.64856007512515e-06
0.0575 8.6634674442852e-06
0.0625 1.12831253004778e-05
0.0675 1.468493488727e-05
0.0725 1.90961638762875e-05
0.0775 2.48070457205261e-05
0.0825 3.21869583875013e-05
0.0875 4.17042759332731e-05
0.0925 5.39505554185066e-05
0.0975 6.96697826532025e-05
0.1025 8.97934458085532e-05
0.1075 0.000115482226727408
0.1125 0.000148175084444427
0.1175 0.000189646440684595
0.1225 0.000242072048313612
0.1275 0.000308103915386144
0.1325 0.000390954354824754
0.1375 0.000494488820231745
0.1425 0.000623326663673109
0.1475 0.000782948310420776
0.1525 0.000979806597068939
0.1575 0.00122143918446957
0.1625 0.00151657807137136
0.1675 0.00187525135065397
0.1725 0.00230887153522797
0.1775 0.00283030411612859
0.1825 0.0034539095903976
0.1875 0.00419555210119987
0.1925 0.00507256814837626
0.1975 0.00610368961503879
0.2025 0.00730891664370317
0.2075 0.00870933767097807
0.2125 0.0103268961317184
0.2175 0.0121841058611737
0.2225 0.0143037199016964
0.2275 0.0167083600710161
0.2325 0.019420117068359
0.2375 0.0224601328850072
0.2425 0.0258481786771399
0.2475 0.0296022419282834
0.2525 0.0337381366152874
0.2575 0.0382691492023285
0.2625 0.0432057316947753
0.2675 0.0485552508169737
0.2725 0.0543217998019709
0.2775 0.0605060764824809
0.2825 0.0671053285351274
0.2875 0.0741133640202822
0.2925 0.0815206229144609
0.2975 0.0893143032548566
0.3025 0.0974785338790117
0.3075 0.105994584596552
0.3125 0.114841104009903
0.3175 0.123994375135092
0.3225 0.133428579482419
0.3275 0.14311606134221
0.3325 0.15302758564526
0.3375 0.163132584823495
0.3425 0.173399392375569
0.3475 0.183795463015846
0.3525 0.194287580911644
0.3575 0.20484205808796
0.3625 0.215424924140179
0.3675 0.226002105694562
0.3725 0.236539589774906
0.3775 0.247003560212418
0.3825 0.257360492132517
0.3875 0.267577188789006
0.3925 0.27762075033611
0.3975 0.287458477611135
0.4025 0.29705773556341
0.4075 0.306385826775564
0.4125 0.315409947230083
0.4175 0.324097302489628
0.4225 0.332415442078596
0.4275 0.340332819416506
0.4325 0.347819512527544
0.4375 0.354847966262696
0.4425 0.361393560352427
0.4475 0.367434777728012
0.4525 0.372952742570127
0.4575 0.377929944441451
0.4625 0.382348184917035
0.4675 0.386186386143682
0.4725 0.389419923077351
0.4775 0.39202389014746
0.4825 0.393981526145045
0.4875 0.395294659731944
0.4925 0.395987920090632
0.4975 0.396099553510851
0.5025 0.395658924537269
0.5075 0.39466232310167
0.5125 0.39306310382351
0.5175 0.390783104604446
0.5225 0.387738696177698
0.5275 0.383869350009942
0.5325 0.37915535793569
0.5375 0.373617298320719
0.5425 0.367302888216957
0.5475 0.360271898211032
0.5525 0.352586214718681
0.5575 0.344306225940405
0.5625 0.33549144247301
0.5675 0.326203073298632
0.5725 0.316507208304638
0.5775 0.306477786978538
0.5825 0.296198456559623
0.5875 0.285762280876216
0.5925 0.275268549641514
0.5975 0.264816735902625
0.6025 0.254498649957448
0.6075 0.244390585790918
0.6125 0.234547407439115
0.6175 0.225000010230103
0.6225 0.215756632540806
0.6275 0.206807460279548
0.6325 0.198131206765991
0.6375 0.189702054908685
0.6425 0.181495514906113
0.6475 0.173492239017786
0.6525 0.165679453456796
0.6575 0.158050243972552
0.6625 0.150601349509476
0.6675 0.143330322405805
0.6725 0.136232902283778
0.6775 0.129301263579978
0.6825 0.122523500768609
0.6875 0.115884389975564
0.6925 0.109367184684881
0.6975 0.102956019193393
0.7025 0.0966384282148705
0.7075 0.090407535240329
0.7125 0.0842635834372528
0.7175 0.07821463818847
0.7225 0.0722764396450493
0.7275 0.0664714987964292
0.7325 0.0608275996641119
0.7375 0.0553758965491429
0.7425 0.0501487921833653
0.7475 0.0451777666759582
0.7525 0.0404913111795752
0.7575 0.0361131088901486
0.7625 0.0320605950931976
0.7675 0.0283440078275513
0.7725 0.0249660026502126
0.7775 0.0219218473852424
0.7825 0.0192001439110504
0.7875 0.0167839596465161
0.7925 0.0146522079473016
0.7975 0.0127811044103527
0.8025 0.0111455452888791
0.8075 0.00972029585256481
0.8125 0.00848092702643994
0.8175 0.00740448530431059
0.8225 0.00646991562426667
0.8275 0.00565827701947578
0.8325 0.00495279813990158
0.8375 0.00433881807780411
0.8425 0.00380365146419993
0.8475 0.00333640876386272
0.8525 0.00292779503725072
0.8575 0.00256990400714959
0.8625 0.00225601919397449
0.8675 0.00198042995414445
0.8725 0.00173826718373137
0.8775 0.0015253610091346
0.8825 0.00133812084964607
0.8875 0.00117343675199983
0.8925 0.00102859984928664
0.8975 0.000901239170925305
0.9025 0.000789271789969033
0.9075 0.000690863374632628
0.9125 0.000604396527075725
0.9175 0.000528444750057864
0.9225 0.00046175039200118
0.9275 0.000403205410113744
0.9325 0.000351834208625823
0.9375 0.000306778127598819
0.9425 0.000267281370897916
0.9475 0.000232678279182708
0.9525 0.000202381893958196
0.9575 0.000175873743991994
0.9625 0.000152694736778497
0.9675 0.000132436973771126
0.9725 0.000114736250561974
0.9775 9.92650028115161e-05
0.9825 8.5725682258205e-05
0.9875 7.38455366845868e-05
0.9925 6.3377104025941e-05
0.9975 5.41183395192608e-05
};

\nextgroupplot[
minor xtick={},
minor ytick={},
tick align=outside,
tick pos=left,
x grid style={white!69.0196078431373!black},
xlabel={\(x\)},
xmin=-0.04725, xmax=1.04725,
xtick style={color=black},
xtick={0,1},
y grid style={white!69.0196078431373!black},
ylabel={\(E\)},
ymin=-0.00848958699034735, ymax=0.524213116033832,
ytick style={color=black},
ytick={0.0157241722380244,0.49999935680546}
]
\addplot [thick, black]
table {%
0.0025 0.49999935680546
0.0075 0.499999184516617
0.0125 0.499998958044162
0.0175 0.499998661709873
0.0225 0.499998275147288
0.0275 0.499997771866551
0.0325 0.499997117461528
0.0375 0.499996267334379
0.0425 0.499995163800781
0.0475 0.499993732414377
0.0525 0.499991877316449
0.0575 0.4999894753775
0.0625 0.499986368851917
0.0675 0.499982356215721
0.0725 0.49997718080134
0.0775 0.499970516783883
0.0825 0.499961952012844
0.0875 0.499950967124919
0.0925 0.499936910322577
0.0975 0.499918967165467
0.1025 0.499896124705878
0.1075 0.499867129315394
0.1125 0.499830437609235
0.1175 0.499784159990916
0.1225 0.499725996526799
0.1275 0.499653165131781
0.1325 0.499562322416104
0.1375 0.499449478017839
0.1425 0.499309903829107
0.1475 0.499138040211023
0.1525 0.498927402066165
0.1575 0.498670488467809
0.1625 0.498358700386789
0.1675 0.49798227184801
0.1725 0.497530220513646
0.1775 0.496990324141894
0.1825 0.496349129517518
0.1875 0.495592000206122
0.1925 0.494703208776299
0.1975 0.493666077917781
0.2025 0.492463173153719
0.2075 0.491076547643305
0.2125 0.489488036992126
0.2175 0.487679599178207
0.2225 0.48563369185145
0.2275 0.483333676590288
0.2325 0.480764237426086
0.2375 0.477911799279573
0.2425 0.474764931059029
0.2475 0.471314718149885
0.2525 0.46755508990689
0.2575 0.463483089491295
0.2625 0.459099075851118
0.2675 0.454406850638818
0.2725 0.449413706176894
0.2775 0.444130393984271
0.2825 0.438571016642011
0.2875 0.432752848712706
0.2925 0.426696094883513
0.2975 0.420423595376289
0.3025 0.41396048990446
0.3075 0.407333852041546
0.3125 0.400572305819732
0.3175 0.393705635744175
0.3225 0.386764400258094
0.3275 0.37977955711779
0.3325 0.37278210725787
0.3375 0.365802761703839
0.3425 0.358871634118662
0.3475 0.352017959878469
0.3525 0.345269841390657
0.3575 0.338654018885252
0.3625 0.332195666215149
0.3675 0.325918212210125
0.3725 0.3198431895337
0.3775 0.313990114233691
0.3825 0.308376399499813
0.3875 0.303017305735776
0.3925 0.297925925329712
0.3975 0.29311319447661
0.4025 0.288587917090328
0.4075 0.284356779508706
0.4125 0.280424332635162
0.4175 0.276792923722078
0.4225 0.273462574861494
0.4275 0.270430827458254
0.4325 0.267692594949205
0.4375 0.265240079944094
0.4425 0.263062807795775
0.4475 0.26114780346012
0.4525 0.259479898018716
0.4575 0.258042105589045
0.4625 0.256815966903368
0.4675 0.255781714613968
0.4725 0.254918097933172
0.4775 0.254201786016598
0.4825 0.253606558386228
0.4875 0.253102948359567
0.4925 0.252659208169941
0.4975 0.252243792059708
0.5025 0.251827275102802
0.5075 0.251381012497179
0.5125 0.25087302853746
0.5175 0.250265602223314
0.5225 0.249517693945798
0.5275 0.248590688140476
0.5325 0.247453389289997
0.5375 0.246084019377779
0.5425 0.244469264456546
0.5475 0.242601857795075
0.5525 0.240478180516557
0.5575 0.23809666448092
0.5625 0.235457212698634
0.5675 0.232561605229087
0.5725 0.229414721585933
0.5775 0.226026206372895
0.5825 0.222411969184941
0.5875 0.218594806862423
0.5925 0.214603596555817
0.5975 0.210470926205335
0.6025 0.206229558715456
0.6075 0.201908554426631
0.6125 0.197530029228202
0.6175 0.193107347902876
0.6225 0.188645124924567
0.6275 0.184140895173082
0.6325 0.179587897877868
0.6375 0.174978201200335
0.6425 0.17030540968722
0.6475 0.16556639879422
0.6525 0.160761827791422
0.6575 0.155895505486493
0.6625 0.150972946975817
0.6675 0.145999613634893
0.6725 0.140979352315915
0.6775 0.13591345321238
0.6825 0.130800564862492
0.6875 0.125637490651162
0.6925 0.120420696893854
0.6975 0.115148229071734
0.7025 0.109821679976769
0.7075 0.104447878554068
0.7125 0.099040050356734
0.7175 0.0936183101554151
0.7225 0.0882094555681862
0.7275 0.0828461170190982
0.7325 0.0775653751442742
0.7375 0.0724069844914259
0.7425 0.0674113519275984
0.7475 0.0626174210389261
0.7525 0.0580606172134823
0.7575 0.0537710115820283
0.7625 0.0497718578191883
0.7675 0.0460786334781048
0.7725 0.042698670459725
0.7775 0.0396313901640515
0.7825 0.0368690808793959
0.7875 0.0343980868649908
0.7925 0.032200237264967
0.7975 0.0302543366145871
0.8025 0.028537564241122
0.8075 0.0270266757298571
0.8125 0.0256989512318784
0.8175 0.024532880548202
0.8225 0.0235086068445918
0.8275 0.0226081682913537
0.8325 0.0218155824541386
0.8375 0.0211168160181255
0.8425 0.0204996763355393
0.8475 0.0199536541582351
0.8525 0.0194697402960386
0.8575 0.0190402333814328
0.8625 0.018658551401896
0.8675 0.0183190559463869
0.8725 0.018016894973501
0.8775 0.0177478672179364
0.8825 0.017508309084488
0.8875 0.017295003070728
0.8925 0.017105105455802
0.8975 0.0169360902085604
0.9025 0.0167857057706824
0.9075 0.016651941479692
0.9125 0.016533000800502
0.9175 0.016427279107946
0.9225 0.0163333443900927
0.9275 0.0162499198292296
0.9325 0.0161758677015307
0.9375 0.0161101743871457
0.9425 0.0160519364975342
0.9475 0.0160003482247449
0.9525 0.0159546900284552
0.9575 0.0159143187351992
0.9625 0.0158786590620089
0.9675 0.0158471965183368
0.9725 0.0158194716003297
0.9775 0.0157950751714789
0.9825 0.0157736449012305
0.9875 0.0157548625331191
0.9925 0.0157384513721274
0.9975 0.0157241722380244
};

\nextgroupplot[
minor xtick={},
minor ytick={},
tick align=outside,
tick pos=left,
x grid style={white!69.0196078431373!black},
xlabel={\(x\)},
xmin=-0.04725, xmax=1.04725,
xtick style={color=black},
xtick={0,1},
y grid style={white!69.0196078431373!black},
ylabel={\(T\)},
ymin=0.0713768746069346, ymax=0.345807068364389,
ytick style={color=black},
ytick={0.0838509743231826,0.333332968648141}
]
\addplot [thick, black]
table {%
0.0025 0.333332968648141
0.0075 0.333332873437815
0.0125 0.333332747096767
0.0175 0.333332581634156
0.0225 0.333332366073626
0.0275 0.333332085970293
0.0325 0.333331722532108
0.0375 0.333331251428846
0.0425 0.333330641255421
0.0475 0.333329851577072
0.0525 0.333328830460161
0.0575 0.333327511370302
0.0625 0.333325809296231
0.0675 0.333323615932409
0.0725 0.333320793726332
0.0775 0.333317168568421
0.0825 0.333312520874583
0.0875 0.333306574785996
0.0925 0.33329898518979
0.0975 0.333289322251567
0.1025 0.333277053150165
0.1075 0.333261520721755
0.1125 0.333241918759793
0.1175 0.333217263785764
0.1225 0.333186363209004
0.1275 0.333147779937887
0.1325 0.333099793693454
0.1375 0.333040359512057
0.1425 0.332967064204139
0.1475 0.332877081855701
0.1525 0.332767129805093
0.1575 0.332633426882131
0.1625 0.332471656033655
0.1675 0.332276933747992
0.1725 0.332043788894212
0.1775 0.331766153672492
0.1825 0.331437369293483
0.1875 0.331050208738131
0.1925 0.330596918477999
0.1975 0.330069280359514
0.2025 0.329458693994092
0.2075 0.328756278992114
0.2125 0.327952995295236
0.2175 0.327039778777948
0.2225 0.326007688293798
0.2275 0.324848059522083
0.2325 0.323552660403677
0.2375 0.322113842696095
0.2425 0.320524684256032
0.2475 0.318779117068815
0.2525 0.316872036753694
0.2575 0.314799390220386
0.2625 0.312558239255801
0.2675 0.310146798991954
0.2725 0.307564451359632
0.2775 0.304811734691119
0.2825 0.301890311540001
0.2875 0.29880291749846
0.2925 0.295553294295276
0.2975 0.292146110752802
0.3025 0.288586875284329
0.3075 0.28488184354977
0.3125 0.281037924685536
0.3175 0.277062589210871
0.3225 0.272963781309929
0.3275 0.268749837713882
0.3325 0.264429414874759
0.3375 0.260011425550048
0.3425 0.255504985332093
0.3475 0.250919369106298
0.3525 0.246263976980273
0.3575 0.241548308993782
0.3625 0.236781948019468
0.3675 0.23197455082149
0.3725 0.227135848342482
0.3775 0.222275657929332
0.3825 0.217403912186402
0.3875 0.212530710959784
0.3925 0.207666403714042
0.3975 0.202821708007094
0.4025 0.198007864585037
0.4075 0.193236820148162
0.4125 0.188521416154465
0.4175 0.18387554992158
0.4225 0.179314269260493
0.4275 0.174853771048142
0.4325 0.170511300913763
0.4375 0.166304990547212
0.4425 0.162253707088678
0.4475 0.158377011034739
0.4525 0.154695324211521
0.4575 0.151230411328739
0.4625 0.148006259820289
0.4675 0.145050261852742
0.4725 0.142393992544879
0.4775 0.140071832223203
0.4825 0.138115263781109
0.4875 0.13654307671066
0.4925 0.135353301297746
0.4975 0.134525708993156
0.5025 0.134037850287624
0.5075 0.133884259465143
0.5125 0.13408391745273
0.5175 0.134673491682662
0.5225 0.135695382736727
0.5275 0.137186819518489
0.5325 0.139172255676875
0.5375 0.141662443939813
0.5425 0.144658839823409
0.5475 0.148159596428082
0.5525 0.152164012292966
0.5575 0.156674216123742
0.5625 0.161694388013152
0.5675 0.167228320227044
0.5725 0.17327595679211
0.5775 0.179829311437647
0.5825 0.186868180711133
0.5875 0.194356321890063
0.5925 0.202239025239255
0.5975 0.210443023577552
0.6025 0.21887930135562
0.6075 0.227448626539
0.6125 0.236048767064081
0.6175 0.244581726589205
0.6225 0.252959249467098
0.6275 0.261105380502306
0.6325 0.268955807775381
0.6375 0.276454676036148
0.6425 0.283550178203326
0.6475 0.290190353996531
0.6525 0.296320207445648
0.6575 0.30188069038244
0.6625 0.306809500156532
0.6675 0.311043171854109
0.6725 0.314519702622753
0.6775 0.317180953246601
0.6825 0.318974292846681
0.6875 0.319853300489035
0.6925 0.319777701494179
0.6975 0.318712993042877
0.7025 0.316630337363522
0.7075 0.313507256319531
0.7125 0.309329477688907
0.7175 0.304094014342264
0.7225 0.29781326005002
0.7275 0.290519608049905
0.7325 0.282269878143869
0.7375 0.273148706461814
0.7425 0.263270040211542
0.7475 0.252776017515606
0.7525 0.241832816998447
0.7575 0.230623517796971
0.7625 0.219338551634448
0.7675 0.208164835184748
0.7725 0.19727499863253
0.7775 0.186818159199727
0.7825 0.176913395230085
0.7875 0.167646535340211
0.7925 0.159070249127469
0.7975 0.151206888222841
0.8025 0.144053202739134
0.8075 0.137585981757603
0.8125 0.131767789468376
0.8175 0.126552201165467
0.8225 0.121888197093304
0.8275 0.117723587140582
0.8325 0.114007490096743
0.8375 0.110691978487166
0.8425 0.107733038256199
0.8475 0.105090999032429
0.8525 0.102730579626558
0.8575 0.100620674229507
0.8625 0.0987339828018665
0.8675 0.0970465670001713
0.8725 0.095537391944865
0.8775 0.094187894962654
0.8825 0.0929816057462935
0.8875 0.0919038286828597
0.8925 0.0909413877396712
0.8975 0.0900824273252991
0.9025 0.0893162587176752
0.9075 0.0886332404638173
0.9125 0.0880246819261635
0.9175 0.0874827611425375
0.9225 0.0870004506957645
0.9275 0.0865714478023507
0.9325 0.086190106949525
0.9375 0.0858513749387026
0.9425 0.0855507290903285
0.9475 0.0852841197049094
0.9525 0.085047917800779
0.9575 0.0848388688276685
0.9625 0.0846540526432564
0.9675 0.0844908496644469
0.9725 0.0843469128646704
0.9775 0.0842201453001416
0.9825 0.0841086834242394
0.9875 0.0840108886452995
0.9925 0.0839253566438232
0.9975 0.0838509743231826
};

\nextgroupplot[
minor xtick={},
minor ytick={},
tick align=outside,
tick pos=left,
x grid style={white!69.0196078431373!black},
xlabel={\(x\)},
xmin=-0.04725, xmax=1.04725,
xtick style={color=black},
xtick={0,1},
y grid style={white!69.0196078431373!black},
ylabel={\(u\)},
ymin=-0.0406678313182572, ymax=0.854034455701849,
ytick style={color=black},
ytick={4.54455384007727e-07,0.813366169928208}
]
\addplot [thick, black]
table {%
0.0025 4.54455384007727e-07
0.0075 5.99854873738933e-07
0.0125 7.88339631466431e-07
0.0175 1.03258692374712e-06
0.0225 1.34970625570035e-06
0.0275 1.76218287602788e-06
0.0325 2.29933943703582e-06
0.0375 2.99930649256839e-06
0.0425 3.91158524969883e-06
0.0475 5.10033722480856e-06
0.0525 6.64857827114536e-06
0.0575 8.66349848987616e-06
0.0625 1.12831782233005e-05
0.0675 1.46850249910634e-05
0.0725 1.90963170333847e-05
0.0775 2.48073055339063e-05
0.0825 3.21873980830344e-05
0.0875 4.17050180021604e-05
0.0925 5.39518038750862e-05
0.0975 6.96718756830361e-05
0.1025 8.9796941125211e-05
0.1075 0.00011548803897638
0.1125 0.000148184704717312
0.1175 0.000189662284395183
0.1225 0.000242098001811395
0.1275 0.000308146187143335
0.1325 0.000391022788402061
0.1375 0.000494598899492913
0.1425 0.000623502543487414
0.1475 0.000783227347704643
0.1525 0.000980246048463475
0.1575 0.00122212599050273
0.1625 0.00151764297538003
0.1675 0.00187688901598883
0.1725 0.00231136883552711
0.1775 0.00283407938462145
0.1825 0.00345956632084196
0.1875 0.00420395137910852
0.1925 0.00508492492597372
0.1975 0.00612169877944503
0.2025 0.00733491560039121
0.2075 0.00874651279325981
0.2125 0.0103795408205767
0.2175 0.0122579380234693
0.2225 0.0144062663006578
0.2275 0.0168494141614543
0.2325 0.019612275561005
0.2375 0.0227194143907933
0.2425 0.0261947254108691
0.2475 0.0300611026975421
0.2525 0.0343401263234669
0.2575 0.0390517770257736
0.2625 0.0442141871426311
0.2675 0.0498434342374108
0.2725 0.0559533817309014
0.2775 0.0625555686764391
0.2825 0.0696591486764306
0.2875 0.0772708759610395
0.2925 0.0853951349073816
0.2975 0.0940340078151976
0.3025 0.103187374592782
0.3075 0.112853037153165
0.3125 0.123026860786141
0.3175 0.133702924583531
0.3225 0.144873673202143
0.3275 0.156530062917544
0.3325 0.168661696113355
0.3375 0.181256940082773
0.3425 0.194303028208197
0.3475 0.207786143984538
0.3525 0.221691490495548
0.3575 0.236003349131829
0.3625 0.250705130646986
0.3675 0.265779418133264
0.3725 0.28120799447212
0.3775 0.296971836342925
0.3825 0.313051044401247
0.3875 0.329424668194723
0.3925 0.346070380369977
0.3975 0.362963964732638
0.4025 0.380078612848098
0.4075 0.397384075462164
0.4125 0.414845780424774
0.4175 0.432424089388619
0.4225 0.4500738955799
0.4275 0.467744742961184
0.4325 0.485381570541668
0.4375 0.502926074157356
0.4425 0.520318546133175
0.4475 0.537499847401279
0.4525 0.554412725873456
0.4575 0.571000891097415
0.4625 0.587203486655369
0.4675 0.60294344265022
0.4725 0.618113053146308
0.4775 0.632569130368672
0.4825 0.64615576149264
0.4875 0.658760417462807
0.4925 0.670374985729142
0.4975 0.681108436783545
0.5025 0.691135652823126
0.5075 0.70063302086328
0.5125 0.709756118307717
0.5175 0.718639911884503
0.5225 0.727378052910409
0.5275 0.735993455371453
0.5325 0.744443286646518
0.5375 0.752652460751926
0.5425 0.760550639096775
0.5475 0.768091257075101
0.5525 0.775248100476541
0.5575 0.781999013968189
0.5625 0.788309346299109
0.5675 0.794122666105529
0.5725 0.799359916634862
0.5775 0.803924559319104
0.5825 0.807710580240528
0.5875 0.810611001849387
0.5925 0.812525478875279
0.5975 0.813366169928208
0.6025 0.813061409406614
0.6075 0.811556968251434
0.6125 0.808815036545057
0.6175 0.804811491540168
0.6225 0.799532402037861
0.6275 0.792970878635736
0.6325 0.785125184675626
0.6375 0.775998501115153
0.6425 0.765600070219843
0.6475 0.753946877401046
0.6525 0.741064773766894
0.6575 0.726988080133006
0.6625 0.71175719630256
0.6675 0.6954144111432
0.6725 0.677998760162225
0.6775 0.659541207684257
0.6825 0.640061509346014
0.6875 0.61956781792466
0.6925 0.598059528639963
0.6975 0.575533193478988
0.7025 0.551990751398907
0.7075 0.52744895037151
0.7125 0.501948717614958
0.7175 0.475563325123379
0.7225 0.448404413717744
0.7275 0.420625193837665
0.7325 0.392420379208283
0.7375 0.364022618519407
0.7425 0.335695397854924
0.7475 0.307722639381284
0.7525 0.280395554569145
0.7575 0.253997713837589
0.7625 0.228789696831602
0.7675 0.20499496322723
0.7725 0.1827885990113
0.7775 0.162290270114352
0.7825 0.143562085757483
0.7875 0.126611284586449
0.7925 0.111396917646906
0.7975 0.0978392023055586
0.8025 0.0858300559643484
0.8075 0.0752434670531137
0.8125 0.0659447158879473
0.8175 0.0577978832234139
0.8225 0.0506714670021821
0.8275 0.0444422054634623
0.8325 0.0389973645239041
0.8375 0.034235809004206
0.8425 0.030068173478066
0.8475 0.0264164101816109
0.8525 0.0232129408191157
0.8575 0.0203995889731599
0.8625 0.0179264256224982
0.8675 0.0157506231672803
0.8725 0.0138353827269048
0.8775 0.0121489743477387
0.8825 0.0106639094547621
0.8875 0.00935624906510782
0.8925 0.00820503977570558
0.8975 0.00719186211336211
0.9025 0.00630047207892808
0.9075 0.00551651599836721
0.9125 0.0048273003259967
0.9175 0.00422160098836141
0.9225 0.00368950042617949
0.9275 0.00322224404398222
0.9325 0.0028121108611305
0.9375 0.0024522955231779
0.9425 0.00213680040575031
0.9475 0.0018603373837018
0.9525 0.00161823908612707
0.9575 0.00140637928321893
0.9625 0.00122110161780079
0.9675 0.00105915534817205
0.9725 0.000917636281843072
0.9775 0.00079393105689887
0.9825 0.000685664703666437
0.9875 0.000590659334604659
0.9925 0.000506938519078793
0.9975 0.000432888816900358
};
\end{groupplot}

\end{tikzpicture}

		\caption{Macroscopic quantities of \(\rare\). Density \(\rho\), momentum \(\rho u\), total energy \(E\), teperature \(T\) and velocity \(u\) over time \(t\) and space \(x\) in the top row and specifically at \(t_i=0.055s\) and \(t_i=0.12\) in the middle and bottom row respectively.}
	\label{Fig: Macro_rare}
\end{figure}
\clearpage
\begin{figure}[htp!]
	\centering
	% This file was created by tikzplotlib v0.9.8.
\begin{tikzpicture}

\begin{groupplot}[
group style={group size=3 by 3,
	horizontal sep=3cm,
	%vertical sep=1cm
},
x tick label style={/pgf/number format/fixed},
y tick label style={/pgf/number format/fixed},
width=0.2\textwidth,
height=0.2\textwidth,
y label style={yshift=-1cm},
x label style={yshift=.5cm}
]
\nextgroupplot[
colorbar horizontal,
colorbar style={xtick={-0.32601398229599,0.615161418914795},
	minor xtick={},
	at={(0.5,1.03)},
	xlabel={\(z_0\)},
	x label style={yshift=-1cm},
	anchor=south,
	xticklabel pos=upper,
	height=0.1*\pgfkeysvalueof{/pgfplots/parent axis width},
	tick pos=right
},
colormap/blackwhite,
minor xtick={},
minor ytick={},
point meta max=0.615161418914795,
point meta min=-0.32601398229599,
tick align=outside,
tick pos=left,
x grid style={white!69.0196078431373!black},
xlabel={\(x\)},
xmin=0, xmax=1,
xtick style={color=black},
xtick={0,1},
y grid style={white!69.0196078431373!black},
ylabel={\(t\)},
ymin=0, ymax=0.12,
ytick style={color=black},
ytick={0,0.12}
]
\addplot graphics [includegraphics cmd=\pgfimage,xmin=0, xmax=1, ymin=0, ymax=0.12] {Figures/Chapter_5/code_hydro-000.png};

\nextgroupplot[
colorbar horizontal,
colorbar style={xtick={-0.292637974023819,0.207332760095596},
	minor xtick={},
	at={(0.5,1.03)},
	xlabel={\(z_1\)},
	x label style={yshift=-1cm},
	anchor=south,
	xticklabel pos=upper,
	height=0.1*\pgfkeysvalueof{/pgfplots/parent axis width},
	tick pos=right
},
colormap/blackwhite,
minor xtick={},
minor ytick={},
point meta max=0.207332760095596,
point meta min=-0.292637974023819,
tick align=outside,
tick pos=left,
x grid style={white!69.0196078431373!black},
xlabel={\(x\)},
xmin=0, xmax=1,
xtick style={color=black},
xtick={0,1},
y grid style={white!69.0196078431373!black},
ylabel={\(t\)},
ymin=0, ymax=0.12,
ytick style={color=black},
ytick={0,0.12}
]
\addplot graphics [includegraphics cmd=\pgfimage,xmin=0, xmax=1, ymin=0, ymax=0.12] {Figures/Chapter_5/code_hydro-001.png};

\nextgroupplot[
colorbar horizontal,
colorbar style={xtick={0.0675623863935471,0.549470663070679},
	minor xtick={},
	at={(0.5,1.03)},
	xlabel={\(z_2\)},
	x label style={yshift=-1cm},
	anchor=south,
	xticklabel pos=upper,
	height=0.1*\pgfkeysvalueof{/pgfplots/parent axis width},
	tick pos=right
},
colormap/blackwhite,
minor xtick={},
minor ytick={},
point meta max=0.549470663070679,
point meta min=0.0675623863935471,
tick align=outside,
tick pos=left,
x grid style={white!69.0196078431373!black},
xlabel={\(x\)},
xmin=0, xmax=1,
xtick style={color=black},
xtick={0,1},
y grid style={white!69.0196078431373!black},
ylabel={\(t\)},
ymin=0, ymax=0.12,
ytick style={color=black},
ytick={0,0.12}
]
\addplot graphics [includegraphics cmd=\pgfimage,xmin=0, xmax=1, ymin=0, ymax=0.12] {Figures/Chapter_5/code_hydro-002.png};

\nextgroupplot[
minor xtick={},
minor ytick={},
tick align=outside,
tick pos=left,
x grid style={white!69.0196078431373!black},
xlabel={\(x\)},
xmin=-0.04725, xmax=1.04725,
xtick style={color=black},
xtick={0,1},
y grid style={white!69.0196078431373!black},
ylabel={\(z_0\)},
ymin=-0.372934377193451, ymax=0.662213599681854,
ytick style={color=black},
ytick={-0.325882196426392,0.615161418914795}
]
\addplot [thick, color0]
table {%
0.0025 0.615161418914795
0.0075 0.615161418914795
0.0125 0.615161418914795
0.0175 0.615161418914795
0.0225 0.615161418914795
0.0275 0.615161418914795
0.0325 0.615161418914795
0.0375 0.615161418914795
0.0425 0.615161418914795
0.0475 0.615161418914795
0.0525 0.615161418914795
0.0575 0.615161418914795
0.0625 0.615161418914795
0.0675 0.615161418914795
0.0725 0.615161418914795
0.0775 0.615161418914795
0.0825 0.615161418914795
0.0875 0.615161418914795
0.0925 0.615161418914795
0.0975 0.615161418914795
0.1025 0.615161418914795
0.1075 0.615161418914795
0.1125 0.615161418914795
0.1175 0.615161418914795
0.1225 0.615161418914795
0.1275 0.615161418914795
0.1325 0.615161418914795
0.1375 0.615161418914795
0.1425 0.615161418914795
0.1475 0.615161418914795
0.1525 0.615161418914795
0.1575 0.615161418914795
0.1625 0.615161418914795
0.1675 0.615161418914795
0.1725 0.615161418914795
0.1775 0.615161418914795
0.1825 0.615161418914795
0.1875 0.615161418914795
0.1925 0.615161418914795
0.1975 0.615161418914795
0.2025 0.615161418914795
0.2075 0.615161418914795
0.2125 0.615161418914795
0.2175 0.615161418914795
0.2225 0.615161418914795
0.2275 0.615161418914795
0.2325 0.615161418914795
0.2375 0.615161418914795
0.2425 0.615161418914795
0.2475 0.615161418914795
0.2525 0.615161418914795
0.2575 0.615161418914795
0.2625 0.615161418914795
0.2675 0.615161418914795
0.2725 0.615161418914795
0.2775 0.615161418914795
0.2825 0.615161240100861
0.2875 0.615161001682281
0.2925 0.615160346031189
0.2975 0.615159034729004
0.3025 0.615156471729279
0.3075 0.615150928497314
0.3125 0.615140318870544
0.3175 0.615120053291321
0.3225 0.615082085132599
0.3275 0.615013420581818
0.3325 0.614892721176147
0.3375 0.614686369895935
0.3425 0.614344000816345
0.3475 0.613793134689331
0.3525 0.612933456897736
0.3575 0.611633539199829
0.3625 0.609727799892426
0.3675 0.607018351554871
0.3725 0.603279709815979
0.3775 0.598267912864685
0.3825 0.591731071472168
0.3875 0.583422005176544
0.3925 0.573108792304993
0.3975 0.560586035251617
0.4025 0.545679092407227
0.4075 0.528249382972717
0.4125 0.508195102214813
0.4175 0.485452115535736
0.4225 0.459993302822113
0.4275 0.431840360164642
0.4325 0.401059210300446
0.4375 0.367750465869904
0.4425 0.332085013389587
0.4475 0.294279545545578
0.4525 0.25461220741272
0.4575 0.213450595736504
0.4625 0.171281591057777
0.4675 0.128751426935196
0.4725 0.0867214947938919
0.4775 0.0463449768722057
0.4825 0.00912030786275864
0.4875 -0.0231140479445457
0.4925 -0.048407569527626
0.4975 -0.0653465837240219
0.5025 -0.0740704089403152
0.5075 -0.0765876024961472
0.5125 -0.075501874089241
0.5175 -0.0725427567958832
0.5225 -0.0691191256046295
0.5275 -0.0671962946653366
0.5325 -0.0692226141691208
0.5375 -0.0769496411085129
0.5425 -0.0902383625507355
0.5475 -0.106931164860725
0.5525 -0.123929485678673
0.5575 -0.138686046004295
0.5625 -0.149915426969528
0.5675 -0.15757554769516
0.5725 -0.162444606423378
0.5775 -0.165698036551476
0.5825 -0.168691352009773
0.5875 -0.172974243760109
0.5925 -0.180471166968346
0.5975 -0.193696156144142
0.6025 -0.215641692280769
0.6075 -0.248294934630394
0.6125 -0.287748694419861
0.6175 -0.318276405334473
0.6225 -0.325882196426392
0.6275 -0.322060644626617
0.6325 -0.319426596164703
0.6375 -0.318584084510803
0.6425 -0.318360924720764
0.6475 -0.318304717540741
0.6525 -0.318290770053864
0.6575 -0.318287253379822
0.6625 -0.31828647851944
0.6675 -0.318286299705505
0.6725 -0.318286299705505
0.6775 -0.318286299705505
0.6825 -0.318286299705505
0.6875 -0.318286299705505
0.6925 -0.318286299705505
0.6975 -0.318286299705505
0.7025 -0.318286299705505
0.7075 -0.318286299705505
0.7125 -0.318286299705505
0.7175 -0.318286299705505
0.7225 -0.318286299705505
0.7275 -0.318286299705505
0.7325 -0.318286299705505
0.7375 -0.318286299705505
0.7425 -0.318286299705505
0.7475 -0.318286299705505
0.7525 -0.318286299705505
0.7575 -0.318286299705505
0.7625 -0.318286299705505
0.7675 -0.318286299705505
0.7725 -0.318286299705505
0.7775 -0.318286299705505
0.7825 -0.318286299705505
0.7875 -0.318286299705505
0.7925 -0.318286299705505
0.7975 -0.318286299705505
0.8025 -0.318286299705505
0.8075 -0.318286299705505
0.8125 -0.318286299705505
0.8175 -0.318286299705505
0.8225 -0.318286299705505
0.8275 -0.318286299705505
0.8325 -0.318286299705505
0.8375 -0.318286299705505
0.8425 -0.318286299705505
0.8475 -0.318286299705505
0.8525 -0.318286299705505
0.8575 -0.318286299705505
0.8625 -0.318286299705505
0.8675 -0.318286299705505
0.8725 -0.318286299705505
0.8775 -0.318286299705505
0.8825 -0.318286299705505
0.8875 -0.318286299705505
0.8925 -0.318286299705505
0.8975 -0.318286299705505
0.9025 -0.318286299705505
0.9075 -0.318286299705505
0.9125 -0.318286299705505
0.9175 -0.318286299705505
0.9225 -0.318286299705505
0.9275 -0.318286299705505
0.9325 -0.318286299705505
0.9375 -0.318286299705505
0.9425 -0.318286299705505
0.9475 -0.318286299705505
0.9525 -0.318286299705505
0.9575 -0.318286299705505
0.9625 -0.318286299705505
0.9675 -0.318286299705505
0.9725 -0.318286299705505
0.9775 -0.318286299705505
0.9825 -0.318286299705505
0.9875 -0.318286299705505
0.9925 -0.318286299705505
0.9975 -0.318286299705505
};

\nextgroupplot[
minor xtick={},
minor ytick={},
tick align=outside,
tick pos=left,
x grid style={white!69.0196078431373!black},
xlabel={\(x\)},
xmin=-0.04725, xmax=1.04725,
xtick style={color=black},
xtick={0,1},
y grid style={white!69.0196078431373!black},
ylabel={\(z_1\)},
ymin=-0.302255213260651, ymax=0.231598854064941,
ytick style={color=black},
ytick={-0.277989119291306,0.207332760095596}
]
\addplot [thick, color0]
table {%
0.0025 0.207332730293274
0.0075 0.207332730293274
0.0125 0.207332730293274
0.0175 0.207332730293274
0.0225 0.207332730293274
0.0275 0.207332730293274
0.0325 0.207332730293274
0.0375 0.207332730293274
0.0425 0.207332730293274
0.0475 0.207332730293274
0.0525 0.207332730293274
0.0575 0.207332730293274
0.0625 0.207332730293274
0.0675 0.207332730293274
0.0725 0.207332730293274
0.0775 0.207332730293274
0.0825 0.207332730293274
0.0875 0.207332730293274
0.0925 0.207332730293274
0.0975 0.207332730293274
0.1025 0.207332730293274
0.1075 0.207332730293274
0.1125 0.207332730293274
0.1175 0.207332730293274
0.1225 0.207332730293274
0.1275 0.207332730293274
0.1325 0.207332730293274
0.1375 0.207332730293274
0.1425 0.207332730293274
0.1475 0.207332730293274
0.1525 0.207332730293274
0.1575 0.207332730293274
0.1625 0.207332730293274
0.1675 0.207332730293274
0.1725 0.207332730293274
0.1775 0.207332730293274
0.1825 0.207332730293274
0.1875 0.207332730293274
0.1925 0.207332730293274
0.1975 0.207332730293274
0.2025 0.207332730293274
0.2075 0.207332730293274
0.2125 0.207332730293274
0.2175 0.207332730293274
0.2225 0.207332730293274
0.2275 0.207332730293274
0.2325 0.207332730293274
0.2375 0.207332730293274
0.2425 0.207332730293274
0.2475 0.207332730293274
0.2525 0.207332730293274
0.2575 0.207332730293274
0.2625 0.207332730293274
0.2675 0.207332745194435
0.2725 0.207332760095596
0.2775 0.207332745194435
0.2825 0.207332700490952
0.2875 0.207332640886307
0.2925 0.20733243227005
0.2975 0.207331985235214
0.3025 0.207331150770187
0.3075 0.207329288125038
0.3125 0.207325726747513
0.3175 0.207318842411041
0.3225 0.207306087017059
0.3275 0.207282811403275
0.3325 0.207241624593735
0.3375 0.207170933485031
0.3425 0.207053065299988
0.3475 0.206862241029739
0.3525 0.206562548875809
0.3575 0.206105768680573
0.3625 0.205430030822754
0.3675 0.204458743333817
0.3725 0.203101187944412
0.3775 0.201253235340118
0.3825 0.198799073696136
0.3875 0.195613205432892
0.3925 0.191561907529831
0.3975 0.186505109071732
0.4025 0.18029673397541
0.4075 0.172785222530365
0.4125 0.163813948631287
0.4175 0.15322108566761
0.4225 0.140840202569962
0.4275 0.126512557268143
0.4325 0.110091239213943
0.4375 0.0914223790168762
0.4425 0.0703631341457367
0.4475 0.0467926785349846
0.4525 0.0206294730305672
0.4575 -0.00814769417047501
0.4625 -0.0394640788435936
0.4675 -0.0730969682335854
0.4725 -0.108594633638859
0.4775 -0.145145237445831
0.4825 -0.181416779756546
0.4875 -0.215371757745743
0.4925 -0.244218558073044
0.4975 -0.264957398176193
0.5025 -0.275907129049301
0.5075 -0.277989119291306
0.5125 -0.273333519697189
0.5175 -0.262819558382034
0.5225 -0.246675163507462
0.5275 -0.226068079471588
0.5325 -0.203677892684937
0.5375 -0.182779580354691
0.5425 -0.165828436613083
0.5475 -0.153671145439148
0.5525 -0.145738273859024
0.5575 -0.140901118516922
0.5625 -0.138079196214676
0.5675 -0.13648396730423
0.5725 -0.13561150431633
0.5775 -0.13515642285347
0.5825 -0.13493013381958
0.5875 -0.134800344705582
0.5925 -0.134632766246796
0.5975 -0.134185522794724
0.6025 -0.132856249809265
0.6075 -0.129107683897018
0.6125 -0.119601495563984
0.6175 -0.100368358194828
0.6225 -0.0763163939118385
0.6275 -0.0612251833081245
0.6325 -0.0559118166565895
0.6375 -0.0544787719845772
0.6425 -0.0541174486279488
0.6475 -0.0540275201201439
0.6525 -0.0540052726864815
0.6575 -0.0539997890591621
0.6625 -0.0539984926581383
0.6675 -0.0539981201291084
0.6725 -0.053998090326786
0.6775 -0.0539980754256248
0.6825 -0.0539980605244637
0.6875 -0.0539980605244637
0.6925 -0.0539980605244637
0.6975 -0.0539980605244637
0.7025 -0.0539980605244637
0.7075 -0.0539980605244637
0.7125 -0.0539980605244637
0.7175 -0.0539980605244637
0.7225 -0.0539980605244637
0.7275 -0.0539980605244637
0.7325 -0.0539980605244637
0.7375 -0.0539980605244637
0.7425 -0.0539980605244637
0.7475 -0.0539980605244637
0.7525 -0.0539980605244637
0.7575 -0.0539980605244637
0.7625 -0.0539980605244637
0.7675 -0.0539980605244637
0.7725 -0.0539980605244637
0.7775 -0.0539980605244637
0.7825 -0.0539980605244637
0.7875 -0.0539980605244637
0.7925 -0.0539980605244637
0.7975 -0.0539980605244637
0.8025 -0.0539980605244637
0.8075 -0.0539980605244637
0.8125 -0.0539980605244637
0.8175 -0.0539980605244637
0.8225 -0.0539980605244637
0.8275 -0.0539980605244637
0.8325 -0.0539980605244637
0.8375 -0.0539980605244637
0.8425 -0.0539980605244637
0.8475 -0.0539980605244637
0.8525 -0.0539980605244637
0.8575 -0.0539980605244637
0.8625 -0.0539980605244637
0.8675 -0.0539980605244637
0.8725 -0.0539980605244637
0.8775 -0.0539980605244637
0.8825 -0.0539980605244637
0.8875 -0.0539980605244637
0.8925 -0.0539980605244637
0.8975 -0.0539980605244637
0.9025 -0.0539980605244637
0.9075 -0.0539980605244637
0.9125 -0.0539980605244637
0.9175 -0.0539980605244637
0.9225 -0.0539980605244637
0.9275 -0.0539980605244637
0.9325 -0.0539980605244637
0.9375 -0.0539980605244637
0.9425 -0.0539980605244637
0.9475 -0.0539980605244637
0.9525 -0.0539980605244637
0.9575 -0.0539980605244637
0.9625 -0.0539980605244637
0.9675 -0.0539980605244637
0.9725 -0.0539980605244637
0.9775 -0.0539980605244637
0.9825 -0.0539980605244637
0.9875 -0.0539980605244637
0.9925 -0.0539980605244637
0.9975 -0.0539980605244637
};

\nextgroupplot[
minor xtick={},
minor ytick={},
tick align=outside,
tick pos=left,
x grid style={white!69.0196078431373!black},
xlabel={\(x\)},
xmin=-0.04725, xmax=1.04725,
xtick style={color=black},
xtick={0,1},
y grid style={white!69.0196078431373!black},
ylabel={\(z_2\)},
ymin=0.0442335151135922, ymax=0.561886460334063,
ytick style={color=black},
ytick={0.0677631944417953,0.538356781005859}
]
\addplot [thick, color0]
table {%
0.0025 0.182571709156036
0.0075 0.182571709156036
0.0125 0.182571709156036
0.0175 0.182571709156036
0.0225 0.182571709156036
0.0275 0.182571709156036
0.0325 0.182571709156036
0.0375 0.182571709156036
0.0425 0.182571709156036
0.0475 0.182571709156036
0.0525 0.182571709156036
0.0575 0.182571709156036
0.0625 0.182571709156036
0.0675 0.182571709156036
0.0725 0.182571709156036
0.0775 0.182571709156036
0.0825 0.182571709156036
0.0875 0.182571709156036
0.0925 0.182571709156036
0.0975 0.182571709156036
0.1025 0.182571709156036
0.1075 0.182571709156036
0.1125 0.182571709156036
0.1175 0.182571709156036
0.1225 0.182571709156036
0.1275 0.182571709156036
0.1325 0.182571709156036
0.1375 0.182571709156036
0.1425 0.182571709156036
0.1475 0.182571709156036
0.1525 0.182571709156036
0.1575 0.182571709156036
0.1625 0.182571709156036
0.1675 0.182571709156036
0.1725 0.182571709156036
0.1775 0.182571709156036
0.1825 0.182571709156036
0.1875 0.182571709156036
0.1925 0.182571709156036
0.1975 0.182571709156036
0.2025 0.182571709156036
0.2075 0.182571709156036
0.2125 0.182571709156036
0.2175 0.182571709156036
0.2225 0.182571709156036
0.2275 0.182571709156036
0.2325 0.182571709156036
0.2375 0.182571709156036
0.2425 0.182571709156036
0.2475 0.182571709156036
0.2525 0.182571709156036
0.2575 0.182571709156036
0.2625 0.182571709156036
0.2675 0.182571724057198
0.2725 0.182571768760681
0.2775 0.182571858167648
0.2825 0.18257200717926
0.2875 0.182572290301323
0.2925 0.182573050260544
0.2975 0.182574599981308
0.3025 0.18257774412632
0.3075 0.182584077119827
0.3125 0.182596579194069
0.3175 0.182620331645012
0.3225 0.18266461789608
0.3275 0.182744577527046
0.3325 0.182884618639946
0.3375 0.183123454451561
0.3425 0.183518260717392
0.3475 0.184151038527489
0.3525 0.185133695602417
0.3575 0.186611637473106
0.3625 0.188764408230782
0.3675 0.191801354289055
0.3725 0.195952340960503
0.3775 0.201454371213913
0.3825 0.208534613251686
0.3875 0.217392429709435
0.3925 0.228183776140213
0.3975 0.241006731987
0.4025 0.255895793437958
0.4075 0.272818237543106
0.4125 0.291673004627228
0.4175 0.312292963266373
0.4225 0.334445118904114
0.4275 0.35782864689827
0.4325 0.382078796625137
0.4375 0.406762063503265
0.4425 0.431350976228714
0.4475 0.45524188876152
0.4525 0.477745145559311
0.4575 0.498067408800125
0.4625 0.515308260917664
0.4675 0.528468430042267
0.4725 0.5364910364151
0.4775 0.538356781005859
0.4825 0.533344149589539
0.4875 0.521510243415833
0.4925 0.504368603229523
0.4975 0.485188990831375
0.5025 0.467339038848877
0.5075 0.450638324022293
0.5125 0.429951399564743
0.5175 0.399710059165955
0.5225 0.357254028320312
0.5275 0.304339647293091
0.5325 0.246896177530289
0.5375 0.192673653364182
0.5425 0.147892862558365
0.5475 0.115144431591034
0.5525 0.0935965478420258
0.5575 0.080579474568367
0.5625 0.0732275545597076
0.5675 0.0693797767162323
0.5725 0.0677631944417953
0.5775 0.0678825229406357
0.5825 0.069920688867569
0.5875 0.07475945353508
0.5925 0.0841235220432281
0.5975 0.100742638111115
0.6025 0.128146350383759
0.6075 0.168915659189224
0.6125 0.219017952680588
0.6175 0.260191112756729
0.6225 0.272938042879105
0.6275 0.268014460802078
0.6325 0.263409346342087
0.6375 0.261706620454788
0.6425 0.261225938796997
0.6475 0.261101990938187
0.6525 0.261071085929871
0.6575 0.261063456535339
0.6625 0.261061608791351
0.6675 0.261061161756516
0.6725 0.261061102151871
0.6775 0.261061102151871
0.6825 0.261061072349548
0.6875 0.261061072349548
0.6925 0.261061072349548
0.6975 0.261061072349548
0.7025 0.261061072349548
0.7075 0.261061072349548
0.7125 0.261061072349548
0.7175 0.261061072349548
0.7225 0.261061072349548
0.7275 0.261061072349548
0.7325 0.261061072349548
0.7375 0.261061072349548
0.7425 0.261061072349548
0.7475 0.261061072349548
0.7525 0.261061072349548
0.7575 0.261061072349548
0.7625 0.261061072349548
0.7675 0.261061072349548
0.7725 0.261061072349548
0.7775 0.261061072349548
0.7825 0.261061072349548
0.7875 0.261061072349548
0.7925 0.261061072349548
0.7975 0.261061072349548
0.8025 0.261061072349548
0.8075 0.261061072349548
0.8125 0.261061072349548
0.8175 0.261061072349548
0.8225 0.261061072349548
0.8275 0.261061072349548
0.8325 0.261061072349548
0.8375 0.261061072349548
0.8425 0.261061072349548
0.8475 0.261061072349548
0.8525 0.261061072349548
0.8575 0.261061072349548
0.8625 0.261061072349548
0.8675 0.261061072349548
0.8725 0.261061072349548
0.8775 0.261061072349548
0.8825 0.261061072349548
0.8875 0.261061072349548
0.8925 0.261061072349548
0.8975 0.261061072349548
0.9025 0.261061072349548
0.9075 0.261061072349548
0.9125 0.261061072349548
0.9175 0.261061072349548
0.9225 0.261061072349548
0.9275 0.261061072349548
0.9325 0.261061072349548
0.9375 0.261061072349548
0.9425 0.261061072349548
0.9475 0.261061072349548
0.9525 0.261061072349548
0.9575 0.261061072349548
0.9625 0.261061072349548
0.9675 0.261061072349548
0.9725 0.261061072349548
0.9775 0.261061072349548
0.9825 0.261061072349548
0.9875 0.261061072349548
0.9925 0.261061072349548
0.9975 0.261061072349548
};

\nextgroupplot[
minor xtick={},
minor ytick={},
tick align=outside,
tick pos=left,
x grid style={white!69.0196078431373!black},
xlabel={\(x\)},
xmin=-0.04725, xmax=1.04725,
xtick style={color=black},
xtick={0,1},
y grid style={white!69.0196078431373!black},
ylabel={\(z_0\)},
ymin=-0.372073522210121, ymax=0.66217260658741,
ytick style={color=black},
ytick={-0.325062334537506,0.615161418914795}
]
\addplot [thick, color0]
table {%
0.0025 0.615161418914795
0.0075 0.615161418914795
0.0125 0.615161418914795
0.0175 0.615161418914795
0.0225 0.615161418914795
0.0275 0.615161418914795
0.0325 0.615161418914795
0.0375 0.615161418914795
0.0425 0.615161418914795
0.0475 0.615161418914795
0.0525 0.615161418914795
0.0575 0.615161418914795
0.0625 0.615161418914795
0.0675 0.615161418914795
0.0725 0.615161418914795
0.0775 0.615161418914795
0.0825 0.615161418914795
0.0875 0.615161418914795
0.0925 0.615161418914795
0.0975 0.615161418914795
0.1025 0.615161418914795
0.1075 0.615161418914795
0.1125 0.615161418914795
0.1175 0.615161299705505
0.1225 0.615161180496216
0.1275 0.615160942077637
0.1325 0.615160465240479
0.1375 0.615159749984741
0.1425 0.615158438682556
0.1475 0.615156352519989
0.1525 0.615152835845947
0.1575 0.615146934986115
0.1625 0.615137815475464
0.1675 0.615123510360718
0.1725 0.615100979804993
0.1775 0.615066468715668
0.1825 0.615014672279358
0.1875 0.614937782287598
0.1925 0.614825367927551
0.1975 0.614663898944855
0.2025 0.614435911178589
0.2075 0.614118695259094
0.2125 0.61368465423584
0.2175 0.613101124763489
0.2225 0.612328708171844
0.2275 0.611323297023773
0.2325 0.61003577709198
0.2375 0.60841304063797
0.2425 0.606399178504944
0.2475 0.603937685489655
0.2525 0.600972294807434
0.2575 0.59744918346405
0.2625 0.59331864118576
0.2675 0.588535726070404
0.2725 0.58306211233139
0.2775 0.576865673065186
0.2825 0.569921910762787
0.2875 0.562213182449341
0.2925 0.553728401660919
0.2975 0.54446268081665
0.3025 0.534416735172272
0.3075 0.523595869541168
0.3125 0.512009859085083
0.3175 0.499671757221222
0.3225 0.486597836017609
0.3275 0.472806692123413
0.3325 0.458319544792175
0.3375 0.443162560462952
0.3425 0.427363395690918
0.3475 0.410948991775513
0.3525 0.393947720527649
0.3575 0.37638908624649
0.3625 0.358309268951416
0.3675 0.339745938777924
0.3725 0.320734739303589
0.3775 0.301312983036041
0.3825 0.281521111726761
0.3875 0.261402130126953
0.3925 0.241002902388573
0.3975 0.2203738540411
0.4025 0.199570849537849
0.4075 0.178654804825783
0.4125 0.157693535089493
0.4175 0.136762514710426
0.4225 0.1159458309412
0.4275 0.0953384265303612
0.4325 0.0750483721494675
0.4375 0.0552000440657139
0.4425 0.0359301790595055
0.4475 0.0173937156796455
0.4525 -0.00023215264081955
0.4575 -0.0167528688907623
0.4625 -0.0319570004940033
0.4675 -0.0456220656633377
0.4725 -0.0575291961431503
0.4775 -0.0674925297498703
0.4825 -0.0754015147686005
0.4875 -0.0812662392854691
0.4925 -0.0852521508932114
0.4975 -0.0876731723546982
0.5025 -0.0889311879873276
0.5075 -0.0894230455160141
0.5125 -0.089460089802742
0.5175 -0.0892373323440552
0.5225 -0.088846281170845
0.5275 -0.0883028954267502
0.5325 -0.087570384144783
0.5375 -0.0865739434957504
0.5425 -0.0852183848619461
0.5475 -0.0834217071533203
0.5525 -0.0811733156442642
0.5575 -0.078609511256218
0.5625 -0.0760809630155563
0.5675 -0.0741699188947678
0.5725 -0.0736264139413834
0.5775 -0.075203150510788
0.5825 -0.0794200748205185
0.5875 -0.0863757580518723
0.5925 -0.0956881940364838
0.5975 -0.106561422348022
0.6025 -0.117985263466835
0.6075 -0.129008218646049
0.6125 -0.138900235295296
0.6175 -0.147224217653275
0.6225 -0.153829634189606
0.6275 -0.158788338303566
0.6325 -0.162311002612114
0.6375 -0.164670005440712
0.6425 -0.166141852736473
0.6475 -0.166973575949669
0.6525 -0.16736675798893
0.6575 -0.167475089430809
0.6625 -0.167410597205162
0.6675 -0.167253091931343
0.6725 -0.167062059044838
0.6775 -0.166888669133186
0.6825 -0.16679073870182
0.6875 -0.166850045323372
0.6925 -0.167198851704597
0.6975 -0.168060079216957
0.7025 -0.169811502099037
0.7075 -0.173086121678352
0.7125 -0.178918823599815
0.7175 -0.188923522830009
0.7225 -0.205374881625175
0.7275 -0.230723485350609
0.7325 -0.265269815921783
0.7375 -0.301685094833374
0.7425 -0.323430180549622
0.7475 -0.325062334537506
0.7525 -0.321111142635345
0.7575 -0.31913423538208
0.7625 -0.318520963191986
0.7675 -0.31835275888443
0.7725 -0.318307876586914
0.7775 -0.318296134471893
0.7825 -0.318293035030365
0.7875 -0.318292200565338
0.7925 -0.318291962146759
0.7975 -0.318291902542114
0.8025 -0.318291902542114
0.8075 -0.318291902542114
0.8125 -0.318291902542114
0.8175 -0.318291902542114
0.8225 -0.318291902542114
0.8275 -0.318291902542114
0.8325 -0.318291902542114
0.8375 -0.318291902542114
0.8425 -0.318291902542114
0.8475 -0.318291902542114
0.8525 -0.318291902542114
0.8575 -0.318291902542114
0.8625 -0.318291902542114
0.8675 -0.318291902542114
0.8725 -0.318291902542114
0.8775 -0.318291902542114
0.8825 -0.318291902542114
0.8875 -0.318291902542114
0.8925 -0.318291902542114
0.8975 -0.318291902542114
0.9025 -0.318291902542114
0.9075 -0.318291902542114
0.9125 -0.318291902542114
0.9175 -0.318291902542114
0.9225 -0.318291902542114
0.9275 -0.318291902542114
0.9325 -0.318291902542114
0.9375 -0.318291902542114
0.9425 -0.318291902542114
0.9475 -0.318291902542114
0.9525 -0.318291902542114
0.9575 -0.318291902542114
0.9625 -0.318291902542114
0.9675 -0.318291902542114
0.9725 -0.318291902542114
0.9775 -0.318291902542114
0.9825 -0.318291902542114
0.9875 -0.318291902542114
0.9925 -0.318291902542114
0.9975 -0.318291902542114
};

\nextgroupplot[
minor xtick={},
minor ytick={},
tick align=outside,
tick pos=left,
x grid style={white!69.0196078431373!black},
xlabel={\(x\)},
xmin=-0.04725, xmax=1.04725,
xtick style={color=black},
xtick={0,1},
y grid style={white!69.0196078431373!black},
ylabel={\(z_1\)},
ymin=-0.31763651072979, ymax=0.232331296801567,
ytick style={color=black},
ytick={-0.292637974023819,0.207332760095596}
]
\addplot [thick, color0]
table {%
0.0025 0.207332730293274
0.0075 0.207332730293274
0.0125 0.207332730293274
0.0175 0.207332730293274
0.0225 0.207332730293274
0.0275 0.207332730293274
0.0325 0.207332730293274
0.0375 0.207332730293274
0.0425 0.207332730293274
0.0475 0.207332730293274
0.0525 0.207332730293274
0.0575 0.207332730293274
0.0625 0.207332730293274
0.0675 0.207332730293274
0.0725 0.207332730293274
0.0775 0.207332730293274
0.0825 0.207332730293274
0.0875 0.207332730293274
0.0925 0.207332730293274
0.0975 0.207332730293274
0.1025 0.207332760095596
0.1075 0.207332760095596
0.1125 0.207332745194435
0.1175 0.207332730293274
0.1225 0.207332670688629
0.1275 0.207332581281662
0.1325 0.20733243227005
0.1375 0.207332223653793
0.1425 0.20733180642128
0.1475 0.20733106136322
0.1525 0.207329839468002
0.1575 0.207327783107758
0.1625 0.207324713468552
0.1675 0.207319706678391
0.1725 0.207311898469925
0.1775 0.207299947738647
0.1825 0.207281798124313
0.1875 0.207254990935326
0.1925 0.207215666770935
0.1975 0.207158982753754
0.2025 0.207078605890274
0.2075 0.206966638565063
0.2125 0.206812739372253
0.2175 0.206605046987534
0.2225 0.206329047679901
0.2275 0.205968037247658
0.2325 0.205503284931183
0.2375 0.204913854598999
0.2425 0.204177513718605
0.2475 0.203270584344864
0.2525 0.202168613672256
0.2575 0.200846880674362
0.2625 0.199280753731728
0.2675 0.197445869445801
0.2725 0.195318907499313
0.2775 0.192877233028412
0.2825 0.190099328756332
0.2875 0.186964839696884
0.2925 0.183454528450966
0.2975 0.179549962282181
0.3025 0.175233602523804
0.3075 0.170488834381104
0.3125 0.165299594402313
0.3175 0.159650251269341
0.3225 0.153525829315186
0.3275 0.146911665797234
0.3325 0.139793366193771
0.3375 0.132160320878029
0.3425 0.124004364013672
0.3475 0.115314096212387
0.3525 0.106078758835793
0.3575 0.0962881296873093
0.3625 0.0859334096312523
0.3675 0.0750062465667725
0.3725 0.0635004192590714
0.3775 0.0514114797115326
0.3825 0.0387374609708786
0.3875 0.0254790186882019
0.3925 0.0116402134299278
0.3975 -0.00277106463909149
0.4025 -0.0177418366074562
0.4075 -0.0332534834742546
0.4125 -0.0492801144719124
0.4175 -0.065787561237812
0.4225 -0.0827309265732765
0.4275 -0.100052870810032
0.4325 -0.117678664624691
0.4375 -0.135512173175812
0.4425 -0.153437256813049
0.4475 -0.171308010816574
0.4525 -0.188941359519958
0.4575 -0.20611160993576
0.4625 -0.222544282674789
0.4675 -0.237913459539413
0.4725 -0.251850873231888
0.4775 -0.263977140188217
0.4825 -0.273962110280991
0.4875 -0.281611174345016
0.4925 -0.286950200796127
0.4975 -0.290254831314087
0.5025 -0.29198369383812
0.5075 -0.292637974023819
0.5125 -0.292625963687897
0.5175 -0.2921983897686
0.5225 -0.291449189186096
0.5275 -0.290340214967728
0.5325 -0.288715332746506
0.5375 -0.286302387714386
0.5425 -0.282720059156418
0.5475 -0.277515798807144
0.5525 -0.270250976085663
0.5575 -0.260628014802933
0.5625 -0.248629808425903
0.5675 -0.234615355730057
0.5725 -0.219327390193939
0.5775 -0.203774005174637
0.5825 -0.189000219106674
0.5875 -0.17587211728096
0.5925 -0.164911717176437
0.5975 -0.156272351741791
0.6025 -0.149770706892014
0.6075 -0.145060747861862
0.6125 -0.14174810051918
0.6175 -0.139467626810074
0.6225 -0.13792285323143
0.6275 -0.136891007423401
0.6325 -0.136212468147278
0.6375 -0.135775327682495
0.6425 -0.135502099990845
0.6475 -0.135338872671127
0.6525 -0.135248273611069
0.6575 -0.13520410656929
0.6625 -0.135188430547714
0.6675 -0.135189205408096
0.6725 -0.135198622941971
0.6775 -0.135211497545242
0.6825 -0.135224729776382
0.6875 -0.135235875844955
0.6925 -0.135242432355881
0.6975 -0.135240912437439
0.7025 -0.135223776102066
0.7075 -0.135173499584198
0.7125 -0.135044723749161
0.7175 -0.134716898202896
0.7225 -0.133867174386978
0.7275 -0.131657004356384
0.7325 -0.126099586486816
0.7375 -0.113539911806583
0.7425 -0.0919987037777901
0.7475 -0.0703980699181557
0.7525 -0.0591211691498756
0.7575 -0.0554092302918434
0.7625 -0.0543886199593544
0.7675 -0.0541185364127159
0.7725 -0.0540473535656929
0.7775 -0.0540286377072334
0.7825 -0.0540236905217171
0.7875 -0.054022379219532
0.7925 -0.0540220364928246
0.7975 -0.0540219619870186
0.8025 -0.0540219321846962
0.8075 -0.0540219321846962
0.8125 -0.0540219321846962
0.8175 -0.0540219321846962
0.8225 -0.0540219321846962
0.8275 -0.0540219321846962
0.8325 -0.0540219321846962
0.8375 -0.0540219321846962
0.8425 -0.0540219321846962
0.8475 -0.0540219321846962
0.8525 -0.0540219321846962
0.8575 -0.0540219321846962
0.8625 -0.0540219321846962
0.8675 -0.0540219321846962
0.8725 -0.0540219321846962
0.8775 -0.0540219321846962
0.8825 -0.0540219321846962
0.8875 -0.0540219321846962
0.8925 -0.0540219321846962
0.8975 -0.0540219321846962
0.9025 -0.0540219321846962
0.9075 -0.0540219321846962
0.9125 -0.0540219321846962
0.9175 -0.0540219321846962
0.9225 -0.0540219321846962
0.9275 -0.0540219321846962
0.9325 -0.0540219321846962
0.9375 -0.0540219321846962
0.9425 -0.0540219321846962
0.9475 -0.0540219321846962
0.9525 -0.0540219321846962
0.9575 -0.0540219321846962
0.9625 -0.0540219321846962
0.9675 -0.0540219321846962
0.9725 -0.0540219321846962
0.9775 -0.0540219321846962
0.9825 -0.0540219321846962
0.9875 -0.0540219321846962
0.9925 -0.0540219321846962
0.9975 -0.0540219321846962
};

\nextgroupplot[
minor xtick={},
minor ytick={},
tick align=outside,
tick pos=left,
x grid style={white!69.0196078431373!black},
xlabel={\(x\)},
xmin=-0.04725, xmax=1.04725,
xtick style={color=black},
xtick={0,1},
y grid style={white!69.0196078431373!black},
ylabel={\(z_2\)},
ymin=0.0458814971148968, ymax=0.573451099544764,
ytick style={color=black},
ytick={0.0698619335889816,0.549470663070679}
]
\addplot [thick, color0]
table {%
0.0025 0.182571709156036
0.0075 0.182571709156036
0.0125 0.182571709156036
0.0175 0.182571709156036
0.0225 0.182571709156036
0.0275 0.182571709156036
0.0325 0.182571709156036
0.0375 0.182571709156036
0.0425 0.182571709156036
0.0475 0.182571709156036
0.0525 0.182571709156036
0.0575 0.182571709156036
0.0625 0.182571709156036
0.0675 0.182571709156036
0.0725 0.182571709156036
0.0775 0.182571709156036
0.0825 0.182571709156036
0.0875 0.182571709156036
0.0925 0.182571709156036
0.0975 0.182571724057198
0.1025 0.182571738958359
0.1075 0.182571768760681
0.1125 0.182571843266487
0.1175 0.182571947574615
0.1225 0.182572156190872
0.1275 0.18257237970829
0.1325 0.182572871446609
0.1375 0.182573780417442
0.1425 0.182575285434723
0.1475 0.182577788829803
0.1525 0.182581812143326
0.1575 0.182588458061218
0.1625 0.18259896337986
0.1675 0.182615622878075
0.1725 0.182641386985779
0.1775 0.182680815458298
0.1825 0.18274013698101
0.1875 0.182827979326248
0.1925 0.182955920696259
0.1975 0.183139503002167
0.2025 0.183398276567459
0.2075 0.18375737965107
0.2125 0.184247195720673
0.2175 0.184904366731644
0.2225 0.185771226882935
0.2275 0.186895474791527
0.2325 0.188329592347145
0.2375 0.190129041671753
0.2425 0.192350938916206
0.2475 0.195051550865173
0.2525 0.198284268379211
0.2575 0.202097713947296
0.2625 0.206533402204514
0.2675 0.211624875664711
0.2725 0.217395722866058
0.2775 0.22385972738266
0.2825 0.231020033359528
0.2875 0.238869786262512
0.2925 0.247393280267715
0.2975 0.256566107273102
0.3025 0.266356915235519
0.3075 0.276727914810181
0.3125 0.287635952234268
0.3175 0.299033612012863
0.3225 0.310869514942169
0.3275 0.32308965921402
0.3325 0.335636556148529
0.3375 0.348450154066086
0.3425 0.36146941781044
0.3475 0.374631017446518
0.3525 0.387870281934738
0.3575 0.40112030506134
0.3625 0.414308577775955
0.3675 0.42735943198204
0.3725 0.440199106931686
0.3775 0.452750593423843
0.3825 0.464934259653091
0.3875 0.476667314767838
0.3925 0.487863630056381
0.3975 0.49843367934227
0.4025 0.508285045623779
0.4075 0.517321765422821
0.4125 0.525445520877838
0.4175 0.532555937767029
0.4225 0.538552343845367
0.4275 0.543334901332855
0.4325 0.546806573867798
0.4375 0.548875451087952
0.4425 0.549470663070679
0.4475 0.54854553937912
0.4525 0.546089291572571
0.4575 0.542148768901825
0.4625 0.536848485469818
0.4675 0.530415236949921
0.4725 0.523197591304779
0.4775 0.515665411949158
0.4825 0.508369505405426
0.4875 0.501842558383942
0.4925 0.496460407972336
0.4975 0.492326587438583
0.5025 0.489261537790298
0.5075 0.486909836530685
0.5125 0.484882086515427
0.5175 0.482833236455917
0.5225 0.4804467856884
0.5275 0.477357000112534
0.5325 0.473055273294449
0.5375 0.466818243265152
0.5425 0.457693666219711
0.5475 0.444577872753143
0.5525 0.426406115293503
0.5575 0.402436554431915
0.5625 0.372558861970901
0.5675 0.337515860795975
0.5725 0.298940807580948
0.5775 0.259134024381638
0.5825 0.220628619194031
0.5875 0.18567419052124
0.5925 0.155788779258728
0.5975 0.131626158952713
0.6025 0.113057434558868
0.6075 0.0993962436914444
0.6125 0.0897032469511032
0.6175 0.0830200016498566
0.6225 0.0785082131624222
0.6275 0.0755113959312439
0.6325 0.073546901345253
0.6375 0.0722715556621552
0.6425 0.0714484304189682
0.6475 0.0709158331155777
0.6525 0.0705645531415939
0.6575 0.0703225582838058
0.6625 0.0701448768377304
0.6675 0.0700081288814545
0.6725 0.0699082463979721
0.6775 0.0698619335889816
0.6825 0.0699136853218079
0.6875 0.0701505690813065
0.6925 0.0707285553216934
0.6975 0.071918711066246
0.7025 0.0741836577653885
0.7075 0.0782992243766785
0.7125 0.0855330377817154
0.7175 0.0978615432977676
0.7225 0.118079617619514
0.7275 0.14926415681839
0.7325 0.19212532043457
0.7375 0.238622009754181
0.7425 0.268727511167526
0.7475 0.272511333227158
0.7525 0.266507714986801
0.7575 0.26281675696373
0.7625 0.261538416147232
0.7675 0.261171072721481
0.7725 0.261071711778641
0.7775 0.261045306921005
0.7825 0.261038392782211
0.7875 0.261036574840546
0.7925 0.26103612780571
0.7975 0.26103600859642
0.8025 0.26103600859642
0.8075 0.261035948991776
0.8125 0.261035948991776
0.8175 0.261035948991776
0.8225 0.261035948991776
0.8275 0.261035948991776
0.8325 0.261035948991776
0.8375 0.261035948991776
0.8425 0.261035948991776
0.8475 0.261035948991776
0.8525 0.261035948991776
0.8575 0.261035948991776
0.8625 0.261035948991776
0.8675 0.261035948991776
0.8725 0.261035948991776
0.8775 0.261035948991776
0.8825 0.261035948991776
0.8875 0.261035948991776
0.8925 0.261035948991776
0.8975 0.261035948991776
0.9025 0.261035948991776
0.9075 0.261035948991776
0.9125 0.261035948991776
0.9175 0.261035948991776
0.9225 0.261035948991776
0.9275 0.261035948991776
0.9325 0.261035948991776
0.9375 0.261035948991776
0.9425 0.261035948991776
0.9475 0.261035948991776
0.9525 0.261035948991776
0.9575 0.261035948991776
0.9625 0.261035948991776
0.9675 0.261035948991776
0.9725 0.261035948991776
0.9775 0.261035948991776
0.9825 0.261035948991776
0.9875 0.261035948991776
0.9925 0.261035948991776
0.9975 0.261035948991776
};
\end{groupplot}

\end{tikzpicture}

	\caption{Intrinsic variables \(h_0(x,t)\), \(h_1(x,t)\) and \(h_2(x,t)\) of \(\idhy\) obtained from the FCNN. Top row depicts \(\idhy\) over the whole \((x,t)\) domain, middle and bottom row for \(t=0.055\) and \(t=0.12\) respectively.}
	\label{Fig: Code_hy}
\end{figure}
\begin{figure}[hbp!]
	% This file was created by tikzplotlib v0.9.8.
\begin{tikzpicture}

\begin{groupplot}[
group style={group size=5 by 3,
	horizontal sep=1.5cm,
%vertical sep=1cm
},
x tick label style={/pgf/number format/fixed},
y tick label style={/pgf/number format/fixed},
width=0.2\textwidth,
height=0.2\textwidth,
y label style={yshift=-1cm},
x label style={yshift=.5cm}
]
\nextgroupplot[
colorbar horizontal,
colorbar style={xtick={-0.1379664093256,0.16521643102169},
	minor xtick={},
	at={(0.5,1.03)},
	xlabel={\(z_0\)},
	x label style={yshift=-1cm},
	anchor=south,
	xticklabel pos=upper,
	height=0.1*\pgfkeysvalueof{/pgfplots/parent axis width},
	tick pos=rig
},
colormap/blackwhite,
minor xtick={},
minor ytick={},
point meta max=0.16521643102169,
point meta min=-0.1379664093256,
tick align=outside,
tick pos=left,
x grid style={white!69.0196078431373!black},
xlabel={\(x\)},
xmin=0, xmax=1,
xtick style={color=black},
xtick={0,1},
y grid style={white!69.0196078431373!black},
ylabel={\(t\)},
ymin=0, ymax=0.12,
ytick style={color=black},
ytick={0,0.12}
]
\addplot graphics [includegraphics cmd=\pgfimage,xmin=0, xmax=1, ymin=0, ymax=0.12] {Figures/Chapter_5/code_rare-000.png};

\nextgroupplot[
colorbar horizontal,
colorbar style={xtick={-0.10020612180233,0.155585333704948},
	minor xtick={},
	at={(0.5,1.03)},
	xlabel={\(z_0\)},
	x label style={yshift=-1cm},
	anchor=south,
	xticklabel pos=upper,
	height=0.1*\pgfkeysvalueof{/pgfplots/parent axis width},
	tick pos=rig
},
colormap/blackwhite,
minor xtick={},
minor ytick={},
point meta max=0.155585333704948,
point meta min=-0.10020612180233,
tick align=outside,
tick pos=left,
x grid style={white!69.0196078431373!black},
xlabel={\(x\)},
xmin=0, xmax=1,
xtick style={color=black},
xtick={0,1},
y grid style={white!69.0196078431373!black},
ylabel={\(t\)},
ymin=0, ymax=0.12,
ytick style={color=black},
ytick={0,0.12}
]
\addplot graphics [includegraphics cmd=\pgfimage,xmin=0, xmax=1, ymin=0, ymax=0.12] {Figures/Chapter_5/code_rare-001.png};

\nextgroupplot[
colorbar horizontal,
colorbar style={xtick={-0.0112405344843864,0.156611829996109},
	minor xtick={},
	at={(0.5,1.03)},
	xlabel={\(z_0\)},
	x label style={yshift=-1cm},
	anchor=south,
	xticklabel pos=upper,
	height=0.1*\pgfkeysvalueof{/pgfplots/parent axis width},
	tick pos=rig
},
colormap/blackwhite,
minor xtick={},
minor ytick={},
point meta max=0.156611829996109,
point meta min=-0.0112405344843864,
tick align=outside,
tick pos=left,
x grid style={white!69.0196078431373!black},
xlabel={\(x\)},
xmin=0, xmax=1,
xtick style={color=black},
xtick={0,1},
y grid style={white!69.0196078431373!black},
ylabel={\(t\)},
ymin=0, ymax=0.12,
ytick style={color=black},
ytick={0,0.12}
]
\addplot graphics [includegraphics cmd=\pgfimage,xmin=0, xmax=1, ymin=0, ymax=0.12] {Figures/Chapter_5/code_rare-002.png};

\nextgroupplot[
colorbar horizontal,
colorbar style={xtick={-0.108655221760273,0.294203132390976},
	minor xtick={},
	at={(0.5,1.03)},
	xlabel={\(z_0\)},
	x label style={yshift=-1cm},
	anchor=south,
	xticklabel pos=upper,
	height=0.1*\pgfkeysvalueof{/pgfplots/parent axis width},
	tick pos=rig
},
colormap/blackwhite,
minor xtick={},
minor ytick={},
point meta max=0.294203132390976,
point meta min=-0.108655221760273,
tick align=outside,
tick pos=left,
x grid style={white!69.0196078431373!black},
xlabel={\(x\)},
xmin=0, xmax=1,
xtick style={color=black},
xtick={0,1},
y grid style={white!69.0196078431373!black},
ylabel={\(t\)},
ymin=0, ymax=0.12,
ytick style={color=black},
ytick={0,0.12}
]
\addplot graphics [includegraphics cmd=\pgfimage,xmin=0, xmax=1, ymin=0, ymax=0.12] {Figures/Chapter_5/code_rare-003.png};

\nextgroupplot[
colorbar horizontal,
colorbar style={xtick={-0.316409736871719,-0.0237041935324669},
	minor xtick={},
	at={(0.5,1.03)},
	xlabel={\(z_0\)},
	x label style={yshift=-1cm},
	anchor=south,
	xticklabel pos=upper,
	height=0.1*\pgfkeysvalueof{/pgfplots/parent axis width},
	tick pos=rig
},
colormap/blackwhite,
minor xtick={},
minor ytick={},
point meta max=-0.0237041935324669,
point meta min=-0.316409736871719,
tick align=outside,
tick pos=left,
x grid style={white!69.0196078431373!black},
xlabel={\(x\)},
xmin=0, xmax=1,
xtick style={color=black},
xtick={0,1},
y grid style={white!69.0196078431373!black},
ylabel={\(t\)},
ymin=0, ymax=0.12,
ytick style={color=black},
ytick={0,0.12}
]
\addplot graphics [includegraphics cmd=\pgfimage,xmin=0, xmax=1, ymin=0, ymax=0.12] {Figures/Chapter_5/code_rare-004.png};

\nextgroupplot[
minor xtick={},
minor ytick={},
tick align=outside,
tick pos=left,
x grid style={white!69.0196078431373!black},
xlabel={\(x\)},
xmin=-0.04725, xmax=1.04725,
xtick style={color=black},
xtick={0,1},
y grid style={white!69.0196078431373!black},
ylabel={\(z_0\)},
ymin=-0.138996778428555, ymax=0.179700104892254,
ytick style={color=black},
ytick={-0.124510556459427,0.165213882923126}
]
\addplot [thick, color0]
table {%
0.0025 0.165213853120804
0.0075 0.165213853120804
0.0125 0.165213853120804
0.0175 0.165213853120804
0.0225 0.165213853120804
0.0275 0.165213853120804
0.0325 0.165213853120804
0.0375 0.165213853120804
0.0425 0.165213853120804
0.0475 0.165213853120804
0.0525 0.165213853120804
0.0575 0.165213853120804
0.0625 0.165213853120804
0.0675 0.165213853120804
0.0725 0.165213853120804
0.0775 0.165213853120804
0.0825 0.165213853120804
0.0875 0.165213853120804
0.0925 0.165213853120804
0.0975 0.165213853120804
0.1025 0.165213853120804
0.1075 0.165213853120804
0.1125 0.165213853120804
0.1175 0.165213853120804
0.1225 0.165213853120804
0.1275 0.165213853120804
0.1325 0.165213853120804
0.1375 0.165213853120804
0.1425 0.165213853120804
0.1475 0.165213853120804
0.1525 0.165213853120804
0.1575 0.165213853120804
0.1625 0.165213853120804
0.1675 0.165213853120804
0.1725 0.165213853120804
0.1775 0.165213853120804
0.1825 0.165213853120804
0.1875 0.165213853120804
0.1925 0.165213882923126
0.1975 0.165213868021965
0.2025 0.165213868021965
0.2075 0.165213868021965
0.2125 0.165213853120804
0.2175 0.165213823318481
0.2225 0.165213793516159
0.2275 0.165213778614998
0.2325 0.165213733911514
0.2375 0.165213629603386
0.2425 0.16521355509758
0.2475 0.165213361382484
0.2525 0.165213122963905
0.2575 0.165212750434875
0.2625 0.165212169289589
0.2675 0.165211319923401
0.2725 0.165210038423538
0.2775 0.165208041667938
0.2825 0.165205091238022
0.2875 0.165200635790825
0.2925 0.165194004774094
0.2975 0.1651840955019
0.3025 0.165169566869736
0.3075 0.165148109197617
0.3125 0.165116965770721
0.3175 0.165071785449982
0.3225 0.165006995201111
0.3275 0.164914831519127
0.3325 0.164784967899323
0.3375 0.164603725075722
0.3425 0.164353162050247
0.3475 0.164010420441628
0.3525 0.163546353578568
0.3575 0.162924751639366
0.3625 0.162101149559021
0.3675 0.161021903157234
0.3725 0.159623056650162
0.3775 0.158135205507278
0.3825 0.156250417232513
0.3875 0.153849303722382
0.3925 0.150821417570114
0.3975 0.147042259573936
0.4025 0.142375037074089
0.4075 0.136673375964165
0.4125 0.129786819219589
0.4175 0.121568016707897
0.4225 0.111882381141186
0.4275 0.100613750517368
0.4325 0.087685838341713
0.4375 0.0730844736099243
0.4425 0.0568690150976181
0.4475 0.0392049886286259
0.4525 0.0204017646610737
0.4575 0.0009489506483078
0.4625 -0.0184667594730854
0.4675 -0.0380144789814949
0.4725 -0.0565802454948425
0.4775 -0.0730570107698441
0.4825 -0.0869079530239105
0.4875 -0.0979045331478119
0.4925 -0.106133505702019
0.4975 -0.112008586525917
0.5025 -0.116120651364326
0.5075 -0.119105339050293
0.5125 -0.121370226144791
0.5175 -0.123027101159096
0.5225 -0.124110251665115
0.5275 -0.124510556459427
0.5325 -0.123983353376389
0.5375 -0.12218676507473
0.5425 -0.118736326694489
0.5475 -0.11328461766243
0.5525 -0.105521857738495
0.5575 -0.0955346971750259
0.5625 -0.0836039483547211
0.5675 -0.0697289854288101
0.5725 -0.054807037115097
0.5775 -0.0397517383098602
0.5825 -0.0250949785113335
0.5875 -0.0111851096153259
0.5925 0.00168458372354507
0.5975 0.0133372321724892
0.6025 0.023677796125412
0.6075 0.0326724201440811
0.6125 0.0404212921857834
0.6175 0.0469857305288315
0.6225 0.0523831695318222
0.6275 0.0567114874720573
0.6325 0.0600726380944252
0.6375 0.0625609159469604
0.6425 0.0644036903977394
0.6475 0.0657281801104546
0.6525 0.066651850938797
0.6575 0.0672768279910088
0.6625 0.0676870048046112
0.6675 0.067948043346405
0.6725 0.0681170076131821
0.6775 0.068218782544136
0.6825 0.0682777389883995
0.6875 0.0683101639151573
0.6925 0.0683266595005989
0.6975 0.0683340355753899
0.7025 0.0683363229036331
0.7075 0.0683360621333122
0.7125 0.0683348923921585
0.7175 0.0683337301015854
0.7225 0.0683324560523033
0.7275 0.0683312937617302
0.7325 0.0683303475379944
0.7375 0.0683296173810959
0.7425 0.0683290585875511
0.7475 0.0683287009596825
0.7525 0.0683284923434258
0.7575 0.0683283433318138
0.7625 0.0683282464742661
0.7675 0.0683281794190407
0.7725 0.0683281496167183
0.7775 0.0683281347155571
0.7825 0.0683281123638153
0.7875 0.0683281198143959
0.7925 0.0683281198143959
0.7975 0.0683281198143959
0.8025 0.0683281123638153
0.8075 0.0683281049132347
0.8125 0.0683281272649765
0.8175 0.0683281123638153
0.8225 0.0683281272649765
0.8275 0.0683281198143959
0.8325 0.0683281272649765
0.8375 0.0683281198143959
0.8425 0.0683281198143959
0.8475 0.0683281123638153
0.8525 0.0683281049132347
0.8575 0.0683281049132347
0.8625 0.0683280974626541
0.8675 0.0683281123638153
0.8725 0.0683281123638153
0.8775 0.0683281123638153
0.8825 0.0683281123638153
0.8875 0.0683281198143959
0.8925 0.0683281198143959
0.8975 0.0683281198143959
0.9025 0.0683281198143959
0.9075 0.0683281198143959
0.9125 0.0683281198143959
0.9175 0.0683281198143959
0.9225 0.0683281198143959
0.9275 0.0683281198143959
0.9325 0.0683281198143959
0.9375 0.0683281198143959
0.9425 0.0683281198143959
0.9475 0.0683281198143959
0.9525 0.0683281198143959
0.9575 0.0683281198143959
0.9625 0.0683281198143959
0.9675 0.0683281198143959
0.9725 0.0683281198143959
0.9775 0.0683281198143959
0.9825 0.0683281198143959
0.9875 0.0683281198143959
0.9925 0.0683281198143959
0.9975 0.0683281198143959
};

\nextgroupplot[
minor xtick={},
minor ytick={},
tick align=outside,
tick pos=left,
x grid style={white!69.0196078431373!black},
xlabel={\(x\)},
xmin=-0.04725, xmax=1.04725,
xtick style={color=black},
xtick={0,1},
y grid style={white!69.0196078431373!black},
ylabel={\(z_1\)},
ymin=-0.108842796832323, ymax=0.168177149444818,
ytick style={color=black},
ytick={-0.096250981092453,0.155585333704948}
]
\addplot [thick, color0]
table {%
0.0025 0.00732426904141903
0.0075 0.00732426904141903
0.0125 0.00732426904141903
0.0175 0.00732426904141903
0.0225 0.00732426904141903
0.0275 0.00732426904141903
0.0325 0.00732426904141903
0.0375 0.00732426904141903
0.0425 0.00732426904141903
0.0475 0.00732426904141903
0.0525 0.00732426904141903
0.0575 0.00732426904141903
0.0625 0.00732426904141903
0.0675 0.00732426904141903
0.0725 0.00732426904141903
0.0775 0.00732426904141903
0.0825 0.00732426904141903
0.0875 0.00732426904141903
0.0925 0.00732426904141903
0.0975 0.00732426904141903
0.1025 0.00732426904141903
0.1075 0.00732426904141903
0.1125 0.00732426904141903
0.1175 0.00732426904141903
0.1225 0.00732426904141903
0.1275 0.00732426904141903
0.1325 0.00732426904141903
0.1375 0.00732426904141903
0.1425 0.00732426904141903
0.1475 0.00732426904141903
0.1525 0.00732425786554813
0.1575 0.00732425786554813
0.1625 0.00732425786554813
0.1675 0.00732425786554813
0.1725 0.00732425041496754
0.1775 0.00732425041496754
0.1825 0.00732425414025784
0.1875 0.00732425972819328
0.1925 0.00732422061264515
0.1975 0.00732419639825821
0.2025 0.00732415728271008
0.2075 0.00732412002980709
0.2125 0.00732402317225933
0.2175 0.00732392445206642
0.2225 0.0073237307369709
0.2275 0.00732352025806904
0.2325 0.00732313096523285
0.2375 0.00732258148491383
0.2425 0.00732171162962914
0.2475 0.00732053257524967
0.2525 0.00731874443590641
0.2575 0.0073161143809557
0.2625 0.00731223076581955
0.2675 0.00730662606656551
0.2725 0.00729838013648987
0.2775 0.00728645361959934
0.2825 0.00726930983364582
0.2875 0.00724475830793381
0.2925 0.007209662348032
0.2975 0.0071599967777729
0.3025 0.00709025375545025
0.3075 0.00699298270046711
0.3125 0.00685843266546726
0.3175 0.00667397491633892
0.3225 0.00642337836325169
0.3275 0.00608645752072334
0.3325 0.00563804432749748
0.3375 0.00504777394235134
0.3425 0.00427963770925999
0.3475 0.00329176336526871
0.3525 0.00203699618577957
0.3575 0.00046335905790329
0.3625 -0.00148444622755051
0.3675 -0.00386305525898933
0.3725 -0.0067279078066349
0.3775 -0.0100881867110729
0.3825 -0.0140168741345406
0.3875 -0.0185472518205643
0.3925 -0.0236932300031185
0.3975 -0.0294486619532108
0.4025 -0.035783801227808
0.4075 -0.0426417700946331
0.4125 -0.0499370507895947
0.4175 -0.0575538165867329
0.4225 -0.0653466731309891
0.4275 -0.0728999972343445
0.4325 -0.0798108130693436
0.4375 -0.0859197676181793
0.4425 -0.090902179479599
0.4475 -0.0944407731294632
0.4525 -0.096250981092453
0.4575 -0.0961086750030518
0.4625 -0.0938714891672134
0.4675 -0.0892391353845596
0.4725 -0.0822390168905258
0.4775 -0.0729197412729263
0.4825 -0.0613682828843594
0.4875 -0.0482962317764759
0.4925 -0.0357777811586857
0.4975 -0.026931282132864
0.5025 -0.024791345000267
0.5075 -0.0280589647591114
0.5125 -0.032212283462286
0.5175 -0.0341588519513607
0.5225 -0.0314053632318974
0.5275 -0.02320546656847
0.5325 -0.0102621763944626
0.5375 0.00596499256789684
0.5425 0.0237473733723164
0.5475 0.041409358382225
0.5525 0.0570433028042316
0.5575 0.0703300163149834
0.5625 0.0812361985445023
0.5675 0.0892677754163742
0.5725 0.0953684449195862
0.5775 0.100566402077675
0.5825 0.105241864919662
0.5875 0.109682023525238
0.5925 0.114042431116104
0.5975 0.118377044796944
0.6025 0.122664511203766
0.6075 0.126839131116867
0.6125 0.130834519863129
0.6175 0.134573310613632
0.6225 0.137980595231056
0.6275 0.141015619039536
0.6325 0.143669113516808
0.6375 0.145962506532669
0.6425 0.147887662053108
0.6475 0.149483010172844
0.6525 0.150791272521019
0.6575 0.151855200529099
0.6625 0.152714371681213
0.6675 0.153402969241142
0.6725 0.153902888298035
0.6775 0.154298171401024
0.6825 0.15460866689682
0.6875 0.15485081076622
0.6925 0.15503816306591
0.6975 0.155181854963303
0.7025 0.155291050672531
0.7075 0.155373230576515
0.7125 0.155433848500252
0.7175 0.155477344989777
0.7225 0.155509054660797
0.7275 0.155531957745552
0.7325 0.155548289418221
0.7375 0.155559852719307
0.7425 0.155567958950996
0.7475 0.155573502182961
0.7525 0.155577287077904
0.7575 0.15557986497879
0.7625 0.15558160841465
0.7675 0.155582800507545
0.7725 0.155583590269089
0.7775 0.155584126710892
0.7825 0.155584499239922
0.7875 0.155584737658501
0.7925 0.155584916472435
0.7975 0.155585050582886
0.8025 0.155585110187531
0.8075 0.155585169792175
0.8125 0.155585214495659
0.8175 0.155585244297981
0.8225 0.155585274100304
0.8275 0.155585274100304
0.8325 0.155585289001465
0.8375 0.155585303902626
0.8425 0.155585318803787
0.8475 0.155585318803787
0.8525 0.155585303902626
0.8575 0.155585318803787
0.8625 0.155585318803787
0.8675 0.155585318803787
0.8725 0.155585318803787
0.8775 0.155585333704948
0.8825 0.155585333704948
0.8875 0.155585333704948
0.8925 0.155585333704948
0.8975 0.155585333704948
0.9025 0.155585333704948
0.9075 0.155585333704948
0.9125 0.155585333704948
0.9175 0.155585333704948
0.9225 0.155585333704948
0.9275 0.155585333704948
0.9325 0.155585333704948
0.9375 0.155585333704948
0.9425 0.155585333704948
0.9475 0.155585333704948
0.9525 0.155585333704948
0.9575 0.155585333704948
0.9625 0.155585333704948
0.9675 0.155585333704948
0.9725 0.155585333704948
0.9775 0.155585333704948
0.9825 0.155585333704948
0.9875 0.155585333704948
0.9925 0.155585333704948
0.9975 0.155585333704948
};

\nextgroupplot[
minor xtick={},
minor ytick={},
tick align=outside,
tick pos=left,
x grid style={white!69.0196078431373!black},
xlabel={\(x\)},
xmin=-0.04725, xmax=1.04725,
xtick style={color=black},
xtick={0,1},
y grid style={white!69.0196078431373!black},
ylabel={\(z_2\)},
ymin=-0.0193834684789181, ymax=0.163240783661604,
ytick style={color=black},
ytick={-0.0110823661088943,0.15493968129158}
]
\addplot [thick, color0]
table {%
0.0025 0.0644568651914597
0.0075 0.0644568651914597
0.0125 0.0644568651914597
0.0175 0.0644568651914597
0.0225 0.0644568651914597
0.0275 0.0644568651914597
0.0325 0.0644568651914597
0.0375 0.0644568651914597
0.0425 0.0644568651914597
0.0475 0.0644568651914597
0.0525 0.0644568651914597
0.0575 0.0644568651914597
0.0625 0.0644568651914597
0.0675 0.0644568651914597
0.0725 0.0644568651914597
0.0775 0.0644568651914597
0.0825 0.0644568651914597
0.0875 0.0644568651914597
0.0925 0.0644568651914597
0.0975 0.0644568651914597
0.1025 0.0644568651914597
0.1075 0.0644568651914597
0.1125 0.0644568651914597
0.1175 0.0644568651914597
0.1225 0.0644568651914597
0.1275 0.0644568651914597
0.1325 0.0644568651914597
0.1375 0.0644568651914597
0.1425 0.0644568651914597
0.1475 0.0644568651914597
0.1525 0.0644568651914597
0.1575 0.0644568651914597
0.1625 0.0644568651914597
0.1675 0.0644568651914597
0.1725 0.0644568502902985
0.1775 0.0644568502902985
0.1825 0.0644568502902985
0.1875 0.0644568502902985
0.1925 0.0644568204879761
0.1975 0.0644568055868149
0.2025 0.0644567757844925
0.2075 0.0644567757844925
0.2125 0.0644567161798477
0.2175 0.0644566565752029
0.2225 0.064456544816494
0.2275 0.0644564107060432
0.2325 0.0644561946392059
0.2375 0.0644558817148209
0.2425 0.0644554197788239
0.2475 0.0644547194242477
0.2525 0.0644537210464478
0.2575 0.0644521564245224
0.2625 0.0644499361515045
0.2675 0.0644466355443001
0.2725 0.0644417852163315
0.2775 0.0644347742199898
0.2825 0.0644245743751526
0.2875 0.0644099414348602
0.2925 0.0643890425562859
0.2975 0.0643593072891235
0.3025 0.0643174350261688
0.3075 0.0642590299248695
0.3125 0.0641781538724899
0.3175 0.0640673041343689
0.3225 0.0639170631766319
0.3275 0.0637156367301941
0.3325 0.0634488463401794
0.3375 0.0630998909473419
0.3425 0.0626495629549026
0.3475 0.062076672911644
0.3525 0.0613588429987431
0.3575 0.0604735910892487
0.3625 0.059400413185358
0.3675 0.058122918009758
0.3725 0.0566318705677986
0.3775 0.054625790566206
0.3825 0.0523416511714458
0.3875 0.0498371012508869
0.3925 0.0471740216016769
0.3975 0.0444465950131416
0.4025 0.0417856797575951
0.4075 0.0393620431423187
0.4125 0.0373877659440041
0.4175 0.036114975810051
0.4225 0.0358311831951141
0.4275 0.0368649214506149
0.4325 0.0395531877875328
0.4375 0.0442033261060715
0.4425 0.0510787330567837
0.4475 0.0603338703513145
0.4525 0.071941927075386
0.4575 0.0856187343597412
0.4625 0.10077016055584
0.4675 0.115812286734581
0.4725 0.129781574010849
0.4775 0.141715750098228
0.4825 0.150471910834312
0.4875 0.15493968129158
0.4925 0.15409043431282
0.4975 0.147509455680847
0.5025 0.1338721960783
0.5075 0.113612838089466
0.5125 0.089301310479641
0.5175 0.0646857768297195
0.5225 0.0429983213543892
0.5275 0.0255106836557388
0.5325 0.012058898806572
0.5375 0.00204916298389435
0.5425 -0.00501344352960587
0.5475 -0.00946371257305145
0.5525 -0.0110823661088943
0.5575 -0.0106191262602806
0.5625 -0.00862734019756317
0.5675 -0.00506298243999481
0.5725 -0.000621721148490906
0.5775 0.00403313338756561
0.5825 0.00863170623779297
0.5875 0.0130030587315559
0.5925 0.017010360956192
0.5975 0.0205758213996887
0.6025 0.0236635655164719
0.6075 0.0262690484523773
0.6125 0.0284500420093536
0.6175 0.0302460119128227
0.6225 0.0316733047366142
0.6275 0.0327796339988708
0.6325 0.03361576795578
0.6375 0.03423210978508
0.6425 0.0346779003739357
0.6475 0.0349956899881363
0.6525 0.0352211892604828
0.6575 0.0353827476501465
0.6625 0.035501517355442
0.6675 0.035593293607235
0.6725 0.0357106626033783
0.6775 0.0358078181743622
0.6825 0.0358894914388657
0.6875 0.0359586998820305
0.6925 0.0360174253582954
0.6975 0.0360670834779739
0.7025 0.0361087769269943
0.7075 0.0361434817314148
0.7125 0.0361703857779503
0.7175 0.0361891835927963
0.7225 0.0362045541405678
0.7275 0.0362169742584229
0.7325 0.036226861178875
0.7375 0.0362346619367599
0.7425 0.0362407267093658
0.7475 0.0362448692321777
0.7525 0.0362478643655777
0.7575 0.0362500920891762
0.7625 0.0362517535686493
0.7675 0.036252960562706
0.7725 0.0362538397312164
0.7775 0.0362544730305672
0.7825 0.0362549349665642
0.7875 0.036255270242691
0.7925 0.0362555012106895
0.7975 0.0362556800246239
0.8025 0.0362557917833328
0.8075 0.0362558737397194
0.8125 0.0362559333443642
0.8175 0.0362559705972672
0.8225 0.0362560003995895
0.8275 0.0362560227513313
0.8325 0.0362560376524925
0.8375 0.0362560525536537
0.8425 0.0362560451030731
0.8475 0.0362560600042343
0.8525 0.0362560600042343
0.8575 0.0362560674548149
0.8625 0.0362560674548149
0.8675 0.0362560674548149
0.8725 0.0362560749053955
0.8775 0.0362560674548149
0.8825 0.0362560674548149
0.8875 0.0362560674548149
0.8925 0.0362560674548149
0.8975 0.0362560674548149
0.9025 0.0362560674548149
0.9075 0.0362560674548149
0.9125 0.0362560674548149
0.9175 0.0362560674548149
0.9225 0.0362560674548149
0.9275 0.0362560674548149
0.9325 0.0362560674548149
0.9375 0.0362560674548149
0.9425 0.0362560674548149
0.9475 0.0362560674548149
0.9525 0.0362560674548149
0.9575 0.0362560674548149
0.9625 0.0362560674548149
0.9675 0.0362560674548149
0.9725 0.0362560674548149
0.9775 0.0362560674548149
0.9825 0.0362560674548149
0.9875 0.0362560674548149
0.9925 0.0362560674548149
0.9975 0.0362560674548149
};

\nextgroupplot[
minor xtick={},
minor ytick={},
tick align=outside,
tick pos=left,
x grid style={white!69.0196078431373!black},
xlabel={\(x\)},
xmin=-0.04725, xmax=1.04725,
xtick style={color=black},
xtick={0,1},
y grid style={white!69.0196078431373!black},
ylabel={\(z_3\)},
ymin=-0.128624464944005, ymax=0.314337748661637,
ytick style={color=black},
ytick={-0.108489818871021,0.294203102588654}
]
\addplot [[thick, color0]
table {%
0.0025 0.294203102588654
0.0075 0.294203102588654
0.0125 0.294203102588654
0.0175 0.294203102588654
0.0225 0.294203102588654
0.0275 0.294203102588654
0.0325 0.294203102588654
0.0375 0.294203102588654
0.0425 0.294203102588654
0.0475 0.294203102588654
0.0525 0.294203102588654
0.0575 0.294203102588654
0.0625 0.294203102588654
0.0675 0.294203102588654
0.0725 0.294203102588654
0.0775 0.294203102588654
0.0825 0.294203102588654
0.0875 0.294203102588654
0.0925 0.294203102588654
0.0975 0.294203102588654
0.1025 0.294203102588654
0.1075 0.294203102588654
0.1125 0.294203102588654
0.1175 0.294203102588654
0.1225 0.294203102588654
0.1275 0.294203102588654
0.1325 0.294203102588654
0.1375 0.294203102588654
0.1425 0.294203102588654
0.1475 0.294203102588654
0.1525 0.294203102588654
0.1575 0.294203102588654
0.1625 0.294203102588654
0.1675 0.294203102588654
0.1725 0.294203102588654
0.1775 0.294203102588654
0.1825 0.294203102588654
0.1875 0.294203102588654
0.1925 0.294203102588654
0.1975 0.294203102588654
0.2025 0.294203102588654
0.2075 0.294203072786331
0.2125 0.294203042984009
0.2175 0.294203042984009
0.2225 0.294203013181686
0.2275 0.294202983379364
0.2325 0.294202893972397
0.2375 0.294202774763107
0.2425 0.294202625751495
0.2475 0.294202327728271
0.2525 0.29420194029808
0.2575 0.294201344251633
0.2625 0.294200479984283
0.2675 0.294199168682098
0.2725 0.294197261333466
0.2775 0.294194519519806
0.2825 0.294190526008606
0.2875 0.294184863567352
0.2925 0.294176757335663
0.2975 0.294165343046188
0.3025 0.294149458408356
0.3075 0.294127434492111
0.3125 0.294097274541855
0.3175 0.294056355953217
0.3225 0.294001251459122
0.3275 0.29392796754837
0.3325 0.293831348419189
0.3375 0.293705403804779
0.3425 0.293542683124542
0.3475 0.293334633111954
0.3525 0.293071091175079
0.3575 0.292740225791931
0.3625 0.292328238487244
0.3675 0.291819274425507
0.3725 0.291194379329681
0.3775 0.290740311145782
0.3825 0.290205359458923
0.3875 0.289534687995911
0.3925 0.288685947656631
0.3975 0.287603497505188
0.4025 0.286216080188751
0.4075 0.284435153007507
0.4125 0.282154858112335
0.4175 0.279252767562866
0.4225 0.275593221187592
0.4275 0.271080374717712
0.4325 0.265612840652466
0.4375 0.259054183959961
0.4425 0.251334607601166
0.4475 0.242450207471848
0.4525 0.232501298189163
0.4575 0.221731781959534
0.4625 0.210552006959915
0.4675 0.198719590902328
0.4725 0.186999797821045
0.4775 0.176122844219208
0.4825 0.166277796030045
0.4875 0.157244056463242
0.4925 0.148406833410263
0.4975 0.139034509658813
0.5025 0.12706583738327
0.5075 0.111156441271305
0.5125 0.0917116329073906
0.5175 0.0707171857357025
0.5225 0.0500350296497345
0.5275 0.0304419472813606
0.5325 0.0119398236274719
0.5375 -0.00560826435685158
0.5425 -0.0221667774021626
0.5475 -0.0375488698482513
0.5525 -0.0516094900667667
0.5575 -0.0641505047678947
0.5625 -0.0750830844044685
0.5675 -0.0843462944030762
0.5725 -0.0919790491461754
0.5775 -0.0980321243405342
0.5825 -0.102569542825222
0.5875 -0.105682916939259
0.5925 -0.107549138367176
0.5975 -0.108397595584393
0.6025 -0.108489818871021
0.6075 -0.108087763190269
0.6125 -0.107395507395267
0.6175 -0.106594957411289
0.6225 -0.10583757609129
0.6275 -0.105202719569206
0.6325 -0.104714512825012
0.6375 -0.104351073503494
0.6425 -0.104143179953098
0.6475 -0.104055918753147
0.6525 -0.104053542017937
0.6575 -0.104104697704315
0.6625 -0.10418463498354
0.6675 -0.104275785386562
0.6725 -0.104372054338455
0.6775 -0.104460291564465
0.6825 -0.104537479579449
0.6875 -0.104602836072445
0.6925 -0.104656860232353
0.6975 -0.104700766503811
0.7025 -0.104735977947712
0.7075 -0.104763947427273
0.7125 -0.104784019291401
0.7175 -0.104795955121517
0.7225 -0.104805380105972
0.7275 -0.104812808334827
0.7325 -0.104818627238274
0.7375 -0.104823186993599
0.7425 -0.104826733469963
0.7475 -0.104829251766205
0.7525 -0.104831136763096
0.7575 -0.104832574725151
0.7625 -0.104833655059338
0.7675 -0.104834504425526
0.7725 -0.104835115373135
0.7775 -0.104835592210293
0.7825 -0.104835934937
0.7875 -0.104836188256741
0.7925 -0.104836367070675
0.7975 -0.104836516082287
0.8025 -0.104836612939835
0.8075 -0.104836687445641
0.8125 -0.104836739599705
0.8175 -0.104836776852608
0.8225 -0.10483679920435
0.8275 -0.104836814105511
0.8325 -0.104836843907833
0.8375 -0.104836843907833
0.8425 -0.104836851358414
0.8475 -0.104836858808994
0.8525 -0.104836866259575
0.8575 -0.104836858808994
0.8625 -0.104836866259575
0.8675 -0.104836866259575
0.8725 -0.104836866259575
0.8775 -0.104836866259575
0.8825 -0.104836866259575
0.8875 -0.104836866259575
0.8925 -0.104836866259575
0.8975 -0.104836866259575
0.9025 -0.104836866259575
0.9075 -0.104836866259575
0.9125 -0.104836866259575
0.9175 -0.104836866259575
0.9225 -0.104836866259575
0.9275 -0.104836866259575
0.9325 -0.104836866259575
0.9375 -0.104836866259575
0.9425 -0.104836866259575
0.9475 -0.104836866259575
0.9525 -0.104836866259575
0.9575 -0.104836866259575
0.9625 -0.104836866259575
0.9675 -0.104836866259575
0.9725 -0.104836866259575
0.9775 -0.104836866259575
0.9825 -0.104836866259575
0.9875 -0.104836866259575
0.9925 -0.104836866259575
0.9975 -0.104836866259575
};

\nextgroupplot[
minor xtick={},
minor ytick={},
tick align=outside,
tick pos=left,
x grid style={white!69.0196078431373!black},
xlabel={\(x\)},
xmin=-0.04725, xmax=1.04725,
xtick style={color=black},
xtick={0,1},
y grid style={white!69.0196078431373!black},
ylabel={\(z_4\)},
ymin=-0.330900359898806, ymax=-0.0121066533029079,
ytick style={color=black},
ytick={-0.316409736871719,-0.0265972763299942}
]
\addplot [thick, color0]
table {%
0.0025 -0.316409707069397
0.0075 -0.316409707069397
0.0125 -0.316409707069397
0.0175 -0.316409707069397
0.0225 -0.316409707069397
0.0275 -0.316409707069397
0.0325 -0.316409707069397
0.0375 -0.316409707069397
0.0425 -0.316409707069397
0.0475 -0.316409707069397
0.0525 -0.316409707069397
0.0575 -0.316409707069397
0.0625 -0.316409707069397
0.0675 -0.316409707069397
0.0725 -0.316409707069397
0.0775 -0.316409707069397
0.0825 -0.316409707069397
0.0875 -0.316409707069397
0.0925 -0.316409707069397
0.0975 -0.316409707069397
0.1025 -0.316409707069397
0.1075 -0.316409707069397
0.1125 -0.316409707069397
0.1175 -0.316409707069397
0.1225 -0.316409707069397
0.1275 -0.316409707069397
0.1325 -0.316409707069397
0.1375 -0.316409707069397
0.1425 -0.316409707069397
0.1475 -0.316409707069397
0.1525 -0.316409707069397
0.1575 -0.316409707069397
0.1625 -0.316409707069397
0.1675 -0.316409707069397
0.1725 -0.316409707069397
0.1775 -0.316409707069397
0.1825 -0.316409707069397
0.1875 -0.316409736871719
0.1925 -0.316409707069397
0.1975 -0.316409736871719
0.2025 -0.316409647464752
0.2075 -0.316409647464752
0.2125 -0.31640961766243
0.2175 -0.316409558057785
0.2225 -0.316409438848495
0.2275 -0.316409349441528
0.2325 -0.316409170627594
0.2375 -0.316408902406693
0.2425 -0.316408485174179
0.2475 -0.316407889127731
0.2525 -0.316407024860382
0.2575 -0.316405773162842
0.2625 -0.316403865814209
0.2675 -0.316401124000549
0.2725 -0.316397219896317
0.2775 -0.316391497850418
0.2825 -0.316383332014084
0.2875 -0.316371649503708
0.2925 -0.316354900598526
0.2975 -0.316331386566162
0.3025 -0.316298365592957
0.3075 -0.316252410411835
0.3125 -0.316188931465149
0.3175 -0.316101968288422
0.3225 -0.315983921289444
0.3275 -0.315825223922729
0.3325 -0.315614134073257
0.3375 -0.315336138010025
0.3425 -0.314974099397659
0.3475 -0.314508020877838
0.3525 -0.313914924860001
0.3575 -0.313169181346893
0.3625 -0.312242776155472
0.3675 -0.311105877161026
0.3725 -0.309727609157562
0.3775 -0.307444632053375
0.3825 -0.304686576128006
0.3875 -0.301470816135406
0.3925 -0.297776967287064
0.3975 -0.293597310781479
0.4025 -0.28893831372261
0.4075 -0.283821314573288
0.4125 -0.278280645608902
0.4175 -0.272360056638718
0.4225 -0.26610654592514
0.4275 -0.259570747613907
0.4325 -0.252784758806229
0.4375 -0.245738625526428
0.4425 -0.238380283117294
0.4475 -0.230591058731079
0.4525 -0.222168296575546
0.4575 -0.212814837694168
0.4625 -0.20214107632637
0.4675 -0.18906857073307
0.4725 -0.17320029437542
0.4775 -0.154196187853813
0.4825 -0.131743788719177
0.4875 -0.106391549110413
0.4925 -0.0805178582668304
0.4975 -0.0583427175879478
0.5025 -0.0425669774413109
0.5075 -0.0327327325940132
0.5125 -0.0269745513796806
0.5175 -0.0265972763299942
0.5225 -0.0319302454590797
0.5275 -0.042245052754879
0.5325 -0.0561828836798668
0.5375 -0.0721094608306885
0.5425 -0.0883113741874695
0.5475 -0.103185623884201
0.5525 -0.115286633372307
0.5575 -0.124035090208054
0.5625 -0.129315316677094
0.5675 -0.131327480077744
0.5725 -0.130633056163788
0.5775 -0.127894192934036
0.5825 -0.123684495687485
0.5875 -0.118535727262497
0.5925 -0.112874403595924
0.5975 -0.107056811451912
0.6025 -0.101374685764313
0.6075 -0.0960611402988434
0.6125 -0.0913240611553192
0.6175 -0.0872838497161865
0.6225 -0.0839738398790359
0.6275 -0.0813927352428436
0.6325 -0.0794999599456787
0.6375 -0.0782299786806107
0.6425 -0.0774455070495605
0.6475 -0.0770415887236595
0.6525 -0.0769172012805939
0.6575 -0.0769832357764244
0.6625 -0.0771668404340744
0.6675 -0.0774120837450027
0.6725 -0.0776635855436325
0.6775 -0.0779141783714294
0.6825 -0.0781476050615311
0.6875 -0.0783551633358002
0.6925 -0.0785334855318069
0.6975 -0.078682541847229
0.7025 -0.0788044333457947
0.7075 -0.0789021998643875
0.7125 -0.0789797902107239
0.7175 -0.0790410041809082
0.7225 -0.0790880024433136
0.7275 -0.0791236907243729
0.7325 -0.0791504979133606
0.7375 -0.0791704803705215
0.7425 -0.0791852623224258
0.7475 -0.0791962891817093
0.7525 -0.0792044252157211
0.7575 -0.0792103409767151
0.7625 -0.0792146474123001
0.7675 -0.0792177468538284
0.7725 -0.0792200118303299
0.7775 -0.0792216062545776
0.7825 -0.0792227983474731
0.7875 -0.0792236328125
0.7925 -0.0792242437601089
0.7975 -0.0792246758937836
0.8025 -0.0792250037193298
0.8075 -0.0792252421379089
0.8125 -0.0792253911495209
0.8175 -0.0792255252599716
0.8225 -0.0792256146669388
0.8275 -0.0792256742715836
0.8325 -0.0792257189750671
0.8375 -0.0792257487773895
0.8425 -0.0792257636785507
0.8475 -0.0792257934808731
0.8525 -0.0792258083820343
0.8575 -0.0792258083820343
0.8625 -0.0792258083820343
0.8675 -0.0792258232831955
0.8725 -0.0792258232831955
0.8775 -0.0792258232831955
0.8825 -0.0792258232831955
0.8875 -0.0792258381843567
0.8925 -0.0792258381843567
0.8975 -0.0792258381843567
0.9025 -0.0792258381843567
0.9075 -0.0792258381843567
0.9125 -0.0792258381843567
0.9175 -0.0792258381843567
0.9225 -0.0792258381843567
0.9275 -0.0792258381843567
0.9325 -0.0792258381843567
0.9375 -0.0792258381843567
0.9425 -0.0792258381843567
0.9475 -0.0792258381843567
0.9525 -0.0792258381843567
0.9575 -0.0792258381843567
0.9625 -0.0792258381843567
0.9675 -0.0792258381843567
0.9725 -0.0792258381843567
0.9775 -0.0792258381843567
0.9825 -0.0792258381843567
0.9875 -0.0792258381843567
0.9925 -0.0792258381843567
0.9975 -0.0792258381843567
};

\nextgroupplot[
minor xtick={},
minor ytick={},
tick align=outside,
tick pos=left,
x grid style={white!69.0196078431373!black},
xlabel={\(x\)},
xmin=-0.04725, xmax=1.04725,
xtick style={color=black},
xtick={0,1},
y grid style={white!69.0196078431373!black},
ylabel={\(z_0\)},
ymin=-0.153125420957804, ymax=0.180372834950686,
ytick style={color=black},
ytick={-0.1379664093256,0.165213823318481}
]
\addplot [thick, color0]
table {%
0.0025 0.165213823318481
0.0075 0.165213823318481
0.0125 0.165213763713837
0.0175 0.165213704109192
0.0225 0.165213659405708
0.0275 0.165213644504547
0.0325 0.165213584899902
0.0375 0.165213480591774
0.0425 0.165213376283646
0.0475 0.165213212370872
0.0525 0.165212988853455
0.0575 0.165212735533714
0.0625 0.165212363004684
0.0675 0.165211886167526
0.0725 0.165211245417595
0.0775 0.165210455656052
0.0825 0.165209352970123
0.0875 0.16520793735981
0.0925 0.165206104516983
0.0975 0.165203660726547
0.1025 0.165200486779213
0.1075 0.165196344256401
0.1125 0.165191009640694
0.1175 0.165184050798416
0.1225 0.165175169706345
0.1275 0.165163695812225
0.1325 0.165149033069611
0.1375 0.165130272507668
0.1425 0.165106475353241
0.1475 0.165076240897179
0.1525 0.165038138628006
0.1575 0.164990231394768
0.1625 0.164930388331413
0.1675 0.164855867624283
0.1725 0.164763554930687
0.1775 0.164649784564972
0.1825 0.164510205388069
0.1875 0.164339855313301
0.1925 0.164133101701736
0.1975 0.163883462548256
0.2025 0.163583546876907
0.2075 0.163225367665291
0.2125 0.162799865007401
0.2175 0.162297084927559
0.2225 0.161706283688545
0.2275 0.161015719175339
0.2325 0.160213112831116
0.2375 0.159285068511963
0.2425 0.158217966556549
0.2475 0.15702086687088
0.2525 0.155809432268143
0.2575 0.154430106282234
0.2625 0.152866959571838
0.2675 0.151103392243385
0.2725 0.14912448823452
0.2775 0.14691436290741
0.2825 0.144457146525383
0.2875 0.141737371683121
0.2925 0.138740062713623
0.2975 0.13545086979866
0.3025 0.131856471300125
0.3075 0.127944707870483
0.3125 0.123704455792904
0.3175 0.119126550853252
0.3225 0.11420339345932
0.3275 0.10892965644598
0.3325 0.103302486240864
0.3375 0.097321555018425
0.3425 0.0909897685050964
0.3475 0.0843132585287094
0.3525 0.077301912009716
0.3575 0.0699696838855743
0.3625 0.0623346194624901
0.3675 0.0544193908572197
0.3725 0.0462512485682964
0.3775 0.0378623083233833
0.3825 0.029289610683918
0.3875 0.0205751769244671
0.3925 0.0117662344127893
0.3975 0.00291521847248077
0.4025 -0.00616323947906494
0.4075 -0.0153933390974998
0.4125 -0.02456334233284
0.4175 -0.0336076095700264
0.4225 -0.0424578487873077
0.4275 -0.051045797765255
0.4325 -0.0593060553073883
0.4375 -0.0671793073415756
0.4425 -0.0746151059865952
0.4475 -0.0815736725926399
0.4525 -0.0880272835493088
0.4575 -0.0939595401287079
0.4625 -0.0993651300668716
0.4675 -0.104249432682991
0.4725 -0.108622595667839
0.4775 -0.112533450126648
0.4825 -0.116006299853325
0.4875 -0.119103237986565
0.4925 -0.121884599328041
0.4975 -0.124404907226562
0.5025 -0.126703456044197
0.5075 -0.12880177795887
0.5125 -0.130657091736794
0.5175 -0.132401958107948
0.5225 -0.133938983082771
0.5275 -0.135251000523567
0.5325 -0.136340722441673
0.5375 -0.137181237339973
0.5425 -0.137736842036247
0.5475 -0.1379664093256
0.5525 -0.13782973587513
0.5575 -0.137292265892029
0.5625 -0.136328488588333
0.5675 -0.134922862052917
0.5725 -0.133068963885307
0.5775 -0.130767643451691
0.5825 -0.128025367856026
0.5875 -0.124852418899536
0.5925 -0.121262073516846
0.5975 -0.117270275950432
0.6025 -0.112895414233208
0.6075 -0.108158484101295
0.6125 -0.103083267807961
0.6175 -0.0976563841104507
0.6225 -0.0919305086135864
0.6275 -0.0859608203172684
0.6325 -0.0797815024852753
0.6375 -0.0734305381774902
0.6425 -0.0669493600726128
0.6475 -0.0603148937225342
0.6525 -0.0534566789865494
0.6575 -0.0466519743204117
0.6625 -0.0399407297372818
0.6675 -0.033358246088028
0.6725 -0.0269324779510498
0.6775 -0.020682618021965
0.6825 -0.0146191194653511
0.6875 -0.00874465703964233
0.6925 -0.00305675715208054
0.6975 0.00244968384504318
0.7025 0.00777920335531235
0.7075 0.0129333212971687
0.7125 0.0179088190197945
0.7175 0.0226970091462135
0.7225 0.0272842198610306
0.7275 0.0316525995731354
0.7325 0.0357819348573685
0.7375 0.0397694855928421
0.7425 0.0435127168893814
0.7475 0.0469665974378586
0.7525 0.0501179769635201
0.7575 0.0529594197869301
0.7625 0.055489644408226
0.7675 0.0577139630913734
0.7725 0.0596437454223633
0.7775 0.0612959042191505
0.7825 0.0626814961433411
0.7875 0.063813291490078
0.7925 0.0647565573453903
0.7975 0.0655250325798988
0.8025 0.0661441460251808
0.8075 0.0666377991437912
0.8125 0.0670276284217834
0.8175 0.0673326849937439
0.8225 0.0675726309418678
0.8275 0.0677590817213058
0.8325 0.0679017305374146
0.8375 0.0680099055171013
0.8425 0.0680911391973495
0.8475 0.0681531354784966
0.8525 0.0682035684585571
0.8575 0.0682413801550865
0.8625 0.0682694539427757
0.8675 0.068290077149868
0.8725 0.0683049932122231
0.8775 0.0683156177401543
0.8825 0.0683230012655258
0.8875 0.0683279931545258
0.8925 0.0683312118053436
0.8975 0.0683331564068794
0.9025 0.0683341771364212
0.9075 0.0683345720171928
0.9125 0.0683348402380943
0.9175 0.0683347955346107
0.9225 0.0683345198631287
0.9275 0.068334124982357
0.9325 0.06833366304636
0.9375 0.0683331415057182
0.9425 0.0683326348662376
0.9475 0.0683321356773376
0.9525 0.0683316588401794
0.9575 0.0683312490582466
0.9625 0.0683308467268944
0.9675 0.0683305189013481
0.9725 0.0683302581310272
0.9775 0.0683300420641899
0.9825 0.0683298856019974
0.9875 0.0683298408985138
0.9925 0.0683300048112869
0.9975 0.0683304071426392
};

\nextgroupplot[
minor xtick={},
minor ytick={},
tick align=outside,
tick pos=left,
x grid style={white!69.0196078431373!black},
xlabel={\(x\)},
xmin=-0.04725, xmax=1.04725,
xtick style={color=black},
xtick={0,1},
y grid style={white!69.0196078431373!black},
ylabel={\(z_1\)},
ymin=-0.103315606713295, ymax=0.167906925082207,
ytick style={color=black},
ytick={-0.0909873098134995,0.155578628182411}
]
\addplot [thick, color0]
table {%
0.0025 0.00732404552400112
0.0075 0.0073239728808403
0.0125 0.00732391700148582
0.0175 0.00732376426458359
0.0225 0.0073236022144556
0.0275 0.00732341594994068
0.0325 0.00732313096523285
0.0375 0.00732281245291233
0.0425 0.00732234492897987
0.0475 0.00732176750898361
0.0525 0.00732097588479519
0.0575 0.00731997564435005
0.0625 0.00731867551803589
0.0675 0.00731696747243404
0.0725 0.00731472671031952
0.0775 0.00731184706091881
0.0825 0.0073081236332655
0.0875 0.00730329938232899
0.0925 0.00729712471365929
0.0975 0.00728912651538849
0.1025 0.0072789378464222
0.1075 0.00726587697863579
0.1125 0.00724918767809868
0.1175 0.00722798332571983
0.1225 0.00720119848847389
0.1275 0.00716738402843475
0.1325 0.00712498091161251
0.1375 0.0070718415081501
0.1425 0.00700569711625576
0.1475 0.00692368857562542
0.1525 0.00682246312499046
0.1575 0.00669820420444012
0.1625 0.0065462663769722
0.1675 0.00636162795126438
0.1725 0.00613829493522644
0.1775 0.00586976483464241
0.1825 0.00554861687123775
0.1875 0.00516672991216183
0.1925 0.00471534766256809
0.1975 0.0041848998516798
0.2025 0.00356539338827133
0.2075 0.00284620374441147
0.2125 0.00201642513275146
0.2175 0.00106506049633026
0.2225 -1.89021229743958e-05
0.2275 -0.00124629959464073
0.2325 -0.00262732803821564
0.2375 -0.00417163595557213
0.2425 -0.00588762760162354
0.2475 -0.00777946785092354
0.2525 -0.00983509421348572
0.2575 -0.0120775885879993
0.2625 -0.014489758759737
0.2675 -0.0170408077538013
0.2725 -0.0197721868753433
0.2775 -0.0226782895624638
0.2825 -0.0257510095834732
0.2875 -0.0289794094860554
0.2925 -0.0323500595986843
0.2975 -0.0358466394245625
0.3025 -0.0394504405558109
0.3075 -0.04313999786973
0.3125 -0.0468916483223438
0.3175 -0.0506793297827244
0.3225 -0.0544751919806004
0.3275 -0.0582494027912617
0.3325 -0.0619708634912968
0.3375 -0.0656070113182068
0.3425 -0.0691246688365936
0.3475 -0.0724900662899017
0.3525 -0.0756694674491882
0.3575 -0.0786294937133789
0.3625 -0.0813378691673279
0.3675 -0.083763599395752
0.3725 -0.085877925157547
0.3775 -0.0876545757055283
0.3825 -0.0890706777572632
0.3875 -0.0901070386171341
0.3925 -0.090749055147171
0.3975 -0.0909873098134995
0.4025 -0.0907614827156067
0.4075 -0.0900776386260986
0.4125 -0.0889802426099777
0.4175 -0.0874869525432587
0.4225 -0.0856241136789322
0.4275 -0.0834273099899292
0.4325 -0.0809405595064163
0.4375 -0.0782148092985153
0.4425 -0.0753058791160583
0.4475 -0.0722712725400925
0.4525 -0.0691672712564468
0.4575 -0.0660486072301865
0.4625 -0.0629714727401733
0.4675 -0.0600023902952671
0.4725 -0.0572211630642414
0.4775 -0.0547393597662449
0.4825 -0.052627008408308
0.4875 -0.0509090386331081
0.4925 -0.0495080538094044
0.4975 -0.0482665188610554
0.5025 -0.0470221303403378
0.5075 -0.0456837080419064
0.5125 -0.0442121662199497
0.5175 -0.0427012853324413
0.5225 -0.0410706363618374
0.5275 -0.0392193160951138
0.5325 -0.0370095185935497
0.5375 -0.0342900045216084
0.5425 -0.0309572480618954
0.5475 -0.0269748605787754
0.5525 -0.0223618596792221
0.5575 -0.017166830599308
0.5625 -0.0114446468651295
0.5675 -0.00524622574448586
0.5725 0.00137970596551895
0.5775 0.00837685167789459
0.5825 0.0156712383031845
0.5875 0.0231646429747343
0.5925 0.0307354778051376
0.5975 0.0382466800510883
0.6025 0.0455592125654221
0.6075 0.0525467842817307
0.6125 0.0591091923415661
0.6175 0.0649483054876328
0.6225 0.0701716244220734
0.6275 0.0749168395996094
0.6325 0.0792082250118256
0.6375 0.0830864459276199
0.6425 0.0866010785102844
0.6475 0.0897011682391167
0.6525 0.0922592878341675
0.6575 0.0946737676858902
0.6625 0.0969814658164978
0.6675 0.0992155522108078
0.6725 0.101405516266823
0.6775 0.103576764464378
0.6825 0.10575045645237
0.6875 0.1079431027174
0.6925 0.110166400671005
0.6975 0.112426787614822
0.7025 0.114725708961487
0.7075 0.11705955862999
0.7125 0.119420155882835
0.7175 0.121795371174812
0.7225 0.124169856309891
0.7275 0.126525908708572
0.7325 0.128844425082207
0.7375 0.131128251552582
0.7425 0.133342579007149
0.7475 0.135464206337929
0.7525 0.137477368116379
0.7575 0.139369159936905
0.7625 0.141129910945892
0.7675 0.142753690481186
0.7725 0.144238129258156
0.7775 0.145584285259247
0.7825 0.146800085902214
0.7875 0.14788843691349
0.7925 0.148843318223953
0.7975 0.149692520499229
0.8025 0.150445327162743
0.8075 0.151111170649529
0.8125 0.151699170470238
0.8175 0.152217999100685
0.8225 0.152656257152557
0.8275 0.153033703565598
0.8325 0.153366550803185
0.8375 0.153660207986832
0.8425 0.153919339179993
0.8475 0.154145419597626
0.8525 0.15433757007122
0.8575 0.154506221413612
0.8625 0.154654130339622
0.8675 0.154783681035042
0.8725 0.15489698946476
0.8775 0.154995858669281
0.8825 0.155081927776337
0.8875 0.155156672000885
0.8925 0.155221357941628
0.8975 0.155277192592621
0.9025 0.155325248837471
0.9075 0.155366495251656
0.9125 0.155400842428207
0.9175 0.155430063605309
0.9225 0.155454933643341
0.9275 0.155476033687592
0.9325 0.155493870377541
0.9375 0.155508950352669
0.9425 0.155521646142006
0.9475 0.15553230047226
0.9525 0.155541270971298
0.9575 0.155548766255379
0.9625 0.155555039644241
0.9675 0.155560225248337
0.9725 0.155564561486244
0.9775 0.155568182468414
0.9825 0.155571207404137
0.9875 0.155573800206184
0.9925 0.155576139688492
0.9975 0.155578628182411
};

\nextgroupplot[
minor xtick={},
minor ytick={},
tick align=outside,
tick pos=left,
x grid style={white!69.0196078431373!black},
xlabel={\(x\)},
xmin=-0.04725, xmax=1.04725,
xtick style={color=black},
xtick={0,1},
y grid style={white!69.0196078431373!black},
ylabel={\(z_2\)},
ymin=-0.0121045127511024, ymax=0.15848150998354,
ytick style={color=black},
ytick={-0.00435060262680054,0.150727599859238}
]
\addplot [thick, color0]
table {%
0.0025 0.0644567683339119
0.0075 0.0644567012786865
0.0125 0.0644566640257835
0.0175 0.064456582069397
0.0225 0.0644564852118492
0.0275 0.0644563883543015
0.0325 0.0644562095403671
0.0375 0.0644560232758522
0.0425 0.0644557774066925
0.0475 0.0644554495811462
0.0525 0.0644550621509552
0.0575 0.0644544661045074
0.0625 0.0644537359476089
0.0675 0.0644527673721313
0.0725 0.0644515380263329
0.0775 0.0644499287009239
0.0825 0.064447820186615
0.0875 0.0644451305270195
0.0925 0.0644416436553001
0.0975 0.064437210559845
0.1025 0.0644315034151077
0.1075 0.0644242092967033
0.1125 0.0644149705767632
0.1175 0.0644032433629036
0.1225 0.0643884092569351
0.1275 0.0643697679042816
0.1325 0.0643464922904968
0.1375 0.064317412674427
0.1425 0.0642814040184021
0.1475 0.0642369464039803
0.1525 0.0641824007034302
0.1575 0.0641158074140549
0.1625 0.0640349686145782
0.1675 0.0639373809099197
0.1725 0.0638203173875809
0.1775 0.0636808425188065
0.1825 0.0635155737400055
0.1875 0.0633211880922318
0.1925 0.0630940943956375
0.1975 0.0628305897116661
0.2025 0.06252720952034
0.2075 0.0621804669499397
0.2125 0.0617871731519699
0.2175 0.0613446570932865
0.2225 0.0608508735895157
0.2275 0.0603045523166656
0.2325 0.0597053878009319
0.2375 0.0590542815625668
0.2425 0.0583534650504589
0.2475 0.0575833395123482
0.2525 0.0566196478903294
0.2575 0.0556071922183037
0.2625 0.0545564666390419
0.2675 0.0534810312092304
0.2725 0.052392240613699
0.2775 0.0513065978884697
0.2825 0.05024279281497
0.2875 0.0492219477891922
0.2925 0.0482671111822128
0.2975 0.0474036335945129
0.3025 0.0466582924127579
0.3075 0.0460595786571503
0.3125 0.0456371232867241
0.3175 0.0454213917255402
0.3225 0.0454432293772697
0.3275 0.0457335114479065
0.3325 0.0463225245475769
0.3375 0.0472395643591881
0.3425 0.0485121607780457
0.3475 0.0501654893159866
0.3525 0.0522217005491257
0.3575 0.0546991638839245
0.3625 0.057611633092165
0.3675 0.0609675012528896
0.3725 0.064768947660923
0.3775 0.0690111070871353
0.3825 0.0736811235547066
0.3875 0.0787574276328087
0.3925 0.084208570420742
0.3975 0.0899923294782639
0.4025 0.095890961587429
0.4075 0.101847529411316
0.4125 0.107881873846054
0.4175 0.113901361823082
0.4225 0.119804367423058
0.4275 0.125482723116875
0.4325 0.130825772881508
0.4375 0.135724529623985
0.4425 0.140075817704201
0.4475 0.143785715103149
0.4525 0.146771624684334
0.4575 0.148962438106537
0.4625 0.150297731161118
0.4675 0.150727599859238
0.4725 0.150212347507477
0.4775 0.148756206035614
0.4825 0.146380126476288
0.4875 0.143177896738052
0.4925 0.139284700155258
0.4975 0.134845897555351
0.5025 0.129963010549545
0.5075 0.124648556113243
0.5125 0.11880848556757
0.5175 0.112364046275616
0.5225 0.105219081044197
0.5275 0.0974133461713791
0.5325 0.0891104191541672
0.5375 0.0805323272943497
0.5425 0.0718985721468925
0.5475 0.0633838474750519
0.5525 0.0551114454865456
0.5575 0.0471684187650681
0.5625 0.0396250635385513
0.5675 0.032546803355217
0.5725 0.0259969905018806
0.5775 0.0200337991118431
0.5825 0.0147047638893127
0.5875 0.0100420266389847
0.5925 0.0060594379901886
0.5975 0.00275204330682755
0.6025 9.76324081420898e-05
0.6075 -0.00194019079208374
0.6125 -0.00340783596038818
0.6175 -0.00415234267711639
0.6225 -0.00435060262680054
0.6275 -0.00417722016572952
0.6325 -0.00368179380893707
0.6375 -0.00291087478399277
0.6425 -0.00190810859203339
0.6475 -0.000649631023406982
0.6525 0.000938087701797485
0.6575 0.00259964168071747
0.6625 0.00430665165185928
0.6675 0.00603395700454712
0.6725 0.0077604204416275
0.6775 0.00946932286024094
0.6825 0.0111484974622726
0.6875 0.0127898082137108
0.6925 0.0143883526325226
0.6975 0.0159413740038872
0.7025 0.0174470990896225
0.7075 0.0189037621021271
0.7125 0.0203088447451591
0.7175 0.0216588005423546
0.7225 0.0229489207267761
0.7275 0.0241737067699432
0.7325 0.0253273397684097
0.7375 0.0264577120542526
0.7425 0.0275222361087799
0.7475 0.0285041332244873
0.7525 0.0294007137417793
0.7575 0.0302109867334366
0.7625 0.0309358239173889
0.7675 0.0315778702497482
0.7725 0.032141350209713
0.7775 0.0326317846775055
0.7825 0.0330555289983749
0.7875 0.033425934612751
0.7925 0.0337509363889694
0.7975 0.0340265855193138
0.8025 0.0342604517936707
0.8075 0.0344594120979309
0.8125 0.0346295684576035
0.8175 0.0347761064767838
0.8225 0.0349205136299133
0.8275 0.0350552052259445
0.8325 0.0351743102073669
0.8375 0.0352802872657776
0.8425 0.0353750884532928
0.8475 0.0354618802666664
0.8525 0.035544641315937
0.8575 0.0356197357177734
0.8625 0.0356878936290741
0.8675 0.0357497185468674
0.8725 0.0358057767152786
0.8775 0.0358565300703049
0.8825 0.0359024554491043
0.8875 0.0359439253807068
0.8925 0.0359813347458839
0.8975 0.0360150337219238
0.9025 0.0360453426837921
0.9075 0.0360725522041321
0.9125 0.0360944196581841
0.9175 0.036113940179348
0.9225 0.036131426692009
0.9275 0.0361470580101013
0.9325 0.0361609980463982
0.9375 0.0361733734607697
0.9425 0.0361843705177307
0.9475 0.0361940860748291
0.9525 0.0362026616930962
0.9575 0.0362102016806602
0.9625 0.0362167805433273
0.9675 0.0362222865223885
0.9725 0.0362270772457123
0.9775 0.0362311899662018
0.9825 0.0362347438931465
0.9875 0.0362377613782883
0.9925 0.0362402722239494
0.9975 0.0362423658370972
};

\nextgroupplot[
minor xtick={},
minor ytick={},
tick align=outside,
tick pos=left,
x grid style={white!69.0196078431373!black},
xlabel={\(x\)},
xmin=-0.04725, xmax=1.04725,
xtick style={color=black},
xtick={0,1},
y grid style={white!69.0196078431373!black},
ylabel={\(z_3\)},
ymin=-0.124776527658105, ymax=0.314154513552785,
ytick style={color=black},
ytick={-0.104825116693974,0.294203102588654}
]
\addplot [thick, color0]
table {%
0.0025 0.294203102588654
0.0075 0.294203072786331
0.0125 0.294203042984009
0.0175 0.294203042984009
0.0225 0.294202983379364
0.0275 0.294202953577042
0.0325 0.294202923774719
0.0375 0.294202864170074
0.0425 0.294202774763107
0.0475 0.294202625751495
0.0525 0.294202506542206
0.0575 0.294202268123627
0.0625 0.294201999902725
0.0675 0.294201672077179
0.0725 0.294201195240021
0.0775 0.294200658798218
0.0825 0.294199883937836
0.0875 0.294198840856552
0.0925 0.294197559356689
0.0975 0.294195920228958
0.1025 0.294193744659424
0.1075 0.294191002845764
0.1125 0.294187515974045
0.1175 0.294182986021042
0.1225 0.294177323579788
0.1275 0.294170081615448
0.1325 0.294161021709442
0.1375 0.294149667024612
0.1425 0.294135451316833
0.1475 0.29411780834198
0.1525 0.294095933437347
0.1575 0.294068992137909
0.1625 0.294035971164703
0.1675 0.293995678424835
0.1725 0.293946623802185
0.1775 0.293887376785278
0.1825 0.293816149234772
0.1875 0.293730854988098
0.1925 0.293629288673401
0.1975 0.293509006500244
0.2025 0.293367207050323
0.2075 0.2932009100914
0.2125 0.293006837368011
0.2175 0.292781591415405
0.2225 0.292521297931671
0.2275 0.292221963405609
0.2325 0.29187947511673
0.2375 0.29148930311203
0.2425 0.291046738624573
0.2475 0.290570974349976
0.2525 0.29018896818161
0.2575 0.289755463600159
0.2625 0.289268583059311
0.2675 0.288728326559067
0.2725 0.288119792938232
0.2775 0.287436306476593
0.2825 0.286670506000519
0.2875 0.285814970731735
0.2925 0.28486156463623
0.2975 0.283801913261414
0.3025 0.282627284526825
0.3075 0.281328737735748
0.3125 0.279897183179855
0.3175 0.278323441743851
0.3225 0.276598453521729
0.3275 0.274713486433029
0.3325 0.272660255432129
0.3375 0.270431339740753
0.3425 0.268020153045654
0.3475 0.265421479940414
0.3525 0.26263165473938
0.3575 0.25964891910553
0.3625 0.256473869085312
0.3675 0.253109306097031
0.3725 0.249561011791229
0.3775 0.245837658643723
0.3825 0.24195122718811
0.3875 0.237917304039001
0.3925 0.233754992485046
0.3975 0.229487419128418
0.4025 0.224952042102814
0.4075 0.220190286636353
0.4125 0.215350210666656
0.4175 0.210465967655182
0.4225 0.205574154853821
0.4275 0.200711727142334
0.4325 0.195913732051849
0.4375 0.191210687160492
0.4425 0.186626315116882
0.4475 0.182175606489182
0.4525 0.177863568067551
0.4575 0.173684269189835
0.4625 0.169620096683502
0.4675 0.1656414270401
0.4725 0.161703914403915
0.4775 0.157767236232758
0.4825 0.153771609067917
0.4875 0.149676084518433
0.4925 0.145447283983231
0.4975 0.141057908535004
0.5025 0.136468261480331
0.5075 0.131601125001907
0.5125 0.126335442066193
0.5175 0.120502538979053
0.5225 0.113985069096088
0.5275 0.106728829443455
0.5325 0.0987623408436775
0.5375 0.0901771560311317
0.5425 0.0810870006680489
0.5475 0.0715990960597992
0.5525 0.0618055239319801
0.5575 0.0517881140112877
0.5625 0.0416270270943642
0.5675 0.0314064845442772
0.5725 0.0212162584066391
0.5775 0.0111501514911652
0.5825 0.00130291283130646
0.5875 -0.00823384895920753
0.5925 -0.0173760578036308
0.5975 -0.0260520800948143
0.6025 -0.0342071354389191
0.6075 -0.0418066494166851
0.6125 -0.0488372892141342
0.6175 -0.0553316883742809
0.6225 -0.0612974166870117
0.6275 -0.0667532309889793
0.6325 -0.0717384070158005
0.6375 -0.0762917101383209
0.6425 -0.0804470479488373
0.6475 -0.0842238068580627
0.6525 -0.0876282826066017
0.6575 -0.0906951203942299
0.6625 -0.0934297367930412
0.6675 -0.0958351492881775
0.6725 -0.0979145988821983
0.6775 -0.0996738970279694
0.6825 -0.101123161613941
0.6875 -0.102277524769306
0.6925 -0.1031574010849
0.6975 -0.103788010776043
0.7025 -0.104198276996613
0.7075 -0.104419961571693
0.7125 -0.104486137628555
0.7175 -0.104430116713047
0.7225 -0.104284279048443
0.7275 -0.104079052805901
0.7325 -0.103842087090015
0.7375 -0.103557854890823
0.7425 -0.103274680674076
0.7475 -0.103018462657928
0.7525 -0.102801464498043
0.7575 -0.102631784975529
0.7625 -0.102513514459133
0.7675 -0.10244707763195
0.7725 -0.10242997854948
0.7775 -0.102457299828529
0.7825 -0.102516621351242
0.7875 -0.102589130401611
0.7925 -0.102689646184444
0.7975 -0.102811239659786
0.8025 -0.102946527302265
0.8075 -0.103089153766632
0.8125 -0.103233866393566
0.8175 -0.103376612067223
0.8225 -0.103516563773155
0.8275 -0.103650286793709
0.8325 -0.103775329887867
0.8375 -0.103891082108021
0.8425 -0.103997297585011
0.8475 -0.10409390181303
0.8525 -0.104180939495564
0.8575 -0.104259379208088
0.8625 -0.104329757392406
0.8675 -0.104392692446709
0.8725 -0.104448787868023
0.8775 -0.104498654603958
0.8825 -0.104542873799801
0.8875 -0.104581996798515
0.8925 -0.104616537690163
0.8975 -0.104646988213062
0.9025 -0.104673787951469
0.9075 -0.104697339236736
0.9125 -0.104715034365654
0.9175 -0.104730442166328
0.9225 -0.104743972420692
0.9275 -0.104755841195583
0.9325 -0.104766242206097
0.9375 -0.104775361716747
0.9425 -0.104783371090889
0.9475 -0.104790382087231
0.9525 -0.104796521365643
0.9575 -0.104801908135414
0.9625 -0.104806609451771
0.9675 -0.104810610413551
0.9725 -0.104814104735851
0.9775 -0.104817137122154
0.9825 -0.104819752275944
0.9875 -0.104821972548962
0.9925 -0.104823797941208
0.9975 -0.104825116693974
};

\nextgroupplot[
minor xtick={},
minor ytick={},
tick align=outside,
tick pos=left,
x grid style={white!69.0196078431373!black},
xlabel={\(x\)},
xmin=-0.04725, xmax=1.04725,
xtick style={color=black},
xtick={0,1},
y grid style={white!69.0196078431373!black},
ylabel={\(z_4\)},
ymin=-0.331044857576489, ymax=-0.00906892381608486,
ytick style={color=black},
ytick={-0.316409587860107,-0.0237041935324669}
]
\addplot [thick, color0]
table {%
0.0025 -0.316409587860107
0.0075 -0.316409587860107
0.0125 -0.316409528255463
0.0175 -0.316409468650818
0.0225 -0.316409379243851
0.0275 -0.316409319639206
0.0325 -0.316409200429916
0.0375 -0.316409021615982
0.0425 -0.316408842802048
0.0475 -0.316408544778824
0.0525 -0.316408157348633
0.0575 -0.316407680511475
0.0625 -0.316407054662704
0.0675 -0.31640625
0.0725 -0.316405206918716
0.0775 -0.316403865814209
0.0825 -0.316402077674866
0.0875 -0.316399782896042
0.0925 -0.31639689207077
0.0975 -0.316393136978149
0.1025 -0.3163882791996
0.1075 -0.316382050514221
0.1125 -0.316374123096466
0.1175 -0.316364109516144
0.1225 -0.316351413726807
0.1275 -0.31633535027504
0.1325 -0.316315144300461
0.1375 -0.316289901733398
0.1425 -0.316258400678635
0.1475 -0.316219240427017
0.1525 -0.316170930862427
0.1575 -0.31611156463623
0.1625 -0.316038757562637
0.1675 -0.315950185060501
0.1725 -0.315842777490616
0.1775 -0.315713375806808
0.1825 -0.315558254718781
0.1875 -0.315373301506042
0.1925 -0.315153926610947
0.1975 -0.314895302057266
0.2025 -0.314592093229294
0.2075 -0.314238637685776
0.2125 -0.313828974962234
0.2175 -0.313356757164001
0.2225 -0.312815755605698
0.2275 -0.312199383974075
0.2325 -0.311500996351242
0.2375 -0.310714155435562
0.2425 -0.3098324239254
0.2475 -0.308800727128983
0.2525 -0.307342857122421
0.2575 -0.305739790201187
0.2625 -0.303987413644791
0.2675 -0.302083134651184
0.2725 -0.30002236366272
0.2775 -0.297803491353989
0.2825 -0.295425891876221
0.2875 -0.292890071868896
0.2925 -0.290197670459747
0.2975 -0.287351101636887
0.3025 -0.284353643655777
0.3075 -0.281209170818329
0.3125 -0.277922093868256
0.3175 -0.274496972560883
0.3225 -0.27093842625618
0.3275 -0.267250716686249
0.3325 -0.263437539339066
0.3375 -0.259501844644547
0.3425 -0.255445152521133
0.3475 -0.251267820596695
0.3525 -0.246968120336533
0.3575 -0.242542535066605
0.3625 -0.237985104322433
0.3675 -0.233287572860718
0.3725 -0.228438913822174
0.3775 -0.22342574596405
0.3825 -0.218231916427612
0.3875 -0.212838977575302
0.3925 -0.207226514816284
0.3975 -0.201372385025024
0.4025 -0.195108011364937
0.4075 -0.188418865203857
0.4125 -0.181371182203293
0.4175 -0.173945680260658
0.4225 -0.166130140423775
0.4275 -0.157921925187111
0.4325 -0.14933069050312
0.4375 -0.140380099415779
0.4425 -0.131108671426773
0.4475 -0.121569007635117
0.4525 -0.111827448010445
0.4575 -0.101964697241783
0.4625 -0.0920813381671906
0.4675 -0.0823096632957458
0.4725 -0.0728283151984215
0.4775 -0.0638599470257759
0.4825 -0.0556506663560867
0.4875 -0.0484002903103828
0.4925 -0.0422024801373482
0.4975 -0.0370302200317383
0.5025 -0.0327871814370155
0.5075 -0.0293795317411423
0.5125 -0.0267909914255142
0.5175 -0.0249151736497879
0.5225 -0.0238598436117172
0.5275 -0.0237041935324669
0.5325 -0.0245038494467735
0.5375 -0.0262849107384682
0.5425 -0.0290093719959259
0.5475 -0.0325828418135643
0.5525 -0.0368789508938789
0.5575 -0.0417655855417252
0.5625 -0.047121174633503
0.5675 -0.0528390184044838
0.5725 -0.0588222816586494
0.5775 -0.06497473269701
0.5825 -0.0711922347545624
0.5875 -0.0773582085967064
0.5925 -0.0833451598882675
0.5975 -0.0890217870473862
0.6025 -0.0942638963460922
0.6075 -0.0989661514759064
0.6125 -0.103051543235779
0.6175 -0.106400430202484
0.6225 -0.109047502279282
0.6275 -0.111055284738541
0.6325 -0.112462863326073
0.6375 -0.113323032855988
0.6425 -0.113695338368416
0.6475 -0.113632291555405
0.6525 -0.113177508115768
0.6575 -0.112412765622139
0.6625 -0.111384436488152
0.6675 -0.110132321715355
0.6725 -0.108689546585083
0.6775 -0.107083097100258
0.6825 -0.10533495247364
0.6875 -0.103463649749756
0.6925 -0.10148574411869
0.6975 -0.0994178205728531
0.7025 -0.0972779095172882
0.7075 -0.0950867682695389
0.7125 -0.0928683280944824
0.7175 -0.0906497985124588
0.7225 -0.0884609073400497
0.7275 -0.0863329917192459
0.7325 -0.084297314286232
0.7375 -0.0824276953935623
0.7425 -0.0807198286056519
0.7475 -0.0791851878166199
0.7525 -0.0778399705886841
0.7575 -0.0766939967870712
0.7625 -0.0757505670189857
0.7675 -0.075006440281868
0.7725 -0.0744523927569389
0.7775 -0.0740741044282913
0.7825 -0.0738574862480164
0.7875 -0.0737883821129799
0.7925 -0.0738291069865227
0.7975 -0.0739597827196121
0.8025 -0.074160099029541
0.8075 -0.0744116082787514
0.8125 -0.074698232114315
0.8175 -0.075006365776062
0.8225 -0.0753185972571373
0.8275 -0.0756302177906036
0.8325 -0.0759376212954521
0.8375 -0.0762356296181679
0.8425 -0.0765205398201942
0.8475 -0.0767895355820656
0.8525 -0.0770406648516655
0.8575 -0.0772736147046089
0.8625 -0.0774880945682526
0.8675 -0.077684223651886
0.8725 -0.0778625011444092
0.8775 -0.0780236572027206
0.8825 -0.078168585896492
0.8875 -0.0782983303070068
0.8925 -0.0784139931201935
0.8975 -0.0785167217254639
0.9025 -0.0786076337099075
0.9075 -0.0786878019571304
0.9125 -0.0787590146064758
0.9175 -0.0788214802742004
0.9225 -0.0788761526346207
0.9275 -0.0789239108562469
0.9325 -0.0789654701948166
0.9375 -0.0790016055107117
0.9425 -0.0790329724550247
0.9475 -0.0790601223707199
0.9525 -0.0790835916996002
0.9575 -0.0791038423776627
0.9625 -0.079121321439743
0.9675 -0.0791364312171936
0.9725 -0.0791494101285934
0.9775 -0.0791605114936829
0.9825 -0.0791700184345245
0.9875 -0.0791780799627304
0.9925 -0.0791847556829453
0.9975 -0.0791898220777512
};
\end{groupplot}

\draw ({$(current bounding box.south west)!0.5!(current bounding box.south east)$}|-{$(current bounding box.south west)!0.98!(current bounding box.north west)$}) node[
  scale=0.6,
  anchor=north,
  text=black,
  rotate=0.0
]{Code FCNN & CNN  rare};
\end{tikzpicture}

	\caption{Intrinsic variables \(r_0(x,t)\), \(r_1(x,t)\), \(r_2(x,t)\), \(r_3(x,t)\) and \(r_4(x,t)\) of \(\idrare\) obtained from the FCNN. Top row depicts \(\idrare\) over the whole \((x,t)\) domain, middle and bottom row for \(t=0.055\) and \(t=0.12\) respectively.}
	\label{Fig: Code_rare}
\end{figure}
\clearpage
The intrinsic variables extracted from the CNN and POD are shown in \cref{Fig: CNNPOD}. They are only dependent on \(v\) with\\
\begin{minipage}{0.45\textwidth}
	\begin{equation}
	[h_1(v),\dots,h_p(v)] = \idhy
	\end{equation}
\end{minipage}%
\begin{minipage}{0.45\textwidth}
	\begin{equation}
	\mathrm{and}\quad[r_n(v),\dots,r_p(v)] = \idrare
	\end{equation}
\end{minipage}.\\\\\
Third and fourth row of \cref{Fig: CNNPOD} show the first three and five POD modes of \(\hy\) and \(\rare\) respectively. As expected, the first three modes of both rarefaction levels do not differ significantly to the same extend as the bulk dynamic answer of the two gas flows to the test case doesn't. Pronounced differences of \(\idhy\) and \(\idrare\) extracted from the CNN, senn in the first and second row of \cref{Fig: CNNPOD}, are also not observed. This aligns with previous findings that the CNN fails to differentiate between \(\hy\) and \(\rare\). Qualitatively do the intrinsic variables of the CNN show less complexity compared to those of POD, nonetheless a rough analogy between the two can be observed. Especially \(h_0\)/\(r_0\), the first intrinsic variables, of the CNN resemble those of POD. Furthermore can a small through in \(h_1\)/\(r_1\) of the CNN be identified which is present, although much more pronounced, in \(h_1\)/\(r_1\) of POD. Worth mentioning is that \(h_0\) and \(r_0\), \(h_1\) and \(r_1\) as well as \(h_4\) and \(r_4\) of the CNN encompass the largest peaking of all intrinsic variables.\\    
\begin{figure}[hp!]
	% This file was created by tikzplotlib v0.9.8.
\begin{tikzpicture}

\begin{groupplot}[
group style={group size=5 by 4,
	horizontal sep=1.4cm,
	%vertical sep=1cm
},
x tick label style={/pgf/number format/fixed},
y tick label style={/pgf/number format/fixed},
width=0.2\textwidth,
height=0.2\textwidth,
y label style={yshift=-1cm},
x label style={yshift=.5cm}
]
\nextgroupplot[
minor xtick={},
minor ytick={},
tick align=outside,
tick pos=left,
x grid style={white!69.0196078431373!black},
xlabel={\(v\)},
xmin=-11, xmax=11,
xtick style={color=black},
xtick={-10,10},
y grid style={white!69.0196078431373!black},
ylabel={\(h_0\)},
ymin=-0.575381374359131, ymax=4.12658319473267,
ytick style={color=black},
ytick={-0.361655712127686,3.91285753250122}
]
\addplot [thick, green!50!black]
table {%
-10 -0.361655712127686
-9.48717948717949 -0.361655712127686
-8.97435897435897 -0.361655712127686
-8.46153846153846 -0.361655712127686
-7.94871794871795 -0.361655712127686
-7.43589743589744 -0.361655712127686
-6.92307692307692 -0.361655712127686
-6.41025641025641 -0.361655712127686
-5.8974358974359 -0.361655712127686
-5.38461538461538 -0.361655354499817
-4.87179487179487 -0.361649990081787
-4.35897435897436 -0.361594676971436
-3.84615384615385 -0.361153125762939
-3.33333333333333 -0.358438789844513
-2.82051282051282 -0.345366179943085
-2.30769230769231 -0.294715404510498
-1.79487179487179 -0.110655725002289
-1.28205128205128 0.669368267059326
-0.769230769230769 1.9442058801651
-0.256410256410256 3.06291913986206
0.256410256410256 3.91285753250122
0.769230769230769 3.64447259902954
1.28205128205128 1.7836229801178
1.79487179487179 0.227226570248604
2.30769230769231 -0.243734955787659
2.82051282051282 -0.338675200939178
3.33333333333333 -0.357649087905884
3.84615384615385 -0.361062586307526
4.35897435897436 -0.361584782600403
4.87179487179487 -0.361648976802826
5.38461538461538 -0.361655294895172
5.8974358974359 -0.361655712127686
6.41025641025641 -0.361655712127686
6.92307692307692 -0.361655712127686
7.43589743589744 -0.361655712127686
7.94871794871795 -0.361655712127686
8.46153846153846 -0.361655712127686
8.97435897435897 -0.361655712127686
9.48717948717949 -0.361655712127686
10 -0.361655652523041
};

\nextgroupplot[
minor xtick={},
minor ytick={},
tick align=outside,
tick pos=left,
x grid style={white!69.0196078431373!black},
xlabel={\(v\)},
xmin=-11, xmax=11,
xtick style={color=black},
xtick={-10,10},
y grid style={white!69.0196078431373!black},
ylabel={\(h_1\)},
ymin=-0.818328279256821, ymax=1.42098810076714,
ytick style={color=black},
ytick={-0.716541171073914,1.31920099258423}
]
\addplot [thick, green!50!black]
table {%
-10 -0.66711837053299
-9.48717948717949 -0.66711837053299
-8.97435897435897 -0.66711837053299
-8.46153846153846 -0.66711837053299
-7.94871794871795 -0.66711837053299
-7.43589743589744 -0.66711837053299
-6.92307692307692 -0.66711837053299
-6.41025641025641 -0.66711837053299
-5.8974358974359 -0.66711837053299
-5.38461538461538 -0.667118430137634
-4.87179487179487 -0.667119741439819
-4.35897435897436 -0.667132377624512
-3.84615384615385 -0.667231559753418
-3.33333333333333 -0.667818903923035
-2.82051282051282 -0.670394003391266
-2.30769230769231 -0.678385853767395
-1.79487179487179 -0.694712162017822
-1.28205128205128 -0.716541171073914
-0.769230769230769 -0.597107350826263
-0.256410256410256 0.311019390821457
0.256410256410256 1.31920099258423
0.769230769230769 1.15723311901093
1.28205128205128 0.195197239518166
1.79487179487179 -0.488594591617584
2.30769230769231 -0.631361305713654
2.82051282051282 -0.659785747528076
3.33333333333333 -0.665511786937714
3.84615384615385 -0.666788101196289
4.35897435897436 -0.667062401771545
4.87179487179487 -0.667110979557037
5.38461538461538 -0.667117595672607
5.8974358974359 -0.667118310928345
6.41025641025641 -0.66711837053299
6.92307692307692 -0.66711837053299
7.43589743589744 -0.66711837053299
7.94871794871795 -0.66711837053299
8.46153846153846 -0.66711837053299
8.97435897435897 -0.66711837053299
9.48717948717949 -0.66711837053299
10 -0.66711837053299
};

\nextgroupplot[
minor xtick={},
minor ytick={},
tick align=outside,
tick pos=left,
x grid style={white!69.0196078431373!black},
xlabel={\(v\)},
xmin=-11, xmax=11,
xtick style={color=black},
xtick={-10,10},
y grid style={white!69.0196078431373!black},
ylabel={\(h_2\)},
ymin=-0.892976216971874, ymax=0.302769921720028,
ytick style={color=black},
ytick={-0.838624119758606,0.24841782450676}
]
\addplot [thick, green!50!black]
table {%
-10 0.24841782450676
-9.48717948717949 0.24841782450676
-8.97435897435897 0.24841782450676
-8.46153846153846 0.24841782450676
-7.94871794871795 0.24841782450676
-7.43589743589744 0.24841782450676
-6.92307692307692 0.24841782450676
-6.41025641025641 0.24841782450676
-5.8974358974359 0.24841782450676
-5.38461538461538 0.24841719865799
-4.87179487179487 0.248409450054169
-4.35897435897436 0.24832671880722
-3.84615384615385 0.247652053833008
-3.33333333333333 0.243428766727448
-2.82051282051282 0.223156988620758
-2.30769230769231 0.150357842445374
-1.79487179487179 -0.0262710452079773
-1.28205128205128 -0.174107193946838
-0.769230769230769 -0.324159502983093
-0.256410256410256 -0.562870502471924
0.256410256410256 -0.772495567798615
0.769230769230769 -0.838624119758606
1.28205128205128 -0.762293994426727
1.79487179487179 -0.465265572071075
2.30769230769231 -0.0179085731506348
2.82051282051282 0.184623777866364
3.33333333333333 0.235428333282471
3.84615384615385 0.24618524312973
4.35897435897436 0.248103260993958
4.87179487179487 0.248382151126862
5.38461538461538 0.248414635658264
5.8974358974359 0.248417675495148
6.41025641025641 0.24841782450676
6.92307692307692 0.24841782450676
7.43589743589744 0.24841782450676
7.94871794871795 0.24841782450676
8.46153846153846 0.24841782450676
8.97435897435897 0.24841782450676
9.48717948717949 0.24841782450676
10 0.24841782450676
};

\nextgroupplot[
minor xtick={},
minor ytick={},
tick align=outside,
tick pos=left,
x grid style={white!69.0196078431373!black},
xlabel={\(v\)},
xmin=-11, xmax=11,
xtick style={color=black},
xtick={-10,10},
y grid style={white!69.0196078431373!black},
ylabel={\(h_3\)},
ymin=-1.06899447441101, ymax=0.706329298019409,
ytick style={color=black},
ytick={-0.988297939300537,0.625632762908936}
]
\addplot [thick, green!50!black]
table {%
-10 0.625632762908936
-9.48717948717949 0.625632762908936
-8.97435897435897 0.625632762908936
-8.46153846153846 0.625632762908936
-7.94871794871795 0.625632762908936
-7.43589743589744 0.625632762908936
-6.92307692307692 0.625632762908936
-6.41025641025641 0.625632762908936
-5.8974358974359 0.625632703304291
-5.38461538461538 0.625629782676697
-4.87179487179487 0.625591039657593
-4.35897435897436 0.625186562538147
-3.84615384615385 0.621958613395691
-3.33333333333333 0.602309584617615
-2.82051282051282 0.51115620136261
-2.30769230769231 0.191796913743019
-1.79487179487179 -0.465624570846558
-1.28205128205128 -0.866331577301025
-0.769230769230769 -0.971776247024536
-0.256410256410256 -0.988297939300537
0.256410256410256 -0.9827561378479
0.769230769230769 -0.936582863330841
1.28205128205128 -0.737656593322754
1.79487179487179 -0.286003947257996
2.30769230769231 0.244262918829918
2.82051282051282 0.514772534370422
3.33333333333333 0.601754248142242
3.84615384615385 0.621721804141998
4.35897435897436 0.625139594078064
4.87179487179487 0.625584602355957
5.38461538461538 0.625629186630249
5.8974358974359 0.625632643699646
6.41025641025641 0.625632762908936
6.92307692307692 0.625632762908936
7.43589743589744 0.625632762908936
7.94871794871795 0.625632762908936
8.46153846153846 0.625632762908936
8.97435897435897 0.625632762908936
9.48717948717949 0.625632762908936
10 0.625632762908936
};

\nextgroupplot[
minor xtick={},
minor ytick={},
tick align=outside,
tick pos=left,
x grid style={white!69.0196078431373!black},
xlabel={\(v\)},
xmin=-11, xmax=11,
xtick style={color=black},
xtick={-10,10},
y grid style={white!69.0196078431373!black},
ylabel={\(h_4\)},
ymin=-0.778421953320503, ymax=4.2126619964838,
ytick style={color=black},
ytick={-0.551554501056671,3.98579454421997}
]
\addplot [thick, green!50!black]
table {%
-10 -0.530550360679626
-9.48717948717949 -0.530550360679626
-8.97435897435897 -0.530550360679626
-8.46153846153846 -0.530550360679626
-7.94871794871795 -0.530550360679626
-7.43589743589744 -0.530550360679626
-6.92307692307692 -0.530550360679626
-6.41025641025641 -0.530550360679626
-5.8974358974359 -0.530550360679626
-5.38461538461538 -0.530549824237823
-4.87179487179487 -0.53054279088974
-4.35897435897436 -0.530469238758087
-3.84615384615385 -0.52988076210022
-3.33333333333333 -0.526262581348419
-2.82051282051282 -0.508833527565002
-2.30769230769231 -0.440384864807129
-1.79487179487179 -0.178807854652405
-1.28205128205128 0.902162551879883
-0.769230769230769 2.7100191116333
-0.256410256410256 3.98579454421997
0.256410256410256 2.97932028770447
0.769230769230769 0.661177635192871
1.28205128205128 -0.47057968378067
1.79487179487179 -0.551554501056671
2.30769230769231 -0.507959425449371
2.82051282051282 -0.518422961235046
3.33333333333333 -0.527566909790039
3.84615384615385 -0.530053377151489
4.35897435897436 -0.530489087104797
4.87179487179487 -0.530544579029083
5.38461538461538 -0.530549883842468
5.8974358974359 -0.530550241470337
6.41025641025641 -0.530550360679626
6.92307692307692 -0.530550360679626
7.43589743589744 -0.530550360679626
7.94871794871795 -0.530550360679626
8.46153846153846 -0.530550360679626
8.97435897435897 -0.530550360679626
9.48717948717949 -0.530550360679626
10 -0.530550360679626
};

\nextgroupplot[
minor xtick={},
minor ytick={},
tick align=outside,
tick pos=left,
x grid style={white!69.0196078431373!black},
xlabel={\(v\)},
xmin=-11, xmax=11,
xtick style={color=black},
xtick={-10,10},
y grid style={white!69.0196078431373!black},
ylabel={\(r_0\)},
ymin=-0.571784174442291, ymax=4.05104199647903,
ytick style={color=black},
ytick={-0.361655712127686,3.84091353416443}
]
\addplot [thick, green!50!black]
table {%
-10 -0.361655712127686
-9.48717948717949 -0.361655712127686
-8.97435897435897 -0.361655712127686
-8.46153846153846 -0.361655712127686
-7.94871794871795 -0.361655712127686
-7.43589743589744 -0.361655712127686
-6.92307692307692 -0.361655712127686
-6.41025641025641 -0.361655712127686
-5.8974358974359 -0.361655712127686
-5.38461538461538 -0.361655354499817
-4.87179487179487 -0.361650228500366
-4.35897435897436 -0.361595988273621
-3.84615384615385 -0.361161530017853
-3.33333333333333 -0.35849142074585
-2.82051282051282 -0.345670104026794
-2.30769230769231 -0.296083450317383
-1.79487179487179 -0.116072416305542
-1.28205128205128 0.66789448261261
-0.769230769230769 2.00346708297729
-0.256410256410256 3.15658783912659
0.256410256410256 3.84091353416443
0.769230769230769 3.46383094787598
1.28205128205128 1.70916211605072
1.79487179487179 0.238550215959549
2.30769230769231 -0.232158601284027
2.82051282051282 -0.335449814796448
3.33333333333333 -0.357114434242249
3.84615384615385 -0.36100959777832
4.35897435897436 -0.36158275604248
4.87179487179487 -0.361649334430695
5.38461538461538 -0.361655354499817
5.8974358974359 -0.361655712127686
6.41025641025641 -0.361655712127686
6.92307692307692 -0.361655712127686
7.43589743589744 -0.361655712127686
7.94871794871795 -0.361655712127686
8.46153846153846 -0.361655712127686
8.97435897435897 -0.361655712127686
9.48717948717949 -0.361655712127686
10 -0.361655652523041
};

\nextgroupplot[
minor xtick={},
minor ytick={},
tick align=outside,
tick pos=left,
x grid style={white!69.0196078431373!black},
xlabel={\(v\)},
xmin=-11, xmax=11,
xtick style={color=black},
xtick={-10,10},
y grid style={white!69.0196078431373!black},
ylabel={\(r_1\)},
ymin=-0.820866966247559, ymax=1.4500020980835,
ytick style={color=black},
ytick={-0.717645645141602,1.34678077697754}
]
\addplot [thick, green!50!black]
table {%
-10 -0.66711837053299
-9.48717948717949 -0.66711837053299
-8.97435897435897 -0.66711837053299
-8.46153846153846 -0.66711837053299
-7.94871794871795 -0.66711837053299
-7.43589743589744 -0.66711837053299
-6.92307692307692 -0.66711837053299
-6.41025641025641 -0.66711837053299
-5.8974358974359 -0.66711837053299
-5.38461538461538 -0.667118430137634
-4.87179487179487 -0.66711950302124
-4.35897435897436 -0.667130947113037
-3.84615384615385 -0.667224168777466
-3.33333333333333 -0.667801022529602
-2.82051282051282 -0.670454323291779
-2.30769230769231 -0.679084658622742
-1.79487179487179 -0.697253465652466
-1.28205128205128 -0.717645645141602
-0.769230769230769 -0.557227253913879
-0.256410256410256 0.511874675750732
0.256410256410256 1.34678077697754
0.769230769230769 0.961284756660461
1.28205128205128 0.0930668264627457
1.79487179487179 -0.468465924263
2.30769230769231 -0.619952797889709
2.82051282051282 -0.657537937164307
3.33333333333333 -0.665462613105774
3.84615384615385 -0.666874885559082
4.35897435897436 -0.667088985443115
4.87179487179487 -0.66711562871933
5.38461538461538 -0.6671182513237
5.8974358974359 -0.66711837053299
6.41025641025641 -0.66711837053299
6.92307692307692 -0.66711837053299
7.43589743589744 -0.66711837053299
7.94871794871795 -0.66711837053299
8.46153846153846 -0.66711837053299
8.97435897435897 -0.66711837053299
9.48717948717949 -0.66711837053299
10 -0.66711837053299
};

\nextgroupplot[
minor xtick={},
minor ytick={},
tick align=outside,
tick pos=left,
x grid style={white!69.0196078431373!black},
xlabel={\(v\)},
xmin=-11, xmax=11,
xtick style={color=black},
xtick={-10,10},
y grid style={white!69.0196078431373!black},
ylabel={\(r_2\)},
ymin=-0.868009044229984, ymax=0.301581008732319,
ytick style={color=black},
ytick={-0.814845860004425,0.24841782450676}
]
\addplot [thick, green!50!black]
table {%
-10 0.24841782450676
-9.48717948717949 0.24841782450676
-8.97435897435897 0.24841782450676
-8.46153846153846 0.24841782450676
-7.94871794871795 0.24841782450676
-7.43589743589744 0.24841782450676
-6.92307692307692 0.24841782450676
-6.41025641025641 0.24841782450676
-5.8974358974359 0.24841782450676
-5.38461538461538 0.248417317867279
-4.87179487179487 0.24841046333313
-4.35897435897436 0.248337805271149
-3.84615384615385 0.247739344835281
-3.33333333333333 0.243947863578796
-2.82051282051282 0.225460469722748
-2.30769230769231 0.157747358083725
-1.79487179487179 -0.0110852122306824
-1.28205128205128 -0.16901171207428
-0.769230769230769 -0.369703888893127
-0.256410256410256 -0.604797184467316
0.256410256410256 -0.762204885482788
0.769230769230769 -0.814845860004425
1.28205128205128 -0.752028405666351
1.79487179487179 -0.489206314086914
2.30769230769231 -0.0461685061454773
2.82051282051282 0.180341571569443
3.33333333333333 0.236210316419601
3.84615384615385 0.246669173240662
4.35897435897436 0.248219013214111
4.87179487179487 0.248400092124939
5.38461538461538 0.248416602611542
5.8974358974359 0.24841782450676
6.41025641025641 0.24841782450676
6.92307692307692 0.24841782450676
7.43589743589744 0.24841782450676
7.94871794871795 0.24841782450676
8.46153846153846 0.24841782450676
8.97435897435897 0.24841782450676
9.48717948717949 0.24841782450676
10 0.24841782450676
};

\nextgroupplot[
minor xtick={},
minor ytick={},
tick align=outside,
tick pos=left,
x grid style={white!69.0196078431373!black},
xlabel={\(v\)},
xmin=-11, xmax=11,
xtick style={color=black},
xtick={-10,10},
y grid style={white!69.0196078431373!black},
ylabel={\(r_3\)},
ymin=-1.06886911690235, ymax=0.706323328614235,
ytick style={color=black},
ytick={-0.988178551197052,0.625632762908936}
]
\addplot [thick, green!50!black]
table {%
-10 0.625632762908936
-9.48717948717949 0.625632762908936
-8.97435897435897 0.625632762908936
-8.46153846153846 0.625632762908936
-7.94871794871795 0.625632762908936
-7.43589743589744 0.625632762908936
-6.92307692307692 0.625632762908936
-6.41025641025641 0.625632762908936
-5.8974358974359 0.625632703304291
-5.38461538461538 0.625630021095276
-4.87179487179487 0.625594615936279
-4.35897435897436 0.625216960906982
-3.84615384615385 0.622152805328369
-3.33333333333333 0.603219509124756
-2.82051282051282 0.514187812805176
-2.30769230769231 0.198670998215675
-1.79487179487179 -0.460103809833527
-1.28205128205128 -0.864103972911835
-0.769230769230769 -0.970924854278564
-0.256410256410256 -0.988178551197052
0.256410256410256 -0.983330070972443
0.769230769230769 -0.938991963863373
1.28205128205128 -0.7430779337883
1.79487179487179 -0.277949690818787
2.30769230769231 0.257076323032379
2.82051282051282 0.51932555437088
3.33333333333333 0.602917790412903
3.84615384615385 0.621969938278198
4.35897435897436 0.625182688236237
4.87179487179487 0.625590443611145
5.38461538461538 0.625629782676697
5.8974358974359 0.625632703304291
6.41025641025641 0.625632762908936
6.92307692307692 0.625632762908936
7.43589743589744 0.625632762908936
7.94871794871795 0.625632762908936
8.46153846153846 0.625632762908936
8.97435897435897 0.625632762908936
9.48717948717949 0.625632762908936
10 0.625632762908936
};

\nextgroupplot[
minor xtick={},
minor ytick={},
tick align=outside,
tick pos=left,
x grid style={white!69.0196078431373!black},
xlabel={\(v\)},
xmin=-11, xmax=11,
xtick style={color=black},
xtick={-10,10},
y grid style={white!69.0196078431373!black},
ylabel={\(r_4\)},
ymin=-0.798403313755989, ymax=4.17039846479893,
ytick style={color=black},
ytick={-0.572548687458038,3.94454383850098}
]
\addplot [thick, green!50!black]
table {%
-10 -0.530550360679626
-9.48717948717949 -0.530550360679626
-8.97435897435897 -0.530550360679626
-8.46153846153846 -0.530550360679626
-7.94871794871795 -0.530550360679626
-7.43589743589744 -0.530550360679626
-6.92307692307692 -0.530550360679626
-6.41025641025641 -0.530550360679626
-5.8974358974359 -0.530550360679626
-5.38461538461538 -0.530549883842468
-4.87179487179487 -0.530543625354767
-4.35897435897436 -0.530476450920105
-3.84615384615385 -0.529928088188171
-3.33333333333333 -0.526492834091187
-2.82051282051282 -0.509659171104431
-2.30769230769231 -0.442609369754791
-1.79487179487179 -0.184795916080475
-1.28205128205128 0.883229076862335
-0.769230769230769 2.65313100814819
-0.256410256410256 3.94454383850098
0.256410256410256 3.09617042541504
0.769230769230769 0.821390807628632
1.28205128205128 -0.450838983058929
1.79487179487179 -0.572548687458038
2.30769230769231 -0.517163991928101
2.82051282051282 -0.517864048480988
3.33333333333333 -0.526701331138611
3.84615384615385 -0.529811441898346
4.35897435897436 -0.530449450016022
4.87179487179487 -0.530540108680725
5.38461538461538 -0.5305495262146
5.8974358974359 -0.530550360679626
6.41025641025641 -0.530550360679626
6.92307692307692 -0.530550360679626
7.43589743589744 -0.530550360679626
7.94871794871795 -0.530550360679626
8.46153846153846 -0.530550360679626
8.97435897435897 -0.530550360679626
9.48717948717949 -0.530550360679626
10 -0.530550360679626
};

\nextgroupplot[
minor xtick={},
minor ytick={},
tick align=outside,
tick pos=left,
x grid style={white!69.0196078431373!black},
xlabel={\(v\)},
xmin=-11, xmax=11,
xtick style={color=black},
xtick={-10,10},
y grid style={white!69.0196078431373!black},
ylabel={\(h_0\)},
ymin=-0.568114146636858, ymax=0.0270530546017552,
ytick style={color=black},
ytick={-0.541061092035103,1.11022302462516e-16}
]
\addplot [thick, red]
table {%
-10 1.11022302462516e-16
-9.48717948717949 -1.49756037906857e-20
-8.97435897435897 -1.27810118294275e-18
-8.46153846153846 -1.12744683074863e-16
-7.94871794871795 -7.61031278983007e-15
-7.43589743589744 -3.95050479625332e-13
-6.92307692307692 -1.57714583487996e-11
-6.41025641025641 -4.84275102585206e-10
-5.8974358974359 -1.14380972444433e-08
-5.38461538461538 -2.07830187863026e-07
-4.87179487179487 -2.90552413910108e-06
-4.35897435897436 -3.12601969992907e-05
-3.84615384615385 -0.000258900668414548
-3.33333333333333 -0.00165127621583846
-2.82051282051282 -0.00811512031118912
-2.30769230769231 -0.0307554404646581
-1.79487179487179 -0.0900252664219377
-1.28205128205128 -0.204408297707841
-0.769230769230769 -0.363493236763491
-0.256410256410256 -0.506703365379294
0.256410256410256 -0.541061092035103
0.769230769230769 -0.433340373945296
1.28205128205128 -0.256535943244692
1.79487179487179 -0.111602690436555
2.30769230769231 -0.0365475607273425
2.82051282051282 -0.00929664348004875
3.33333333333333 -0.00185629538558574
3.84615384615385 -0.000289820815853585
4.35897435897436 -3.5225241343174e-05
4.87179487179487 -3.32774351739608e-06
5.38461538461538 -2.4444710128489e-07
5.8974358974359 -1.39837998972549e-08
6.41025641025641 -6.24428204040681e-10
6.92307692307692 -2.18275076466274e-11
7.43589743589744 -5.9915325787844e-13
7.94871794871795 -1.2952475837196e-14
8.46153846153846 -2.21019702020631e-16
8.97435897435897 -2.9804272176168e-18
9.48717948717949 -3.1755125641467e-20
10 -2.66916408904971e-22
};

\nextgroupplot[
minor xtick={},
minor ytick={},
tick align=outside,
tick pos=left,
x grid style={white!69.0196078431373!black},
xlabel={\(v\)},
xmin=-11, xmax=11,
xtick style={color=black},
xtick={-10,10},
y grid style={white!69.0196078431373!black},
ylabel={\(h_1\)},
ymin=-0.534209545581082, ymax=0.600709312993124,
ytick style={color=black},
ytick={-0.4826223247368,0.549122092148842}
]
\addplot [thick, red]
table {%
-10 -6.52256026967279e-16
-9.48717948717949 -1.11023299841881e-16
-8.97435897435897 -6.00279985198224e-18
-8.46153846153846 -2.43089976887143e-16
-7.94871794871795 -1.63779640628311e-14
-7.43589743589744 -8.49092822076895e-13
-6.92307692307692 -3.38441280150102e-11
-6.41025641025641 -1.03712137046538e-09
-5.8974358974359 -2.44324195657029e-08
-5.38461538461538 -4.42430689096351e-07
-4.87179487179487 -6.15715736041037e-06
-4.35897435897436 -6.58295067224784e-05
-3.84615384615385 -0.000540376231108169
-3.33333333333333 -0.00340181835101554
-2.82051282051282 -0.0163877027864573
-2.30769230769231 -0.0601457264406097
-1.79487179487179 -0.166593776856988
-1.28205128205128 -0.340778444520313
-0.769230769230769 -0.4826223247368
-0.256410256410256 -0.350573153470886
0.256410256410256 0.16305672323786
0.769230769230769 0.549122092148842
1.28205128205128 0.395734276311465
1.79487179487179 0.107797605642259
2.30769230769231 0.00113955872203914
2.82051282051282 -0.00628828145589854
3.33333333333333 -0.00183037661651124
3.84615384615385 -0.000293980239536051
4.35897435897436 -3.06998183837196e-05
4.87179487179487 -1.99051118798304e-06
5.38461538461538 -4.83099134786566e-08
5.8974358974359 4.76124359665473e-09
6.41025641025641 6.42627485384192e-10
6.92307692307692 4.09436697221996e-11
7.43589743589744 1.72243753098337e-12
7.94871794871795 5.18312293205685e-14
8.46153846153846 1.15181996987467e-15
8.97435897435897 1.92044935175516e-17
9.48717948717949 2.42388524543346e-19
10 2.32839870106148e-21
};

\nextgroupplot[
minor xtick={},
minor ytick={},
tick align=outside,
tick pos=left,
x grid style={white!69.0196078431373!black},
xlabel={\(v\)},
xmin=-11, xmax=11,
xtick style={color=black},
xtick={-10,10},
y grid style={white!69.0196078431373!black},
ylabel={\(h_2\)},
ymin=-0.502335108871341, ymax=0.505441492195811,
ytick style={color=black},
ytick={-0.456527081550107,0.459633464874577}
]
\addplot [thick, red]
table {%
-10 -1.55431223447522e-15
-9.48717948717949 1.11022009439296e-16
-8.97435897435897 1.09213069552502e-16
-8.46153846153846 -4.11302163870102e-16
-7.94871794871795 -2.73855527116848e-14
-7.43589743589744 -1.41767600983015e-12
-6.92307692307692 -5.64079853813417e-11
-6.41025641025641 -1.724830854549e-09
-5.8974358974359 -4.05240621292349e-08
-5.38461538461538 -7.31333772813112e-07
-4.87179487179487 -1.01335627152193e-05
-4.35897435897436 -0.000107733076686539
-3.84615384615385 -0.000877781684076761
-3.33333333333333 -0.00547097471652983
-2.82051282051282 -0.0260021653540801
-2.30769230769231 -0.0936925974540094
-1.79487179487179 -0.251576815346058
-1.28205128205128 -0.456527081550107
-0.769230769230769 -0.317891765968506
-0.256410256410256 0.401768320918204
0.256410256410256 0.459633464874577
0.769230769230769 -0.213550725589105
1.28205128205128 -0.380666111280626
1.79487179487179 -0.215754658190286
2.30769230769231 -0.0852674972409338
2.82051282051282 -0.0264365709515214
3.33333333333333 -0.00630008262332984
3.84615384615385 -0.00113692600891485
4.35897435897436 -0.00015652926761505
4.87179487179487 -1.66465605259822e-05
5.38461538461538 -1.38045539282732e-06
5.8974358974359 -8.97854083246136e-08
6.41025641025641 -4.59329613409237e-09
6.92307692307692 -1.84947282959922e-10
7.43589743589744 -5.85480723152104e-12
7.94871794871795 -1.45400512715473e-13
8.46153846153846 -2.82487700068086e-15
8.97435897435897 -4.28096738270601e-17
9.48717948717949 -5.04652899526609e-19
10 -4.61634328769642e-21
};

\nextgroupplot[group/empty plot]

\nextgroupplot[group/empty plot]


\nextgroupplot[
minor xtick={},
minor ytick={},
tick align=outside,
tick pos=left,
x grid style={white!69.0196078431373!black},
xlabel={\(v\)},
xmin=-11, xmax=11,
xtick style={color=black},
xtick={-10,10},
y grid style={white!69.0196078431373!black},
ylabel={\(r_0\)},
ymin=-0.568489815679664, ymax=0.0270709436037935,
ytick style={color=black},
ytick={-0.54141887207587,0}
]
\addplot [thick, red]
table {%
-10 0
-9.48717948717949 -1.35474782018883e-20
-8.97435897435897 -9.90271652612958e-19
-8.46153846153846 -8.92501919419628e-17
-7.94871794871795 -6.13123795399933e-15
-7.43589743589744 -3.23890285902636e-13
-6.92307692307692 -1.31573449493248e-11
-6.41025641025641 -4.1103201448489e-10
-5.8974358974359 -9.87520176004384e-09
-5.38461538461538 -1.82481350067199e-07
-4.87179487179487 -2.59388614078416e-06
-4.35897435897436 -2.83678330767366e-05
-3.84615384615385 -0.00023875982336884
-3.33333333333333 -0.00154710573534998
-2.82051282051282 -0.00772206569570178
-2.30769230769231 -0.0297128044659424
-1.79487179487179 -0.0882688920176303
-1.28205128205128 -0.203366224923946
-0.769230769230769 -0.366429605310788
-0.256410256410256 -0.512939617362043
0.256410256410256 -0.54141887207587
0.769230769230769 -0.426446045857636
1.28205128205128 -0.252073919585735
1.79487179487179 -0.111760694088678
2.30769230769231 -0.037521169685504
2.82051282051282 -0.00969933524075994
3.33333333333333 -0.00194793173252855
3.84615384615385 -0.000303861356000786
4.35897435897436 -3.66940917363907e-05
4.87179487179487 -3.42083998438241e-06
5.38461538461538 -2.45808543581406e-07
5.8974358974359 -1.36022059621898e-08
6.41025641025641 -5.79342505107048e-10
6.92307692307692 -1.89852903301188e-11
7.43589743589744 -4.78567475195194e-13
7.94871794871795 -9.27751003490909e-15
8.46153846153846 -1.38299554270409e-16
8.97435897435897 -1.58512970257808e-18
9.48717948717949 -1.3967807303635e-20
10 -9.4620347185224e-23
};

\nextgroupplot[
minor xtick={},
minor ytick={},
tick align=outside,
tick pos=left,
x grid style={white!69.0196078431373!black},
xlabel={\(v\)},
xmin=-11, xmax=11,
xtick style={color=black},
xtick={-10,10},
y grid style={white!69.0196078431373!black},
ylabel={\(r_1\)},
ymin=-0.54532168320864, ymax=0.571892898016068,
ytick style={color=black},
ytick={-0.494539202243881,0.521110417051309}
]
\addplot [thick, red]
table {%
-10 -4.16333634234434e-17
-9.48717948717949 -8.09916070616939e-22
-8.97435897435897 -8.390769320896e-18
-8.46153846153846 -2.14090842255578e-16
-7.94871794871795 -1.48308606537555e-14
-7.43589743589744 -7.83032926451485e-13
-6.92307692307692 -3.1785640400241e-11
-6.41025641025641 -9.91975037034677e-10
-5.8974358974359 -2.37989915028907e-08
-5.38461538461538 -4.38887934179193e-07
-4.87179487179487 -6.22009714402372e-06
-4.35897435897436 -6.77214817434677e-05
-3.84615384615385 -0.000566022270992918
-3.33333333333333 -0.00362675699763037
-2.82051282051282 -0.017765291951657
-2.30769230769231 -0.0661334050964958
-1.79487179487179 -0.184513167351946
-1.28205128205128 -0.372227607149716
-0.769230769230769 -0.494539202243881
-0.256410256410256 -0.313280694847605
0.256410256410256 0.181261452178741
0.769230769230769 0.521110417051309
1.28205128205128 0.395921892017147
1.79487179487179 0.136019356899409
2.30769230769231 0.0176012620713995
2.82051282051282 -0.00242110635155434
3.33333333333333 -0.00142198161140801
3.84615384615385 -0.00030200303193007
4.35897435897436 -4.14338453814823e-05
4.87179487179487 -4.09615698250206e-06
5.38461538461538 -3.02274074506188e-07
5.8974358974359 -1.688325723549e-08
6.41025641025641 -7.18307613734838e-10
6.92307692307692 -2.33561849387153e-11
7.43589743589744 -5.81497256308826e-13
7.94871794871795 -1.10979414819641e-14
8.46153846153846 -1.62480087729132e-16
8.97435897435897 -1.8256921932754e-18
9.48717948717949 -1.57494360584451e-20
10 -1.04330277404064e-22
};

\nextgroupplot[
minor xtick={},
minor ytick={},
tick align=outside,
tick pos=left,
x grid style={white!69.0196078431373!black},
xlabel={\(v\)},
xmin=-11, xmax=11,
xtick style={color=black},
xtick={-10,10},
y grid style={white!69.0196078431373!black},
ylabel={\(r_2\)},
ymin=-0.476965095817056, ymax=0.509856746864573,
ytick style={color=black},
ytick={-0.432109557513345,0.465001208560863}
]
\addplot [thick, red]
table {%
-10 4.44089209850063e-16
-9.48717948717949 5.55108426406762e-17
-8.97435897435897 2.18788122980875e-16
-8.46153846153846 -3.26692353403019e-16
-7.94871794871795 -2.17120975612721e-14
-7.43589743589744 -1.14543566128252e-12
-6.92307692307692 -4.64533108543993e-11
-6.41025641025641 -1.44794132939433e-09
-5.8974358974359 -3.46807956301729e-08
-5.38461538461538 -6.38115456572969e-07
-4.87179487179487 -9.01505193960147e-06
-4.35897435897436 -9.77096593180638e-05
-3.84615384615385 -0.000811307444058508
-3.33333333333333 -0.00514759260951107
-2.82051282051282 -0.024840920111239
-2.30769230769231 -0.0903600879356342
-1.79487179487179 -0.242143882580277
-1.28205128205128 -0.432109557513345
-0.769230769230769 -0.293339277076144
-0.256410256410256 0.389979322211522
0.256410256410256 0.465001208560863
0.769230769230769 -0.224702263819344
1.28205128205128 -0.428988444189607
1.79487179487179 -0.228104372874858
2.30769230769231 -0.0687293124772077
2.82051282051282 -0.0140796795474724
3.33333333333333 -0.00217712007813332
3.84615384615385 -0.000259227027058239
4.35897435897436 -2.28213743632893e-05
4.87179487179487 -1.39406510555431e-06
5.38461538461538 -5.02244288248085e-08
5.8974358974359 -1.37861138096342e-10
6.41025641025641 1.01400387030769e-10
6.92307692307692 6.65102271994134e-12
7.43589743589744 2.46359492935696e-13
7.94871794871795 6.19282948874081e-15
8.46153846153846 1.1171164898547e-16
8.97435897435897 1.48201353068632e-18
9.48717948717949 1.46487032866199e-20
10 1.08721363805544e-22
};

\nextgroupplot[
minor xtick={},
minor ytick={},
tick align=outside,
tick pos=left,
x grid style={white!69.0196078431373!black},
xlabel={\(v\)},
xmin=-11, xmax=11,
xtick style={color=black},
xtick={-10,10},
y grid style={white!69.0196078431373!black},
ylabel={\(r_3\)},
ymin=-0.458475523716077, ymax=0.605638774403177,
ytick style={color=black},
ytick={-0.410106691983383,0.557269942670484}
]
\addplot [thick, red]
table {%
-10 -5.55111512312578e-17
-9.48717948717949 2.77555710831279e-16
-8.97435897435897 -4.16814592554486e-16
-8.46153846153846 -5.24406233845294e-16
-7.94871794871795 -2.31846010822523e-14
-7.43589743589744 -1.24543766410088e-12
-6.92307692307692 -5.04051023091615e-11
-6.41025641025641 -1.56675830008607e-09
-5.8974358974359 -3.73829731975076e-08
-5.38461538461538 -6.8412123391717e-07
-4.87179487179487 -9.58974108758543e-06
-4.35897435897436 -0.000102742401400365
-3.84615384615385 -0.000838218497067562
-3.33333333333333 -0.00517409725986912
-2.82051282051282 -0.0238886758731109
-2.30769230769231 -0.0807686073321761
-1.79487179487179 -0.190953504634852
-1.28205128205128 -0.263757631863766
-0.769230769230769 -0.00244100552988856
-0.256410256410256 0.382315411567851
0.256410256410256 -0.215018610607292
0.769230769230769 -0.410106691983383
1.28205128205128 0.38382662805783
1.79487179487179 0.557269942670484
2.30769230769231 0.253360080078215
2.82051282051282 0.0637936604354777
3.33333333333333 0.0109409395745161
3.84615384615385 0.00143441092448841
4.35897435897436 0.000150638483105252
4.87179487179487 1.27316000981608e-05
5.38461538461538 8.58652711991343e-07
5.8974358974359 4.57775837614438e-08
6.41025641025641 1.91403302558729e-09
6.92307692307692 6.23723463569772e-11
7.43589743589744 1.57655781924584e-12
7.94871794871795 3.08008053826984e-14
8.46153846153846 4.63900539262411e-16
8.97435897435897 5.37643272450758e-18
9.48717948717949 4.78859085839096e-20
10 3.27480488585219e-22
};

\nextgroupplot[
minor xtick={},
minor ytick={},
tick align=outside,
tick pos=left,
x grid style={white!69.0196078431373!black},
xlabel={\(v\)},
xmin=-11, xmax=11,
xtick style={color=black},
xtick={-10,10},
y grid style={white!69.0196078431373!black},
ylabel={\(r_4\)},
ymin=-0.543699521933286, ymax=0.475884749016531,
ytick style={color=black},
ytick={-0.497354782344658,0.429540009427903}
]
\addplot [thick, red]
table {%
-10 1.52655665885959e-15
-9.48717948717949 -1.94289074803414e-16
-8.97435897435897 2.49317093868918e-16
-8.46153846153846 -7.43164901300426e-16
-7.94871794871795 -1.25699829137171e-13
-7.43589743589744 -6.50034626552667e-12
-6.92307692307692 -2.61979665044585e-10
-6.41025641025641 -8.09786637899976e-09
-5.8974358974359 -1.91770336817243e-07
-5.38461538461538 -3.47348779759653e-06
-4.87179487179487 -4.79904550114141e-05
-4.35897435897436 -0.000503528159057139
-3.84615384615385 -0.00398174937196514
-3.33333333333333 -0.0234080926181439
-2.82051282051282 -0.0996414630189462
-2.30769230769231 -0.290127708528144
-1.79487179487179 -0.497354782344658
-1.28205128205128 -0.250382026650705
-0.769230769230769 0.429540009427903
-0.256410256410256 0.141394879769582
0.256410256410256 -0.399214720393455
0.769230769230769 0.280127575271265
1.28205128205128 0.0582375070074312
1.79487179487179 -0.311400262760249
2.30769230769231 -0.216499954851486
2.82051282051282 -0.0735561111136329
3.33333333333333 -0.0169239399072133
3.84615384615385 -0.00291384143782801
4.35897435897436 -0.000382366571877707
4.87179487179487 -3.81761927290871e-05
5.38461538461538 -2.8981499062208e-06
5.8974358974359 -1.67529968361744e-07
6.41025641025641 -7.3872776536039e-09
6.92307692307692 -2.48855917793557e-10
7.43589743589744 -6.41165875432445e-12
7.94871794871795 -1.26444800041156e-13
8.46153846153846 -1.90979927767441e-15
8.97435897435897 -2.21008718822092e-17
9.48717948717949 -1.9601981286969e-19
10 -1.33280429881029e-21
};
\end{groupplot}

\end{tikzpicture}

	\caption{Intrinsic variables \(h_0(v)\), \(h_1(v)\), \(h_2(v)\), \(h_3(v)\) and \(h_4(v)\) of \(\idhy\) in the top row and \(r_0(v)\), \(r_1(v)\), \(r_2(v)\), \(r_3(v)\) and \(r_4(v)\) of \(\idrare\) of in the second row extracted from the CNN. Third and fourth row show \(h_0(v)\), \(h_1(v)\), \(h_2(v)\) of \(\idhy\) and \(r_0(v)\), \(r_1(v)\) , \(r_2(v)\), \(r_3(v)\) and \(r_4(v)\) of \(\idrare\) extracted from POD respectively.}
	\label{Fig: CNNPOD}
\end{figure}

With POD one usually exploits the intrinsic variables within a Galerkin framework as in \cite{Bernard} to produce new states. The same can be done with the intrinsic variables obtained from autoencoders as in \cite{Carlberg}. Both won't be discussed in this contribution. Rather new states are obtained by performing an interpolation in time \(t\) of \(\idhy\) and \(\idrare\). This approach tests a different kind of ability to generalize about the FOM solution. So far there have been ested two kinds of generalization: The first can be encountered during training, where a split into train- and validation set probes the ability to fit unseen examples. A second approach into generalization is tried with the CNN, where both rarefaction levels as training examples probes the ability to fit both in the same model. The third insight into the ability to generalize is tested using the FCNN. By the temporal interpolation in the intrinsic variables it is probed if new states can be generated that meet the condition of MOR that the distance \(||f - \tilde{f}||\) is small.\\
For this thesis an additional solution \(\hy^*\) of the BGK model in Sod's shock tube with \(\Kn=0.00001\) is provided with a temporal resolution of 241 snapshots. This resolution is 9.64 times finer than the original one. Hence an interpolation using cubic splines in the temporal axis of \(\idhy\) is performed. Thus resulting intrinsic variables called \(\idhy^*\), which, needless to say, also have a temporal resolution of 241, are then fed into the decoder of the FCNN generating \(\widetilde{\hy}^*\). A comparison of \(\hy^*\) against \(\widetilde{\hy}^*\) as truth against prediction in terms of macroscopic quantities density \(\rho\), momentum \(\rho u\) and total energy \(E\) is provided in \cref{Fig: IntHy}.
\begin{figure}[H]
	% This file was created by tikzplotlib v0.9.8.
\begin{tikzpicture}

\begin{groupplot}[group style={group size=3 by 1},
legend cell align={left},
legend style={fill opacity=0.1, draw opacity=1, text opacity=1, at={(1,1)}, anchor=north east, draw=none},
tick align=outside,
tick pos=left,
x grid style={white!69.0196078431373!black},
xlabel={\(x\)},
xmin=-0.04725, xmax=1.04725,
xtick style={color=black},
y grid style={white!69.0196078431373!black},
ytick style={color=black},
width=.35\textwidth,
height=.4\textwidth,
y label style={yshift=-2em}
]
\nextgroupplot[
ylabel={\(\rho\)},
ytick={0.2,0.4,0.8,1},
ymin=0.0812408233619941, ymax=1.04383396967043,
]
\addplot [semithick, color0, mark=pentagon, mark size=2, mark options={solid},mark repeat=5]
table {%
0.0025 1.00007958960132
0.0075 1.00007958960132
0.0125 1.00007958960132
0.0175 1.00007958960132
0.0225 1.00007958960132
0.0275 1.00007958960132
0.0325 1.00007958960132
0.0375 1.00007958960132
0.0425 1.00007958960132
0.0475 1.00007958960132
0.0525 1.00007958960132
0.0575 1.00007958960132
0.0625 1.00007958960132
0.0675 1.00007958960132
0.0725 1.00007958960132
0.0775 1.00007958960132
0.0825 1.00007958960132
0.0875 1.00007958960132
0.0925 1.00007958960132
0.0975 1.00007960560096
0.1025 1.00007967294275
0.1075 1.00007964882388
0.1125 1.00007973574732
0.1175 1.00007942697807
0.1225 1.00007952906535
0.1275 1.00007935247227
0.1325 1.00007910507478
0.1375 1.0000786045486
0.1425 1.00007773758281
0.1475 1.00007644805925
0.1525 1.00007424166057
0.1575 1.00007047887858
0.1625 1.00006458671113
0.1675 1.00005566690952
0.1725 1.00004139821195
0.1775 1.00001937338839
0.1825 0.999986432031501
0.1875 0.999937488403667
0.1925 0.999865611912762
0.1975 0.999762540533591
0.2025 0.999616785088856
0.2075 0.999413966793886
0.2125 0.999135611671512
0.2175 0.998761416889719
0.2225 0.998265151151335
0.2275 0.997618268225603
0.2325 0.996788949596336
0.2375 0.99574209013388
0.2425 0.994440960625299
0.2475 0.992848293362174
0.2525 0.990926713533989
0.2575 0.988643694819403
0.2625 0.98597447156029
0.2675 0.982894724536353
0.2725 0.979384886899208
0.2775 0.97543120528311
0.2825 0.971025352350884
0.2875 0.966164945468956
0.2925 0.960852690965192
0.2975 0.955096454813896
0.3025 0.948908228748352
0.3075 0.942303409679839
0.3125 0.935299842942792
0.3175 0.927917993992475
0.3225 0.920179602285666
0.3275 0.912107081055203
0.3325 0.903720666985319
0.3375 0.895043348699038
0.3425 0.886099242316696
0.3475 0.876913020961165
0.3525 0.867506658352671
0.3575 0.8578984900834
0.3625 0.848106184178299
0.3675 0.838090457212586
0.3725 0.827832277131951
0.3775 0.817345481359971
0.3825 0.806643968094992
0.3875 0.795742298087786
0.3925 0.784654691798187
0.3975 0.773418127866418
0.4025 0.76206761590187
0.4075 0.750619441273673
0.4125 0.739091075712631
0.4175 0.727500537444857
0.4225 0.715867658268065
0.4275 0.704210523511352
0.4325 0.692526782779235
0.4375 0.680846175469963
0.4425 0.66921524980196
0.4475 0.657692996940784
0.4525 0.646356617796227
0.4575 0.635305164135372
0.4625 0.624665874322266
0.4675 0.61460028374918
0.4725 0.605298140195823
0.4775 0.596974775589291
0.4825 0.589842921078701
0.4875 0.584063832062555
0.4925 0.579672229440244
0.4975 0.576530995545067
0.5025 0.574369383563039
0.5075 0.572842073294033
0.5125 0.571619951938634
0.5175 0.570435925138936
0.5225 0.56906307830009
0.5275 0.567255114691168
0.5325 0.564675626033231
0.5375 0.560842670806613
0.5425 0.555107028704735
0.5475 0.546692667435545
0.5525 0.534817340636247
0.5575 0.5188380042211
0.5625 0.498502658421844
0.5675 0.473995488276239
0.5725 0.446001658485437
0.5775 0.415641278464789
0.5825 0.38431216452763
0.5875 0.353492774178785
0.5925 0.324502245123317
0.5975 0.298354483407366
0.6025 0.275683994082713
0.6075 0.256797161618465
0.6125 0.241638212449143
0.6175 0.229901758262591
0.6225 0.221134202819148
0.6275 0.214821467045285
0.6325 0.210452181777347
0.6375 0.207558450317622
0.6425 0.205741919376864
0.6475 0.204680016666904
0.6525 0.204124898744949
0.6575 0.203894491451661
0.6625 0.203860315972972
0.6675 0.203935182367526
0.6725 0.204061199146803
0.6775 0.204198292814041
0.6825 0.204315014755273
0.6875 0.204378813575093
0.6925 0.204344825913122
0.6975 0.204142378236774
0.7025 0.203654290704554
0.7075 0.202686372769895
0.7125 0.200921816195532
0.7175 0.197857485664529
0.7225 0.19278071511726
0.7275 0.184850776625979
0.7325 0.173437997270236
0.7375 0.159227317397964
0.7425 0.144888745564241
0.7475 0.134016824746112
0.7525 0.128206411531143
0.7575 0.125959403239509
0.7625 0.125261302826069
0.7675 0.125066164599088
0.7725 0.125013926119164
0.7775 0.125000055502295
0.7825 0.124996353614482
0.7875 0.124995377277988
0.7925 0.124995147432354
0.7975 0.124995057285105
0.8025 0.124995140387734
0.8075 0.124995080209968
0.8125 0.124995080209968
0.8175 0.124995080209968
0.8225 0.124995080209968
0.8275 0.124995080209968
0.8325 0.124995080209968
0.8375 0.124995080209968
0.8425 0.124995080209968
0.8475 0.124995080209968
0.8525 0.124995080209968
0.8575 0.124995080209968
0.8625 0.124995080209968
0.8675 0.124995080209968
0.8725 0.124995080209968
0.8775 0.124995080209968
0.8825 0.124995080209968
0.8875 0.124995080209968
0.8925 0.124995080209968
0.8975 0.124995080209968
0.9025 0.124995080209968
0.9075 0.124995080209968
0.9125 0.124995080209968
0.9175 0.124995080209968
0.9225 0.124995080209968
0.9275 0.124995080209968
0.9325 0.124995080209968
0.9375 0.124995080209968
0.9425 0.124995080209968
0.9475 0.124995080209968
0.9525 0.124995080209968
0.9575 0.124995080209968
0.9625 0.124995080209968
0.9675 0.124995080209968
0.9725 0.124995080209968
0.9775 0.124995080209968
0.9825 0.124995080209968
0.9875 0.124995080209968
0.9925 0.124995080209968
0.9975 0.124995080209968
};
\addlegendentry{prediction}
\addplot [semithick, black, mark=+, mark size=2, mark options={solid},
dashed,%only marks,
mark repeat=5]
table {%
0.0025 0.999999994499998
0.0075 0.999999994499989
0.0125 0.999999994499973
0.0175 0.999999994499938
0.0225 0.999999994499854
0.0275 0.999999994499657
0.0325 0.999999994499225
0.0375 0.999999994498307
0.0425 0.999999994496346
0.0475 0.999999994492153
0.0525 0.999999994483302
0.0575 0.999999994464812
0.0625 0.999999994426618
0.0675 0.999999994348593
0.0725 0.999999994191032
0.0775 0.999999993876523
0.0825 0.999999993255951
0.0875 0.999999992045765
0.0925 0.999999989713722
0.0975 0.999999985273699
0.1025 0.999999976922728
0.1075 0.999999961408873
0.1125 0.999999932947061
0.1175 0.999999881389466
0.1225 0.999999789188686
0.1275 0.999999626442624
0.1325 0.99999934295137
0.1375 0.999998855715861
0.1425 0.99999802963589
0.1475 0.99999664829786
0.1525 0.999994370682796
0.1575 0.999990668414943
0.1625 0.999984736919399
0.1675 0.999975372768066
0.1725 0.999960808885077
0.1775 0.999938499600449
0.1825 0.999904849332842
0.1875 0.999854882524753
0.1925 0.99978185882593
0.1975 0.999676846595887
0.2025 0.999528279209094
0.2075 0.999321531266986
0.2125 0.999038563648431
0.2175 0.998657694619451
0.2225 0.99815355591569
0.2275 0.997497285204109
0.2325 0.996656988391704
0.2375 0.995598477836268
0.2425 0.994286259099913
0.2475 0.992684705039226
0.2525 0.990759328214853
0.2575 0.988478046531036
0.2625 0.985812336037212
0.2675 0.982738179020057
0.2725 0.979236741747293
0.2775 0.975294749126552
0.2825 0.970904557028253
0.2875 0.96606395178504
0.2925 0.960775726924111
0.2975 0.955047098184107
0.3025 0.948889019907117
0.3075 0.942315460872332
0.3125 0.9353426880196
0.3175 0.927988594730014
0.3225 0.920272098375855
0.3275 0.912212621103049
0.3325 0.903829659028516
0.3375 0.895142438489912
0.3425 0.886169653587621
0.3475 0.876929276719067
0.3525 0.867438432750232
0.3575 0.857713327521522
0.3625 0.847769222220737
0.3675 0.837620446519981
0.3725 0.82728044509819
0.3775 0.816761854164244
0.3825 0.806076606850691
0.3875 0.795236068929107
0.3925 0.784251209358396
0.3975 0.773132813964529
0.4025 0.761891755435828
0.4075 0.750539339336509
0.4125 0.739087754739222
0.4175 0.727550670369378
0.4225 0.715944034148932
0.4275 0.704287157239119
0.4325 0.692604194420389
0.4375 0.680926170788684
0.4425 0.669293745622052
0.4475 0.657760931521731
0.4525 0.646399957544768
0.4575 0.635307281677215
0.4625 0.624610228710984
0.4675 0.614472544087769
0.4725 0.605094996557836
0.4775 0.596704251904308
0.4825 0.589521611837022
0.4875 0.583707948506557
0.4925 0.57929759656908
0.4975 0.576156313265513
0.5025 0.574000965899689
0.5075 0.57248017604117
0.5125 0.571264200514682
0.5175 0.570086379629996
0.5225 0.568721456981151
0.5275 0.566925162056573
0.5325 0.564365789892603
0.5375 0.560570505963129
0.5425 0.554906052745071
0.5475 0.546613496573865
0.5525 0.534908968811207
0.5575 0.519142277507728
0.5625 0.49897861904467
0.5675 0.474548678169257
0.5725 0.446512146926428
0.5775 0.41600335371414
0.5825 0.384466860249091
0.5875 0.353428287708994
0.5925 0.324264601694908
0.5975 0.298031573575595
0.6025 0.275379773169268
0.6075 0.256558296419699
0.6125 0.241481616612008
0.6175 0.229826468110973
0.6225 0.221130573161533
0.6275 0.214876144173828
0.6325 0.210551636151645
0.6375 0.207691966758486
0.6425 0.205900478048754
0.6475 0.20485677362768
0.6525 0.204314507250902
0.6575 0.204092812694663
0.6625 0.204064483070227
0.6675 0.204143251128216
0.6725 0.204271659470887
0.6775 0.204410148437548
0.6825 0.204527208694704
0.6875 0.204589750704457
0.6925 0.204552164518092
0.6975 0.204341776813738
0.7025 0.203837533803555
0.7075 0.202838125078778
0.7125 0.201017075159804
0.7175 0.197870633082619
0.7225 0.192691646081915
0.7275 0.18466735979763
0.7325 0.173289805104828
0.7375 0.159203830364094
0.7425 0.144954928763386
0.7475 0.134090935533034
0.7525 0.128237750573878
0.7575 0.125970038823052
0.7625 0.125265821320929
0.7675 0.125069121636156
0.7725 0.125016331920844
0.7775 0.125002354274959
0.7825 0.124998671121747
0.7875 0.124997702861611
0.7925 0.124997448751868
0.7975 0.124997382171878
0.8025 0.124997364756704
0.8075 0.124997360209753
0.8125 0.124997359024918
0.8175 0.124997358716831
0.8225 0.124997358636905
0.8275 0.124997358616222
0.8325 0.124997358610884
0.8375 0.12499735860951
0.8425 0.124997358609158
0.8475 0.124997358609067
0.8525 0.124997358609045
0.8575 0.124997358609039
0.8625 0.124997358609038
0.8675 0.124997358609038
0.8725 0.124997358609037
0.8775 0.124997358609037
0.8825 0.124997358609037
0.8875 0.124997358609037
0.8925 0.124997358609037
0.8975 0.124997358609037
0.9025 0.124997358609037
0.9075 0.124997358609037
0.9125 0.124997358609037
0.9175 0.124997358609037
0.9225 0.124997358609037
0.9275 0.124997358609037
0.9325 0.124997358609037
0.9375 0.124997358609037
0.9425 0.124997358609037
0.9475 0.124997358609037
0.9525 0.124997358609037
0.9575 0.124997358609037
0.9625 0.124997358609037
0.9675 0.124997358609037
0.9725 0.124997358609037
0.9775 0.124997358609037
0.9825 0.124997358609037
0.9875 0.124997358609037
0.9925 0.124997358609037
0.9975 0.124997358609037
};
\addlegendentry{truth}

\nextgroupplot[
ylabel={\(\rho u\)},
ytick = {0,0.1,0.3,0.4},
ymin=-0.0210742313599136, ymax=0.437621533883013,
]
\addplot [semithick, color0, mark=pentagon, mark size=2, mark options={solid},mark repeat=5]
table {%
0.0025 -0.000224018744984643
0.0075 -0.000224018744984643
0.0125 -0.000224018744984643
0.0175 -0.000224018744984643
0.0225 -0.000224018744984643
0.0275 -0.000224018744984643
0.0325 -0.000224018744984643
0.0375 -0.000224018744984643
0.0425 -0.000224018744984643
0.0475 -0.000224018744984643
0.0525 -0.000224018744984643
0.0575 -0.000224018744984643
0.0625 -0.000224018744984643
0.0675 -0.000224018744984643
0.0725 -0.000224018744984643
0.0775 -0.000224018744984643
0.0825 -0.000224018744984643
0.0875 -0.000224018744984643
0.0925 -0.000224018744984643
0.0975 -0.000224114204017176
0.1025 -0.000224420971017424
0.1075 -0.000224423848871476
0.1125 -0.000224399111572556
0.1175 -0.000224021377914952
0.1225 -0.000223996242614871
0.1275 -0.000223399761430633
0.1325 -0.000222882727388758
0.1375 -0.000222388961103418
0.1425 -0.000220583597507155
0.1475 -0.000218405245650912
0.1525 -0.000214517724002303
0.1575 -0.000207951256962543
0.1625 -0.000198811069898842
0.1675 -0.000182532549448376
0.1725 -0.000157972483380186
0.1775 -0.000120629754814146
0.1825 -6.45468077751703e-05
0.1875 1.94128097132156e-05
0.1925 0.000142050020835177
0.1975 0.000317399813851215
0.2025 0.000566233682376917
0.2075 0.000911136198722704
0.2125 0.00138317272628902
0.2175 0.00201805094390059
0.2225 0.00285827773621694
0.2275 0.00395250206330534
0.2325 0.00535272705479511
0.2375 0.00711846857027321
0.2425 0.00930975956459987
0.2475 0.0119877343535197
0.2525 0.0152145700462701
0.2575 0.0190384658616776
0.2625 0.023486006956628
0.2675 0.0285888621560336
0.2725 0.0343668356271487
0.2775 0.0408301942981457
0.2825 0.0479735901058079
0.2875 0.0557848471250038
0.2925 0.0642390732036719
0.2975 0.0733041575213734
0.3025 0.0829408320345379
0.3075 0.093101483810572
0.3125 0.103735478966086
0.3175 0.114789414264818
0.3225 0.126204462695237
0.3275 0.137922470816743
0.3325 0.149856288696278
0.3375 0.161922343022342
0.3425 0.17405417015264
0.3475 0.186189567965098
0.3525 0.198268231517027
0.3575 0.210235498357924
0.3625 0.222050954078593
0.3675 0.233822526473728
0.3725 0.245508581475077
0.3775 0.257068285670135
0.3825 0.268466084457423
0.3875 0.279669139654188
0.3925 0.290645325132257
0.3975 0.30128351815843
0.4025 0.311509890508708
0.4075 0.321307295897787
0.4125 0.330665317075539
0.4175 0.339573258160456
0.4225 0.348027166971018
0.4275 0.356030595024101
0.4325 0.363641410975208
0.4375 0.370859526743636
0.4425 0.3776673689889
0.4475 0.384045174629705
0.4525 0.389965828769702
0.4575 0.395394642704727
0.4625 0.400289485355001
0.4675 0.404598553234369
0.4725 0.408279565124794
0.4775 0.4112957652881
0.4825 0.413636596928851
0.4875 0.415293668469434
0.4925 0.416305254573296
0.4975 0.416742638645411
0.5025 0.416771726371971
0.5075 0.416527856893603
0.5125 0.416087128785646
0.5175 0.415465345538933
0.5225 0.414619796982088
0.5275 0.413435572513614
0.5325 0.411703304513104
0.5375 0.409091789631392
0.5425 0.405106248364163
0.5475 0.399101061056479
0.5525 0.39036861576975
0.5575 0.378290466986318
0.5625 0.363146681986418
0.5675 0.345323436214576
0.5725 0.32549162805932
0.5775 0.304513083434557
0.5825 0.283249913988369
0.5875 0.262475023127174
0.5925 0.242767803704885
0.5975 0.224524474631891
0.6025 0.208051999031094
0.6075 0.193581449267863
0.6125 0.181341328977864
0.6175 0.171418746395887
0.6225 0.163721182217289
0.6275 0.15801678880118
0.6325 0.153986948144191
0.6375 0.151285444173278
0.6425 0.149584317095146
0.6475 0.148598559150083
0.6525 0.148099745476024
0.6575 0.147915445005698
0.6625 0.14792264607018
0.6675 0.148034800207918
0.6725 0.148191477993727
0.6775 0.148346908903325
0.6825 0.148456643561773
0.6875 0.148466094832618
0.6925 0.148289537902271
0.6975 0.147783888314429
0.7025 0.146705594838524
0.7075 0.14462977813771
0.7125 0.140840847912772
0.7175 0.134184715481149
0.7225 0.123067795069719
0.7275 0.105610533503771
0.7325 0.0802160459178515
0.7375 0.049383167051561
0.7425 0.0224540554733486
0.7475 0.00768127561627308
0.7525 0.00231781407386183
0.7575 0.000701356647177434
0.7625 0.000246654138973484
0.7675 0.00012330573050479
0.7725 9.08364029228138e-05
0.7775 8.20810216900367e-05
0.7825 7.98726666741706e-05
0.7875 7.89232503624513e-05
0.7925 7.88321693419298e-05
0.7975 7.90195053972418e-05
0.8025 7.88600600342628e-05
0.8075 7.87138405539246e-05
0.8125 7.87138405539246e-05
0.8175 7.87138405539246e-05
0.8225 7.87138405539246e-05
0.8275 7.87138405539246e-05
0.8325 7.87138405539246e-05
0.8375 7.87138405539246e-05
0.8425 7.87138405539246e-05
0.8475 7.87138405539246e-05
0.8525 7.87138405539246e-05
0.8575 7.87138405539246e-05
0.8625 7.87138405539246e-05
0.8675 7.87138405539246e-05
0.8725 7.87138405539246e-05
0.8775 7.87138405539246e-05
0.8825 7.87138405539246e-05
0.8875 7.87138405539246e-05
0.8925 7.87138405539246e-05
0.8975 7.87138405539246e-05
0.9025 7.87138405539246e-05
0.9075 7.87138405539246e-05
0.9125 7.87138405539246e-05
0.9175 7.87138405539246e-05
0.9225 7.87138405539246e-05
0.9275 7.87138405539246e-05
0.9325 7.87138405539246e-05
0.9375 7.87138405539246e-05
0.9425 7.87138405539246e-05
0.9475 7.87138405539246e-05
0.9525 7.87138405539246e-05
0.9575 7.87138405539246e-05
0.9625 7.87138405539246e-05
0.9675 7.87138405539246e-05
0.9725 7.87138405539246e-05
0.9775 7.87138405539246e-05
0.9825 7.87138405539246e-05
0.9875 7.87138405539246e-05
0.9925 7.87138405539246e-05
0.9975 7.87138405539246e-05
};
\addlegendentry{pred.}
\addplot [semithick, black, mark=+, mark size=2, mark options={solid},dashed,
mark repeat=5]
table {%
0.0025 6.30909939231123e-15
0.0075 2.52743809394443e-14
0.0125 6.71665948905054e-14
0.0175 1.47797315006362e-13
0.0225 3.07924882400649e-13
0.0275 6.6172472758689e-13
0.0325 1.42564590009118e-12
0.0375 3.07415834705836e-12
0.0425 6.61513596957766e-12
0.0475 1.41586575371753e-11
0.0525 3.00685203765747e-11
0.0575 6.32804197847002e-11
0.0625 1.31840368981028e-10
0.0675 2.71816268178485e-10
0.0725 5.54376848420904e-10
0.0775 1.11825149293422e-09
0.0825 2.23051508115254e-09
0.0875 4.3988436607807e-09
0.0925 8.57587718345818e-09
0.0975 1.6525966137083e-08
0.1025 3.14735316685944e-08
0.1075 5.92319220162265e-08
0.1125 1.1013832102345e-07
0.1175 2.02317177002482e-07
0.1225 3.67093736874415e-07
0.1275 6.57821440259511e-07
0.1325 1.16402287545304e-06
0.1375 2.03362895936117e-06
0.1425 3.50728899710936e-06
0.1475 5.97025139178314e-06
0.1525 1.00291758876392e-05
0.1575 1.66233526126904e-05
0.1625 2.71819698904176e-05
0.1675 4.38409222097891e-05
0.1725 6.97336015392367e-05
0.1775 0.000109369362893115
0.1825 0.000169109904238073
0.1875 0.000257746595168556
0.1925 0.000387169949609523
0.1975 0.000573105626212421
0.2025 0.000835870200910254
0.2075 0.00120107654392011
0.2125 0.0017001966437239
0.2175 0.00237087431188884
0.2225 0.00325687722538598
0.2275 0.00440759236330102
0.2325 0.00587700372294412
0.2375 0.00772214475261246
0.2425 0.0100010837355784
0.2475 0.0127705675659017
0.2525 0.0160835049890066
0.2575 0.0199865026149812
0.2625 0.024517668517973
0.2675 0.0297048683221924
0.2725 0.0355645634604718
0.2775 0.0421012918433731
0.2825 0.0493077803908931
0.2875 0.0571656182273979
0.2925 0.0656463765219179
0.2975 0.0747130389685672
0.3025 0.0843216044817968
0.3075 0.0944227366293171
0.3125 0.104963357117122
0.3175 0.115888107826333
0.3225 0.127140633030916
0.3275 0.138664657456851
0.3325 0.150404855140034
0.3375 0.162307518113764
0.3425 0.174321043116104
0.3475 0.186396259514057
0.3525 0.198486623435526
0.3575 0.210548302610127
0.3625 0.222540174456852
0.3675 0.234423757164425
0.3725 0.2461630903626
0.3775 0.257724578799727
0.3825 0.269076809419216
0.3875 0.280190349466021
0.3925 0.291037530789031
0.3975 0.301592223328621
0.4025 0.311829598863364
0.4075 0.321725884407147
0.4125 0.33125810319655
0.4175 0.340403800052193
0.4225 0.34914074723873
0.4275 0.357446627257301
0.4325 0.36529869126666
0.4375 0.37267339798658
0.4425 0.379546051649685
0.4475 0.385890485406314
0.4525 0.391678889528999
0.4575 0.396881977899246
0.4625 0.401469838716726
0.4675 0.405414024841958
0.4725 0.408691630863368
0.4775 0.411292024184161
0.4825 0.413226029159821
0.4875 0.414535185870344
0.4925 0.415295776454039
0.4975 0.415611531428894
0.5025 0.41559377314098
0.5075 0.415336270392398
0.5125 0.414895446220246
0.5175 0.41427973531504
0.5225 0.413441913351203
0.5275 0.412265556853831
0.5325 0.410543069713507
0.5375 0.407952684100831
0.5425 0.404050220477402
0.5475 0.39829414322411
0.5525 0.390116259686097
0.5575 0.379034706827778
0.5625 0.364785475907473
0.5675 0.347433283903082
0.5725 0.327421672581876
0.5775 0.305538966019578
0.5825 0.282804899447141
0.5875 0.260309643125283
0.5925 0.239050561287535
0.5975 0.219807541815168
0.6025 0.20307934526991
0.6075 0.189081107060286
0.6125 0.177786616825983
0.6175 0.168992999905332
0.6225 0.162388541129918
0.6275 0.15761173233599
0.6325 0.154296586879928
0.6375 0.152103711892583
0.6425 0.150738704005009
0.6475 0.149960167756696
0.6525 0.149579858547135
0.6575 0.149457421963541
0.6625 0.149491943424773
0.6675 0.1496120131206
0.6725 0.149765299386181
0.6775 0.149907790936039
0.6825 0.149991956492704
0.6875 0.149952035270804
0.6925 0.149683390951819
0.6975 0.149011228649705
0.7025 0.147642219487674
0.7075 0.145092233973356
0.7125 0.140589880124687
0.7175 0.132983873988083
0.7225 0.120763965024808
0.7275 0.10247138733323
0.7325 0.0779153045242086
0.7375 0.0500927744393442
0.7425 0.0255869633865413
0.7475 0.0101029770077908
0.7525 0.00323067177688417
0.7575 0.000916685479303123
0.7625 0.000247535672051547
0.7675 6.57015394421608e-05
0.7725 1.733476779243e-05
0.7775 4.56166541654529e-06
0.7825 1.19832317168437e-06
0.7875 3.14296210643595e-07
0.7925 8.22998342934205e-08
0.7975 2.15132336229347e-08
0.8025 5.61307170845078e-09
0.8075 1.46156535792315e-09
0.8125 3.79741250295264e-10
0.8175 9.8431384383556e-11
0.8225 2.54493088990962e-11
0.8275 6.56196969727746e-12
0.8325 1.68704996773948e-12
0.8375 4.3243891523643e-13
0.8425 1.10497770422186e-13
0.8475 2.8177175461474e-14
0.8525 7.14463267733202e-15
0.8575 1.78835622885588e-15
0.8625 4.08881035414389e-16
0.8675 5.23823250656242e-17
0.8725 2.23070011976512e-18
0.8775 -1.70270269849063e-18
0.8825 1.91820139524602e-19
0.8875 -2.40410875285755e-18
0.8925 -2.60371400131214e-18
0.8975 1.34734437976866e-19
0.9025 1.3517563413719e-19
0.9075 1.3561001001166e-19
0.9125 1.3561001001166e-19
0.9175 1.3561001001166e-19
0.9225 1.3561001001166e-19
0.9275 1.3561001001166e-19
0.9325 1.3561001001166e-19
0.9375 1.3561001001166e-19
0.9425 1.3561001001166e-19
0.9475 1.3561001001166e-19
0.9525 1.3561001001166e-19
0.9575 1.3561001001166e-19
0.9625 1.3561001001166e-19
0.9675 1.3561001001166e-19
0.9725 1.3561001001166e-19
0.9775 1.3561001001166e-19
0.9825 1.3561001001166e-19
0.9875 1.3561001001166e-19
0.9925 1.3561001001166e-19
0.9975 1.3561001001166e-19
};
\addlegendentry{truth}

\nextgroupplot[
ytick={0,0.1,0.2,0.4,0.5},
ylabel={\(E\)},
ymin=-0.0100275328732878, ymax=0.52428702249394,
y label style={xshift=.8em}
]
\addplot [semithick, color0, mark=pentagon, mark size=2, mark options={solid},mark repeat=5]
table {%
0.0025 0.499566434696068
0.0075 0.499566434696068
0.0125 0.499566434696068
0.0175 0.499566434696068
0.0225 0.499566434696068
0.0275 0.499566434696068
0.0325 0.499566434696068
0.0375 0.499566434696068
0.0425 0.499566434696068
0.0475 0.499566434696068
0.0525 0.499566434696068
0.0575 0.499566434696068
0.0625 0.499566434696068
0.0675 0.499566434696068
0.0725 0.499566434696068
0.0775 0.499566434696068
0.0825 0.499566434696068
0.0875 0.499566434696068
0.0925 0.499566434696068
0.0975 0.499566370348632
0.1025 0.499567510892607
0.1075 0.499567527370009
0.1125 0.499568461188917
0.1175 0.499565495813832
0.1225 0.499567820889932
0.1275 0.499567439440812
0.1325 0.499567908092993
0.1375 0.499566768757936
0.1425 0.499564687891228
0.1475 0.499561228374623
0.1525 0.499559679675419
0.1575 0.499553906343711
0.1625 0.499542880865482
0.1675 0.499527347451757
0.1725 0.499503439223567
0.1775 0.499467262862731
0.1825 0.499412237656599
0.1875 0.499332065979628
0.1925 0.499212074995381
0.1975 0.499043806748767
0.2025 0.498802532356096
0.2075 0.498472301059849
0.2125 0.498020814721587
0.2175 0.497416882395611
0.2225 0.49662143605863
0.2275 0.495597837974689
0.2325 0.494300632658181
0.2375 0.492685228301581
0.2425 0.490716071969207
0.2475 0.488356221513139
0.2525 0.485586777843386
0.2575 0.482366202590156
0.2625 0.478636349469023
0.2675 0.474370902309825
0.2725 0.469563017550409
0.2775 0.464211038835019
0.2825 0.458326442635222
0.2875 0.451927760843783
0.2925 0.445051297040587
0.2975 0.437740810869383
0.3025 0.430049988082055
0.3075 0.422041759481815
0.3125 0.41377940822452
0.3175 0.405341573274708
0.3225 0.396805402694221
0.3275 0.388252174850362
0.3325 0.379800179743692
0.3375 0.371559981380593
0.3425 0.363620787652613
0.3475 0.356180819505493
0.3525 0.349377926001172
0.3575 0.3432583843994
0.3625 0.33782942289075
0.3675 0.332347496805064
0.3725 0.326394980320559
0.3775 0.320012823852642
0.3825 0.313236933914264
0.3875 0.306129853124111
0.3925 0.298745550819665
0.3975 0.291790480170699
0.4025 0.285852767651042
0.4075 0.281006252061158
0.4125 0.277299036681199
0.4175 0.274759348070896
0.4225 0.273400023151207
0.4275 0.272964188700684
0.4325 0.271815780753738
0.4375 0.269828438346021
0.4425 0.267249841352473
0.4475 0.264350149833369
0.4525 0.261423391158425
0.4575 0.258766695461318
0.4625 0.256654295660602
0.4675 0.25529988899042
0.4725 0.254513976549939
0.4775 0.254076380414675
0.4825 0.253857447711592
0.4875 0.253797470865467
0.4925 0.253808752077065
0.4975 0.253763714009828
0.5025 0.253699368179262
0.5075 0.25362855235414
0.5125 0.253532433385479
0.5175 0.253393969678747
0.5225 0.253173501656991
0.5275 0.252807950506907
0.5325 0.252185353063773
0.5375 0.251113538453846
0.5425 0.249354667481387
0.5475 0.246752242224898
0.5525 0.243418878017469
0.5575 0.239465106067682
0.5625 0.234279419366452
0.5675 0.227783804517479
0.5725 0.220192690876746
0.5775 0.211972309824358
0.5825 0.203765609306105
0.5875 0.196264027025048
0.5925 0.190010807978054
0.5975 0.184577986106031
0.6025 0.179033526307619
0.6075 0.173685600452327
0.6125 0.168782162839598
0.6175 0.164524354824771
0.6225 0.16104175884175
0.6275 0.158359583934681
0.6325 0.156417219529006
0.6375 0.15509608100915
0.6425 0.154262423100925
0.6475 0.153786435607038
0.6525 0.153558933343705
0.6575 0.153494577775065
0.6625 0.15352806838995
0.6675 0.153617849069447
0.6725 0.153725001225156
0.6775 0.15382214803754
0.6825 0.1538752986271
0.6875 0.153841953133464
0.6925 0.153640439458696
0.6975 0.153146652727694
0.7025 0.152149559024472
0.7075 0.150290444535956
0.7125 0.146908177847255
0.7175 0.140638036539332
0.7225 0.129557863401295
0.7275 0.11102893867151
0.7325 0.0820449516669314
0.7375 0.046488003712627
0.7425 0.0207998444859785
0.7475 0.0142594923706771
0.7525 0.014927622493576
0.7575 0.0156025802899868
0.7625 0.0158366322815483
0.7675 0.0158999766958207
0.7725 0.0159181611108628
0.7775 0.0159240230115524
0.7825 0.0159260359236937
0.7875 0.0159249571562401
0.7925 0.0159248001028088
0.7975 0.0159241176643776
0.8025 0.0159246474218944
0.8075 0.0159247248240804
0.8125 0.0159247248240804
0.8175 0.0159247248240804
0.8225 0.0159247248240804
0.8275 0.0159247248240804
0.8325 0.0159247248240804
0.8375 0.0159247248240804
0.8425 0.0159247248240804
0.8475 0.0159247248240804
0.8525 0.0159247248240804
0.8575 0.0159247248240804
0.8625 0.0159247248240804
0.8675 0.0159247248240804
0.8725 0.0159247248240804
0.8775 0.0159247248240804
0.8825 0.0159247248240804
0.8875 0.0159247248240804
0.8925 0.0159247248240804
0.8975 0.0159247248240804
0.9025 0.0159247248240804
0.9075 0.0159247248240804
0.9125 0.0159247248240804
0.9175 0.0159247248240804
0.9225 0.0159247248240804
0.9275 0.0159247248240804
0.9325 0.0159247248240804
0.9375 0.0159247248240804
0.9425 0.0159247248240804
0.9475 0.0159247248240804
0.9525 0.0159247248240804
0.9575 0.0159247248240804
0.9625 0.0159247248240804
0.9675 0.0159247248240804
0.9725 0.0159247248240804
0.9775 0.0159247248240804
0.9825 0.0159247248240804
0.9875 0.0159247248240804
0.9925 0.0159247248240804
0.9975 0.0159247248240804
};
\addlegendentry{prediction}
\addplot [semithick, black, mark=+, mark size=2, mark options={solid},dashed,mark repeat=5]
table {%
0.0025 0.499999997249975
0.0075 0.499999997249952
0.0125 0.49999999724991
0.0175 0.49999999724983
0.0225 0.499999997249673
0.0275 0.499999997249343
0.0325 0.499999997248633
0.0375 0.499999997247108
0.0425 0.499999997243849
0.0475 0.499999997236918
0.0525 0.49999999722231
0.0575 0.499999997191854
0.0625 0.499999997129034
0.0675 0.499999997000896
0.0725 0.499999996742452
0.0775 0.499999996227163
0.0825 0.499999995211631
0.0875 0.499999993233655
0.0925 0.499999989426775
0.0975 0.499999982187882
0.1025 0.499999968590229
0.1075 0.499999943362711
0.1125 0.49999989714242
0.1175 0.499999813531151
0.1225 0.499999664219035
0.1275 0.499999401043895
0.1325 0.499998943288962
0.1375 0.499998157735899
0.1425 0.499996827939107
0.1475 0.499994607841606
0.1525 0.499990953221392
0.1575 0.499985022618125
0.1625 0.499975537529369
0.1675 0.499960590121808
0.1725 0.499937386002681
0.1775 0.499901910468537
0.1825 0.499848509989294
0.1875 0.499769387444734
0.1925 0.499654020597409
0.1975 0.499488528760043
0.2025 0.499255032002629
0.2075 0.498931068658278
0.2125 0.498489156918177
0.2175 0.497896600156329
0.2225 0.497115637784343
0.2275 0.496104028964957
0.2325 0.494816122733982
0.2375 0.493204416224345
0.2425 0.491221538805241
0.2475 0.488822534483274
0.2525 0.48596726066466
0.2575 0.482622690300756
0.2625 0.478764904226945
0.2675 0.474380592224794
0.2725 0.469467939022164
0.2775 0.46403684380784
0.2825 0.458108495428486
0.2875 0.451714388097888
0.2925 0.444894905918178
0.2975 0.437697625542276
0.3025 0.430175486289908
0.3075 0.422384960603958
0.3125 0.414384331017942
0.3175 0.406232148833711
0.3225 0.397985919383409
0.3275 0.389701032374154
0.3325 0.381429935098179
0.3375 0.373221531647741
0.3425 0.365120782228682
0.3475 0.357168472254166
0.3525 0.349401119983176
0.3575 0.341850992947928
0.3625 0.334546206351425
0.3675 0.327510880277021
0.3725 0.320765336408751
0.3775 0.314326318665062
0.3825 0.308207225493483
0.3875 0.302418344457086
0.3925 0.296967082133645
0.3975 0.291858184256781
0.4025 0.287093942489105
0.4075 0.28267438527183
0.4125 0.278597450881622
0.4175 0.27485914117029
0.4225 0.271453654478241
0.4275 0.268373495894285
0.4325 0.265609562373013
0.4375 0.263151199234399
0.4425 0.260986223393296
0.4475 0.25910090775627
0.4525 0.257479921790285
0.4575 0.256106228028066
0.4625 0.254960948347607
0.4675 0.254023244450158
0.4725 0.253270307401023
0.4775 0.252677601550096
0.4825 0.252219485718782
0.4875 0.251870132202037
0.4925 0.251604330194165
0.4975 0.251397729084703
0.5025 0.25122669129834
0.5075 0.251068332781992
0.5125 0.250900333775037
0.5175 0.250698706378175
0.5225 0.250431897599141
0.5275 0.250051148444404
0.5325 0.249478662542547
0.5375 0.248596912673306
0.5425 0.247244381726547
0.5475 0.245223917214396
0.5525 0.242327893303772
0.5575 0.238378924611544
0.5625 0.233277606118821
0.5675 0.227043160220145
0.5725 0.219832439446087
0.5775 0.211928643264648
0.5825 0.203701164573315
0.5875 0.195547680233291
0.5925 0.187834659876278
0.5975 0.18085111996356
0.6025 0.174784098494021
0.6075 0.169716410533201
0.6125 0.165641200169253
0.6175 0.162485367499619
0.6225 0.160134752274621
0.6275 0.158456399070025
0.6325 0.157315740975606
0.6375 0.156588304507358
0.6425 0.156166477645325
0.6475 0.155962258709567
0.6525 0.155906982680582
0.6575 0.155948945374948
0.6625 0.156049656989133
0.6675 0.156179160619016
0.6725 0.156310449279297
0.6775 0.156412496184095
0.6825 0.156440733162282
0.6875 0.156322878793174
0.6925 0.155936716265043
0.6975 0.155074755703434
0.7025 0.15338930096658
0.7075 0.150312941539415
0.7125 0.144962247510296
0.7175 0.136077279136682
0.7225 0.122162286306937
0.7275 0.102173607088272
0.7325 0.0770700772857907
0.7375 0.0513960997098126
0.7425 0.0317715372289992
0.7475 0.0212761317050686
0.7525 0.0172929078157419
0.7575 0.0160879851700508
0.7625 0.0157570212643047
0.7675 0.0156687593965459
0.7725 0.0156454185249256
0.7775 0.0156392644262287
0.7825 0.015637644580833
0.7875 0.0156372188316351
0.7925 0.0156371070941755
0.7975 0.0156370778143173
0.8025 0.0156370701545981
0.8075 0.0156370681544042
0.8125 0.0156370676331101
0.8175 0.0156370674975363
0.8225 0.0156370674623578
0.8275 0.0156370674532522
0.8325 0.0156370674509016
0.8375 0.0156370674502965
0.8425 0.0156370674501412
0.8475 0.0156370674501014
0.8525 0.0156370674500913
0.8575 0.0156370674500887
0.8625 0.015637067450088
0.8675 0.0156370674500878
0.8725 0.0156370674500877
0.8775 0.0156370674500877
0.8825 0.0156370674500877
0.8875 0.0156370674500877
0.8925 0.0156370674500877
0.8975 0.0156370674500877
0.9025 0.0156370674500877
0.9075 0.0156370674500877
0.9125 0.0156370674500877
0.9175 0.0156370674500877
0.9225 0.0156370674500877
0.9275 0.0156370674500877
0.9325 0.0156370674500877
0.9375 0.0156370674500877
0.9425 0.0156370674500877
0.9475 0.0156370674500877
0.9525 0.0156370674500877
0.9575 0.0156370674500877
0.9625 0.0156370674500877
0.9675 0.0156370674500877
0.9725 0.0156370674500877
0.9775 0.0156370674500877
0.9825 0.0156370674500877
0.9875 0.0156370674500877
0.9925 0.0156370674500877
0.9975 0.0156370674500877
};
\addlegendentry{truth}
\end{groupplot}

\end{tikzpicture}

	\caption{Macroscopic quantities density \(\rho\), momentum \(\rho u\) and total Energy \(E\), displayed in this order, at time \(t=0.12\) against predictions generated from interpolation. Temporal interpolation with cubic splines is perofrmed in \(\idhy\) from 25 snaptshots to 241 snapshots.}
	\label{Fig: IntHy}
\end{figure}
The generated snapshots match the macroscopic quantities almost exactly, similar to the results obtained without interpolation. Nonetheless, the resulting \(\L2\)-error for \(\widetilde{\hy}^*\) increases to \(\L2=0.0012\), which is 1.5 times higher than the original \(\L2\)-error.\\

In summary, the reconstructions obtained from the CNN do not meet conservation of mass, momentum and total energy, which devalues their physical applicableness. Nonetheless, the CNN uses with 8246, the second highest amount of parameters. The attempt to learn both levels of rarefaction with the CNN subsequently also failed. Specifically, did the CNN reconstruct \(\hy\) from \(\rare\) as input. In addition, the intrinsic variables of the CNN show, to some degree, similarities to POD basis modes. \\
The FCNN uses with 2683 to 3725 parameters the least amount, while achieving the lowest \(\L2\)-error in the range of \num{8e-4} to \num{9e-4}. At the same time for \(\hy\) less parameters than for \(\rare\) are needed to achieve comparable results using the FCNN. Conservation of mass momentum and total energy is met, hence macroscopic quantities of considerable accuracy can be obtained which show a good fit to the FOM solution's. The question of the interpretability of the intrinsic variables is up to discussion owing to their entangled nature, especially for \(\rare\). Generalization was successfully tested, by generating new snapshots of \(\hy\) but at the same time increasing the \(\L2\)-error to \num{1.2e-3}.\\
Both neural networks are tested against POD. By fixing \(p\) the algorithm is artificially limited and performs under it's abilities. Though, POD uses with this adjustement five to six times more parameters than the FCNN. It's determinstic character enables POD to achieve any possible accuracy, which was not observed with the neural networks.    
\subsection{Discussion and Outlook}
An interesting finding during the hyperparameters search is the different number of parameters, that the FCNN requires to achieve comparable results for the slip- and the continuum flow. To recap, a continuum flow can be described in terms of three macroscopic quantities, a rarefied flow is described as a probabillity distribution of the macroscopic velocities of all involved particles. In turn, the necessity of using more parameters for the rarefied flow arises. This can be interpreted as a validation of the FCNN, as physical properties are reflected.\\
POD and the FCNN have different qualities. POD uses up to 6.7 times more parameters than the FCNN. This suggests, that the quality of the parameters, that the FCNN has learnt surpasses those of POD. On the other hand, the reconstruction loss of the FCNN was not able to cross a threshold, which POD had no struggle with.   