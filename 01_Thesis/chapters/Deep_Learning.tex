% !TEX root = master.tex

\chapter{Dimensionality reduction algorithms}
\label{Ch:DimRedAl}
%\pagenumbering{arabic}


\section{Proper orthogonal decomposition (POD)}
\label{Sec: POD}
\section{Autoencoders}
\label{Sec:AE}
In this section deep learning with the focus on fully connected autoencoders and convolutional autoencoders will be introduced. Starting of with addressing important terminology whilst presenting the concept of autoencoders and continuing with the introduction of the fully connected and convolutional forwardpass , this section closes with ADAM \cite{the} an update to the backpropagation algorithm as well as important training methods.\\ 
The term deep learning situated around the much broader field of artificial intelligence stems from the use of deep feed forward networks also called multi layer perceptrons (MLP) or feed forward neural networks. Deep recurrent neural networks (RNN) are also used in this field but won't be covered in this thesis. In contrast to RNNs, information flows forward through these networks, which explains the name feed foward. In the following I will use the abbreviation MLP when talking about the aforementioned algorithm. Network refers to the typical composition of many different functions.\\The task of any MLP is to approximate a function \(f^\star(\mathbf{x};\mathbf{\Theta}) \approx f(\mathbf{x})\) through learning the values of the parameters \(\mathbf{\Theta}\). As mentioned before  \(f^\star\) is a composition of functions eg. \cref{Eq. Composition}.
\begin{equation}
	f^\star(\mathbf{x};\mathbf{\Theta})= f^3(f^2(f^1(\mathbf{x};\mathbf{\Theta}^1),\mathbf{\Theta}^2),\mathbf{\Theta}^3)
	\label{Eq. Composition}
\end{equation}
 In \cref{Eq. Composition} each function \(f^i\) is called layer. In this example \(f^1\) is called the input layer, \(f^3\) the output layer and \(f^2\) a hidden layer. Hence a layer is a vector-to-vector function. In this context a unit describes the corresponding vector-to-scalar functions of one layer. The width of a layer is referred to as the dimension of the vector valued input. The depth of the network describes the number of composed functions. In autoencoders the dimensions of input and output layer are identical.\\Autoencoders are a special kind of neural network, that have a central hidden layer, that outputs a code \(c\) which should contain useful features of the input \(x\) while usually being of a lower dimension. Hereafter the code will be addressed as intrinsic variables highlighting the property of containing useful features of the input \(x\). Autoencoders can be split in two parts, the encoder \(h(x) = c\) which compresses the input and outputs the intrinsic variables \(c\) and the decoder \(g(h) = \hat{x}\) which reconstructs the input from the intrinsic variables to output \(\hat{x}\). The goal of autoencoders conflicts with the training objective. The former is to produce a code that describes the intrinsic features of the input, while the latter is to minimize the difference between \(x\) and \(\hat{x}\). Therefore autoencoders need to be restrained from learning the identity function perfectly which in turn should drive the model to choose which instance to copy.\\
Obviously deep learning emphasizes the focus on the depth of a model. This is because linear models with just one layer can only approximate linear functions. Adding a non-linear activation function to the output of the proposed model wouldn't be sufficient in modeling any nonlinear behavior of the function. However the universal approximation theorem \cite{Hornik1989} states that MLPs with at least one hidden layer and any non-linear activation function can approximate any function given that enough hidden layers can be provided. In conclusion, MLPs are universal approximators \cite{Goodfellow}.\\
There are several types of layers, that can be used in MLPs. For this thesis it is sufficient to treat so called \textit{fully connected layers} and \textit{convolutional layers}.
Fully connected layers are called \texttt{Linear}\cite{bibid} in \texttt{PyTorch}\cite{bibid} because they compute a linear transformation of the input \cref{Eq. Linear Transformation}, where \(x\) is the input vector, \(A\) is the weight matrix and \(b\) is a bias vector. This is the forward pass of a linear layer. The learnable parameters \(\Theta\) are in this case the values in \(A\) and \(b\). For a linear layer which takes a vector of size \(i\) as input and outputs a vector of size \(o\) there are \(l = i \times o + o\) learnable parameters. 
\begin{equation}
	y = xA^T + b \label{Eq. Linear Transformation}
\end{equation}
Continuing with the forward pass in convolutional layers, which are called \texttt{Conv2D}\cite{bibid} in \texttt{Pytorch}. Convolutional layers are usually applied when the input data has a known grid-like topology \cite{Goodfellow}.While for fully connected layers, the input size is fixed, convolutional layers can be applied to inputs of various sizes. Furthermore, they are sparse by construction and share parameters making them equivariant \cite{Goodfellow}.  Even tough the name implies the use of the convolutional operation in \cref{Eq. Convolution}, \texttt{PyTorch} and many other neural network libraries instead use the cross correlation, which is an operation closely related to a convolution\cite{Goodfellow}\cite{Pytorch website}. The convolution in \ref{Eq. Convolution} operates on two functions \(x\) and \(w\), where the latter is a weighting function of the former which makes \(s\) the weighted average of the input \(x\). In \cref{Eq. Discrete Convolution} \(x\) is represented by \(I(m,n)\), and \(w\) by \(K(m,n)\) respectively. Note that while \cref{Eq. Convolution} is scalar valued and \cref{Eq. Discrete Convolution} is two dimensional, only for illustrative reasons. 
\begin{align}
	s(t) &= (x * w)(t) = \int x(a)w(t-a)\,da \label{Eq. Convolution}\\
	S(i,j) &= (I * K)(i,j) = \sum_{m}\sum_{n}I(m,n)K(i-m,j-n)
	\label{Eq. Discrete Convolution}
\end{align}
One drawback in implementing the discrete convolution \cref{Eq. Discrete Convolution} is that there can be invalid values for \(m\) and \(n\). This can be solved by exploiting the commutative property of the convolution, which results for the discrete convolution in a flipped kernel relative to the input, \cref{Eq. Discrete Convolution Flip}. Without the need of flipping the kernel which is not always possible, i.e. exploiting the commutative property, cross correlation is adopted \cref{Eq. Discrete Cross Correlation}.
\begin{align}
	S(i,j) &= (K * I)(i,j) = \sum_{m}\sum_{n}I(i-m,j-n)K(m,n) 
	\label{Eq. Discrete Convolution Flip}\\
	S(i,j) &= (I * K)(i,j) = \sum_{m}\sum_{n}I(i+m,j+n)K(m,n) 
	\label{Eq. Discrete Cross Correlation}
\end{align}
The given equations illustrate the movement of the kernel over a two dimensional input, but leaving out two basic accompanying designs. First is the so called stride of the kernel, which results in a increased down sampling with increased stride size. Considering strides of the kernel along one dimension of the input results in a shrinkage of that dimension by 
\begin{equation}\label{Eq:Downsampling}
	o = \frac{i -k}{s} + 1.
\end{equation} 
Here \(o\), \(i\), \(s\) and \(k\) are the output, input, stride and kernel size of one dimension respectively \cite{dumoulin2018guide}. Second are the so called channels, which allow convolutional layers to extract a different feature for every channel from the same input. For a two dimensional input like images this could be different features for the same location on the image, like edges and color in the RGB color space \cite{Goodfellow}. Adding strides and kernel to \cref{Eq. Discrete Cross Correlation} yields
\begin{equation}
	\mathsf{S}_{i,j,k} = c(\mathsf{K},\mathsf{I},s)_{i,j,k}\sum_{l,m,n}\left[\mathsf{I}_{l,(j-1)\times s+m,(k -1)\times s+n}\mathsf{K}_{i,l,m,n}\right]. \label{Eq. Cross correlation stride channel}
\end{equation}
The kernel \(\mathsf{K}\) is a four dimensional tensor with \(i\) indexing into the output channels of \(S\), \(l\) indexing into the input channels of \(I\), \(m\) and \(n\) indexing into the rows and colums. The input \(\mathsf{I}\) and output \(S\) are three dimensional tensors with \(j\) and \(k\) indexing into the rows and columns. Note that in \texttt{PyTorch} \cref{Eq. Cross correlation stride channel} is a so called valid cross correlation (valid convolution) \cite{bibid} meaning the kernel will only move over input units for which  all \(m\) and \(n\) are inside the rows and columns of the input. Zero padding the input can prevent the kernel from omitting corners in that case.
\\
For autoencoders the necessity to perform a transposition of the applied layers arises, which is straight forward for fully connected layers. Convolutional layers with strides greater than unity on the other hand  the transposition needs the kernel to be padded with zeros to realize an upsampling of the input data \cite{dumoulin2018guide}. Note that the padding of the kernel with zeros is only used to illustrate how the upsampling works. In \cref{Eq: Transposed convolution} taken from \cite{Goodfellow} multiplications with zero are omitted. The size of one dimension during upsampling can be calculated with the transposition of \cref{Eq:Downsampling}.
\begin{equation}\label{Eq: Transposed convolution}
	t(\mathsf{K},\mathsf{H},s)_{i,j,k} = \sum_{\substack{l,m\\s.t\\(l-1)\times s+m=j}}
												  \sum_{\substack{n,p\\s.t\\(n-1)\times s+p=k}}
												  \sum_q \mathsf{K}_{q,i,m,p}\mathsf{H}_{q,l,n} 
\end{equation}
\noindent
Forward-propagation is then referred to as the compositional evaluation of each layer. For the example in \cref{Eq. Composition} the outputs of each layer would then be:  \(f^1(\mathbf{x},\mathbf{\Theta}^1) = \mathbf{p}\), \(f^2(\mathbf{p},\mathbf{\Theta}^2) = \mathbf{q}\) and \(f^3(\mathbf{q},\mathbf{\Theta}^3) = \mathbf{\hat{y}}\).\\
Subsequently the cost function \(J(\Theta)\), which will be discussed later on, can be computed. Afterwards back-propagation returns the gradients w.r.t. the layer parameters \(\Theta\) to finally compute updated parameters \(\mathbf{\Theta}\).  The name back-propagation refers to the use of the chain rule of calculus to obtain the gradients of each layer. Assuming again the example in \cref{Eq. Composition} back-propagation would be equations \cref{Eq. Backpropagation1} to \cref{Eq. Backpropagation3}:
\begin{align}
	\frac{\partial J}{\partial \mathbf{\Theta}^3} &= \frac{\partial J}{\partial\hat{y}_i}\frac{\partial\hat{y}}{\partial \Theta^3}
\label{Eq. Backpropagation3}\\
	\frac{\partial J}{\partial \Theta^2} &= \frac{\partial J}{\partial\hat{y}}\frac{\partial\hat{y}}{\partial q}\frac{\partial q}{\partial \Theta^2}
\label{Eq. Backpropagation2}\\
	\frac{\partial J}{\partial \mathbf{\Theta}^1} &= \frac{\partial J}{\partial\hat{y}}\frac{\partial\hat{y}}{\partial q}\frac{\partial q}{\partial p}\frac{\partial p}{\partial \Theta^1}
\label{Eq. Backpropagation1}
\end{align}
While the term back-propagation is solely used for the method to compute the gradients for each layer in a backward fashion, meaning form the last layer to the first, the update of the parameters is done in an optimization step.
-L2-error as perfomance metrics
\subsection{Autoencoders}
Autoencoders have many hyperparameters determining their capability for compression and subsequent reconstruction. These parameters include : \textit{depth}, \textit{width of layers}, \textit{activation functions}, \textit{batch-size}, \textit{learning rate}, \textit{number of filters}, \textit{stride width}, \textit{kernel size}. Their finding is discussed in this section, for both $\hy$ and $\rare$.\\
For the fully connected autoencoder the order of determining the hyperparameters is as follows:
Depth $\rightarrow$ Hidden Units $\rightarrow$ Batch Size $\rightarrow$  Activation Fuctions.
For the convolutional autoencoder the order of determining the hyperparameters is as follows:
Depth $\rightarrow$ Channels $\rightarrow$ Batch Size $\rightarrow$ Activation Functions.
Both models are evaluated on the course of the training. Figures showing the training process for all experiments are provided in the appendix. After training the L2-Error\cref{labellist} is evaluated on the unshuffled, complete dataset as described in \cref{Sec: Data Sampling}.\\
Important to mention is that the difficulty of finding the right set of hyperparameters for MLPs led to an intensive search prior to the creation of this contribution. The difficulty lies in the fact that there is little systematic knowledge about how the hyperparameters interact in the model and answer to a variety of input data. Goodfellow et al. point out that with a combination of intuition, certain methods and first of all experience practitioners find hyperparameters that work well \cite{Goodfellow}. As for this thesis a list of the prior search is provided in \cref{Sec:AppendixA}. This list does not claim to cover the complete search and can by no means give enough insight to enable reproducability by another person.\\
For this reason an insight into the methods employed to find the set of hyperparameters used in this contribution are described in the following. 

The convolutional autoencoder architecture 1.1 is a composition of six convolutional and three fully conected layers, \cref{f_Conv}.
\begin{equation}
y_p = f_{C}^9(f_{C}^8(f_{C}^7(f_{F}^6(f_{F}^5(f_{F}^4(f_{C}^3(f_{C}^2(f_{C}^1(y_0)))))))))
\label{f_Conv}
\end{equation}