% !TEX root = master.tex

\documentclass[%
pagesize,oneside,% for thesis use : 
%twoside,
%titlepage=false,%
11pt,
%toc=listof,%toc=bibliography,%
%chapterprefix=true,%
%bibliography=totocnumbered%
]%{article}
{book}
\usepackage[english]{babel}

\usepackage{dsfont} %% for \1
\usepackage{exscale}
\usepackage{mathrsfs} %mathscr{}
\usepackage[T1]{fontenc}
\usepackage[utf8]{inputenc}
\usepackage{lmodern}
\usepackage{graphicx}
\usepackage[utf8]{inputenc}
\usepackage{footnote}
\usepackage{hyperref}
\usepackage{fullpage}
\usepackage{amsmath,amssymb,amsfonts}
\usepackage{multicol}
\usepackage{wrapfig}
\usepackage{float}
\usepackage{subcaption}
\usepackage{pgfplots}
\usepackage{pgfplotstable}
\usepackage{tikz}
\usetikzlibrary{external,positioning,backgrounds}
\tikzexternalize[prefix=Figures/]
\usetikzlibrary{shapes,arrows,snakes,chains}
\usetikzlibrary{matrix}
\usetikzlibrary{decorations}
\pgfplotsset{compat=newest}
\usepgfplotslibrary{groupplots}
\usepackage{tikzscale}
\usepackage{cleveref}
\usepackage{booktabs}
\usepackage{graphicx}
\usepackage{caption}
\usepackage{tocbibind}
\bibliographystyle{JHEP}

%%%%%%%%%%%%%%%%%%%%New Commands%%%%%%%%%%%%%%%%%%%%%%%%%%%%

\DeclareMathOperator*{\argmin}{\arg\!\min}
\DeclareMathOperator*{\hy}{\textbf{\textsf{H}}}
\DeclareMathOperator*{\rare}{\textbf{\textsf{R}}}
\DeclareMathOperator*{\idhy}{\textbf{\textsf{h}}}
\DeclareMathOperator*{\idrare}{\textbf{\textsf{r}}}
\DeclareMathOperator*{\pstar}{p^{\star}}
\DeclareMathOperator*{\Kn}{\mathit{Kn}}
\DeclareMathOperator*{\L2}{L_2}
\DeclareMathOperator*{\a1}{\alpha_{1}}
\DeclareMathOperator*{\a2}{\alpha_{2}}
\DeclareMathOperator*{\a3}{\alpha_{3}}
\DeclareMathOperator*{\mae}{\Pi_{FCNN}}
\DeclareMathOperator*{\mconv}{\Pi_{CNN}}
\DeclareMathOperator*{\mrare}{\Pi_{r}}
\DeclareMathOperator*{\mhy}{\Pi_{h}}
\DeclareMathOperator*{\cumu}{\Epsilon}




\newcommand{\colvec}[1]{\ensuremath{\begin{pmatrix}#1\end{pmatrix}}}
%\newcommand{\pod1}{\(\gamma_1\)}
%\newcommand{\pod2}{\(\gamma_2\)}
%\newcommand{\pod3}{\(\gamma_3\)}
%\newcommand{\1pod}{\(\delta_1\)}
%\newcommand{\2pod}{\(\delta_2\)}
%\newcommand{\3pod}{\(\delta_3\)}
%\newcommand{\conv1}{\(\beta_1\)}
%\newcommand{\conv2}{\(\beta_2\)}
%\newcommand{\conv2}{\(\beta_3\)}

%%%%%%%%%%%%%%%%%Tikz Styles%%%%%%%%%%%%%%%%%%%%%%%%%%%%%%%%%

\tikzstyle{block} = [rectangle, draw, 
text width=5em, text centered, rounded corners, minimum height=4em]
\tikzstyle{cube} = [rectangle, minimum width=1cm, minimum height=1cm, draw=black]
\tikzstyle{rec} = [rectangle,minimum height=4em,text centered, fill=blue!20,draw=black]
\tikzstyle{circ} = [circle,minimum size =0.4cm,text centered, draw=black,inner sep = 0pt]
\tikzstyle{arrow} = [thick,->,>=stealth]
\tikzstyle{base} = [draw, on chain, on grid, align=center, minimum height=4ex]
\tikzstyle{proc} = [base, rectangle, text width=8em]
\tikzstyle{test} = [base, diamond, aspect=2, text width=5em]
\tikzstyle{term} = [proc, rounded corners]
%%%%%%%%%%%%%%%%%%%%%%%Colors%%%%%%%%%%%%%%%%%%%%%%%%%%%%%%%%
\definecolor{color0}{rgb}{0.12156862745098,0.466666666666667,0.705882352941177}
%%%%%%%%%%%header einstellungen%%%%%%%%%%%%%%%%%%%
\usepackage{fancyhdr}
%\pagestyle{fancy}
% \fancyhead{}
% \fancyhead[ER]{\rightmark}
% \fancyhead[OL]{\leftmark} 

%\renewcommand{\sectionmark}[1]{\markboth{#1}{#1}} % set the \leftmark % set the \leftmark
%\fancyhf{}
%\fancyhead[EL,OR]{\thepage}
%\fancyhead[ER]{\rightmark}
%\fancyhead[OL]{\leftmark} 

%%%%%%%%%%%%%%%%%%%chapter logo %%%%%%%%%%%%%%%%%%%%%%%%%%5
\usepackage{titlesec,titletoc}

%Einstellung serifenlose überschriften
\titleformat{\chapter}[block]{\vspace{0cm}\sffamily \bfseries \LARGE}{\thechapter}{ 0.5cm}{ }[\vspace{-0.65cm}\rule{\textwidth}{1pt}]
\titleformat*{\section}{\sffamily \bfseries \Large}
\titleformat*{\subsection}{\sffamily \bfseries \large}
\titleformat*{\subsubsection}{\sffamily \bfseries \normalsize}

%%%%Keine Identierung
\setlength\parindent{0pt}




