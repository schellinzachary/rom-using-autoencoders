% !TEX root = master.tex


%-Eingabe der Metadaten des Titelblattes--------------------------

%-Daten des Autors / Authors Data---------------------------------

\newcommand{\dcauthorpre}{} 
\newcommand{\dcauthorsurname}{Schellin} 
\newcommand{\dcauthorname}{Zachary} 
\newcommand{\dcauthoradd}{11.02.1991, Berlin}

%-Titel und Untertitel / Title and subtitle-----------------------

\newcommand{\dctitle}{\LARGE\textsc{Using Autoencoders for reduced order modeling of the BGK model}} 
\newcommand{\dcsubtitle}{~}  
% Falls dcsubtitle NICHT verwendet werden soll, {\dcsubtitle}{~} eingeben.

%-Eingabe der Betreuuernahmen / Names of the consultants---------

\newcommand{\dcconsulta}{Prof.~Dr.~Julius Reiss} 
\newcommand{\dcconsultb}{Dr.~Stefan Schaefer}

%-Eingabe der Gutachternamen / Names of the approvals-------------

\newcommand{\dcapprovala}{Prof.~Dr.~Julius Reiss} 
\newcommand{\dcapprovalb}{Dr.~Stefan Schaefer} 
\newcommand{\dcapprovalc}{} 

%-Information zur Universitaet------------------------------------

\newcommand{\dcdegree}{Bachelor of Science (B. Sc.)} 
\newcommand{\dcsubject}{Physikalische Ingenieurswissenschaften} 
\newcommand{\dcfaculty}{Fakultät Verkehrs- und Maschinensysteme V}
\newcommand{\dcinstitute}{Institut für Numerische Fluiddynamik}
\newcommand{\dcuniversity}{Technische Universität Berlin}
\newcommand{\dcdean}{}
\newcommand{\dcpresident}{}

%-Pruefungsdaten: eingereicht und mdl. Pruefung-------------------
%-data of submission and oral exam--------------------------------

\newcommand{\dcdatesubmitted}{19. Februar 2021} %auch wenn nicht auf dem Titelblatt, bitte erf�llen!
\newcommand{\dcdateexam}{} 

%-deutsche Schlagwoerter / german keywords------------------------

\newcommand{\dckeydea}{Schlagwort 1}
\newcommand{\dckeydeb}{Schlagwort 2}
\newcommand{\dckeydec}{Schlagwort 3}
\newcommand{\dckeyded}{Schlagwort 4}

% Folgende Zeile bitte nicht aendern!
\newcommand{\dckeywordsde}{\vfill \raggedright {\textbf{Schlagw\"orter:}}\\ \dckeydea, \dckeydeb, \dckeydec, \dckeyded \\}

%-englische Schlagwoerter / english keywords----------------------

\newcommand{\dckeyena}{keyword 1}
\newcommand{\dckeyenb}{keyword 2}
\newcommand{\dckeyenc}{keyword 3}
\newcommand{\dckeyend}{keyword 4}

% Folgende Zeile bitte nicht aendern!
\newcommand{\dckeywordsen}{\vfill \raggedright {\textbf{Keywords:}}\\ \dckeyena, \dckeyenb, \dckeyenc, \dckeyend \\}

\newcommand{\dcpdfsubject}{Dissertation}