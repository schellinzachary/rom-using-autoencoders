%flowchart training
\begin{tikzpicture}[%
 >=triangle 60,              
start chain=going below,   
node distance=6mm and 60mm, 
every join/.style={norm},
]
\tikzset{
	base/.style={draw, on chain, on grid, align=center, minimum height=4ex},
	proc/.style={base, rectangle, text width=8em},
	test/.style={base, diamond, aspect=2, text width=5em},
	term/.style={proc, rounded corners},
	% coord node style is used for placing corners of connecting lines
	coord/.style={coordinate, on chain, on grid, node distance=6mm and 25mm},
	% nmark node style is used for coordinate debugging marks
	nmark/.style={draw, cyan, circle, font={\sffamily\bfseries}},
	% -------------------------------------------------
	% Connector line styles for different parts of the diagram
	norm/.style={->, draw},
	free/.style={->, draw},
	cong/.style={->, draw},
	it/.style={font={\small\itshape}}
}
	% Start by placing the nodes
	\node [proc] (start) {\footnotesize \(Epoch = 0\)\\ \(Batch=0\)};
	% Use join to connect a node to the previous one 
	\node [test, join]      {\footnotesize \(Epoch < Epoch_{max}\)};
	\node [proc, join] (p1) {Get quota $k > 1$};
	\node [proc, join]      {Open queue};
	\node [proc, join]      {Dispatch message};
	\node [test, join] (t1) {Got msg?};
\end{tikzpicture}