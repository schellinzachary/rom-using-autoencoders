% This file was created by tikzplotlib v0.9.6.
\begin{tikzpicture}
\definecolor{color0}{rgb}{0.12156862745098,0.466666666666667,0.705882352941177}

\begin{groupplot}[group style={group size=2 by 1,horizontal sep=1.8cm}]
\nextgroupplot[
legend cell align={left},
legend style={draw=none,at={(0.03,0.03)}, anchor=south west},
log basis y={10},
tick align=outside,
tick pos=left,
x grid style={white!69.0196078431373!black},
xmajorgrids,
xmin=-0.55, xmax=33.55,
xminorgrids,
xtick style={color=black},
y grid style={white!69.0196078431373!black},
ymajorgrids,
ymin=4.38349387313967e-16, ymax=1.61134858880557,
yminorgrids,
ymode=log,
ytick style={color=black},
xlabel={Number of intrinsic varibales},
ylabel={\(L_2\)-Error},
width=0.47\textwidth,
height =.45\textwidth,
clip=false,
y label style={yshift=-.7em}
]
\addplot [semithick, red, mark=o, mark size=2, mark options={solid}, dashed]
table {%
1 0.188112310801957
2 0.0750338979596223
3 0.020528730333796635
4 0.00808627149114823
8 0.000193252431896578
16 3.06183124046159e-08
32 2.23530701528268e-15
};
\addlegendentry{POD}
\addplot [semithick, color0, mark=pentagon, mark size=2, mark options={solid}, only marks]
table {%
1 0.00824632961302996
2 0.0060168607160449
3 0.0016505243
4 0.00198765122331679
8 0.00124555476941168
16 0.00101344427093863
32 0.000832061574328691
};
\addlegendentry{FCNN}
\addplot [semithick, green!50!black, mark=triangle, mark size=2, mark options={solid,rotate=180}, only marks]
table {%
1 0.315989553928375
2 0.0128393778577447
3 0.00947935
4 0.0101089663803577
8 0.00921880733221769
16 0.00887134857475758
32 0.00860222987830639
};
\addlegendentry{CNN}
\draw[thick](3,40e-17)--(3,1.5);
\draw [thick,<-] (axis cs:3,1.5)-- +(10pt,10pt) node[right] {\(p^\star\)};
\nextgroupplot[
legend cell align={left},
legend style={draw=none, at={(0.03,0.03)}, anchor=south west},
log basis y={10},
tick align=outside,
tick pos=left,
x grid style={white!69.0196078431373!black},
xmajorgrids,
xmin=-0.55, xmax=33.55,
xminorgrids,
xtick style={color=black},
y grid style={white!69.0196078431373!black},
ymajorgrids,
ymin=1.70008814466799e-15, ymax=0.821373329691319,
yminorgrids,
ymode=log,
ytick style={color=black},
xlabel={Number of intrinsic varibales},
ylabel={\(L_2\)-Error},
width=0.47\textwidth,
height =.45\textwidth,
clip=false,
y label style={yshift=-.7em}
]
\addplot [semithick, red, mark=o, mark size=2, mark options={solid}, dashed]
table {%
1 0.176637499442346
2 0.0853532495802733
4 0.015335605212791
5 0.008731715326052242
8 0.00145958547754045
16 3.54159334428613e-07
32 7.90549608414532e-15
};
\addlegendentry{POD}
\addplot [semithick, color0, mark=pentagon, mark size=2, mark options={solid}, only marks]
table {%
1 0.00949728023260832
2 0.00740677583962679
4 0.0032303836196661
5 0.0018827654
8 0.00190045870840549
16 0.00116818200331181
32 0.00140941923018545
};
\addlegendentry{FCNN}
\addplot [semithick, green!50!black, mark=triangle, mark size=2, mark options={solid,rotate=180}, only marks]
table {%
1 0.150382950901985
2 0.0282952953130007
4 0.00750687718391418
5 0.009697513
8 0.00584819912910461
16 0.00925309397280216
32 0.00917380768805742
};
\addlegendentry{CNN}
\draw[thick](5,16.5e-16)--(5,0.85);
\draw [thick,<-] (axis cs:5,0.85)-- +(10pt,10pt) node[right] {\(p^\star\)};
\end{groupplot}

\end{tikzpicture}
