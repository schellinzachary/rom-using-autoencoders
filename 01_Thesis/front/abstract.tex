% !TEX root = master.tex


%-englische-Zusammenfassung---------------------------------------
%
%
{
\pagestyle{empty}
\begin{center}
{\sffamily \bfseries\Large Abstract}\\
\end{center}%\hspace*{\fill}\\[1.5cm]
\vspace{1cm}
%
%\addcontentsline{toc}{chapter}{Abstract}
%\setcounter{page}{2} % Nach Bedarf anpassen!
Neural networks, in particular autoencoders, in the context of model order reduction (MOR) of rarefied gases are evaluated against state-of-the-art proper orthogonal decomposition (POD). Both methods are examined on the solution of the BGK model in Sod's shock tube for a continuum- and a slightly rarefied gas. Therefore, a convolutional neural network (CNN) as well as a fully connected neural network (FCNN) are designed and the finding of appropriate hyperparameters is discussed. The FCNN surpasses POD only in the number of parameters used to achieve reconstruction losses of \num{8e-4} and \num{9e-4} for the two flows by a factor of 5.6 and 6.7 respectively. The CNN fails to reproduce usable results, giving rise to further the exploration chosen hyperparameters. Moreover, the similarity between macroscopic quantities and intrinsic variables extracted from the FCNN demonstrate physical interpretabillity. Beyond that, the generation of new snapshots of solutions of the flow succeeded using the decoder of the FCNN through a temporal interpolation in the intrinsic variables.
\vspace{1.5cm}

%-deutsche Zusammenfassung----------------------------------------
%
\begin{center}
{\sffamily \bfseries\Large Zusammenfassung}\\
\end{center}%\hspace*{\fill}\\[1.5cm]
\vspace{1cm}
%
Neuronale netzte im speziellen Autoencoder und proper orthogonal decomposition (POD), dem Stand der Technik, werden im Kontext von Modellreduktion einander gegenübergestellt. Beide Methoden werden an Lösungen der BGK-Gleichung in Sods Sto\ss rohr für eine kontinuierliche- und eine leicht verdünnte Gasströmung untersucht. Hierfür ist ein Convolutional Neural Network (CNN) und ein Fully Connected Neural Network (FCNN) designt und das Auffinden von Hyperparametern diskutiert. Es wird gezeigt, dass das FCNN die POD lediglich in der Anzahl an verwendeten Parametern für Rekonstruktionsfehler von \num{8e-4} und \num{9e-4} übertrifft. Das CNN schafft es nicht, brauchbare Ergebnisse zu erzielen. Dies wirft jedoch die Frage nach geeigneteren Hyperparametern auf und motiviert für eine weitere Untersuchung dieser. Auch kann die physikalische Interpretierbarkeit der intrinsischen Variablen anhand der Ähnlichkeit zu Makroskopischen Grö\ss en gezeigt werden. Darüber hinaus können mithilfe des Decoders neue Zeitschritte der Lösung durch eine zeitliche Interpolation in den intrinsischen Variablen erzeugt werden.    
% hier werden die deutschen Schlagwörter aus Metadaten übernommen
%\dckeywordsde

\newpage
\pagestyle{plain}
}